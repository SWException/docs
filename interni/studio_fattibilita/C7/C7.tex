\section{Studio C7}

Il capitolato è stato proposto da Zextras una softwarehouse dedita allo sviluppo software rivolto principalmente ad utilizzatori business.
Sviluppa inoltre alcuni moduli impiegati nella versione a pagamento del famoso server di posta elettronica “Zimbra”

\subsection{Informazioni generali}
\begin{itemize}
	\item \textbf{Nome:} Soluzioni di sincronizzazione desktop
	\item \textbf{Proponente$_G$:} Zextras
	\item \textbf{Committente$_G$:} Prof. Tullio Vardanega e Prof. Riccardo Cardin
\end{itemize}

\subsection{Descrizione del capitolato}
Il capitolato propone lo sviluppo di un applicativo client-server in grado di sincronizzare in varie postazioni file e cartelle selezionati dall’utente ovvero una soluzione ai più noti Dropbox e Google Drive ma dedicata alla sincronizzazione nel cloud di Zextras Drive il quale offre una perfetta integrazione con Zimbra.

\subsection{Finalità del progetto}
Sviluppo di un’interfaccia multipiattaforma per l’utilizzo di un algoritmo solido ed efficiente in grado di garantire il salvataggio in cloud di file e cartelle selezionate.
Si richiede inoltre di visualizzare i cambiamenti al contenuto sincronizzato attraverso la connessione a Zextras Drive. E’ requisito fondamentale che la soluzione sviluppata non dipenda da framework terzi.


\subsection{Tecnologie interessate}
Il proponente consiglia di utilizzare Qt Framework basato su C++ per lo sviluppo dell’interfaccia e Python per il Backend

\subsection{Aspetti positivi}
Un aspetto sicuramente positivo è l’utilizzo del framework Qt unitamente a C++  per lo sviluppo dell’interfaccia grafica in quanto già noto ai componenti del gruppo.


\subsection{Criticità e fattori di rischio}
Tra i fattori di rischio emerge la necessità di dover sviluppare un applicativo che deve operare con files e cartelle residenti nel fileystem locale della macchina in diversi sistemi operativi.
La soluzione sviluppata deve inoltre interfacciarsi con un applicativo già sviluppato e questo potrebbe causare delle difficoltà nella fase di sviluppo.

\subsection{Conclusioni}
Il capitolato proposto benchè si differenzi particolarmente rispetto agli altri in quanto orientato ad un applicativo client-server che lavora su files e cartelle locali non ha particolarmente attirato l’attenzione dei componenti del gruppo. Sia per il contesto d'utilizzo verso cui è orientato l’applicativo richiesto che per le tecnologie impiegate.

\newpage
