%%%%%%%%%%%%%%%%%%%%%%%%%%%%%% Sets the document class for the document
% Openany is added to remove the book style of starting every new chapter on an odd page (not needed for reports)
\documentclass[12pt,italian,a4paper]{article}
%%%%%%%%%%%%%%%%%%%%%%%%%%%%%% Loading packages that alter the style
\usepackage[]{graphicx}
\usepackage[]{color}
\usepackage{alltt}
\usepackage[T1]{fontenc}
\usepackage[utf8]{inputenc}

% Math
\usepackage{amsmath}
\usepackage{amssymb}

% Fonts
\renewcommand{\rmdefault}{ptm}
\renewcommand{\sfdefault}{phv}


\setcounter{secnumdepth}{3}
\setcounter{tocdepth}{3}
\setlength{\parskip}{\smallskipamount}
\setlength{\parindent}{12pt}
\usepackage{indentfirst}

% Set page margins
\usepackage[top=80pt,bottom=60pt,left=78pt,right=78pt]{geometry}
\usepackage{subcaption}
% Package used for placeholder text
\usepackage{lipsum}
\usepackage{booktabs}
\usepackage{multirow}
% Prevents LaTeX from filling out a page to the bottom
\raggedbottom

% Adding both languages
\usepackage[italian]{babel}

% All page numbers positioned at the bottom of the page
\usepackage{fancyhdr}
\fancyhf{} % clear all header and footers
\fancyfoot[C]{\thepage}
%\fancyhead[L]{{\selectfont \leftmark}}
\renewcommand{\headrulewidth}{0pt} % remove the header rule
\pagestyle{fancy}

% Changes the style of chapter headings
\usepackage{titlesec}
\titleformat{\chapter}
{\normalfont\LARGE\bfseries}{\thechapter.}{1em}{}
% Change distance between chapter header and text
\titlespacing{\chapter}{0pt}{40pt}{2\baselineskip}

% Adds table captions above the table per default
\usepackage{float}
\floatstyle{plaintop}
\restylefloat{table}

% Adds space between caption and table
\usepackage[tableposition=top]{caption}

% add cc license
\usepackage[
type={CC},
modifier={by-nc-sa},
version={4.0},
]{doclicense}

% Adds hyperlinks to references and ToC
\usepackage[backref=page]{hyperref} % hyperlinks
\renewcommand*{\backref}[1]{}
\renewcommand*{\backrefalt}[4]{{\footnotesize [%
		\ifcase #1 Not cited.%
		\or Cited on page~#2%
		\else Cited on pages #2%
		\fi%
		]}}
\usepackage[capitalise]{cleveref}

% Uncomment the line below this block to set all hyperlink color to black
\hypersetup{
	colorlinks,
	linkcolor={blue},
	citecolor={green!90!black},
	urlcolor={red!70!black}
}
%\hypersetup{hidelinks,linkcolor = black} % Changes the link color to black and hides the hideous red border that usually is created

% Set specific color for hyperref
\usepackage{xcolor}


% tcolorbox; Notice! add "-shell-escape" to the compile command
\usepackage{tcolorbox}

% If multiple images are to be added, a folder (path) with all the images can be added here 
\graphicspath{ {figures/} }

% Separates the first part of the report/thesis in Roman numerals
%\frontmatter
\patchcmd{\chapter}{\thispagestyle{plain}}{\thispagestyle{fancy}}{}{}{}
% Uncomment to stop the new chapter start at a new page
%\usepackage{etoolbox}
%\makeatletter
%\patchcmd{\chapter}{\if@openright\cleardoublepage\else\clearpage\fi}{}{}{}
%\makeatother

%%%%%%%%%%%%%%%%%%%%%%%%%%%%%% Starts the document
\begin{document}
	
	%%% Selects the language to be used for the first couple of pages
	\selectlanguage{italian}
	
	%%%%% Adds the title page
	\begin{titlepage}
		\clearpage\thispagestyle{empty}
		\centering
		\vspace{1cm}
		
		% Titles
		% Information about the University
		{\
			\textsc{Ingegneria del Software - Università degli Studi di Padova}
		}
		\vspace{2.5cm}
		
		% INSERIRE QUI TITOLO DOCUMENTO
		\rule{\linewidth}{2mm} \\[0.8cm]
		{ \LARGE \sc Verbale Interno - 7 dicembre 2020}\\[0.55cm]
		\rule{\linewidth}{0.6mm} \\[3.4cm]
		
		\hspace{2cm}
		% aggiungere i campi necessari per ogni documento
		\begin{tabular}{l p{5cm}}
			\textbf{Gruppo} & SWException \\[10pt]
			\textbf{Studenti} & Marco Canovese \\ & Nicole Davanzo \\ & Ivan Furlan \\ & Gianmarco Guazzo \\ & Stefano Lazzaroni \\ & Francesco Trolese \\ & Michele Veronesi \\[10pt]
			\textbf{Email} & \texttt{swexception@outlook.com} \\[10pt]
			\textbf{Data} & \today \\
			\textbf{Redattore} & Michele Veronesi       
		\end{tabular}
		
		
		\vfill
		\centering \includegraphics[scale=0.17]{logo.jpg}
	\end{titlepage}
	%\tableofcontents
	
	\newpage
	
	\pagenumbering{arabic}
	
	\section*{Ordine del giorno}
	In data 7 dicembre 2020 si è tenuto un meeting con tutti i componenti del gruppo, al fine di discutere i seguenti punti:
	\begin{enumerate}
		\item Scelta definitiva del capitolato per cui concorrere nella gara d'appalto alla Revisione di Accettazione
		\item Caratteristiche del documento interno \textit{Studio di Fattibilità}
		\item Suddivisione del carico di lavoro per la produzione del documento contenente le \textit{Norme di Progetto}
		\item Stesura del Glossario
		\item Varie ed eventuali
	\end{enumerate}
	
	\section*{Punto 1}
	La scelta dei componenti del gruppo è ricaduta all'unanimità sul capitolato C2 dell'azienda RedBabel, \textit{EmporioLambda: piattaforma di e-commerce in stile Serverless}.\\
	Di conseguenza le attività inerenti al progetto didattico verteranno d'ora in poi all'aggiudicazione di tale capitolato alla Revisione di Accettazione del 18 Gennaio 2021.\\
	Maggiori informazioni sulla motivazione di tale scelta saranno presenti nello Studio di Fattibilità, tutt'ora in fase di redazione.
	
	\section*{Punto 2}
	È stata definita la struttura del documento Studio di Fattibilità, visto che il carico di lavoro per la sua stesura era già stato definito con la riunione verbalizzata nel \verb|vInterno-02-12-2020|.\\
	In particolare, per ogni capitolato in esame dovranno essere riportate le seguenti informazioni inerenti al suo studio:
	\begin{itemize}
		\item informazioni generali
		\item descrizione del capitolato
		\item finalità del progetto
		\item tecnologie interessate
		\item aspetti positivi
		\item criticità e fattori di rischio
		\item conclusioni
	\end{itemize}
	
	\section*{Punto 3}
	È stato suddiviso il carico di lavoro per la stesura delle Norme di Progetto in questo modo:
	\begin{itemize}
		\item \textit{processi primari:} incaricati Stefano Lazzaroni e Marco Canovese
		\item \textit{processi organizzativi:} incaricati Ivan Furlan e Gianmarco Guazzo
		\item \textit{processi di supporto:} incaricati Francesco Trolese, Nicole Davanzo e Michele Veronesi
	\end{itemize}
	La deadline stabilita per la consegna di una prima bozza del documenti è il giorno 16 dicembre 2020. Il giorno successivo inizierà il processo di verifica di tale documento in seguito ad un meeting apposito.

	\section*{Punto 4}
	Per quanto riguarda la stesura del glossario, è stato stabilito che ogni componente è incaricato ad integrarlo man mano che vengono redatte le parti di loro competenza di tutti gli altri documenti.
	
	\section*{Punto 5}
	Per quanto riguarda la deadline del 13 dicembre 2020 per la stesura dello Studio di Fattibilità definita nel \verb|vInterno-02-12-20| è stato fissato un ulteriore meeting per l'avvio della verifica di tale documento il giorno 14 dicembre 2020.
	
	\section*{Conclusioni}
	In seguito alla riunione l'intero gruppo ha preso contatti con il proponente del capitolato C2 scelto tramite l'applicazione Slack$_G$ (come specificato nel documento di introduzione). È stata quindi fissata una riunione con i rappresentanti dell'azienda RedBabel il giorno 10 dicembre 2020 alle ore 17 CET con tutti i componenti del gruppo \verb|SWException|.
	
\end{document}
