\subsection{Gestione di processo}
\subsubsection{Scopo}
Secondo lo standard ISO/12207-1995 la gestione di processo contiene le attività e i compiti generici che vengono impiegati per gestire i vari processi.\\
Il Responsabile di progetto è responsabile della gestione di: del prodotto, del progetto, e delle attività o processi ad esso applicabili.
\subsubsection{Aspettative}
Le aspettative per questo processo sono:
\begin{itemize}
    \item coordinare e semplificare la comunicazione tra i membri del team, assegnando alle singole componenti ruoli e compiti da svolgere;
    \item avere il controllo sul progetto, monitorando il lavoro del team ed in che stato sono i compiti da svolgere;
    \item avere una buona pianificazione delle attività da seguire.
\end{itemize}
\subsubsection{Descrizione}
Questo processo consiste nelle seguenti attività:
\begin{itemize}
    \item iniziazione e definizione dello scopo;
    \item inizializzazione dei processi;
    \item pianificazione di risorse, tempi e costi; 
    \item assegnazione dei ruoli e dei compiti;
    \item esecuzione e controllo;
    \item revisione e valutazione.    
\end{itemize}
\subsubsection{Coordinamento}
L'attività di coordinamento consente di gestire la comunicazione sia interna che esterna, e le riunioni. Ciò viene fatto attraverso comportamenti precisi che ogni membro del gruppo deve avere per l'intera durata del processo.
\paragraph{Comunicazioni}
Oltre che la comunicazione all'interno del gruppo (Comunicazione interna), durante lo sviluppo del progetto il team si troverà a comunicare con diversi attori esterni:
\begin{itemize}
    \item proponente: azienda
    \item committenti: Prof. e Prof.
\end{itemize}
\subparagraph{Comunicazione interna}
Viene utilizzata come piattaforma principale Slack per comunicazioni che abbiano lo scopo di coordinamento o decisionale, mentre per comunicazioni meno importanti e più informali può essere utilizzato un gruppo Telegram.
Le discussioni su Slack dovranno essere svolte all'interno del canale appropriato, e qual ora si ritenga necessario crearne uno nuovo va fatta segnalazione al Responsabile di progetto che provvederà a valutarne la creazione.\\
I canali di discussione sono organizzati nel seguente modo:
\begin{itemize}
\item un canale per ogni documento dove si discute del contenuto dello stesso;
\item {\#glossario:} per discutere ciò da aggiungere al glossario e delle relative definizioni;
\item {\#scadenze:} per scrivere le scadenze future;
\item {\#proponente:} discutere di argomenti che probabilmente avranno un coinvolgimento del proponente per porgli delle domande;
\item {\#varie:} canale per avere comunicazioni di carattere generale.
\end{itemize}
Un'altra piattaforma di cui ci si avvale è Discord, la quale attraverso le chat vocali può essere utilizzata per svolgere dei compiti o attività con altri membri. In caso di problemi è possibile utilizzare Zoom, in quanto applicazione utilizzata e conosciuta da tutto il team.
\subparagraph{Comunicazione esterna}
Il gruppo è provvisto di un indirizzo email che verrà utilizzato per comunicazioni esterne, il particolare per comunicazioni verso i committenti. Il controllo e l'utilizzo della casella di posta elettronica è vincolato al solo Responsabile di progetto, il quale ha anche il dovere di informare gli altri membri alla ricezione di eventuali comunicazioni.
I membri del team sono altresì presenti all'interno dello Slack del proponente. I canali in questo caso sono due:
\begin{itemize}
	\item {\#swe-2020\_2021:} canale di discussione in cui sono presenti anche gli altri gruppi che svolgono lo stesso capitolato
	\item {\#swexception:} canale del gruppo in cui poter fare domande al proponente.
\end{itemize}
\paragraph{Riunioni}
Per ogni riunione, sia essa interna od esterna, verrà nominata una persona incaricata di prendere appunti e far rispettare l'ordine del giorno. Al termine della riunione avrà anche il compito di redarre il verbale e farlo avere agli altri membri entro breve tempo.
\subparagraph{Riunioni interne}
È consentita la partecipazione ai soli membri del team, e per essere svolta deve essere presente più del 50\% delle componenti del gruppo. Vengono svolte su Discord in una chat vocale dedicata.
\subparagraph{Riunioni esterne}
Sono svolte con i proponenti o committenti, e si terranno su GMeet per riunioni con i proponenti e su Zoom per quelle con i committenti. Il link per il collegamento verrà comunicato dai soggetti esterni nel momento in cui si pianifica un incontro con essi.
\subsubsection{Pianificazione}

\paragraph{Ruoli di progetto}
\subparagraph{Responsabile di Progetto}
\subparagraph{Amministratore di Progetto}
\subparagraph{Analista}
\subparagraph{Progettista}
\subparagraph{Programmatore}
\subparagraph{Verificatore}
\paragraph{Ciclo di vita di un'attività}
\paragraph{Assegnazione dei compiti}
\paragraph{Metriche di processo}