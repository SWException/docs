\section{Processi supporto}
\subsection{Documentazione}
\subsubsection{Scopo}
\subsubsection{Modalità}
Il team ha deciso di eseguire la stesura della documentazione grazie l'ausilio di \latex. Abbiamo optato per l'utilizzo in locale da parte di ogni membro dell'editor Texmaker v5.0.4 e come compilatore TexLive 2020.
\subsubsection{Compilazione}
Per la compilazione di un qualsiasi documento del progetto è necessario aprire tramite Texmaker il file main.tex della documentazione che si vuole produrre. Successivamente, assicurandosi che il menu a tendina relativo al compilatore da usare sia settato a "Compilazione rapida", tramite lo shortcut F1 (su piattaforme Windows e Linux) si procede alla compilazione di tutto il documento.
\subsubsection{Tipologie}
Ogni documento ha un preciso ruolo all'interno della documentazione. E' però necessario suddividerli in aree distinte:
