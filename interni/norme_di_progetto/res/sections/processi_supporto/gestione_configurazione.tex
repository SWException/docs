\subsection{Gestione della configurazione}
\subsubsection{Scopo}
L'obiettivo del processo è quello di regolare la produzione di documenti e codice sorgente.
Dal momento che la documentazione viene scritta in \LaTeX, come descritto nella sezione relativa
alla documentazione, quest'ultima può essere gestita con alcuni degli strumenti con
cui viene gestito il codice sorgente.

\subsubsection{Descrizione}
Tale processo descrive tutti gli strumenti utilizzati per la produzione di documenti, codice e diagrammi,
oltre che le modalità di versionamento e coordinamento del gruppo.

\subsubsection{Aspettative}
I risultati attesi da questo processo sono:
\begin{itemize}
    \item sistematizzare la produzione di codice e documentazione
    \item rendere uniforme l'utilizzo degli strumenti di versionamento coinvolti nel progetto
    \item classificare i prodotti dei vari processi implementati
\end{itemize}

\subsubsection{Versionamento}
\paragraph{Codice di versione per documenti e software}
Il numero di versione di ogni componente del prodotto e di ogni documento è definito nel seguente formato:
\begin{center}
    \textbf{X.Y.Z}
\end{center}
dove
\begin{itemize}
    \item \textbf{X} indica il numero di versione approvato dal \glock{Responsabile di Progetto}. Inizialmente è a 0.
    \item \textbf{Y} indica il numero di versione approvato dai \glock{Verificatori}. Inizialmente a 0 e riparte ad ogni
                       incremento di \textbf{X}.
    \item \textbf{Z} è il numero che identifica la versione in redazione. Viene quindi incrementato dai redattori
                       ad ogni aggiunta/modifica. Inizialmente a 0 e riparte ad ogni incremento di \textbf{Z}
\end{itemize}

\paragraph{Tecnologie coinvolte}
Per il processo di versionamento ci si affiderà al software \glock{Git} offerto come servizio dalla piattaforma \glock{GitHub}.\\
In particolare, è stata creata un'organizzazione su tale piattaforma con il nome \verb|SWException| in cui tutti i membri del gruppo
hanno il medesimo accesso in scrittura e lettura per tutte le repository esistenti.\\
Per il versionamento della documentazione viene usata una repository dal nome \verb|swe-docs|, in cui vengono riposti tutti i
documenti creati durante lo svolgimento del progetto didattico.\\
Successivamente, visto che si andrà a creare un'applicazione a \glock{microservizi}, e dunque composta da parti fortemente indipendenti,
verrà creata una repository per ogni modulo del prodotto.

\paragraph{Struttura delle repository}
Il lavoro in tutte le repository dovrà rispettare il \glock{GitFlow}, con o senza apposito plugin. In particolare saranno presenti i
seguenti branch:
\begin{itemize}
    \item \textbf{master:} utilizzato per ospitare le versioni di rilascio. In questo ramo dovrà esserci sempre una versione del prodotto utilizzabile
                           e pronto ad essere rilasciato in \glock{ambiente di produzione}.
    \item \textbf{develop:} questo è il ramo utilizzato per le operazioni di sviluppo. In particolare qui convergeranno i vari \verb|feature-branch|
                            utilizzati per la creazione delle funzionalità del modulo in oggetto. In questo ramo deve essere presente sempre e solo codice
                            compilabile.
    \item \textbf{feature/nome-feature:} per la creazione di ogni feature da parte dei vari membri addetti. In questi rami 
\end{itemize}