\section{Processi supporto}
\subsection{Documentazione}
%aggiunto da @michele creando lo scheletro
% \subsubsection{Modalità}
% Il team ha deciso di eseguire la stesura della documentazione grazie l'ausilio di latex. Abbiamo optato per l'utilizzo in locale da parte di ogni membro dell'editor Texmaker v5.0.4 e come compilatore TexLive 2020.
% \subsubsection{Compilazione}
% Per la compilazione di un qualsiasi documento del progetto è necessario aprire tramite Texmaker il file main.tex della documentazione che si vuole produrre. Successivamente, assicurandosi che il menu a tendina relativo al compilatore da usare sia settato a "Compilazione rapida", tramite lo shortcut F1 (su piattaforme Windows e Linux) si procede alla compilazione di tutto il documento.
% \subsubsection{Tipologie}
% Ogni documento ha un preciso ruolo all'interno della documentazione. E' però necessario suddividerli in aree distinte:

\subsubsection{Scopo}
In questo capitolo ci occuperemo di illustrare tutte le regole e gli standard che disciplinano la documentazione. Questo permetterà che ogni documento risulti dal punto di vista grafico e organizzativo coerente.
\subsubsection{Descrizione}
Le prossime sezioni illustreranno nei dettagli le norme di redazione di ogni documento.
Le regole di seguito riportate devo essere seguite da tutti i membri del gruppo durante tutte le fasi di lavoro.

\subsubsection{Aspettative}
Il team si aspetta che questo processo porti a una redazione ottimale della documentazione necessaria per lo svolgimento del progetto.

\subsubsection{Ciclo di vita}
Ogni documento passa 3 fase  diverse nel suo ciclo di vita:
\begin{itemize}
\item\textbf{redazione}: si tratta della vera e propria creazione del documento, il quale deve rispettare tutte le norme imposte in questo documento, partendo dalla struttura del template alla modalità di scrittura. Verranno incaricati dei redattori che si occuperanno di questo compito.
\item\textbf{verifica}: in questa fase vengono assegnati dei verificatori che hanno il compito di controllare che il lavoro fatto dai redattori sia coerente con le norme e gli obiettivi riguardanti la tipologia di documento redatto. I verificatori hanno inoltre il compito di riferire al responsabile di progetto le eventuali modifiche da apportare. Il responsabile a sua volta avrà il compito di informare i redattori su quali punti è necessario apportare delle migliorie. 
\item\textbf{approvazione}:il responsabile di progetto deve approvare il documento inviatogli con esito positivo dai verificatori e procedere al rilascio di esso.
\end{itemize}
Un ulteriore fase che un documento può riportare nel proprio ciclo di vita è il \textbf{versionamento}, ovvero nel caso in cui siano necessarie delle modifiche dopo il rilascio del documento da parte del responsabile. Nel caso in cui questo avvenga il documento modificato deve in seguito rivivere le fasi di verifica e approvazione.


\subsubsection{Classificazione dei documenti}
I documenti prodotti dal team sono di due tipologie:
\begin{itemize}
\item\textbf{Formali}: sono tutti quei documenti che richiedo la verifica e l'approvazione da parte del responsabile di progetto. 
\item\textbf{Informali}: sono quelli che permettono ai membri del gruppo di condividere informazioni sulle decisioni prese.
\end{itemize}
I documenti formali e informali sono a loro volta suddivisi in:
\begin{itemize}
\item\textbf{Interni}
\item\textbf{Esterni}
\end{itemize}

\paragraph{Documenti Formali} 
I documenti interni sono tutti quelli che interessano i membri del team e che gli aiutano nelle scelte e nella programmazione della redazione dei successivi documenti. 
Tra questi documenti troviamo:
\begin{itemize}
\item\textbf{Studio di fattibilità}: vengono presentati brevemente i diversi \glock{capitolati} proposti, indicando per ognuno la sua finalità, le tecnologie da utilizzare, gli aspetti positivi e le criticità. Viene inoltre esposto la motivazione della decisione del capitolato;
\item\textbf{Norme di progetto}: si tratta del seguente documento, esso contiene tutte le norme e le regole che tutti i membri del team devono seguire.
\end{itemize}
Quelli esterni invece sono di interesse per \glock{committenti} e \glock{proponente}, infatti vengono consegnati loro nell'ultima versione.
Tra questi troviamo:
\begin{itemize}
\item\textbf{Piano di Progetto}: %da specificare più avanti in modo da non scrivere cose sbagliate
\item\textbf{Piano di Qualifica}:%da specificare più avanti in modo da non scrivere cose sbagliate
\item\textbf{Glossario}: contiene un elenco di tutti quei termini che ricorrono nei documenti e che necessitano di una definizione esplicita;
\item\textbf{Analisi dei Requisiti}:%da specificare più avanti in modo da non scrivere cose sbagliate
\end{itemize}
Quando si tratta di documenti formali può accadere che questi abbiano più versioni, in questo caso si considera come corrente quella approvata dal Responsabile del progetto più recentemente.

\paragraph{Documenti Informali} 
I documenti che appartengono a questa tipologia sono i \textbf{verbali}.
Questa documentazione non necessita la verifica e l'approvazione da parte del Responsabile di progetto, quindi viene redatta una sola volta.\\
I verbali \textbf{interni} sono di interesse per i componenti del team, aiutano a ricapitolare le decisioni prese durante i meeting.\\
Quelli \textbf{esterni} invece riguardano le riunioni a cui partecipano anche i committenti e/o il proponente.

\subsubsection{Directory di un documento}
La documentazione è suddivisa in diverse directory, ognuna delle quali riporta il nome del documento contenuto al suo interno. Queste directory sono a loro volta contenute e suddivise nella directory \textit{Interni} e in quella \textit{Esterni}, che stanno ad indicare la tipologia di documento.

\subsubsection{Struttura di un documento}
\paragraph{Template} 
Per la stesura della documentazione il team ha deciso di utilizzare \glock{LATEX}, che permette di organizzare efficacemente le informazioni e allo stesso tempo di standardizzare le norme tipografiche e di formattazione.

\paragraph{Struttura dei documenti formali}
Ogni directory di un determinato\\ documento contiene un file chiamato \textbf{"main.text"} che ha il compito di raggruppare tutte le sezioni ed i comandi necessari per la compilazione.\\
All'interno della cartella \textit{res} sono presenti i seguenti file:
\begin{itemize}
	\item \textbf{"configurazione.text"}: dirige l'importazione dei \glock{pacchetti};
	\item \textbf{"frontespizio.text"}: viene gestita l'impostazione di tutte le informazioni principali del documento;
	\item \textbf{"registro.text"}:gestiste la tabella delle modifiche, con l'inserimento del versionamento;
	\item \textbf{"sezioni.text"}:controlla la gestione delle sezioni del documento.
\end{itemize}
Inoltre si trovano all'interno della \textit{res}, per quanto riguarda la documentazione formale, altre due directory :
\begin{itemize}
	\item \textbf{images}: contiene le immagini da inserire;
	\item\textbf{sections}: contiene le varie sezioni del contenuto del documento.
\end{itemize}

\subparagraph{Frontespizio}
Come già anticipato, il frontespizio permette di illustrare tutti i dati principali del documento.\\
Sono sempre presenti il logo, il nome del gruppo ed il nome del documento.
Le informazioni sono così impostate:
\begin{itemize}
	\item\textbf{versione}: indica la versione attuale del documento;
	\item\textbf{uso}: indica l'utilità del documento, può essere "interno" o "esterno";
	\item\textbf{stato}: indica lo stato in cui si trova il documento, può essere "in redazione" o "approvato";
	\item\textbf{destinatari}: indica a chi è destinato il documento;
	\item\textbf{redattori}: riporta i nomi di chi ha provveduto alla redazione;
	\item\textbf{verificatori}: riporta chi ha verificato il documento in questione.
\end{itemize}

\subparagraph{Registro delle modifiche}
Questo registro permette di tracciare attraverso una rappresentazione tabellare tutte le modifiche che sono state apportate al documento. \\
Per ogni cambiamento la tabella riporta:
\begin{itemize}
	\item\textbf{versione}: indica la versione del documento relativa alla modifica;
	\item\textbf{descrizione}: illustra brevemente il cambiamento apportato;
	\item\textbf{data}: giorno in cui la modifica è stata effettuata;
	\item\textbf{autore}: chi ha fatto la modifica;
	\item\textbf{ruolo}: il ruolo di chi ha effettuato la modifica.
\end{itemize}

\subparagraph{Indice}
Il team ha concordato che l'inserimento di un indice avrebbe favorito la lettura del documento, aiutando il lettore ad orientarsi.
Esso si presenta con una classica struttura gerartica, infatti sono riportate tutte le sezione, ognuna delle quali presenta le relative sottosezioni, i paragrafi e i sotto paragrafi.

\subparagraph{Contenuto principale}
La formattazione del contenuto principale delle pagine della documentazione è così stilato:
\begin{itemize}
	\item a sinistra in alto è presente il nome del documento e la sua versione;
	\item a destra il capitolo in cui ci si trova;
	\item segue tutto il corpo del documento;
	\item nel footer è indicato il numero della pagina attuale su il numero totale delle pagine del documento in questione. 
\end{itemize}

\paragraph{Glossario}
Il glossario permette di fornire una spiegazione funzionale per tutti quei termini che possono risultare sconosciuti per il lettore.\\
Per la sua struttura si è deciso di adottare un ordine lessicografico.\\
Ulteriori informazioni sono riportante in seguito nel paragrafo \textit{Termini di glossario}.

\paragraph{Struttura dei verbali} 
I verbali presentano una struttura simile a quella dei documenti formali, però il file e la cartella \textit{sections} sono sostituiti da \textbf{"contenuto.text"} che svolge la stessa funzione, ossia incorporare il corpo del documento.\\
Inoltre è presente il file \textbf{"tracciamento.text"}, il quale permette di riportare l'elenco delle decisioni.\\
Come già riportato precedentemente i verbali essendo documenti informali non richiedo la fase di verifica e approvazione, questo comporta che non è presente nessun versionamento e nemmeno un registro delle modifiche.\\
Le informazioni che presenta il frontespizio dei verbali sono:
\begin{itemize}
	\item\textbf{data}: riporta la data in cui si è svolto l'incontro
	\item\textbf{luogo}: indica la piattaforma o il luogo in cui si è svolto il meeting
	\item\textbf{orario}: presenta l'orario in cui si è svolta la seduta
	\item\textbf{redattore}: viene indicato il nome del componente del team incaricato di redarre il verbale
\end{itemize}
Come per gli altri documenti è presente anche il logo del team e il titolo del documento.\\
Nel corpo del verbale vengono esposti brevemente tutti i punti di discussione trattati durante il meeting.\\
Alla fine del documento è inserita una tabella per il tracciamento delle decisioni raggiunte. Ognuna di esse è identificata attraverso un codice ed una descrizione.\\
Il codice identificativo è riportato seguendo la seguente dicitura \textbf{VX-Y.Z}, dove:
\begin{itemize}
    \item\textbf{V} : specifica che si tratta di un verbale
	\item\textbf{X} : indica la tipologia di verbale se interno "\textbf{I}" o esterno "\textbf{E}";
	\item\textbf{Y} : indica il numero del verbale;
	\item\textbf{Z} : indica il numero della decisione presa.
\end{itemize}
    
\subsubsection{Norme tipografiche} \label{_normetipografiche}
Per evitare che i diversi file differiscano nell'ortografia, nella tipografia e nello stile vengono descritte delle norme che permettono di ottenere l'uniformità tra i diversi documenti.

\paragraph{Convenzioni per la denominazione}
Per i nomi delle directory che contengono la documentazione si segue la convenzione \textit{PascalCase}: le parole che compongono un nome vengono unite e ciascuna viene riportata con l'iniziale maiuscola. I nomi dei file \glock{\textit{.tex}} e delle directory a una profondità maggiore delle directory con il nome dello specifico documento (e.g. \textit{res, sections, main.tex} sono definiti da soli caratteri minuscoli e, se formati da più parole, queste ultime sono concatenate senza spazi a separarle. 

\paragraph{Stile del testo}
Gli stili di testo utilizzati nella documentazione sono:
\begin{itemize}
\item\textbf{grassetto} per tutti i titoli, i sottotitoli, per le parole che vengono ritenute più importanti e per quelle subito seguite da una definizione;
\item \textit{corsivo} per tutti i nomi propri (di persona, organizzazione o tecnologia), i termini tecnici e le citazioni;
\item \texttt{monospace} per gli \glock{snippet} di codice;
\item MAIUSCOLO per acronimi, iniziali di noi propri e ove previsto dalle convenzioni per la denominazione sopra riportate.
\end{itemize}

\paragraph{Termini di glossario}
I termini non immediatamente comprensibili (che nella documentazione spesso corrispondono ai termini appartenenti alla sfera semantica dell'informatica) sono contrassegnati dal pedice \glock{} (e.g. \glock{parola}) la prima volta che appaiono in ogni sezione. Se un termine appare più volte all'interno della stessa sezione solo la prima occorrenza dovrà essere seguita dal pedice \glock{}. Il simbolo indica che il termine contrassegnato può essere trovato, seguito dalla sua definizione, all'interno del \dext{Glossario}. In questo documento le voci sono riportate in ordine alfabetico.  

\paragraph{Elementi testuali}
Nella fase di redazione della documentazione è necessario attenersi alle convenzioni stilistiche esposte di seguito.
\subparagraph{Elenchi puntati ed elenchi numerati}
Ogni elemento di un elenco puntato è scandito dal simbolo "\textbullet". Per il secondo livello di annidamento si utilizza il simbolo "--", mentre per il terzo livello "$\ast$". Di seguito la rappresentazione di questa convenzione.
\begin{itemize}
    \item elemento;
    \begin{itemize}
        \item sotto-elemento;
        \begin{itemize}
            \item sotto-sotto-elemento.
        \end{itemize}
    \end{itemize}
\end{itemize}
Gli elementi del primo livello di un elenco numerato sono identificati da numeri arabi seguiti da punto fermo, gli elementi del secondo livello da lettere dell'alfabeto tra parentesi tonde e quelli del terzo livello da numeri romani minuscoli seguiti da  un punto fermo. Di seguito si trova un esempio di quanto descritto.
\begin{enumerate}
    \item elemento 1;
    \begin{enumerate}
        \item elemento 1.1;
        \begin{enumerate}
            \item elemento 1.1.1.
        \end{enumerate}
    \end{enumerate}
\end{enumerate}
Negli item di un elenco nella forma "\textit{nome dell'elemento - descrizione dell'elemento}"  il primo elemento è rappresentato con testo in grassetto, il secondo con caratteri normali.
\subparagraph{Formato di data e ora}
Per l'indicazione delle date e degli orari si seguono le convenzioni stabilite dallo standard \glock{ISO 8601}. \\
Il formato utilizzato per le date è:
\begin{center}
    [YYYY]-[MM]-[DD]
\end{center}
    dove:
    \begin{itemize}
        \item \textbf{[YYYY]} corrisponde al numero dell'anno secondo il calendario gregoriano;
        \item \textbf{[MM]} corrisponde al numero del mese;
        \item \textbf{[DD]} corrisponde al numero del giorno.
    \end{itemize}
Il formato con cui vengono indicati gli orari è:
\begin{center}
    [HH]:[MM]
\end{center} 
dove:
\begin{itemize}
    \item \textbf{[HH]} rapresenta le ore;
    \item \textbf{[MM]} rapresenta i minuti.
\end{itemize}
Le ore e i minuti devono comparire sempre in doppia cifra (qualora ore e/o minuti siano in singola cifra si antepone uno zero). 
\subparagraph{Sigle}
Le sigle sono indicate con le iniziali di ogni parola maiuscole. Se nella sigla compare l'iniziale di una preposizione,articolo o congiunzione, questa è riportata in carattere minuscolo.
Le sigle utilizzate nei documenti del progetto sono:
\begin{itemize}
    \item relative ai nomi dei documenti, rientrano in questa categoria:
    \begin{itemize}
        \item \textbf{Analisi dei Requisiti}: AdR;
        \item \textbf{Piano di Progetto}: PdP;
        \item \textbf{Norme di Progetto}: NdP;
        \item \textbf{Piano di Qualifica}: PdQ;
        \item \textbf{Studio di Fattibilità}: SdF;
        \item \textbf{Verbali Interni}: VI;
        \item \textbf{Verbali Esterni}: VE;
        \item \textbf{Glossario}: G.
        % da aggiungere i manuali per questa revisione?
    \end{itemize}
    \item Relative alle \glock{revisioni} del progetto previste dai committenti:
    \begin{itemize}
        \item \textbf{Revisione dei Requisiti}: RR;
        \item \textbf{Revisione di Progettazione}: RP;
        \item \textbf{Revisione di Qualifica}: RQ;
        \item \textbf{Revisione di Accettazione}: RA.
    \end{itemize}
    \item Relative ai ruoli del progetto:
    \begin{itemize}
        \item \textbf{Analista}: AN;
        \item \textbf{Verificatore}: VE;
        \item \textbf{Amministratore}: AM;
        \item \textbf{Responsabile di progetto}: RE;
        \item \textbf{Programmatore}: PR;
        \item \textbf{Progettista}: PT.
    \end{itemize}
    Altre sigle utilizzate sono:
    \begin{itemize}
        \item \textbf{VCS}: \glock{Version Control System}.
        %inserire qui altre sigle usate
    \end{itemize}
\end{itemize}

\paragraph{Elementi grafici}
Qui di seguito vengono esposte le norme per la rappresentazione di elementi grafici all'interno della documentazione.
\subparagraph{Immagini}
Le immagini compaiono tutte centrate rispetto al testo e sono tutte accompagnate da una didascalia che descrive il loro contenuto.
\subparagraph{Tabelle}
Le tabelle sono centrate rispetto al testo. Ognuna di esse è accompagnata da una didascalia che inizia con l'identificativo della tabella:
\begin{center}
    Tabella[X]
\end{center}
dove [X] indica il numero assoluto progressivo della tabella all'interno del documento. Ad esso segue il testo descrittivo della didascalia. A tale convenzione fanno eccezione la tabella del registro delle modifiche presente in ogni documento e il registro delle decisioni presente nei documenti di tipo \textit{Verbale} che sono sprovviste di didascalia.
\subparagraph{Grafici UML}
I grafici in linguaggio \glock{UML} sono inseriti come immagini.

\subsubsection{Metriche}
\begin{itemize}
    \item Indice di Gulpease: leggibilità del testo;
    \item indice Correttezza ortografica (CORT).
    % da trattare approfonditamente se non sono stati trattati nella gestione della qualità
\end{itemize}

\subsubsection{Strumenti}
\paragraph{Latex}
Per la scrittura dei documenti si è scelto di usare \LaTeX, linguaggio di markup che si basa sul programma di tipografia digitale \TeX . Questo permette di redigere documenti templatizzati, coesi, coerenti e in maniera collaborativa.
\paragraph{Editor di testo}
Viene lasciata libertà sulla scelta dell'editor di testo da utilizzare poiché i componenti del gruppo hanno manifestato preferenze diverse. Gli strumenti che prevalgono sono \textit{TexMaker} e \textit{Visual Studio Code}, utilizzato con l'ausilio del plugin \textit{LaTeX Workshop}.
\paragraph{Diagrammi}
Per i diagrammi sono stati usati diversi software a seconda del tipo di diagramma da rappresentare.
\subparagraph{\textit{Draw.io}}
Web app gratuita integrabile con \glock{Google Drive}, per realizzare casi d'uso, diagrammi ER e grafici UML. Rende possibile anche il versionamento dei file prodotti.
\subparagraph{\textit{Gantt project}}
Applicazione open source per costruire diagrammi di Gantt, usati largamente nella fase di pianificazioni e presenti nel \textit{Piano di Progetto}.

\subsubsection{Verifica ortografica}
Per la verifica ortografica si utilizzano gli strumenti di correzione ortografica messi a disposizione dagli editor di testo che segnalano in maniera automatica gli errori sottolineando con colore rosso le parole non appartenenti al dizionario delle lingue selezionate come lingue del documento.
