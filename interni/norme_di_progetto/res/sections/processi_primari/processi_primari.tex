\section{Processi primari}


\subsection{Processi di fornitura}
Secondo lo standard ISO/IEC/IEEE 12207:2017 lo scopo del processo di fornitura è quello di consegnare all'acquirente un prodotto o servizio che soddisfa i requisiti richiesti. Il fornitore determina l'esistenza di un acquirente che necessità di un prodotto o servizio e definisce una strategia di fornitura di quanto richiesto dopo averne analizzato i rischi e le criticità mediante la redazione di uno studio di fattibilità.
Il fornitore deve inoltre definire un accordo contrattuale che sancisca i rapporti con il committente ed in particolare l'accettazione da parte di quest'ultimo dei requisiti e delle tempistiche di consegna. Si potrà quindi dare avviso alla fase esecutiva stabilendo le procedure e risorse che andranno definite nel piano di progetto.
Il processo di fornitura è composto dalle seguenti fasi:
\begin{itemize}
\item Avvio
\item Approntamento di risposte alle richieste
\item Contrattazione
\item Pianificazione
\item Esecuzione e controllo
\item Revisione e valutazione
\item Consegna e completamento
\end{itemize}
\subsubsection{Scopo} 
\subsubsection{Descrizione}
Questa sezione include le norme che i membri del gruppo SWException devono rispettare in tutte le fasi di progettazione, sviluppo e consegna del prodotto EmporioLambada al fine di diventare fornitori nei confronti del proponente Red Babel e dei committenti Prof. Tullio Vardanega e Prof. Riccardo Cardin.
\subsubsection{Aspettative}
Il gruppo intende avviare e intrattenere un costante dialogo con il proponente per avere un feedback continuativo sul lavoro svolto ed in particolare:
\begin{itemize}
\item Determinare i bisogni che il committente si prefigge di soddisfare mediante il prodotto finale
\item Stabilire vincoli sui requisiti richiesti
\item Stimare le tempistiche di lavoro
\item Effettuare una verifica continua del lavoro
\item Prevenire in anticipo eventuali ambiguità o incertezze in merito al prodotto
\item Concordare scelte relative alla qualifica del prodotto
\end{itemize}

\subsubsection{Studio di fattibilità}
A seguito della presentazione dei capitolati d'appalto da parte dei proponenti il Responsbile di Progetto organizza riunioni tra i membri del gruppo al fine di condividere internamente opinioni sui capitolati stessi.
Lo Studio di Fattibilità è redatto dagli analisti e per ogni capitolato deve indicare:
\begin{itemize}
\item \textbf{Informazioni Generali:} ovvero un elenco di informazioni di base che identifichino il progetto, il proponente e il committente
\item Descrizione del capitolato:  evidenziando le caratteristiche principali per il prodotto e gli obiettivi dello stesso
\item Tecnologie interessate: le tecnologie da impiegare nello svolgimento del progetto
\item Aspetti Positivi
\item Criticità e fattori di rischio
\item Conclusioni: riassunto delle ragioni per cui il gruppo ha deciso di accettare o meno il capitolato in esame
\end{itemize}

\subsubsection{Piano di Progetto}
Deve essere redatto un Piano di Progetto da seguire durante lo svolgimento del progetto, in particolare deve contenere:

\begin{itemize}
\item Analisi dei Rischi: vengono analizzati approfonditamente i rischi che potrebbero presentarsi e vengono preventivate delle modalità di mitigazione degli stessi. Viene inoltre stimata la probabilità con la quale questi possono presentarsi e il loro livello di gravità
\item Modello di Sviluppo:

\item Pianificazione:


\item Preventivo e consutivo:  

\end{itemize}



\subsubsection{Strumenti}




\subsection{Processi di sviluppo}
\subsubsection{Scopo}
L'obiettivo del processo di sviluppo, in accordo con quanto scritto nello standard ISO/IEC/IEEE 12207:2017, è quello di trasformare i requisiti, l'architettura e il design in azioni che permettono la creazione di un prodotto che rispetti i requisiti prestabiliti.

\subsubsection{Descrizione}
Questo processo definisce le seguenti attività:
\begin{itemize}
  \item Analisi dei requisiti
  \item Progettazione dell'architettura
  \item Codifica
  \item Test
  \item Validazione
\end{itemize}

\subsubsection{Analisi dei requisiti}
\paragraph{Scopo}
Lo scopo del processo di analisi dei requisiti è quello di individuare tutte le necessità dello Stakeholder e convertirli in requisiti espliciti e impliciti, nonché diretti e indiretti. Questa attività è svolta dagli analisti che come risultato redigono un documento in cui all'interno vi sono:
\begin{itemize}
  \item Definire lo scopo del prodotto che si andrà a realizzare
  \item Definire le funzionalità e i requisiti concordati con lo Stakeholder
  \item Fornire ai progettisti dei riferimenti affidabili e precisi per permettere loro una progettazione architetturale accurata.
  \item Definire una base per integrare i raffinamenti che permettono un miglioramento continuo del processo di sviluppo e del prodotto
  \item Fornire ai verificatori i riferimenti per il processo di verifica
  \item Fornire una stima oraria del lavoro per definire una stima dei costi.
\end{itemize}

\paragraph{Descrizione}
Le informazioni sopracitate sono state ricavate grazie a:
\begin{itemize}
  \item Capitolato d'appalto
  \item Verbali Esterni
  \item Verbali Interni
  \item Casi d'uso
\end{itemize}

\paragraph{Aspettative}
L'obiettivo è quello di creare un documento formale e completo contenente i requisiti richiesti e concordati con lo Stakeholder.
\paragraph{Struttura}
\paragraph{Classificazione dei requisiti}
\paragraph{Classificazione dei casi d'uso}
\paragraph{Tracciamento di requisiti e casi d'uso}
\paragraph{Metriche}

\subsubsection{Progettazione dell'architettura}
\paragraph{Scopo}
\paragraph{Descrizione}
\paragraph{Aspettative}
\paragraph{Design Patterns}
\paragraph{Diagrammi UML}
\paragraph{Test}

\subsubsection{Codifica}
\paragraph{Scopo}
\paragraph{Descrizione}
\paragraph{Aspettative}
\paragraph{Stile di codifica}

\subsubsection{Test}
\paragraph{Scopo}
\paragraph{Descrizione}
\paragraph{Aspettative}

\subsubsection{Validazione}
\paragraph{Scopo}
\paragraph{Descrizione}
\paragraph{Aspettative}

\subsubsection{Metriche}
\paragraph{???? tipo di metrica che andremo ad usare}

\subsubsection{Strumenti}
\paragraph{Scopo}









