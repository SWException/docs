\section{Processi primari}


\subsection{Processi di fornitura}
\subsubsection{Scopo} 
\subsubsection{Descrizione}
\subsubsection{Aspettative}
\subsubsection{Studio di fattibilità}
\subsubsection{Strumenti}




\subsection{Processi di sviluppo}
\subsubsection{Scopo}
L'obiettivo del processo di sviluppo, in accordo con quanto scritto nello standard ISO/IEC/IEEE 12207:2017, è quello di trasformare i requisiti, l'architettura e il design in azioni che permettono la creazione di un prodotto che rispetti i requisiti prestabiliti.

\subsubsection{Descrizione}
Questo processo definisce le seguenti attività:
\begin{itemize}
  \item Analisi dei requisiti
  \item Progettazione dell'architettura
  \item Codifica
  \item Test
  \item Validazione
\end{itemize}

\subsubsection{Analisi dei requisiti}
\paragraph{Scopo}
Lo scopo del processo di analisi dei requisiti è quello di individuare tutte le necessità dello Stakeholder e convertirli in requisiti espliciti e impliciti, nonché diretti e indiretti. Questa attività è svolta dagli analisti che come risultato redigono un documento in cui all'interno vi sono:
\begin{itemize}
  \item Definire lo scopo del prodotto che si andrà a realizzare
  \item Definire le funzionalità e i requisiti concordati con lo Stakeholder
  \item Fornire ai progettisti dei riferimenti affidabili e precisi per permettere loro una progettazione architetturale accurata.
  \item Definire una base per integrare i raffinamenti che permettono un miglioramento continuo del processo di sviluppo e del prodotto
  \item Fornire ai verificatori i riferimenti per il processo di verifica
  \item Fornire una stima oraria del lavoro per definire una stima dei costi.
\end{itemize}

\paragraph{Descrizione}
Le informazioni sopracitate sono state ricavate grazie a:
\begin{itemize}
  \item Capitolato d'appalto
  \item Verbali Esterni
  \item Verbali Interni
  \item Casi d'uso
\end{itemize}

\paragraph{Aspettative}
L'obiettivo è quello di creare un documento formale e completo contenente i requisiti richiesti e concordati con lo Stakeholder.
\paragraph{Struttura}
\paragraph{Classificazione dei requisiti}
\paragraph{Classificazione dei casi d'uso}
\paragraph{Tracciamento di requisiti e casi d'uso}
\paragraph{Metriche}

\subsubsection{Progettazione dell'architettura}
\paragraph{Scopo}
\paragraph{Descrizione}
\paragraph{Aspettative}
\paragraph{Design Patterns}
\paragraph{Diagrammi UML}
\paragraph{Test}

\subsubsection{Codifica}
\paragraph{Scopo}
\paragraph{Descrizione}
\paragraph{Aspettative}
\paragraph{Stile di codifica}

\subsubsection{Test}
\paragraph{Scopo}
\paragraph{Descrizione}
\paragraph{Aspettative}

\subsubsection{Validazione}
\paragraph{Scopo}
\paragraph{Descrizione}
\paragraph{Aspettative}

\subsubsection{Metriche}
\paragraph{???? tipo di metrica che andremo ad usare}

\subsubsection{Strumenti}
\paragraph{Scopo}









