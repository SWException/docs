\appendix
\section{Metriche di qualità}

\subsection{Metriche per la qualità del processo}

\subsection{Metriche per la qualità del prodotto}
Per quantificare la qualità del prodotto vengono adottate le metriche descritte in questa sezione.
\subsubsection{Indice di Gulpease} 
Indice che individua il grado di leggibilità di un testo in lingua italiana mediante la formula:
\begin{center}
    \(GULP=89+\frac{300(totale\; frasi)-10(totale\; lettere)}{totale\; parole}\)
\end{center}
I valori ottenuti sono da interpretare secondo questi criteri:
\begin{itemize}
    \item \textbf{GULP < 80}: leggibilità difficile per un utente con licenza elementare;
    \item \textbf{GULP < 60}: leggibilità difficile per un utente con licenza media;
    \item \textbf{GULP < 40}: leggibilità difficile per un utente con diploma di scuola secondaria di secondo grado;
    \item \textbf{GULP \(\sim\) 0} leggibilità difficile per chiunque.
\end{itemize}

\subsubsection{Correttezza ortografica (CORT)}
Questa metrica permette di misurare la correttezza lessicografica del documento mediante un numero intero ottenuto come:
\begin{center}
    \textit{CORT=\# errori ortografici}
\end{center}
Se CORT=0 allora il documento non ha errori ortografici, se CORT>0 il documento presenta errori.