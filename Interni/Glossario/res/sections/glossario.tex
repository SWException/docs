\section*{A}
\subsection*{Applicazione mobile}
Programma che specializza il funzionamento di un computer in una determinata attività,
pensato per dispositivi adatti alla mobilità, quali smartphone e tablet.

\subsection*{Amministratore}
Tipologia di utente dell'applicazione EmporioLambda con la possibilità di
\begin{itemize}
    \item rilasciare l'applicazione nell'infrastruttura cloud AWS;
    \item gestire la configurazione dei servizi di terze parti integrati.
\end{itemize}
Nello svolgimento di questo progetto è rappresentato dal gruppo SWException.

\subsection*{AWS Lambda}
Servizio di elaborazione serverless che consente di eseguire codice senza fare il provisioning o gestire i server, creare una logica di scalabilità del cluster in grado di riconoscere il carico di lavoro, mantenere integrazioni di eventi o gestire i runtime.

\subsection*{AWS CloudWatch}
Servizio di monitoraggio e osservabilità che fornisce dati e approfondimenti utilizzabili per monitorare il software, rispondere ai cambiamenti delle prestazioni a livello di sistema, ottimizzare l'utilizzo delle risorse e ottenere una visione unificata dell'integrità operativa.

\subsection*{AWS DynamoDB}
E' un database di documenti e valori-chiave durevole, multi-regione e completamente gestito con sicurezza integrata, backup e ripristino e memorizzazione nella cache in memoria per software.

\subsection*{AWS S3}
Amazon Simple Storage Service è un servizio di storage di oggetti che offre scalabilità, disponibilità dei dati, sicurezza e prestazioni leader del settore.

\subsection*{AWS API Gateway}
E' un servizio gestito che semplifica agli sviluppatori la creazione, la pubblicazione, la manutenzione, il monitoraggio e la protezione delle API su qualsiasi scala.

\subsection*{Auth0}
Standard open-source per la delega di accesso, comunemente utilizzato come un modo per gli utenti di Internet di concedere a siti Web o applicazioni l'accesso alle proprie informazioni su altri siti Web, ma senza fornire loro le password.

\section*{B}
Empty

\section*{C}
\subsubsection*{Carrello}
Pagina dell'e-commerce che contiene tutti i prodotti che l'utente ha aggiunto per acquistare.

\subsubsection*{Checkout}
Insieme di passaggi che l'utente deve compiere per acquistare i prodotti precedentemente inseriti nel carrello.

\subsection*{Cliente}
Tipologia di utente dell'applicazione EmporioLambda che rappresenta l'utente finale, che usa
il prodotto per acquistare beni o servizi dal commerciante.

\subsubsection*{CMS}
\textit{Content Management System:} Software usato per gestire la creazione e modifica dei contenuti di un sito.

\subsection*{Commerciante}
Tipologia di utente dell'applicazione EmporioLambda che acquista il prodotto per specializzarlo
alle sue necessità, mettendo in vendita i propri beni o servizi offerti ai clienti.

\subsection*{Committente}
Persone o gruppo di persone incaricato ad individuare il/i capitolato/i da presentare al fornitore.
Chi commette, cioè ordina ad altri l'esecuzione di un lavoro o di una prestazione


\section*{D}
\subsubsection*{Dashboard del commerciante}
Pagina del sito web accessibile solo da utenti con profilo di tipo commerciante. Offre funzionalità di modifica dei dati sui prodotti
presenti nell'applicazione e fornisce una visione generale sugli ordini in corso.

\section*{E}
\subsection*{EML-FE}
\textit{EmporioLambda Front-End:} servizio per il front-end menzionato nel capitolato d'appalto C2.

\subsection*{EML-BE}
\textit{EmporioLambda Back-End:} servizio per la parte di back-end del capitolato C2.

\subsection*{EML-I}
\textit{EmporioLambda Integration:} Servizi di terze parti integrati nel modulo di back-end.

\subsection*{EML-MON}
\textit{EmporioLambda Monitoring:} insieme di strumenti usati dall'amministratore dell'applicazione EmporioLambda
per monitorare lo stato della stessa.

\section*{F}
Empty

\section*{G}
\subsection*{Git}
Version Control Manager usato per il versionamento di codice sorgente e documenti latex.

\subsection*{GitHub}
Piattaforma che offre il software Git as-a-service, insieme a molte altre funzionalità.

\subsection*{Google Drive}
Servizio di cloud storage che permette la memorizzazione e sincronizzazione di file e cartelle. Può essere usato via web o con app desktop e mobile.

\subsection*{Google Meet}
Piattaforma per videocall e conferenze di proprietà di Google.

\section*{H}
\subsubsection*{Homepage}
Pagina principale di un sito. Qui devono essere presenti le principali informazioni utili ad una navigazione proficua del
sito da parte dell'utente.

\section*{I}
\subsubsection*{Issue Tracking System}
Strumento per tenere traccia di attività, miglioramenti e bug per quanto riguarda il progetto.

\section*{J}
Empty

\section*{K}
Empty

\section*{L}
\subsection*{Latex}
Sistema di composizione matematico che include funzionalità progettate per la produzione di documentazione tecnica e scientifica.

\section*{M}
\subsection*{Microservizi}
I microservizi sono un approccio architetturale alla realizzazione di applicazioni. 
Quello che distingue l'architettura basata su microservizi dagli approcci monolitici tradizionali 
è la suddivisione del prodotto nelle sue funzioni di base. Ciascuna funzione, denominata servizio, può essere 
compilata e implementata in modo indipendente. Pertanto, i singoli servizi possono funzionare, o meno, senza compromettere gli altri.

\subsection*{MIT}
Licenza di software libero creata dal Massachusetts Institute of Technology (MIT).
E' una licenza permissiva, cioè permette il riutilizzo nel software proprietario sotto la condizione che la licenza sia distribuita con tale software.

\section*{N}

\subsubsection*{Next.js}
Framework React web di sviluppo front-end che abilita funzionalità come il rendering lato server e la generazione di siti web statici per applicazioni web basate su React.

\section*{O}
\subsubsection*{Outlook}
Piattaforma di gestione delle informazioni personali di Microsoft composta da servizi di posta Web, calendario, contatti e attività.

\section*{P}
\subsubsection*{PDP}
\textit{Product Detail Pages:} insieme di pagine dell'applicazione EmporioLambda, ognuna di esse contiene informazioni
riguardo un singolo prodotto/servizio in vendita nell'e-commerce.

\subsubsection*{PLP}
\textit{Product Listing Page:} pagina del sito che contiene l'elenco di tutti i prodotti/servizi venduti
nell'e-commerce, derivato da un'istanza di EmporioLambda.

\subsubsection*{Profilo}
Insieme di informazioni che identificano un univoco utente dell'applicazione.

\subsection*{Proponente}
Ente o azienda che propone un capitolato d'appalto.

\section*{Q}
Empty

\section*{R}
\subsubsection*{Revisione di progetto}
Milestone fissata dai committenti per valutare i progressi del team. Una revisione permette di capire dove si sta sbagliando in tempo per non pregiudicare l'esito finale del progetto.

\section*{S}
\subsubsection*{SEO}
\textit{Search Engine Optimization:} Insieme delle attività volte a migliorare il posizionamento (ranking) di un sito o di una pagina web per determinati fattori nei risultati forniti
da un motore di ricerca (Search Engine Result Page o SERP)

\subsection*{Server}
In una rete, qualunque computer che offre un servizio agli altri calcolatori (client), come l'accesso a risorse condivise (dischi, stampanti ecc.), la ricerca su basi di dati o altre funzioni applicative.

\subsection*{Serverless}
Framework web gratuito e open source scritto utilizzando Node.js. E' il primo framework sviluppato per la creazione di applicazioni su AWS Lambda.

\subsection*{Servizio}
Singolo componente in un'architettura a microservizi.

\subsection*{Slack}
Applicazione di messaggistica per ambienti enterprise.

\subsection*{Snippet}
Frammento di codice inserito all'interno di un documento.

\section*{T}
\subsubsection*{Typescript}
Linguaggio di programmazione per ambiente web sviluppato e mantenuto da Microsoft. E' un sovrainsieme della sintassi di JavaScript,
a cui aggiunge opzionalmente un sistema di tipi statico. Viene usato sia lato client che server.
\subsubsection*{Telegram}
Strumento di messaggistica istantanea utilizzato per la comunicazione tra coppie di utenti o in gruppi.

\subsection*{Tex}
Programma e linguaggio di markup per la stesura di testi scientifici e matematici da cui è stato ricavato \glock{Latex}. L'estensione dei file Tex e Latex è \textit{.tex}.

\section*{U}

\subsection*{UML}
United Modelling Language. Linguaggio di modellazione e specifica basato sul paradigma orientato agli oggetti. Con i suoi diagrammi fornisce uno strumento  ricco nella semantica e nella sintassi, per l'architettura, la progettazione e l'implementazione di sistemi software complessi.

\section*{V}
Empty

\section*{W}
Empty

\section*{X}
Empty

\section*{Y}
Empty

\section*{Z}
\subsubsection*{Zoom}
Servizio di videocall e chat online che attraverso una piattaforma software viene utilizzato per conferenze e lavoro.
