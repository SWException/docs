\section{Studio C6} \label{_c6}
Il capitolato C6 è stato presentato dall'azienda ZERO12 con sede a Padova. Quest'azienda si occupa di sviluppare soluzioni software
innovative basate su tecnologia \textit{Cloud Amazon Web Services}.

\subsection{Informazioni generali}
\begin{itemize}
	\item \textbf{Nome:} RGP: Realtime Gaming Platform;
	\item \textbf{Proponente}: ZERO12;
	\item \textbf{Committente}: Prof. Tullio Vardanega e Prof. Riccardo Cardin.
\end{itemize}

\subsection{Descrizione del capitolato}
Il capitolato in questione richiede la realizzazione di una realtime game platform su mobile che presenta le seguenti caratteristiche:
\begin{itemize}
	\item piattaforma a scorrimento verticale;
	\item \glock{multiplayer};
	\item utilizzare la modalità fantasma per il multiplayer;
	\item il gioco deve essere infinito;
	\item \glock{powerup} e nemici uguali per il multiplayer.
\end{itemize}


\subsection{Finalità del progetto}
Il focus del progetto è dato alla componente server basata su microservizi, mentre l'applicazione mobile viene realizzata solo per testare l'applicativo.

\subsection{Tecnologie interessate}
Le tecnologie richieste dal proponente per la realizzazione del prodotto finale sono:
\begin{itemize}
	\item \textit{DynamoDB}: database NoSQL da utilizzare per la conservazione di tag o altre informazioni di supporto;
	\item \textit{Nodejs}: permette di sviluppare le API Restful JSON a supporto applicativo.
\end{itemize}
Inoltre l'azienda richiede l'utilizzo, a discrezione del gruppo, di una delle seguenti tecnologie per lo sviluppo del gioco:
\begin{itemize}
	\item \textit{AWS Gamelift}: è un servizio gestito di Amazon per lo sviluppo di giochi online;
	\item \textit{AWS appsync}: anche questo è un servizio gestito di Amazon, permette però di sviluppare rapidamente delle API GraphQL.
\end{itemize}
Per lo sviluppo dell'applicazione mobile invece è richiesto, a seconda dell'ambiente di sviluppo, l'utilizzo di:
\begin{itemize}
	\item \textit{Swift}: linguaggio di programmazione orientato agli oggetti per lo sviluppo di app in ambito iOS/MacOS;
	\item \textit{Kotlin}: linguaggio di programmazione general-purpose per lo sviluppo di app in ambito Android.
\end{itemize}


\subsection{Aspetti positivi}
Sicuramente uno degli aspetti positivi di questo capitolato è la disponibilità del proponente a fornire dei seminari per approfondire le tecnologie da utilizzare.
Inoltre, diversi membri del gruppo si sono dimostrati interessati all'apprendimento dei servizi di \textit{AWS} e della realizzazione di un videogioco.


\subsection{Criticità e fattori di rischio}
Il maggior fattore di rischio di questo capitolato è la mancanza di conoscenza dei servizi \textit{AWS}, infatti nessuno dei componenti del gruppo ha esperienze pregresse con essi.


\subsection{Conclusioni}
Sebbene il progetto sembrava interessare diversi componenti del gruppo, abbiamo deciso di non selezionarlo in quanto preferiamo realizzare una diversa tipologia di prodotto.
\newpage
