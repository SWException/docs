\section{Studio C4} \label{_c4}
Il capitolato C4 è stato presentato da Zucchetti Spa, azienda italiana con sede a Lodi. L'attività è specializzata nello sviluppo di software per la pubblica amministrazione, ma da alcuni anni ha esteso la sua presenza anche in altri settori come robotica e automazione industriale.

\subsection{Informazioni generali}
\begin{itemize}
    \item \textbf{Nome:} HDViz - Visualizzatore di dati con molte dimensioni;
    \item \textbf{Proponente:} Zucchetti Spa;
    \item \textbf{Committente:} Prof. Tullio Vardanega e Prof. Riccardo Cardin.
\end{itemize}

\subsection{Descrizione del capitolato}
Il capitolato propone la creazione di una applicazione che aiuti l'utente nella fase esplorativa dell'analisi dei dati grazie alla visualizzazione multidimensionale degli stessi.
Come scritto nel capitolato d'appalto il prodotto finale dovrà prevedere almeno le seguenti visualizzazioni fornite dalla libreria \textit{\glock{D3.js}}:
\begin{itemize}
    \item \textit{Scatter plot Matrix};
    \item \textit{Force Field};
    \item \textit{Heat Map};
    \item \textit{Proiezione Lineare Multi Asse}.
\end{itemize}
Inoltre HDViz dovrà obbligatoriamente ordinare i punti nel grafico "Heat map" per evidenziare i "cluster" presenti nei dati e sarà necessario che l'applicazione ottenga i dati sia tramite \glock{query} sia tramite file \glock{CSV} (con i dati precedentemente preparati).

\subsection{Finalità del progetto}
L'obiettivo di questo progetto è quello di creare un software d'ausilio che permetta all'utente che lo utilizza l'identificazione con un colpo d'occhio di dati fuori scala o errati.

\subsection{Tecnologie interessate}
L'applicazione sarà suddivisa in due parti:
\begin{itemize}
    \item Frontend: sviluppato con tecnologie \textit{\glock{HTML}/\glock{CSS}/JavaScript} sfruttando la libreria D3.js che permette la visualizzazione dinamica ed interattiva di dati partendo da dati organizzati;
    \item Backend: un server sviluppato in \textit{Java} con \textit{\glock{TomCat}} oppure in \textit{JavaScript} tramite \textit{Node.js}. Il servizio dovrà fornire supporto per le presentazioni del frontend fornendo i dati prelevati da un database che può essere \textit{\glock{SQL}} o \textit{\glock{NoSQL}}.
\end{itemize}

\subsection{Aspetti positivi}
Il capitolato ha particolarmente attratto l'interesse del gruppo sia per il prestigio riconosciuto al proponente Zucchetti Spa che per la possibilità di approfondire tematiche relative all'elaborazione dei dati mediante librerie di terze parti anche alla luce delle informazioni raccolte durante il seminario tecnico.

\subsection{Criticità e fattori di rischio}
Le tecnologie richieste sono per lo più sconosciute alla maggior parte dei membri (D3.js, Node.js). Inoltre riteniamo che il capitolato proposto si focalizzi maggiormente sull'aspetto algoritmico-matematico dell'analisi e visualizzazione dei dati, piuttosto che nello sviluppo dell'applicativo web in sé.

\subsection{Conclusioni}
Il progetto, nonostante fosse di interesse comune, non è stato scelto in quanto si è deciso di dedicarsi ad un altro capitolato che raccoglieva maggiori stimoli e interessi da parte dei membri.
