\section{Studio C5}
Il capitolato C1 è stato presentato da Sanmarco Informatica, un’azienda con varie sedi dislocate in Veneto, Lombardia, Friuli-Venezia Giulia ed Emilia-Romagna. Si occupa di consulenza e sviluppo software con una forte specializzazione verso la progettazione e realizzazione di soluzioni a supporto della riorganizzazione di tutti i processi aziendali e professionali.
\subsection{Informazioni generali}
\begin{itemize}
    \item \textbf{Nome:} PORTACS: piattaforma di controllo mobilità autonoma;
    \item \textbf{Proponente:} Sanmarco Informatica;
    \item \textbf{Committente:} Prof. Tullio Vardanega e Prof. Riccardo Cardin.
\end{itemize}

\subsection{Descrizione del capitolato}
Il capitolato richiede di realizzare un \glock{\textit{real time POI oriented anti-collision system}} che permetta l’automazione nella gestione di robot, automobili e mezzi industriali che si muovono all’interno di un ambiente mappato. L’obiettivo è coordinare gli spostamenti di tutti i veicoli evitando collisioni ed eventualmente fare in modo che ogni unità raggiunga la sua destinazione compiendo il percorso più breve possibile.
L’azienda presenta quattro principali ambiti di utilizzo dell’applicativo:
\begin{enumerate}
    \item l’automazione di robot camerieri, che devono muoversi tra i tavoli di un ristorante evitando gli ostacoli che si presentano lungo il percorso e portare ai clienti gli ordini provenienti dalla cucina;
    \item la gestione di auto a guida autonoma, che una volta fissati destinazione e punti intermedi del percorso necessita di un sistema che permetta di variare la sua velocità di crociera in funzione dei limiti vigenti e del traffico, riconoscendo lo stato dei semafori e la presenza di ostacoli;
    \item la geolocalizzazione dei mezzi e degli operatori all’interno di un magazzino; i mezzi si devono muovere sulle corsie lungo le quali è previsto il loro spostamento evitando collisioni. Ogni operatore, a piedi o su un mezzo, ha un compito da portare a termine e varie ubicazioni intermedie da raggiungere per poter completare la sua attività. I responsabili devono essere in grado di vedere la posizione di tutte le unità che operano nel magazzino;
    \item la geolocalizzazione nei trasporti, che necessita di un sistema real time per il tracciamento dei mezzi e dello stato delle consegne. È necessario calcolare per ogni mezzo il percorso da seguire per effettuare tutte le consegne della giornata, tornando infine al deposito di partenza.
\end{enumerate}

\subsection{Finalità del progetto}
L’azienda richiede che il software prodotto sia in grado di accettare in input:
\begin{itemize}
    \item una “scacchiera” (o mappa) con la definizione delle percorrenze, dei relativi vincoli (sensi unici, numero massimo di unità contemporanee) e definizione dei \textit{POI} (Point Of Interest);
    \item definizione delle N unità provviste di identificativo di sistema, velocità massima, posizione iniziale e lista dei \textit{POI} da attraversare già ordinata.
\end{itemize}
La User Interface che rappresenta ogni singola unità dovrà mostrare la direzione verso la quale viene indicato all'unità di spostarsi dal sistema centrale, il pulsante di stop/start e l’indicatore di velocità attuale.
Il sistema deve indicare ad ogni unità la prossima “mossa” da fare, in funzione del prossimo \textit{POI} da raggiungere, in modo da evitare collisioni e da  (possibilmente) seguire il percorso ottimo.
Ogni unità deve inviare al sistema centrale costantemente la propria posizione, direzione e velocità, in modo tale che il sistema centrale piloti e coordini tutte le unità per evitare incidenti e ingorghi.
Il sistema dovrà essere corredato da una visualizzazione in real-time della mappa e della relativa posizione delle unità.

\subsection{Tecnologie interessate}
Non sono state indicate particolari tecnologie per la realizzazione del capitolato. Le note tecniche da evidenziare sono:
\begin{itemize}
    \item non vi è necessità di gestire la geolocalizzazione delle unità che sarà simulata per l’invio della posizione corrente;
    \item oltre al codice sorgente dovranno essere consegnati anche vari file \glock{\textit{Docker}} per le varie componenti del sistema.
\end{itemize}

\subsection{Aspetti positivi}
\begin{itemize}
    \item Alcuni membri del gruppo hanno dimostrato interesse verso questo capitolato, ritenendo utile acquisire competenze in ambito real-time monitoring and decision making;
    \item il progetto ha dato fin da subito l’impressione di essere alla portata delle abilità del gruppo;
    \item l’azienda proponente nel corso della presentazione ha evidenziato l’opportunità di una possibile offerta di lavoro al termine del progetto.
\end{itemize}

\subsection{Criticità e fattori di rischio}
\begin{itemize}
    \item Nella presentazione non è stato chiarito in maniera specifica a quali tecnologie specifiche bisognerà fare affidamento per lo sviluppo del software. Troppa libertà nelle scelte implementative potrebbe rivelarsi un’arma a doppio taglio;
    \item più gruppi rispetto ai posti disponibili hanno mostrato fin da subito interesse per il capitolato;
\end{itemize}

\subsection{Conclusioni}
In sede di valutazione dei capitolati il progetto \textit{PORTACS} , nonostante abbia suscitato interesse per gli ambiti coinvolti nella sua realizzazione, è stato escluso in virtù della poca chiarezza sulle tecnologie da impiegare. Anche l’interesse dimostrato da molti altri gruppi ha contribuito ad orientare la scelta verso un altro capitolato.


