\section{Studio C3} \label{_c3}
Il capitolato C3 è stato presentato dall'azienda SyncLab che ha 5 sedi diffuse sul territorio (Roma, Napoli, Milano, Padova e Verona). L'azienda è una software house che sviluppa principalmente nei seguenti settori: mobile, videosorveglianza e sicurezza delle infrastrutture informatiche aziendali.

\subsection{Informazioni generali}
\begin{itemize}
    \item \textbf{Nome:} GDP - Gathering Detection Platform;
    \item \textbf{Proponente:} SyncLab;
    \item \textbf{Committente:} Prof. Tullio Vardanega e Prof. Riccardo Cardin.
\end{itemize}

\subsection{Descrizione del capitolato}
L'applicazione da sviluppare consiste in una piattaforma che permetta di analizzare dei dati in ingresso e visualizzarli su una heatmap. L'applicazione deve acquisire informazioni riguardanti la presenza di un certo numero di persone in un luogo attraverso sensori o sorgenti di vario tipo (come telecamere, conta persone, trasporti pubblici con capienze medie per corsa, …), permettere di visualizzare i dati in una mappa in tempo reale e deve poter fare previsioni sulla situazione futura attraverso algoritmi di machine learning. Inoltre deve essere in grado di mantenere uno storico dei dati per poterli eventualmente analizzare in futuro.\\
Oltre ai requisiti richiesti è fatta espressamente richiesta di soddisfare i seguenti vincoli:
\begin{itemize}
    \item implementare test con copertura $\geq 80\%$;
    \item avere in documentazione i motivi delle scelte implementative e progettuali effettuate;
    \item avere in documentazione i problemi ancora aperti, ed eventuali soluzioni proposte per essi.
\end{itemize}
Il proponente non impone tecnologie specifiche, ma ne raccomanda fortemente l’utilizzo di alcune.

\subsection{Finalità del progetto}
L'obbiettivo è avere una piattaforma che permetta attraverso l'iterazione con la mappa di riconoscere degli assembramenti e prevederne la formazione futura in una certa zona.

\subsection{Tecnologie interessate}
Le tecnologie fortemente richieste dal proponente per la realizzazione del prodotto finale sono le seguenti:
\begin{itemize}
    \item \textit{Java}: è un linguaggio di programmazione ad alto livello, orientato agli oggetti, che si appoggia sull'omonima piattaforma software di esecuzione, specificamente progettato per essere il più possibile indipendente dalla piattaforma hardware di esecuzione.
    \item \textit{Angular}: è un framework open source per lo sviluppo di applicazioni web, e il linguaggio di programmazione usato è TypeScript. È stato progettato per fornire uno strumento facile e veloce per sviluppare applicazioni che girano su qualunque piattaforma inclusi smartphone e tablet.
    \item \textit{Leaflet}: è una libreria JavaScript open source utilizzata per creare delle applicazioni web che utilizzino delle mappe interattive. Permette di sovrapporre ad esse degli elementi, potendo creare ad esempio una heatmap.
    \item \textit{Pattern Publisher/Subscriber}: è un design pattern utilizzato per la comunicazione asincrona fra diversi processi, oggetti o altri agenti, e consiste nel creare un intermediario tra mittenti e destinatari.
    \item \textit{Protocollo MQTT}: è un protocollo ISO standard (ISO/IEC PRF 20922) di messaggistica leggera di tipo Publisher/Subscriber. È stato progettato per le situazioni in cui è richiesto un basso impatto e dove la banda è limitata.
    \item \textit{Apache Kafka}: è una piattaforma open source di stream processing. Mira a creare una piattaforma a bassa latenza ed alta velocità per la gestione di feed dati in tempo reale.
\end{itemize}

\subsection{Aspetti positivi}
\begin{itemize}
    \item L’utilizzo delle tecnologie consigliate sono un ottima opportunità di formazione per il gruppo.
    \item La realizzazione di un progetto software che faccia uso di machine learning è abbastanza accattivante in quanto essa è una tecnologia molto attuale, e il suo utilizzo in un progetto potrà essere utile in futuro nell’entrare nel mondo del lavoro.
\end{itemize}

\subsection{Criticità e fattori di rischio}
\begin{itemize}
    \item Studiare e poi implementare il progetto richiesto potrebbe richiedere molto tempo vista la complessità generale.
    \item Le tecnologie “preferenziali” che vengono indicate sono in parte non conosciute dal gruppo.
    \item Sono presenti argomenti e tecnologie non ancora sufficientemente conosciute tra i membri del gruppo (come il machine learning). Il loro studio individuale potrebbe non essere abbastanza efficace, o richiedere più tempo del previsto.
\end{itemize}

\subsection{Conclusioni}
Il capitolato proposto, pur avendo inizialmente attirato l'attenzione dei componenti del gruppo, dopo un'attenta analisi, il seminario proposto dal proponente e la valutazione delle conoscenze attualmente in possesso che il progetto necessita, richiede secondo il team un'eccessiva mole di lavoro.
