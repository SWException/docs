\section{Informazioni}
\textbf{Data}: 2020-12-02\\
\textbf{Orario}: 14:30-16:30\\
\textbf{Luogo}: Zoom\\\\
\textbf{Partecipanti}:\begin{list}{*}{\setlength{\itemsep}{0cm}}
	\item Marco Canovese
	\item Nicole Davanzo
	\item Ivan Furlan
	\item Gianmarco Guazzo
	\item Stefano Lazzaroni
	\item Francesco Trolese
	\item Michele Veronesi
\end{list}

\section{Introduzione}
In data 2020-12-02 si è tenuta in videoconferenza una riunione con tutti i componenti del gruppo per prendere alcunde decisioni fondamentali riguardo il progetto

\subsection{Ordine del giorno}
\begin{enumerate}
    \item Decisioni concernenti l'identità del gruppo
    \item Analisi dei capitolati proposti, opinioni personali e criticità rilevate
    \item Discussione relativamente alla documentazione da produrre per RR
    \item Tecnologie ed applicativi per la produzione e gestione della documentazione di progetto
    \item Decisioni in merito alle metodologie di versionamento
\end{enumerate}

\section{Svolgimento}

\subsection{Decisioni concernenti l'identità del gruppo}
A seguito della discussione avvenuta informalmente tra i membri del gruppo si approva all’unanimità il nome del gruppo proposto da Stefano Lazzaroni ovvero SWExeception. Nicole Davanzo si propone spontaneamente per realizzare un logo distintivo del gruppo, le bozze grafiche saranno valutate nella successiva seduta.
In ottemperanza a quanto previsto dal regolamento del progetto didattico è stato attivato un account email: swexception@outlook.com da utilizzare per le comunicazioni con i proponenti e committenti

\subsection{Analisi dei capitolati proposti, opinioni personali e criticità rilevate}
È stato effettuato nuovamente un rapido brain storming sulla base delle informazioni raccolte relativamente ai tre capitolati che maggiormente si avvicinano agli interessi del gruppo alla luce anche dei seminari tecnici frequentati.
Si è deliberato per iniziare l’attività di studio di fattibilità di tutti i capitolati proposti al fine di poterli valutare più oggettivamente, che dovrà essere completata entro il 13 Dicembre 2020.

\subsection{Discussione relativamente alla documentazione da produrre per RR}
Dopo aver nuovamente preso visione e comprensione del regolamento del progetto didattico è stato effettuato un riepilogo della documentazione da produrre in ingresso alla R.R. entro il giorno 11/01/2021. Si fissa in Lunedì 7/12/2020 il termine ultimo per l’approfondimento individuale della documentazione con particolare attenzione al contenuto del documento ”Norme di Progetto”

\subsection{Tecnologie ed applicativi per la produzione e gestione della documentazione di progetto}

Si è deliberato di adottare il linguaggio di markup LaTeX per la stesura della documentazione di progetto. Michele Veronesi ha relazionato i membri del gruppo relativamente alle possibili soluzioni tecniche da adottare per redarre e versionare la documentazione. Tutti hanno accolto positivamente la scelta di adottare un’unica repository dedicata alla documentazione a cui connettere l’applicativo overleaf per effettuare il merge dei documenti e produrre la versione definitiva.

\subsection{Decisioni in merito alle metodologie di versionamento}
E’ stato configurato un profilo dedicato al gruppo sulla piattaforma di versionamento GitHub a cui sono stati collegati gli account dei componenti del gruppo. E’ stato deciso di adottare l’approccio GitFlow rimandando eventuali altri decisioni specifiche durante la stesura del documento contenente le norme di progetto.


\section{Conclusioni}
Viene convocato un nuovo incontro il 7/12/2020 al fine di condividere opinioni personali alla luce degli approfondimenti sulle norme di progetto.