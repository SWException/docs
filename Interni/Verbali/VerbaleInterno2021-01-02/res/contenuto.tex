\section{Informazioni}
\textbf{Data}: 2021-01-02\\
\textbf{Orario}: 11:30 - 12:15\\
\textbf{Luogo}: Zoom\\\\
\textbf{Partecipanti}:\begin{list}{*}{\setlength{\itemsep}{0cm}}
	\item Marco Canovese
	\item Nicole Davanzo
	\item Ivan Furlan
	\item Gianmarco Guazzo
	\item Francesco Trolese
	\item Michele Veronesi
\end{list}

\section{Introduzione}
In data 2021-01-02 si è tenuta in videoconferenza una riunione con 6 componenti del gruppo al fine di organizzare il lavoro.

\subsection{Ordine del giorno}
\begin{enumerate}
    \item Avvio stesura requisiti e relativa scadenza, contestuale verifica dei casi d'uso
    \item Aggiornamento glossario
    \item Verifica requisiti in analisi dei requisiti
    \item Programmazione stesura piano di qualifica e relativa verifica
    \item Avvio validazione per norme di progetto e studio di fattibilità
    \item Scadenza ultima redazione e verifica verbali
    \item Fissato meeting esterno con i proponenti
\end{enumerate}

\section{Svolgimento}
\subsection{Avvio stesura requisiti e relativa scadenza, contestuale verifica dei casi d'uso}
Visto che la scadenza per la stesura dei casi d'uso è stata rispettata, viene avviata la stesura dei requisiti
e la contestuale verifica dei casi d'uso individuati da Marco Canovese, Michele Veronesi, Stefano Lazzaroni e Nicole Davanzo.
Vengono quindi incaricati per questo lavoro Gianmarco Guazzo, Ivan Furlan e Francesco Trolese.
La scadenza ultima è fissata per il 2021-01-06.

\subsection{Aggiornamento glossario}
Nicole Davanzo si offre per l'aggiorìnamento del glossario. Viene quindi incaricata del suo mantenimento, con l'indicazione
di consegnare una prima bozza completa entro il 2021-01-10.
La verifica partirà contestualmente alla consegna della prima bozza. Al momento manca un assegnatario per questo incarico.

\subsection{Verifica requisiti in analisi dei requisiti}
Vista la decisione VI\_2021-01-02.1 e relativa scadenza, viene assegnato l'incarico per la verifica dei requisiti stesi
a partire dal 2021-01-07. Tale verifica deve essere completata entro il 2021-01-08 prima del meeting con i proponenti.
A tale incarico sono assegnati Marco Canovese e Michele Veronesi.

\subsection{Programmazione stesura piano di qualifica e relativa verifica}
Per la stesura del piano di qualifica viene fissata la scadenza del 2021-01-09. I redattori sono quelli precedentemente assegnati a questo
incarico, ovvero Michele Veronesi, Marco Canovese e Stefano Lazzaroni.
Partirà quindi la verifica del documento il 2021-01-10 da parte dei verificatori Francesco Trolese e Gianmarco Guazzo.

\subsection{Avvio validazione per norme di progetto e studio di fattibilità}
A questo compito viene assegnato Ivan Furlan, il quale avrà il ruolo di responsabile di progetto in questi due documenti.
La scadenza è fissata per il 2021-01-10.

\subsection{Scadenza ultima redazione e verifica verbali}
La scadenza per la verifica di tutti i verbali, interni ed esterni, è fissata per il 2021-01-10.
Il verificatore incaricato è Michele Veronesi, ad esclusione dei verbali da lui redatti. Per questi dovrà
essere identificato un altro verificatore.
Di conseguenza, tutti i verbali devono essere caricati nel branch \verb|feature/verbali| entro il 2021-01-09
dai redattori incaricati.

\subsection{Fissato meeting esterno con i proponenti}
Viene fissato un meeting con i RedBabel il 2021-01-08 alle ore 17.00 in videoconferenza sulla piattaforma Google Meet
al fine di discutere la prima bozza dell'analisi dei requisiti.

\section{Conclusione}
Il gruppo fisserà un ultimo meeting interno prima della revisione dei requisiti dopo il 2021-01-08 per
le verifiche e validazioni finali, oltre che per la presentazione di eventuali problematiche sorte.


%IMPORTANTE: ricordare di compilare il registro delle decisioni prese (file tracciamenti.tex)