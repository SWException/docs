\section{Informazioni}
\textbf{Data}: 2021-04-26 \\
\textbf{Orario}: 18:00 - 19:00\\
\textbf{Luogo}: piattaforma Google Meet \\\\
\textbf{Partecipanti}:\begin{list}{*}{\setlength{\itemsep}{0cm}}
	\item Marco Canovese
	\item Nicole Davanzo
	\item Ivan Furlan
	\item Gianmarco Guazzo
	\item Stefano Lazzaroni
	\item Francesco Trolese
	\item Michele Veronesi
	\item proponente Red Babel
\end{list}

\section{Ordine del giorno}
\begin{enumerate}
	\item Test del prodotto da parte del proponente
\end{enumerate}

\section{Svolgimento}
\subsection{Test del prodotto da parte del proponente} \label{_svolgimento}
Viene consegnato al proponente il puntatore al prodotto ospitato su architettura cloud AWS.
Dopo una serie di test di sistema, questo risulta sufficientemente robusto e i requisiti funzionali obbligatori stabiliti
nell'analisi dei requisiti sono stati completamente soddisfatti.\\
Tuttavia vengono suggerite un paio di migliorie da apportare in vista della Revisione di Accettazione, comunque entro i limiti
delle risorse ancora disponibili:
\begin{enumerate}
	\item nel front-end l'utilizzo della funzione \textit{getStaticProp} per fare il rendering delle pagine con cadenza temporale, e non just-in-time
		  ad ogni richiesta. Questo dovrebbe aumentare la reattività del sistema, tuttavia qualsiasi modifica attuata dal venditore sarà visibile solo quando
		  è il momento di rifare il rendering delle pagine;
	\item per i pagamenti con Stripe nel checkout, l'adozione del flusso webhook, il quale rende il sistema più robusto ai fallimenti di connessione durante il pagamento di un ordine.
\end{enumerate}

\section{Conclusione}
Il gruppo SWException proverà a comprendere ed implementare le due migliorie suggerite dal proponente, tuttavia non ne assicura l'esito positivo viste le scarse risorse disponibili.
Sarà meglio consegnare un prodotto con meno funzionalità ma che sia verificato a dovere, visto che queste non sono specificate nel documento di analisi dei requisiti.
