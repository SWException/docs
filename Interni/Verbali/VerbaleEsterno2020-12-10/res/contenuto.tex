\section{Informazioni}
\textbf{Data}: 2020-12-10\\
\textbf{Orario}: 17:00-17:30\\
\textbf{Luogo}: Zoom\\\\
\textbf{Partecipanti}:\begin{list}{*}{\setlength{\itemsep}{0cm}}
	\item Marco Canovese
	\item Nicole Davanzo
	\item Ivan Furlan
	\item Gianmarco Guazzo
	\item Stefano Lazzaroni
	\item Francesco Trolese
	\item Michele Veronesi
	\item RebBabel
\end{list}

\section{Introduzione}
In data 2020-12-10 si è tenuto il primo meeting con tutti i componenti del gruppo e i rappresentanti dell'organizzazione Red Babel, proponenti del capitolato \textit{C2-Emporio Lambda} sulla piattaforma \glock{\textit{Google Meet}}. \\
Tutti i membri del gruppo \textit{SWException} risultano presenti all'incontro.

\subsection{Ordine del giorno}
\begin{enumerate}
    \item Presentazioni.
    \item Motivazioni che hanno portato il gruppo \textit{SWException} alla scelta del capitolato C2-\textit{Emporio Lambda}.
    \item Consigli sulla comunicazione tra gruppo e proponenti.
    \item Domande del gruppo ai proponenti.
    \item Varie ed eventuali
\end{enumerate}

\section{Svolgimento}

\subsection{Presentazioni}
I membri del gruppo \textit{SWException} si sono presentati ad uno ad uno ponendo un accento sulla loro situazione accademica (eventuali esami arretrati ed esami da svolgere nel corso del presente anno accademico). A loro volta i due responsabili dell'azienda proponente, Milo Ertola e Alessandro Maccagnan hanno fatto una breve presentazione riguardante \emph{Red Babel} e gli altri progetti che seguono al momento.

\subsection{Le motivazioni della scelta}
Il gruppo \textit{SWException}, in seguito alla domanda posta dai proponenti, ha spiegato le ragioni che hanno fatto ricadere la scelta del capitolato su Emporio Lambda, ovvero:
	\begin{itemize}
	\item La compatibilità con le preferenze degli altri gruppi, che ha reso possibile evitare di entrare in conflitto con questi ultimi;
	\item L'interesse dei membri del gruppo verso le tecnologie che verranno impiegate per il compimento del progetto, in particolare l'\glock{\textit{architettura a microservizi}} e gli \glock{\textit{Amazon Web Services}};
	\item La disponibilità dimostrata fin dall'inizio dai committenti ad offrire il loro supporto qualora il gruppo ne necessitasse per chiarimenti sulle tecnologie da utilizzare e consigli riguardi le scelte più critiche.
	\end{itemize}
\subsection{Consigli dei committenti sulla comunicazione}
I committenti hanno sottolineato fin da subito l'importanza di una efficace comunicazione con il gruppo. 
Il loro tipico approccio è quello di tenersi in contatto con i gruppi in maniera costante sui vari canali \glock{\textit{Slack}} e di organizzare quando necessario un meeting (per via telematica).
Tuttavia non vogliono che si approfitti del loro tempo in maniera inappropriata, cioè ponendo domande la cui risposta potrebbe essere facilmente reperita attraverso altre vie (ad es. libri di testo o una ricerca su \textit{Google}).
Viene vivamente sconsigliato al gruppo di interrompere la comunicazione con i committenti per un lungo lasso di tempo perché ciò porta quasi inevitabilmente al fallimento del progetto.

\subsection{Domande}
Alla domanda del gruppo riguardante il pagamento del piano di abbonamento per usufruire degli \glock{\textit{AWS}} viene risposto che i limiti del piano gratuito (ed eventualmente del piano di prova per studenti) sono più che sufficienti a soddisfare il fabbisogno di risorse necessarie al sistema da realizzare.

\section{Conclusioni}
I committenti si dichiarano disponibili a partecipare ad un meeting per future domande che sorgeranno inevitabilmente al momento della redazione dell'\glock{\textit{Analisi dei Requisiti}}. La data non viene però stabilita. 