\section{Informazioni}
\textbf{Data}: 2020-12-28\\
\textbf{Orario}: 17:45-18:45\\
\textbf{Luogo}: Google Meet\\\\
\textbf{Partecipanti}:\begin{list}{*}{\setlength{\itemsep}{0cm}}
	\item Marco Canovese
	\item Nicole Davanzo
	\item Ivan Furlan
	\item Gianmarco Guazzo
	\item Stefano Lazzaroni
	\item Francesco Trolese
	\item Michele Veronesi
	\item RebBabel
\end{list}

\section{Introduzione}
In data 2020-12-28 si è tenuto il secondo incontro con i proponenti del capitolato C2-\textit{Emporio Lambda}.
Tutti i membri del gruppo \textit{SWException} risultano presenti all'incontro che si è svolto in modalità videoconferenza su Google Meet.

\subsection{Ordine del giorno}
\begin{enumerate}
  	\item Discussione sugli attori individuati;
    \item Discussione sui casi d'uso individuati .
 
\end{enumerate}

\section{Svolgimento}


\subsection{Discussione sugli attori individuati}
Il gruppo ha esposto ai proponenti i principali attori individuati nel sistema che si andrà a produrre. 
I proponenti chiariscono che il ruolo di amministratore non è da prevedere all'interno della piattaforma in quanto in un ambiente di produzione questo ruolo viene svolto dal fornitore.

\subsection{Discussione sui casi d'uso individuati}
I membri del gruppo \textit{SWException} hanno presentato rapidamente i casi d'uso individuati sino a quel momento per avviare una discussione critica con i proponenti al fine di individuare eventuali mancanze ed in particolare rimuovere ciò che ritenevano superfluo,  in particolare:
\begin{itemize}

\item Non è necessario prevedere un caso d'uso per l'effettuazione di ordini di prova in quanto il venditore assumerà che tutto funzioni correttamente in quanto già testato dal fornitore;
\item L'account amministratore non dovrà avere accesso al negozio ma solo limitatamente alla parte di backoffice ed è necessario prevedere due pagine di autenticazioni separate per clienti e per venditori;
\item I proponenti chiariscono che è necessario prevedere la possibilità di avere più utenti di ruolo venditore all'interno del sistema che sono creati dall'amministratore; 
\item I proponenti chiariscono che la gestione degli utenti non è richiesta salvo per le operazioni di recupero password dimenticata che devono essere previste da un caso d'uso apposito;
\item È stato richiesto da parte dei proponenti di prevedere che la possibilità di aggiungere prodotti al carrello sia permessa solamente agli utenti autenticati;
\item Viene considerata superflua la gestione dei codici promozionali i proponenti richiedono di rimuovere il caso d'uso;
\item I proponenti consigliano di utilizzare contentfull come CMS;
\item I proponenti preferirebbero una navigazione basata su categorie e/o tag anzichè su una barra di ricerca.

\end{itemize}

\subsection{Conclusioni}
Alla luce del colloquio intercorso con i proponenti il gruppo prende atto delle richieste pervenute e delibera le decisioni delle quali si fornisce riepilogo nella sezione successiva.  Si rimanda ad un momento successivo la deliberazione relativamente a consigli e richieste relative all'implementazione e sviluppo del prodotto.