\section{Informazioni}
\textbf{Data}: 2021-02-15\\
\textbf{Orario}: 13.00-13.30\\
\textbf{Luogo}: Zoom\\\\
\textbf{Partecipanti}:\begin{list}{*}{\setlength{\itemsep}{0cm}}
	\item Marco Canovese
	\item Nicole Davanzo
	\item Ivan Furlan
	\item Gianmarco Guazzo
	\item Stefano Lazzaroni
	\item Francesco Trolese
	\item Michele Veronesi
	\item Prof. Cardin
\end{list}

\section{Introduzione}
In data 2021-02-15 si è tenuto un meeting con tutti i componenti del gruppo \textit{SWException} e il professor Cardin per discutere delle modifiche da aportare all'analisi dei requisiti e per avere dei chiarimenti per il PoC. \\

\subsection{Ordine del giorno}
\begin{enumerate}
    \item Discussione su diversi casi d'uso presentati nell'AdR;
    \item Chiarimenti sulla presentazione della technology baseline.
\end{enumerate}

\section{Svolgimento}

\subsection{Discussione su casi d'uso}
Con il professor Cardin abbiamo discusso di quali modifiche apportare all'AdR. Dopo una breve discussione, è risultato che il problema principale nell'AdR, presentata alla RR, erano i casi d'uso di visualizzazione. 
Essi infatti non indicavano nello specifico i dettagli di visualizzazione.
In seguito, è stato proposto di inserire dei diagrammi che permettono la visualizzazione schematica dei vari casi d'uso.

\section{Chiarimenti sulla presentazione della technology baseline}
Il team ha presentato brevemente al prof Cardin la propria idea per il PoC.

\section{Conclusioni}
In seguito all'incontro è stato deciso di specificare i casi d'uso di visualizzazione e di inserire i vari diagrammi. Inoltre grazie al confronto con il professor Cardin abbiamo potuto confermare la nostra idea iniziale per il PoC.
