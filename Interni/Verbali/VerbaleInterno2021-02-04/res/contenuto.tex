\section{Informazioni}
\textbf{Data}: 2021-02-04\\
\textbf{Orario}: 14:30 - 15:30\\
\textbf{Luogo}: Zoom\\\\
\textbf{Partecipanti}:\begin{list}{*}{\setlength{\itemsep}{0cm}}
	\item Marco Canovese
	\item Nicole Davanzo
	\item Ivan Furlan
	\item Gianmarco Guazzo
	\item Francesco Trolese
	\item Michele Veronesi
	\item Stefano Lazzaroni
\end{list}

\section{Introduzione}
In data 2021-02-04 si è tenuta in videoconferenza una riunione con tutti i componenti del gruppo al fine di definire la struttura del Proof of Concept da presentare nel colloquio di Technology Baseline.

\subsection{Ordine del giorno}
\begin{enumerate}
    \item Definizione delle tecnologie coinvolte nel PoC;
    \item Assegnazione delle tecnologie ai vari componenti del gruppo;
    \item Definizione della struttura del PoC.
\end{enumerate}

\section{Svolgimento}

\subsection{Definizione delle tecnologie coinvolte nel PoC}
Dall'analisi del capitolato C2 e una discussione tra i membri del gruppo basata sulle conoscenze pregresse, anche se scarse, delle tecnologie menzionate dal proponente, si è deciso
di adottare le seguenti tecnologie.

\subsubsection{Tecnologie comuni tra tutti i moduli}
\begin{itemize}
	\item Come linguaggio di programmazione si userà TypeScript, il quale è una versione avanzata di JavaScript con controllo statico dei tipi;
	\item Per la gestione della configurazione e il deploy nell'infrastruttura cloud di AWS ci si affiderà al framework Serverless;
	\item Per l'esecuzione delle API generate si userà il servizio AWS Lambda, il quale calcola il costo di utilizzo solamente in base al tempo di esecuzione delle API;
	\item Per esportare le API ci si affiderà al servizio di AWS API Gateway.
\end{itemize}

\subsubsection{EML-FE}
Per il modulo del front-end si utilizzerà la libreria Next.js. In particolare perché ha una perfetta integrazione con TypeScript,
oltre che una serie di altri vantaggi.

\subsubsection{EML-BE}
Per il modulo del back-end si è deciso di adottare le seguenti tecnologie:
\begin{itemize}
	\item Per la persistenza dei dati ci si affiderà al database DynamoDB di Amazon, il quale si basa sulla tecnologia NoSQL;
	\item Per la gestione delle credenziali degli utilizzatori della piattaforma si utilizzerà Amazon Cognito, questo anche per evitare alcuni problemi di sicurezza;
	\item Per la gestione dei pagamenti si adotterà Stripe, in questo modo non ci si dovrà preoccupare di rispettare gli stretti standard di sicurezza imposti alle piattaforme che elaborano i pagamenti elettronici direttamente.
\end{itemize}

\subsection{Assegnazione delle tecnologie ai vari componenti del gruppo}
Visto e considerato che alcuni membri del gruppo hanno già una certa familiarità con le tecnologie sopra menzionate, si è deciso di suddividere l'apprendimento e approfondimento di ciascuna nel seguente modo:
\begin{itemize}
	\item A Nicole Davanzo, Francesco Trolese e Michele Veronesi viene assegnato il linguaggio TypeScript, il framework Serverless e in generale l'apprendimento dello stile di programmazione asincrono richiesto per il back end;
	\item A Marco Canovese e Ivan Furlan viene affidato l'apprendimento delle tecnologie Cognito e Stripe;
	\item Stefano Lazzaroni e Gianmarco Guazzo vengono incaricati di comprendere l'architettura e le tecnologie per il modulo del front end, viste anche le conoscenze pregresse di Guazzo sulle tecnologie coinvolte in questa parte dell'applicazione.
\end{itemize}

\subsection{Definizione della struttura del PoC}
La struttura del PoC dovrà essere il più semplice possibile, toccando la maggior parte, se non tutte, delle tecnologie sopra menzionate.
Di conseguenza la struttura definita dal gruppo è la seguente:
\begin{itemize}
	\item Una home page con elementi mock (ovvero che non si trovano nella base di dati). Da questa è possibile effettuare il login e la registrazione;
	\item L'implementazione di Amazon Cognito per la gestione delle credenziali, ovvero registrazione, autenticazione e recupero password;
	\item La visualizzazione di un carrello già impostato in DynamoDB e associato ad un preciso utente (già registrato);
	\item La possibilità di eseguire il pagamento con Stripe del suddetto carrello.
\end{itemize}

%IMPORTANTE: ricordare di compilare il registro delle decisioni prese (file tracciamenti.tex)