\section{Informazioni}
\textbf{Data}: 2021-03-10 \\
\textbf{Orario}: 14:30-15:30 \\
\textbf{Luogo}: Piattaforma online Zoom \\\\
\textbf{Partecipanti}:\begin{list}{*}{\setlength{\itemsep}{0cm}}
	\item Marco Canovese
	\item Nicole Davanzo
	\item Ivan Furlan
	\item Gianmarco Guazzo
	\item Stefano Lazzaroni
	\item Francesco Trolese
	\item Michele Veronesi
\end{list}

\section{Ordine del giorno}
\begin{enumerate}
	\item Definizione attività per progettazione architetturale
	\item Definizione team di sviluppo e relativa allocazione risorse
	\item Livello di riuso del PoC
\end{enumerate}

\section{Svolgimento}
\subsection{Definizione attività per progettazione architetturale}
Vengono definite le attività da eseguire al fine di ottenere un'architettura dell'applicazione che si andrà a sviluppare.
Per quanto riguarda il backend sarà necessario identificare quali sono le chiamate API, definendo quali risposte e a che tipo di chiamata
ogni API dovrà rispondere. Sarà inoltre necessario ragionare sull'architettura interna del backend, questa attività viene posticipata a quando il team
avrà più conoscenze circa le architetture serverless a microservizi.\\
Per quanto riguarda il frontend, ci si baserà sulle conoscenze pregresse di un membro del gruppo, il quale ha già usato queste tecnologie in ambito aziendale.

\subsection{Definizione team di sviluppo e relativa allocazione risorse}
Al fine di espletare le attività di cui sopra vengono allocate le seguenti risorse:
\begin{itemize}
	\item quattro membri del gruppo sono designate per la definizione delle scelte architetturali del backend e la definizione dei contratti delle API. Questi sono Canovese, Furlan, Trolese e Veronesi;
	\item i restanti tre membri del gruppo sono invece assegnati alla definizione delle scelte architetturali per il frontend.
\end{itemize}
La priorità per il team addetto al backend sarà la definizione dei contratti API esposte al frontend seguendo il paradigma REST ed API-first.
Queste saranno definite usando lo standard openAPI, in modo da poter generare dei mock automatici e permettere al team frontend di testare le scelte architetturali su queste API "finte".

\subsection{Livello di riuso del PoC}
Il codice prodotto nella Proof of Concept creata per la Technology Baseline verrà quasi completamente riutilizzato, visto che si trattava di una piccola quantità di lavoro, necessaria solamente alla compresione
delle tecnologie, è improbabile che intralci qualche scelta architetturale successiva.

\section{Conclusione}
Il team prevede di tenere un meeting con il proponente al fine di mostrare lo stato di avanzamento dei lavori e ricevere un feedback nella settimana p.v. La data precisa è ancora da definire.