\section{Informazioni}
\textbf{Data}: 2021-01-08\\
\textbf{Orario}: 17:00-17:30\\
\textbf{Luogo}: Zoom\\\\
\textbf{Partecipanti}:\begin{list}{*}{\setlength{\itemsep}{0cm}}
	\item Marco Canovese
	\item Nicole Davanzo
	\item Ivan Furlan
	\item Stefano Lazzaroni
	\item Francesco Trolese
	\item Michele Veronesi
	\item RebBabel
\end{list}

\section{Introduzione}
In data 2021-01-08 si è tenuto il terzo meeting con tutti i componenti del gruppo \textit{SWException} e i rappresentanti dell'organizzazione \textit{RedBabel}, proponenti del capitolato \textit{C2-Emporio Lambda} sulla piattaforma \glock{\textit{Google Meet}}. \\

\subsection{Ordine del giorno}
\begin{enumerate}
    %\item Chiarimenti sui requisiti richiesti dal proponente;
    \item Discussione su diversi casi d'uso presentati nell'AdR;
    \item Discussione sui requisiti prestazionali.
\end{enumerate}

\section{Svolgimento}

%\subsection{Chiarimenti requisiti}
%Durante la stesura dell'analisi dei requisiti sono sorti alcuni dubbi riguardo alcuni requisiti.
%In questo incontro i proponenti hanno chiarito che i requisiti fondamentali che il gruppo deve seguire sono quelli riportati nel capitolo 4 della documentazione fornita dal committente.
%La sezione dedicata alle Background information è invece da considerare come non obbligatori.

\subsection{Discussione su casi d'uso}
Dopo una breve presentazione dei casi d'uso ai RedBabel, il gruppo ha voluto chiarire alcuni punti.
I casi discussi sono:
\begin{itemize}
	%\item Definizione della tipologia di venditore;
	\item Modalità annullamento ordine;
	\item Ricerca per la lista degli ordini del venditore;
	\item Ricerca per la lista degli ordini del cliente;
	\item Possibilità di ordinare plp per diversi tag;
\end{itemize}

%\subsubsection{Definizione tipologia di venditore}
%Dopo una discussione avuta con un altro gruppo, che sta svolgendo lo stesso capitolato, è sorto il dubbio di aver gestito il venditore in modo errato.
%Esposta la nostra incertezza ai proponenti, ci hanno rassicurato che la nostra visione era quella da loro desiderata.
%Si tratta quindi di vedere i diversi account venditore come dipendenti di un'unica azienda, proprietaria del sito.

\subsubsection{Form contatto venditore}
Dopo una discussione sui resi ed annullamenti di un ordine da parte dei clienti, fornitori e proponenti si sono accordati di proseguire con l'idea presentata nel documento \textit{\dext{AnalisiDeiRequisiti\_1.0.0}}.

\subsubsection{Ricerca per la lista degli ordini del venditore}
%Secondo il team \textit{SWException} l'introduzione di una ricerca all'interno della lista degli ordini del venditore riuscirebbe a aumentare l'efficacia del sito.
%Per questo motivo abbiamo proposto ai RedBabel di aggiungerla tra i requisiti richiesti. La risposta ricevuta è stata quella di inserirla tra quelli desiderabili.
Il proponente durante l'incontro ha manifestato l'interesse ad avere una modalità di filtri degli ordini lato venditore. Questo verrà quindi inserito tra i requisiti desiderabili.

\subsubsection{Ricerca per la lista degli ordini del cliente}
%Per quanto riguarda la ricerca per la lista degli ordini del cliente, ovvero lo storico, il team pensava di non inserirla. I proponenti hanno confermato questa decisione.
La ricerca per la lista degli ordini del cliente è invece non necessaria, in quanto basta visualizzare una lista statica non filtrabile degli ordini effettuati dal singolo cliente.

\subsubsection{Possibilità di ordinare plp}
La possibilità di ordinare una plp era un altro punto di discussione che il team ha voluto trattare. Dopo la spiegazione dei dubbi che il team aveva, i proponenti hanno suggerito di inserire l'ordinamento per alcune specifiche che accomunano i diversi siti e-ccomerce sul web, per esempio il prezzo.

\section{Discussione sui requisiti prestazionali}
I proponenti, dopo una breve discussione, hanno confermato che non ci sono requisiti prestazionali richiesti.

\section{Conclusioni}
Grazie a questo meeting il team ha potuto chiarire i diversi dubbi che erano sorti durante l'AdR.