\section{Informazioni}
\textbf{Data}: 2020-12-22\\
\textbf{Orario}: 14:30-18:00\\
\textbf{Luogo}: Zoom\\\\
\textbf{Partecipanti}:\begin{list}{*}{\setlength{\itemsep}{0cm}}
	\item Marco Canovese
	\item Nicole Davanzo
	\item Ivan Furlan
	\item Gianmarco Guazzo
	\item Stefano Lazzaroni
	\item Francesco Trolese
	\item Michele Veronesi
\end{list}

\section{Introduzione}
In data 2020-12-22 si è tenuta in videoconferenza una riunione con tutti i componenti del gruppo. \\
L'incontro è stato anticipato di un giorno rispetto alla data fissata nel precedente verbale per via di qualche impegno sorto ad alcuni membri. 

\subsection{Ordine del giorno}
\begin{enumerate}
    \item Punto della situazione sulle norme di progetto;
    \item Chiarire l'uso delle pull request per effettuare il processo di verifica;
    \item Discussione di gruppo sull'analisi dei requisiti;
    \item Suddivisione e pianificazione del lavoro per i documenti ancora da redarre.
\end{enumerate}

\section{Svolgimento}
\subsection{Punto della situazione sulle norme di progetto} 
Nella prima parte dell'incontro i verificatori delle norme di progetto Michele Veronesi, Stefano Lazzaroni e Gianmarco Guazzo hanno esposto i primi problemi riscontrati nel documento. Non essendo ancora terminata la verifica ulteriori loro segnalazioni saranno fatte successivamente ai diretti interessati.\\
Per il momento il documento sembra esaustivo almeno per le sezioni di principale interesse nella fase corrente e può quindi essere utilizzato come primo riferimento scritto per tutti, seppur ancora non verificato e passibile quindi di modifiche.

\subsection{Chiarire l'uso delle pull request per effettuare il processo di verifica}
Dopo che alcuni abbiano esposto dei chiarimenti sull'utilizzo delle pull requests, e considerata la poca familiarità da parte dei membri del gruppo, Stefano Lazzaroni ha mostrato a tutti un esempio pratico su GitHub così da avere una chiara idea su come utilizzarle per il processo di verifica.\\
Per permettere di avere più verificatori su una stessa pull request è stato deciso di rendere la repository pubblica. Ciò, se dovesse servire, permette anche di avere più assegnatari nelle singole issue.

\subsection{Discussione di gruppo sull'analisi dei requisiti}
Il gruppo ha iniziato una prima analisi del capitolato C2 "EmporioLambda: piattaforma di e-commerce in stile Serverless" stilando una prima lista di casi d'uso e requisiti.\\
Durante tale analisi sono sorte domande sul capitolato ed è stato quindi deciso di contattare l'azienda proponente RedBabel, la quale si è resa disponibile per un incontro in data 2020-12-28 alle ore 17:45.

\subsection{Suddivisione e pianificazione del lavoro per i documenti ancora da redarre} 
Al termine dell'incontro è emerso che sia preferibile concentrarsi prima sul piano di progetto e sull'analisi dei requisiti, e continuare successivamente il piano di qualifica.
È stato quindi deciso di incaricare Gianmarco Guazzo e Francesco Trolese per finire di redarre il piano di progetto entro il 2020-12-28. Gli altri membri del gruppo si occuperanno invece di continuare l'analisi del capitolato iniziata in questa riunione, così che in vista dell'incontro con il proponente ci sia l'opportunità di esporre ulteriori domande che potrebbero sorgere.

\section{Conclusione}
Il prossimo incontro sarà quello fissato con il proponente che come sopra riportato si terrà in data 2020-12-28 alle 17:45.

%IMPORTANTE: ricordare di compilare il registro delle decisioni prese (file tracciamenti.tex)