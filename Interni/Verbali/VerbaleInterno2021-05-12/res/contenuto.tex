\section{Informazioni}
\textbf{Data}: 2021-05-12 \\
\textbf{Orario}: 14:30-15:30 \\
\textbf{Luogo}: piattaforma virtuale Zoom \\\\
\textbf{Partecipanti}:\begin{list}{*}{\setlength{\itemsep}{0cm}}
	\item Marco Canovese
	\item Nicole Davanzo
	\item Ivan Furlan
	\item Gianmarco Guazzo
	\item Stefano Lazzaroni
	\item Francesco Trolese
	\item Michele Veronesi
\end{list}

\section{Ordine del giorno}
\begin{enumerate}
	\item Cambiamento way of working team backend
	\item Decisione architettura singolo componente backend
\end{enumerate}

\section{Svolgimento}
\subsection{Cambiamento way of working team backend}
Dopo un'attenta riflessione scaturita dalla lezione del 2021-03-15 del docente Cardin circa gli stili architetturali monilite e microservizi,
il team dedicato allo sviluppo e alla progettazione si è reso conto di non aver compreso a pieno lo stile architetturale a microservizi richiesto dal proponente.\\
Per cercare di rendere il lavoro più aderente a quanto compreso dalla lezione si decide di tracciare una netta definizione dei microservizi (definiti in base al paradigma
domain driven development, su cui il gruppo ha cercato di acquisire una quanto meno superficiale conoscenza).\\
Di conseguenza il way of working viene riadattato in questo modo: il codice sorgente di ogni microservizio sarà ospitato in una repository apposita, con un proprio processo di
continuous integration indipendente. Grazie alla definizione dei contratti delle API, lo sviluppo delle singole componenti potrà avvenire indipendentemente dalle altre.

\subsection{Decisione architettura singolo componente backend}
Per il singolo componente si seguirà l'architettura a livelli, ovvero presentazione (composto dagli handler delle API), core (contente la business logic) e repository
(per l'interfacciamento con eventuali basi di dati e in generale altre componenti, anche di terze parti).

\section{Conclusione}
Si decide di implementare quanto sopra riportato, e di effettuare un incontro con il proponente e il docente Cardin al fine di avere un riscontro esperto.