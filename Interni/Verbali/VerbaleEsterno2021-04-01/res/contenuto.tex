\section{Informazioni}
\textbf{Data}: 2021-04-01 \\
\textbf{Orario}: 12:00-12:30\\
\textbf{Luogo}: piattaforma Zoom \\\\
\textbf{Partecipanti}:\begin{list}{*}{\setlength{\itemsep}{0cm}}
	\item Marco Canovese
	\item Nicole Davanzo
	\item Ivan Furlan
	\item Gianmarco Guazzo
	\item Stefano Lazzaroni
	\item Francesco Trolese
	\item Michele Veronesi
	\item docente Vardanega
\end{list}

\section{Ordine del giorno}
\begin{enumerate}
	\item Richiesta informazioni manuale manutentore
	\item Richiesta informazioni manuale utente
\end{enumerate}

\section{Svolgimento}
\subsection{Richiesta informazioni manuale manutentore}
Viene posto al docente un quesito per chiarire quali siano le informazioni da includere nella redazione del manuale manutentore.
Al gruppo viene fornita una esaustiva risposta che indica di pensare al documento come indirizzato a tutti gli stakeholder che
abbiano interesse nell'estendere e mantenere il prodotto. Dovrà quindi essere la mappa dell'impero, utile non solo per guardarvi all'interno
ma anche per fare delle conquiste esterne.

\subsection{Richiesta informazioni manuale utente}
Lo stesso quesito viene posto per quanto riguarda il manuale per l'utilizzatore. Dall'incontro con il docente Vardanega di evince che
la profondità di dettaglio da raggiungere dovrà essere tale da non offendere il lettore e dovrà essere abbastanza condensato da non far demordere
un eventuale lettore. È possibile anche produrre materiale audio visivo integrativo.

\section{Conclusione}
Verrà affidata la redazione del manuale utente al team di frontend, mentre per quello manutentore si dividerà in backend e frontend, di conseguenza ogni
parte sarà redatta dal team addetto.