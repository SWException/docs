\section{Informazioni}
\textbf{Data}: 2020-12-07\\
\textbf{Orario}: 14:30-16:30\\
\textbf{Luogo}: Zoom\\\\
\textbf{Partecipanti}:\begin{list}{*}{\setlength{\itemsep}{0cm}}
	\item Marco Canovese
	\item Nicole Davanzo
	\item Ivan Furlan
	\item Gianmarco Guazzo
	\item Stefano Lazzaroni
	\item Francesco Trolese
	\item Michele Veronesi
\end{list}

\section{Introduzione}
In data 2020-12-07 si è tenuto un meeting con tutti i componenti del gruppo, al fine
di decidere il capitolato da sviluppare per la Revisione dei Requisiti e avviare le relative attività.

\subsection{Ordine del giorno}
\begin{enumerate}
    \item Scelta definitiva del capitolato per cui concorrere nella gara d'appalto alla Revisione di Accettazione
    \item Caratteristiche del documento interno \textit{Studio di Fattibilità}
    \item Suddivisione del carico di lavoro per la produzione del documento contenente le \textit{Norme di Progetto}
    \item Stesura del Glossario
    \item Varie ed eventuali
\end{enumerate}

\section{Svolgimento}
\subsection{Scelta definitiva del capitolato per cui concorrere nella gara d'appalto alla Revisione di Accettazione}
	La scelta dei componenti del gruppo è ricaduta all'unanimità sul capitolato C2 dell'azienda RedBabel, \textit{EmporioLambda: piattaforma di e-commerce in stile Serverless}.\\
	Di conseguenza le attività inerenti al progetto didattico verteranno d'ora in poi all'aggiudicazione di tale capitolato alla Revisione di Accettazione del 18 Gennaio 2021.\\
	Maggiori informazioni sulla motivazione di tale scelta saranno presenti nello Studio di Fattibilità, tutt'ora in fase di redazione.
	
	\subsection{Caratteristiche del documento interno Studio di Fattibilità}
	È stata definita la struttura del documento Studio di Fattibilità, visto che il carico di lavoro per la sua stesura era già stato definito con la riunione verbalizzata nel \verb|vInterno-02-12-2020|.\\
	In particolare, per ogni capitolato in esame dovranno essere riportate le seguenti informazioni inerenti al suo studio:
	\begin{itemize}
		\item informazioni generali
		\item descrizione del capitolato
		\item finalità del progetto
		\item tecnologie interessate
		\item aspetti positivi
		\item criticità e fattori di rischio
		\item conclusioni
	\end{itemize}
	
	\subsection{Suddivisione del carico di lavoro per la produzione del documento contenente le Norme di Progetto}
	È stato suddiviso il carico di lavoro per la stesura delle Norme di Progetto in questo modo:
	\begin{itemize}
		\item \textit{processi primari:} incaricati Stefano Lazzaroni e Marco Canovese
		\item \textit{processi organizzativi:} incaricati Ivan Furlan e Gianmarco Guazzo
		\item \textit{processi di supporto:} incaricati Francesco Trolese, Nicole Davanzo e Michele Veronesi
	\end{itemize}
	La deadline stabilita per la consegna di una prima bozza del documenti è il giorno 2020-12-16. Il giorno successivo inizierà il processo di verifica di tale documento in seguito ad un meeting apposito.

	\subsection{Stesura del Glossario}
	Per quanto riguarda la stesura del glossario, è stato stabilito che ogni componente è incaricato ad integrarlo man mano che vengono redatte le parti di loro competenza di tutti gli altri documenti.
	
	\subsection{Varie ed eventuali}
	Per quanto riguarda la deadline del 2020-12-13 per la stesura dello Studio di Fattibilità definita nel \verb|vInterno-02-12-20| è stato fissato un ulteriore meeting per l'avvio della verifica di tale documento il giorno 14 dicembre 2020.
	
	\subsection{Conclusioni}
	In seguito alla riunione l'intero gruppo ha preso contatti con il proponente del capitolato C2 scelto tramite l'applicazione Slack$_G$ (come specificato nel documento di introduzione). È stata quindi fissata una riunione con i rappresentanti dell'azienda RedBabel il giorno 10 dicembre 2020 alle ore 17 CET con tutti i componenti del gruppo \verb|SWException|.
	