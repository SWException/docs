\section{Informazioni}
\textbf{Data}: 2020-12-17\\
\textbf{Orario}: 14:30-16:30\\
\textbf{Luogo}: Zoom\\\\
\textbf{Partecipanti}:\begin{list}{*}{\setlength{\itemsep}{0cm}}
	\item Marco Canovese
	\item Nicole Davanzo
	\item Ivan Furlan
	\item Gianmarco Guazzo
	\item Stefano Lazzaroni
	\item Francesco Trolese
	\item Michele Veronesi
\end{list}

\section{Introduzione}
In data 2020-12-17 alle ore 14.30 si è tenuta in videoconferenza una riunione con tutti i componenti del gruppo per discutere delle norme di progetto, valutarne la prima stesura e definire le prossime scadenze.
\subsection{Ordine del giorno}
\begin{enumerate}
    \item Gestione del registro delle modifiche e scelta convenzione per nomenclatura dei file
    \item Discussione relative alle norme di progetto
    \item Organizzazione prossime scadenze 
\end{enumerate}

\section{Svolgimento}

\subsection{Gestione del registro delle modifiche e scelta convenzione per nomenclatura dei file}
Successivamente è stato discusso l'utilizzo del registro delle modifiche e si è proposto di lavorare attraverso le Issue messe a disposizione dallo strumento GitHub. Michele Veronesi si è proposto di fare il refactoring del documento  dello studio di fattibilità mentre Francesco Trolese si è proposto per il refactoring delle Norme di Progetto. 

È stata poi discusso lo standard da utilizzare per la nomenclatura delle cartelle e dei file all'interno del progetto e delle documentazione. La decisione è ricaduta sull'utilizzo del Pascal Case (NomeProgetto) su tutte le cartelle fino a quella prima dei file di esecuzione, mentre per i file all'interno si userà una nomenclatura Snake Case (nome\_progetto).

\subsection{Discussione relative alle norme di progetto}

Nella seconda fase dell'incontro si è discusso principalmente sulle Norme di Progetto e sulla prima stesura del documento. Ogni componente del gruppo ha esposto il suo lavoro ed eventuali problematiche riscontrate che necessitavano il consiglio del gruppo. 

\subsection{Organizzazione prossime scadenze}
Come ultimo punto della videoconferenza sono state discusse le prossime scadenze relative alla stesura della documentazione e all'Analisi dei Requisiti. Le scadenze scelte sono state:
\begin{itemize}
	\item 2020-12-20 data ultima per poter fare il push delle modifiche sulle Norme di Progetto
	\item 2020-12-21 data di inizio per la verifica del documento Norme di Progetto 
	\item 2020-12-23 ore 14.30 incontro di tutti i componenti del gruppo per discutere ed iniziare le Analisi dei Requisiti per il capitolato scelto
	\item 2020-12-28 data per la prima stesura dei documenti Piano di Progetto e Piano di Qualifica
\end{itemize}

\section{Conclusione}
Come scritto in precedenza il prossimo incontro sarà il giorno 2020-12-23 per l'Analisi dei Requisiti.
