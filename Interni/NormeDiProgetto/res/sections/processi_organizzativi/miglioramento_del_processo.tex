\subsection{Miglioramento del processo}
\subsubsection{Scopo}
Il processo di Miglioramento permette di controllare le dinamiche relative al progetto con l'intento di migliorare l'efficacia e l'efficienza dei suoi processi.
\subsubsection{Descrizione}
Ciascun processo relativo al processo sarà monitorato da chi di dovere durante tutto il ciclo di vita del software. Il gruppo si pone l'obbiettivo di seguito la Pianificazione affiancandola allo sviluppo controllato dei componenti che compongono il software. Il giusto equilibrio tra pianificazione e controllo dello sviluppo sarà un punto fondamentale per il raggiungimento degli obbiettivi fissati. Per assicurare l'efficienza e l'efficacia dei processi verrà eseguita una loro revisione a intervalli appropriati in modo da confermare e conservare il lavoro svolto in precedenza. Il gruppo dovrà poi definire i miglioramenti necessari e attuarli in maniera adeguata nella revisione del processo in questione. I dati e gli storici saranno raccolti e analizzati nel Piano di Qualifica per scovare eventuali problematiche e individuare i punti di forza dei processi impiegati. L'analisi deve essere usata come feedback per migliorare i processi, raccomandando modifiche da applicare alle procedure, agli strumenti e alle tecnologie.
