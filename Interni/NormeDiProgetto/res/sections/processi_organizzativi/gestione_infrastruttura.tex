\subsection{Gestione dell'infrastruttura}
\subsubsection{Scopo}
Il processo di Gestione dell’Infrastruttura permette di stabilire e mantenere un modello necessario per qualsiasi altro processo. Essa può includere hardware, software, strumenti, tecniche, standard e strutture per lo sviluppo, il funzionamento o la
manutenzione.

\subsubsection{Descrizione}
Le prossime sezioni espongono gli strumenti impiegati dal gruppo SWException per quanto riguarda le attività di Coordinamento e Pianificazione del progetto. Viene suddivisa per software unilizzati che comprendono tutte le comunicazioni e le conferenze del gruppo. Per ciascuno strumento viene fornita una breve descrizione e l'utilità all'interno dell'organizzazione dei lavori nel gruppo.

\subsubsection{Aspettative}
Con il processo di Gestione dell’Infrastruttura si ha l'obbiettivo di mantenere la coerenza dei componenti sia di codice che di documentazione, rendendoli disponibili chiunque all'interno del gruppo. Gli strumenti e le tecnologie utilizzate servono a migliorare l'esperienza lavorativa e le comunicazioni tra i componenti del gruppo. 


\subsubsection{Per il Coordinamento}
Di seguito vengono descritti e spiegati gli strumenti impiegati dal gruppo di lavoro per il coordinamento delle attività di sviluppo e comunicazione.

\paragraph{Slack}
Slack è uno strumento collaborativo aziendale, utilizzato per inviare messaggi in modo istantaneo ai membri di un preciso team di lavoro. L’applicativo permette ad un utente di iscriversi
a diversi workspace. Per ogni workspace di appartenenza è concessa la creazione di canali
tematici dove sviluppare delle conversazioni specializzate. Ciò permette di evitare la trattazione di diversi argomenti su un’unica chat generale, come invece avviene in Telegram.Per questo è stato deciso dal gruppo di utilizzare Slack per le comunicazioni ufficiali e per la suddivisione del workspace in canali relativi ai vari ambi da trattare. Slack può inoltre venire integrato con diverse applicazioni come Google Calendar, Outlook Calendar e GitHub.
[..comunicazione azienda]

\paragraph{Telegram}
Telegram è uno strumento di messaggistica istantanea molto utilizzato per la creazione e gestione di gruppi e canali. Fin dal primo giorno di incontro e presentazione digitale, è sempre stato lo strumento più utilizzato per le decisioni e le discussioni relative a qualsiasi ambito del progetto. Il gruppo ha quindi deciso di utilizzare Telegram insieme a Slack come strumenti per la comunicazione interna ed esterna. Alcune funzionalità di cui questo strumento dispone sono:
\begin{itemize}
\item possibilità di accedere da diversi dispositivi contemporaneamente, cosa attuabile
grazie alla sincronizzazione istantanea del cloud;
\item vasta gamma d’impiego grazie alla possibilità di inviare messaggi di testo, messaggi
vocali, videomessaggi e file (con una dimensione massima di 1.5 GB);
\item si possono avere fino a 200 membri in un gruppo, che possono salire fino a 200 mila nel
caso di supergruppo (limite che può essere aumentato). È inoltre possibile impostare
amministratori con permessi selezionabili.
\end{itemize}

\paragraph{Zoom}
Zoom è un servizio di videotelefonia e chat online che tramite una piattaforma software peer-to-peer viene utilizzata per conferenze, lavoro e istruzione a distanza. E' stato utilizzato come primo strumento per le videoconferenze del gruppo. E' uno strumento facile da utilizzare e che come tutti i servizi di meeting permette di condividere lo schermo. Permette di registrare le videoconferenze e di condividerle tra i partecipanti attraverso un link. Sono state create dal gruppo delle coordinate, accessibili attraverso un link zoom appuntato sulla chat di Telegram, per le videochiamate in modo da semplificare la gestione delle stesse e per permettere ai partecipanti di avere una stanza comune per discutere.
le videochiamate (come descritto in 4.1.4.4)

\paragraph{Discord}
Discord è uno strumento gratuito di messaggistica istantanea utilizzato per chattare attraverso tutti i dispositivi come smartphone, tablet e pc. Con Discord gli utenti possono:
\begin{itemize}
\item scriversi messaggi di testo;
\item comunicare tra loro collegando il microfono oppure fare delle videochiamate a gruppi.
\end{itemize}
È quindi uno strumento che permette di comunicare in tempo reale e fare videochat di gruppo. 
le videochiamate (come descritto in §4.1.4.4)

\paragraph{Outlook}
Outlook è un'app Web e mobile di gestione delle informazioni personali di Microsoft composta da servizi di posta Web, calendario, contatti e attività. E' stata utlizzata dal gruppo per la creazione di una email comune a tutti i partecipanti a rappresentanza dei SWException. 

\paragraph{GitHub}
GitHub è un sito Web e un servizio basato su cloud che aiuta gli sviluppatori a memorizzare e gestire il proprio codice, nonché a tracciare e controllare le modifiche al proprio codice. GitHub è essenzialmente sviluppato su due principi connessi tra di loro:
\begin{itemize}
\item Controllo versioni,
\item Git.
\end{itemize}
Il controllo della versione aiuta il gruppo a tenere traccia e gestire le modifiche al codice di un progetto software. Man mano che un progetto software cresce, il controllo della versione diventa essenziale per il completamento dei requisiti. Il controllo della versione consente agli sviluppatori di lavorare in sicurezza attraverso Branch e Merge. Il Branch duplica parte del codice sorgente (chiamato repository). Ogni componente del gruppo può quindi apportare modifiche in modo sicuro a quella parte del codice senza influire sul resto del progetto. Una volta che lo sviluppatore ottiene che la sua parte di codice funzioni correttamente, può unire nuovamente quel codice nel codice sorgente principale per renderlo ufficiale. Il processo di unione è definito merge.
Git è invece un sistema di controllo della versione distribuito: l'intera base di codice e la cronologia sono disponibili sul computer di ogni sviluppatore, il che consente facili branch e merge. 

\paragraph{Google Meets?}

\subsubsection{Per la Pianificazione}
Di seguito vengono invece esposti gli strumenti utilizzati per l’attività di Pianificazione
\paragraph{GanttProject}
Per il supporto all’attività Pianificazione del progetto e alla realizzazione di diagrammi di Gantt all’unanimità si è deciso di utilizzare lo strumento GanttProject, software open-source e multi-piattaforma.

