\subsection{Gestione dell'infrastruttura} \label{_gestioneInfrastruttura}
\subsubsection{Scopo}
Il processo di Gestione dell’Infrastruttura permette di stabilire e mantenere un modello necessario per qualsiasi altro processo. Essa può includere hardware, software, strumenti, tecniche, standard e strutture per lo sviluppo. 

\subsubsection{Descrizione}
Le prossime sezioni espongono gli strumenti impiegati dal gruppo SWException per quanto riguarda le attività di Coordinamento e Pianificazione del progetto. La lista presenterà i software utilizzati per tutte le comunicazioni e le conferenze del gruppo. Per ciascuno strumento viene fornita una breve descrizione e l'utilità all'interno dell'organizzazione dei lavori nel gruppo. 

\subsubsection{Aspettative}
Con il processo di Gestione dell’Infrastruttura si ha l'obbiettivo di mantenere la coerenza dei componenti sia di codice che di documentazione, rendendoli disponibili a chiunque all'interno del gruppo. Gli strumenti e le tecnologie utilizzate servono a migliorare l'esperienza lavorativa e le comunicazioni tra i componenti del gruppo. 


\subsubsection{Per il Coordinamento}
Di seguito viene presentata una breve descrizione degli strumenti utilizzati.

\paragraph{\glock{Slack}}
Slack è uno strumento collaborativo aziendale, utilizzato per inviare messaggi in modo istantaneo ai membri di un preciso team di lavoro. L’applicativo permette ad un utente di iscriversi
a diversi workspace. Per ogni workspace di appartenenza è concessa la creazione di canali
tematici dove sviluppare delle conversazioni specializzate. Ciò permette di evitare la trattazione di diversi argomenti su un’unica chat generale, come invece avviene in Telegram.Per questo è stato deciso dal gruppo di utilizzare Slack per le comunicazioni ufficiali e per la suddivisione del workspace in canali relativi ai vari ambi da trattare. Slack può inoltre venire integrato con diverse applicazioni come Google Calendar, Outlook Calendar e GitHub. Oltre all'utilizzo per comunicazioni interne la piattaforma verrà utilizzata per comunicare col proponente del progetto. 

\paragraph{\glock{Telegram}}
Telegram è uno strumento di messaggistica istantanea molto utilizzato per la creazione e gestione di gruppi e canali. Fin dal primo giorno di incontro e presentazione digitale, è sempre stato lo strumento più utilizzato per le decisioni e le discussioni relative a qualsiasi ambito del progetto. Il gruppo ha quindi deciso di utilizzare Telegram insieme a Slack come strumenti per la comunicazione interna. 

\paragraph{\glock{Zoom}}
Zoom è un servizio di videocall e chat online che tramite una piattaforma software peer-to-peer viene utilizzata per conferenze e lavoro. È stato utilizzato come primo strumento per le videoconferenze del gruppo. È uno strumento facile da utilizzare e come tutti i servizi di meeting permette di condividere lo schermo. Permette di registrare le videoconferenze e di condividerle tra i partecipanti attraverso un link. Sono state create dal gruppo delle coordinate, accessibili attraverso un link zoom appuntato sulla chat di Telegram, per le videochiamate in modo da semplificare la gestione delle stesse e per permettere ai partecipanti di avere una stanza comune per discutere.

\paragraph{\glock{Outlook}}
Outlook è un'app Web e mobile di gestione delle informazioni personali di Microsoft composta da servizi di posta Web, calendario, contatti e attività. È stata utilizzata dal gruppo per la creazione di una email comune a tutti i partecipanti a rappresentanza dei SWException.

\paragraph{\glock{GitHub}}
GitHub è un sito Web e un servizio basato su cloud che aiuta gli sviluppatori a memorizzare e gestire il proprio codice, nonché a tracciare e controllare le modifiche al proprio codice. GitHub è essenzialmente sviluppato su due principi connessi tra di loro:
\begin{itemize}
    \item Controllo versioni,
    \item \glock{Git}.
\end{itemize}
Il controllo della versione aiuta il gruppo a tenere traccia e gestire le modifiche al codice di un progetto software. Man mano che un progetto software cresce, il controllo della versione diventa essenziale per il completamento dei requisiti. Il controllo della versione consente agli sviluppatori di lavorare in sicurezza attraverso \glock{Branch} e \glock{Merge}. Il Branch duplica parte del codice sorgente (chiamato repository). Ogni componente del gruppo può quindi apportare modifiche in modo sicuro a quella parte del codice senza influire sul resto del progetto. Una volta che lo sviluppatore ottiene che la sua parte di codice funzioni correttamente, può unire nuovamente quel codice nel codice sorgente principale per renderlo ufficiale. Il processo di unione è definito merge.
Git è invece un sistema di controllo della versione distribuito: l'intera base di codice e la cronologia sono disponibili sul computer di ogni sviluppatore, il che consente facili branch e merge.

\paragraph{\glock{Google Meet}}
Google Meet è una piattaforma per videocall e conferenze simile a Zoom. Permette di gestire i video meeting attraverso link d'invito e salvataggio sul calendario. E' stato utilizzato dal gruppo per le comunicazioni esterne con il proponente.

\paragraph{\glock{Issue Tracking System}}
L'Issue Tracking System è un ottimo strumento per tenere traccia di attività, miglioramenti e bug per quanto riguarda il progetto. Il gruppo utilizzerà questo sistema per gestire l'avanzamento del software, i compiti assegnati ad ogni componente, le milestone e la verifica del codice. 

\subsubsection{Per la Pianificazione}
Di seguito viene invece espostolo  strumento utilizzato per l’attività di Pianificazione
\paragraph{\glock{GanttProject}}
Per il supporto all’attività Pianificazione del progetto e per la realizzazione di diagrammi è stato deciso di utilizzare di Gantt, software open-source e multi-piattaforma.

