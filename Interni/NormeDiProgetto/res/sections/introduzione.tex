\section{Introduzione} \label{_introduzione}
\subsection{Scopo del documento}
L'obiettivo del presente documento è quello di uniformare il modo di lavorare di tutti i
componenti del gruppo \textit{SWException} definendo quali sono le convenzioni adottate e gli
standard di riferimento in ogni attività per tutti i processi attivi o che dovranno essere
attivati.\\
Tutti i membri del gruppo sono tenuti a restare aggiornati con le future modifiche che saranno apportate al documento,
in modo da svolgere i compiti loro assegnati in modo coerente con quanto previsto.\\
Le decisioni di seguito riportate sul modo di operare vengono adottate seguendo gli standard in seguito elencati,
a seguito di una discussione collettiva in cui almeno il $50\%$ dei componenti approva la scelta.

\subsection{Glossario}
All'interno del documento vi sono parole che richiedono una definizione specifica del loro significato,
queste vengono marcate con il pedice "G" almeno alla loro prima occorrenza, in modo da rimuovere ambiguità
sulla semantica delle frasi. Il file che contiene queste informazioni è \dext{Glossario\_1.0.0}.

\subsection{Riferimenti}
\subsubsection{Riferimenti normativi}
\begin{itemize}
    \item \href{https://www.math.unipd.it/~tullio/IS-1/2020/Progetto/C2.pdf}{Capitolato C2}
\end{itemize}

\subsubsection{Riferimenti informativi}
\begin{itemize}
    \item \href{https://www.math.unipd.it/~tullio/IS-1/2009/Approfondimenti/ISO_12207-1995.pdf}{ISO 12207:1995}
    \item Libro di testo: Ingegneria del Software - $10^A$ edizione - Ian Sommerville
    \item \href{https://www.math.unipd.it/~tullio/IS-1/2020/Dispense/L03.pdf}{Lezione T3}
\end{itemize}
