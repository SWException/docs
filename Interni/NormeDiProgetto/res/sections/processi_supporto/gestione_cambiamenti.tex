\subsection{Gestione dei cambiamenti}
\subsubsection{Scopo}
La finalità di questo processo è assicurare una gestione appropriata, affidabile e tracciata
dei cambiamenti attuati per sopperire a mancanze e adattare tutti gli altri processi durante tutto lo sviluppo del progetto.

\subsubsection{Aspettative}
Con l'avvio del processo di gestione dei cambiamenti, il gruppo di lavoro di aspetta la riduzione dei costi provocati da un cambiamento
del \textit{way of working}, riuscendo ad elaborare risposte in tempi brevi ai vari problemi che possono sorgere a livello gestionale.

\subsubsection{Denominazione dei cambiamenti}
Per garantire il tracciamento, viene assegnato un codice univoco a tutti i cambiamenti, così formato:
\begin{itemize}
    \item Per i prodotti (documenti/artefatti)
        \begin{center}
            \textbf{CA[PR]-[DOC/ART]-[codice prodotto][X]}
        \end{center}
        in particolare \textbf{DOC} per i documenti e \textbf{ART} per il codice sorgente dell'applicazione.
    \item Per i processi
        \begin{center}
            \textbf{CA[PR]-ATT-[nome attività][X]}
        \end{center}
        dove il nome dell'attività viene scritto per intero.
\end{itemize}
In entrambi i casi, il parametro \textbf{[X]} indica un numero intero che incrementa ad ogni cambiamento su quel determinato prodotto/processo,
 con valore iniziale 1.\\
 Il parametro \textbf[PR] sta ad indicare la priorità, rappresentata da un numero compreso nell'insieme
 \[\{x \in \mathbb{N}_0 \ | \  x \% 5 = 0 \} \]
 ovvero l'insieme dei multipli di 5 non negativi. In casi eccezionali, possono essere assegnati numeri interi positivi non multipli di 5,
 in modo da inserire il cambiamento nella giusta posizione. Questa prassi non deve comunque entrare nella normalità, è necessario che si cerchi
 di assegnare le priorità con cautela.

 \subsubsection{Prassi generale}
 Le issue rappresentanti i vari cambiamenti seguono il ciclo di vita generale dei ticket.
 Al termine dello svolgimento di ogni attività, parte la verifica su questa appena terminata, la quale porta ad un'accettazione
del lavoro svolto o ad un rigetto. Nel primo caso si procede con la validazione, nel secondo caso il verificatore riapre il ticket inserendo i commenti
che esplicano il motivo del rifiuto. Nel caso in cui chi ha svolto l'attività non sia in accordo con il verificatore ci sarà un meeting tra i due, se anche
dopo questo i due si trovano in disaccordo interviene il \textit{Responsabile di Progetto}, che prenderà la decisione finale.