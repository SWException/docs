\subsection{Gestione di qualità}

\subsubsection{Scopo}
La finalità è quella di garantire la qualità prestabilita di prodotti e processi da sviluppare, rispettando le richieste del proponente.

\subsubsection{Descrizione}
Per esporre i valori di soglia delle metriche e gli standard da applicare al progetto, è stato introdotto il documento \textit{Piano di qualifica}.
In questa sezione ci proponiamo di illustrare come avviene il processo di gestione della qualità e l'istanziamento di un processo.

\subsubsection{Aspettative}
Dalla gestione della qualità, il team prevede di soddisfare a pieno le aspettative del proponente, rispettando gli standard di qualità con esso  precedentemente concordati.\\
Per ottenere ciò il team si aspetta:
\begin{itemize}
    \item Organizzare le attività e i processi in modo efficace e prolifico, conseguendo la qualità attesa;
    \item Raggiungere la qualità di prodotto prevista dal proponente, verificandone l'effettiva qualità.
\end{itemize}


\subsubsection{Processo gestione della qualità}
Il processo di gestione della qualità si articola nelle seguenti fasi:
\begin{itemize}
    \item\textbf{Studio}: il team individua l'obiettivo che ogni lavoro deve perseguire, studiando la quantità di risorse che esso richiederebbe;
    \item \textbf{Regolamentazione}: vengono stabilite le strategie da adottare per conseguire la qualità prestabilita, organizzando le risorse a disposizione;
    \item\textbf{Attuazione}: viene eseguito quanto scelto in precedenza. Questa fase permette di ottenere dei risultati concreti che possiamo verificare nella fase successiva;
    \item\textbf{Valutazione}: si verifica se i risultati ottenuti rispettano gli standard precedentemente richiesti.
\end{itemize}


\subsubsection{Denominazione metriche}
Per garantire uniformità, si è deciso di denominare le \glock{metriche} nel seguente modo:
\begin{itemize}
    \item Per i prodotti:
          \begin{center}
              \textbf{M[PD]-[DOC/S][X]}
          \end{center}
    \item Per i processi:
          \begin{center}
              \textbf{M[PR]-[DOC/S][X]}
          \end{center}
    \item Per i test:
          \begin{center}
              \textbf{M[TS]-[DOC/S][X]}
          \end{center}
\end{itemize}
La dicitura \textbf{DOC} sta ad indicare i documenti mentre \textbf{S} i prodotti software.
In tutti i casi, il parametro \textbf{[X]} indica un numero intero che stabilisce la metrica, la numerazione inizia da 1.

\subsubsection{Denominazione obiettivi}
Per la denominazione degli obiettivi, il team ha concordato:
\begin{itemize}
    \item Per i prodotti:
          \begin{center}
              \textbf{O[PD]-[DOC/S][X]}
          \end{center}
    \item Per i processi:
          \begin{center}
              \textbf{O[PR]-[DOC/S][X]}
          \end{center}
\end{itemize}
La dicitura \textbf{DOC} sta ad indicare i documenti mentre \textbf{S} i prodotti software.
In tutti i casi, il parametro \textbf{[X]} indica un numero intero che stabilisce la metrica, la numerazione inizia da 1.

\subsubsection{Istanziazione di un processo}
Quando un componente del team istanzia un nuovo processo, bisogna che verifichi i seguenti punti:
\begin{itemize}
    \item Il processo che si sta istanziando deve avere un unico obiettivo,  ovvero non deve sovrapporsi con quelli di altri processi  in sviluppo;
    \item Avviare una preventiva analisi dei rischi;
    \item Controllore e gestire le risorse a disposizione, in modo da aver un utilizzo ottimale.
\end{itemize}
Per gestire nel migliore dei modi le risorse è necessario fin da subito porre una data per il termine del processo.
