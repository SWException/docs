\subsection{Gestione della configurazione}
\subsubsection{Scopo}
L'obiettivo del processo è quello di regolare la produzione di documenti e codice sorgente.
Dal momento che la documentazione viene scritta in \LaTeX, come descritto nella sezione relativa
alla documentazione, quest'ultima può essere gestita con alcuni degli strumenti con
cui viene gestito il codice sorgente.

\subsubsection{Descrizione}
Tale processo descrive tutti gli strumenti utilizzati per la produzione di documenti, codice e diagrammi,
oltre che le modalità di versionamento e coordinamento del gruppo.

\subsubsection{Aspettative}
I risultati attesi da questo processo sono:
\begin{itemize}
    \item sistematizzare la produzione di codice e documentazione
    \item rendere uniforme l'utilizzo degli strumenti di versionamento coinvolti nel progetto
    \item classificare i prodotti dei vari processi implementati
\end{itemize}

\subsubsection{Versionamento}
\paragraph{Codice di versione per documenti e software}
Il numero di versione di ogni componente del prodotto e di ogni documento è definito nel seguente formato:
\begin{center}
    \textbf{[X].[Y].[Z]}
\end{center}
dove
\begin{itemize}
    \item \textbf{[X]} indica il numero di versione approvato dal \glock{Responsabile di Progetto}. Inizialmente è a 0.
    \item \textbf{[Y]} indica il numero di versione approvato dai \glock{Verificatori}. Inizialmente a 0 e riparte ad ogni
                       incremento di \textbf{[X]}.
    \item \textbf{[Z]} è il numero che identifica la versione in redazione. Viene quindi incrementato dai redattori
                       ad ogni aggiunta/modifica. Inizialmente a 0 e riparte ad ogni incremento di \textbf{[Y]}
\end{itemize}

\paragraph{Tecnologie coinvolte}
Per il processo di versionamento ci si affiderà al software \glock{Git} offerto come servizio dalla piattaforma \glock{GitHub}.\\
In particolare, è stata creata un'organizzazione su tale piattaforma con il nome \verb|SWException| in cui tutti i membri del gruppo
hanno il medesimo accesso in scrittura e lettura per tutte le repository esistenti.\\
Per il versionamento della documentazione viene usata una repository dal nome \verb|swe-docs|, in cui vengono riposti tutti i
documenti creati durante lo svolgimento del progetto didattico.\\
Successivamente, visto che si andrà a creare un'applicazione a \glock{microservizi}, e dunque composta da parti fortemente indipendenti,
verrà creata una repository per ogni modulo del prodotto.

\paragraph{Struttura delle repository}
Ogni membro del gruppo, in tutte le repository, dovrà attenersi alle convenzioni del \glock{GitFlow}, con o senza apposito plugin a discrezione dei singoli membri.
In particolare saranno presenti i seguenti branch:
\begin{itemize}
    \item \textbf{master:} utilizzato per ospitare le versioni di rilascio. In questo ramo dovrà esserci sempre una versione del prodotto utilizzabile
                           e pronto ad essere rilasciato in \glock{ambiente di produzione}. E' fatto divieto a tutti i membri del gruppo di effettuare \verb|commit|
                           direttamente su questo ramo, tutte le modifiche dovranno seguire il percorso previsto di GitFlow.
    \item \textbf{develop:} questo è il ramo utilizzato per le operazioni di sviluppo. In particolare qui convergeranno i vari \verb|feature-branch|
                            utilizzati per la creazione delle funzionalità del modulo in oggetto. In questo ramo deve essere presente sempre e solo codice
                            compilabile e feature complete.
    \item \textbf{feature/nome-feature:} per la creazione di ogni feature da parte dei vari membri addetti. In questi rami le feature possono anche essere
                                         incomplete e/o non funzionanti. Una volta completata la funzionalità il ramo deve essere chiuso sia in locale che in remoto e 
                                         convergere in develop.
    \item \textbf{release/versione[X.Y.Z]:} ramo contenente la versione \glock{candidate-release} pronta per essere approvata dal \textit{Responsabile di Progetto} dopo
                                            aver passato tutte le verifiche. Una volta approvata il branch deve terminare sia in locale che in remoto, convergendo in master e in develop.
                                            Contestualmente nel branch master deve essere presente il tag di release.
    \item \textbf{bugfix/nome-feature:} rami uscenti dal develop usati per sistemare errori rilevati successivamente alla chiusura del relativo feature-branch.
                                       Sono usati per la correzione dell'errore trovato, una volta inserita la correzione devono convergere in develop. E' necessario
                                       che il nome del bugfix corrisponda al nome della determinata feature a cui si rivolge la correzione.
\end{itemize}
Riassumendo ogni modifica al codice deve partire da un branch di feature o di bugfix, strettamente collegato ad uno o più membri del gruppo, e convergere in develop solamente dopo
aver superato le verifiche necessarie per quella specifica attività.
Infine può andare in master solamente quando si deciderà di effettuare una release, passando quindi per l'apposito ramo.
Per ulteriori informazioni consultare il tutorial di Atlassian su \href{https://www.atlassian.com/git/tutorials/comparing-workflows/gitflow-workflow}{GitFlow-Workflow}.

Altrettanto importante è il tipo di file che deve essere all'interno delle repository. Queste, in ogni loro branch, potranno ospitare solo file sorgenti, ad esempio
in formato \verb|.tex| per la documentazione. Ogni tipo di file binario, compresi i \verb|.pdf|, prodotti della compilazione dei sorgenti, dovranno essere
ospitati nella specifica \textit{\glock{artifact repository}} alimentata dal processo di \textit{\glock{continuous delivery}}, il quale verrà avviato contestualmente
all'inizio dello sviluppo del prodotto. Non viene invece implementato un processo di CI e CD per la stesura della documentazione, a differenza di tutte le altre repository,
in quanto ritenuto non necessario. Ogni membro dovrà assicurarsi che il codice prodotto sia compilabile con una recente distribuzione di \LaTeX,
sia essa \verb|MikTex| o \verb|TexLive|, prima di far convergere il proprio ramo in develop.
