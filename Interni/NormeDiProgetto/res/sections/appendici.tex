\appendix
\section{Metriche di qualità}


\subsubsection{MPR-6: Requisiti obbligatori soddisfatti (PROS)} \label{_MPR-6}
Indica la percentuale di requisiti obbligatori soddisfatti rispetto a quanti ne sono stati individuati.
\begin{itemize}
    \item \textbf{Formula}: \(\frac{Numero\ requisiti\ obbligatori\ soddisfatti}{Numero\ requisiti\  obbligatori\ individuati}*100\)
\end{itemize}

\subsubsection{MPR-7: Requisiti opzionali non  soddisfatti (RONS )} \label{_MPR-7}
Indica la percentuale di requisiti opzionali che non sono stati soddisfatti rispetto a quanti ne sono stati individuati.
\begin{itemize}
    \item \textbf{Formula}: \(\frac{Numero\ requisiti\ opzionali\ non\ soddisfatti}{Numero\ requisiti\  opzionali\ individuati}*100\)
\end{itemize}

\subsubsection{MPR-8: Requisiti desiderabili non soddisfatti (RDNS)} \label{_MPR-8}
Indica la percentuale di requisiti desiderabili che non sono stati  soddisfatti rispetto a quanti ne sono stati individuati.
\begin{itemize}
    \item \textbf{Formula}: \(\frac{Numero\ requisiti\ desiderabili\ non\ soddisfatti}{Numero\ desiderabili\  obbligatori\ individuati}*100\)
\end{itemize}

\subsubsection{MPR-9: Rischi non previsti ma avvenuti (RNPA)} \label{_MPR-9}
Indica il numero di rischi non previsti ma che si sono presentanti nel corso del progetto.


\subsubsection{MPD-S6: Profondità strutturale dell'interfaccia (PSI)}
Valore corrispondente alla profondità strutturale dell'interfaccia utente che valuta il numero di passaggi da compiere per raggiungere la funzionalità desiderata.