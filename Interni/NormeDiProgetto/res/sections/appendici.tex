\appendix
\section{Metriche di qualità}

\subsection{Metriche per la qualità del processo} \label{_metricheprocesso}

\subsubsection{MPR-DOC1: Budgeted Cost of Work Scheduled (BCWS)} \label{_MPR-DOC1}
Rappresenta il costo pianificato in euro per realizzare le attività di progetto alla data corrente.

\subsubsection{MPR-DOC2: Actual Cost of Work Performed (ACWP)} \label{_MPR-DOC2}
Rappresenta il costo effettivamente sostenuto in euro alla data corrente.

\subsubsection{MPR-DOC3: Budgeted Cost of Work Performed (BCWP)} \label{_MPR-DOC3}
Rappresenta il valore in euro delle attività realizzate alla data corrente.

\subsubsection{MPR-DOC4: Cost Variance (CV)} \label{_MPR-DOC4}
Indica se il valore realmente maturato è maggiore, uguale o minore rispetto al costo effettivo.
\begin{itemize}
    \item \textbf{Formula}: $CV = BCWP - ACWP$;
    \item \textbf{Risultato}: se $CV > 0$ significa che il progetto produce con maggior efficienza (minor costo) rispetto a quanto pianificato, viceversa se negativo.
\end{itemize}

\subsubsection{MPR-DOC5: Schedule Variance (SV)} \label{_MPR-DOC5}
Indica se si è in linea, in anticipo o in ritardo rispetto alla schedulazione delle attività di progetto pianificate nella baseline.
\begin{itemize}
    \item \textbf{Formula}: $SV = BCWP - BCWS$;
    \item \textbf{Risultato}: se $SV > 0$ significa che il progetto sta producendo con maggior velocità a quanto pianificato, viceversa se negativo.
\end{itemize}


\subsection{Metriche per la qualità del prodotto} \label{_metricheprodotto}
Per quantificare la qualità del prodotto vengono adottate le metriche descritte in questa sezione.
\subsubsection{MPD-DOC1: Indice di Gulpease}
Indice che individua il grado di leggibilità di un testo in lingua italiana mediante la formula:
\begin{center}
    \(GULP=89+\frac{300(totale\; frasi)-10(totale\; lettere)}{totale\; parole}\)
\end{center}
I valori ottenuti sono da interpretare secondo questi criteri:
\begin{itemize}
    \item \textbf{GULP < 80}: leggibilità difficile per un utente con licenza elementare;
    \item \textbf{GULP < 60}: leggibilità difficile per un utente con licenza media;
    \item \textbf{GULP < 40}: leggibilità difficile per un utente con diploma di scuola secondaria di secondo grado;
    \item \textbf{GULP \(\sim\) 0} leggibilità difficile per chiunque.
\end{itemize}

\subsubsection{MPD-DOC2: Correttezza ortografica (CORT)}
Questa metrica permette di misurare la correttezza lessicografica del documento mediante un numero intero ottenuto come:
\begin{center}
    \textit{CORT=\# errori ortografici}
\end{center}
Se CORT=0 allora il documento non ha errori ortografici, se CORT>0 il documento presenta errori.