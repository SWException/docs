\appendix
\section{Metriche di qualità}

\subsection{Metriche per la qualità del processo} \label{_metricheprocesso}

\subsubsection{MPR-1: Budgeted Cost of Work Scheduled (BCWS)} \label{_MPR-1}
Rappresenta il costo pianificato in euro per realizzare le attività di progetto alla data corrente.

\subsubsection{MPR-2: Actual Cost of Work Performed (ACWP)} \label{_MPR-2}
Rappresenta il costo effettivamente sostenuto in euro alla data corrente.

\subsubsection{MPR-3: Budgeted Cost of Work Performed (BCWP)} \label{_MPR-3}
Rappresenta il valore in euro delle attività realizzate alla data corrente.

\subsubsection{MPR-4: Cost Variance (CV)} \label{_MPR-4}
Indica se il valore realmente maturato è maggiore, uguale o minore rispetto al costo effettivo.
\begin{itemize}
    \item \textbf{Formula}: $CV = BCWP - ACWP$;
\end{itemize}

\subsubsection{MPR-5: Schedule Variance (SV)} \label{_MPR-5}
Indica se si è in linea, in anticipo o in ritardo rispetto alla schedulazione delle attività di progetto pianificate nella baseline.
\begin{itemize}
    \item \textbf{Formula}: $SV = BCWP - BCWS$;
\end{itemize}

\subsubsection{MPR-6: Requisiti obbligatori soddisfatti (PROS)} \label{_MPR-6}
Indica la percentuale di requisiti obbligatori soddisfatti rispetto a quanti ne sono stati individuati.
\begin{itemize}
    \item \textbf{Formula}: \(\frac{Numero\ requisiti\ obbligatori\ soddisfatti}{Numero\ requisiti\  obbligatori\ individuati}*100\)
\end{itemize}

\subsubsection{MPR-7: Requisiti opzionali non  soddisfatti (RONS )} \label{_MPR-7}
Indica la percentuale di requisiti opzionali che non sono stati soddisfatti rispetto a quanti ne sono stati individuati.
\begin{itemize}
    \item \textbf{Formula}: \(\frac{Numero\ requisiti\ opzionali\ non\ soddisfatti}{Numero\ requisiti\  opzionali\ individuati}*100\)
\end{itemize}

\subsubsection{MPR-8: Requisiti desiderabili non soddisfatti (RDNS)} \label{_MPR-8}
Indica la percentuale di requisiti desiderabili che non sono stati  soddisfatti rispetto a quanti ne sono stati individuati.
\begin{itemize}
    \item \textbf{Formula}: \(\frac{Numero\ requisiti\ desiderabili\ non\ soddisfatti}{Numero\ desiderabili\  obbligatori\ individuati}*100\)
\end{itemize}

\subsubsection{MPR-9: Rischi non previsti ma avvenuti (RNPA)} \label{_MPR-9}
Indica il numero di rischi non previsti ma che si sono presentanti nel corso del progetto.

\subsection{Metriche per la qualità della documentazione} \label{_metricheprodotto}
Per quantificare la qualità della documentazione vengono adottate le metriche descritte in questa sezione.

\subsubsection{MPD-DOC1: Indice di Gulpease (GULP)}
Indice che individua il grado di leggibilità di un testo in lingua italiana mediante la formula:
\begin{center}
    \(GULP=89+\frac{300(totale\; frasi)-10(totale\; lettere)}{totale\; parole}\)
\end{center}

\subsubsection{MPD-DOC2: Correttezza ortografica (CORT)}
Questa metrica permette di misurare la correttezza lessicografica del documento mediante un numero intero ottenuto come:
\begin{center}
    \textit{CORT=\# errori ortografici}
\end{center}

\subsection{Metriche per la qualità del codice sorgente} \label{_metricheCodiceSorgenteApp}
Per misurare la qualità del codice sorgente prodotto vengono adottate le seguenti metriche, come introdotto in \S\ref{_metricheQualitaCodice}

\subsubsection{MPD-S1: Complessità ciclomatica (COCI)}
La complessità ciclomatica di un programma strutturato è definita in riferimento ad un grafo diretto
contenente i blocchi base di un programma con un arco tra due blocchi se il controllo può passare dal
primo al secondo (il "grafo di controllo di flusso"). La complessità è quindi definita come:
\[
    v(G) = e − n + 2 \cdot p
\]
dove:
\begin{itemize}
    \item $v(G)$ è la complessità ciclomatica del grafo G, ed ha un valore in $\mathbb{N}$;
    \item $e$ è il numero di archi del grafo, ha un valore in $\mathbb{N}$;
    \item $n$ è il numero di nodi del grafo, ha un valore in $\mathbb{N}$;
    \item $p$ è il numero di componenti connesse, ha un valore in $\mathbb{N}$.
\end{itemize}

\subsubsection{MPD-S2: Complessità espressioni booleane (COEB)}
Questo è un indice necessario a tenere sotto controllo la complessità delle espressioni booleane e, di conseguenza, la complessità ciclomatica.
Si sintetizza nel massimo numero di operatori logici presenti in una singola espressione booleana.\\
Il suo calcolo è banale, e consiste nel conteggio degli operatori logici.

\subsubsection{MPD-S3: Lunghezza delle righe di codice (LRC)}
Indice che misura la lunghezza di una singola riga di codice. Serve a garantire la leggibilità di esso anche in schermi
con dimensioni e risoluzione minori.\\
Si ottiene contando i caratteri di ogni singola riga, ci si assicura quindi che il valore ottenuto per ognuna stia sotto il valore
definito nel \dext{PianoDiQualifica\_1.0.0}.

\subsubsection{MPD-S4: Code coverage (CODCO)}
Con questo indice si misura la percentuale di righe di codice coperte dai test automatici.\\
È necessario cercare di avere la massima copertura possibile per ogni modulo, in quanto codice non testato
può provocare comportamenti indesiderati e/o imprevisti dell'artefatto. La definizione della soglia minima viene rimandata al \dext{PianoDiQualifica\_1.0.0}.


\subsubsection{MPD-S5: Maturità dei test (MATE)}
Misura la percentuale di casi di test eseguiti con successo rispetto al numero totale previsto per garantire una copertura minima dei requisiti;
\begin{itemize}
    \item \textbf{Formula}: \(\frac{casi\ test\ eseguiti\ con\ successo}{casi\ test\ previsti}*100\)
\end{itemize}

\subsubsection{MPD-S6: Profondità strutturale dell'interfaccia (PSI)}
Valore corrispondente alla profondità strutturale dell'interfaccia utente che valuta il numero di passaggi da compiere per raggiungere la funzionalità desiderata.