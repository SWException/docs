\section{Studio C2}
Il capitolato C2 è stato presentato dall'azienda RebBabel localizzata ad Amsterdam - Paesi Bassi. Quest'ultima si occupa di consulenza ad altre start-up e sviluppo agile di applicativi web, operando in Italia e ad Amsterdam.

\subsection{Informazioni generali}
\begin{itemize}
	\item \textbf{Nome:} EmporioLambda - piattaforma di e-commerce in stile Serverless
	\item \textbf{Proponente$_G$:} RedBabel
	\item \textbf{Committente$_G$:} Prof. Tullio Vardanega e Prof. Riccardo Cardin
\end{itemize}

\subsection{Descrizione del capitolato}
L'applicazione da sviluppare consiste in una piattaforma serverless composta da microservizi, la quale implementa un template per siti e-commerce basata sui servizi AWS.\\
In particolare ci sono 4 moduli di alto livello:
\begin{itemize}
	\item EML-FE$_G$: è il componente per il front-end del sito, sviluppato usando la libreria \verb|Next.js| con il linguaggio \verb|Typescript|.
	\item EML-BE$_G$: è il componente per la parte di back-end della piattaforma, usa lo stesso linguaggio del modulo precedente basandosi sui servizi \verb|AWS Lambda|.
	\item EML-I$_G$: è il modulo per l'integrazione dei servizi di terze parti, utili ad esempio per le operazioni di pagamento e per l'autenticazione degli utenti. Questo deve essere sviluppato tramite il framework \textit{Serverless} nello stesso linguaggio degli altri due.
	\item EML-MON$_G$: modulo per il monitoraggio del sistema da parte dell'amministratore$_G$. I proponenti suggeriscono di implementarlo con \verb|Amazon CloudWatch|.
\end{itemize}

\subsection{Finalità del progetto}
L'obiettivo prefissato dall'azienda proponente è quello di avere un prodotto per provare la produttività delle tecnologie coinvolte, in particolare l'architettura appena menzionata. Per farlo, viene richiesto di realizzare un prototipo del prodotto finale, quindi specializzare la piattaforma creata per un caso d'uso specifico a scelta.

\subsection{Tecnologie interessate}
Le tecnologie richieste dal proponente per la realizzazione del prodotto finale sono le seguenti:
\begin{itemize}
	\item \verb|AWS Lambda|: sistema Amazon che funge come Function-as-a-Service (FaaS) per la realizzazione della parte di back-end, il quale esegue codice sorgente in container gestiti e temporanei (che esistono per una singola invocazione).
	\item 
\end{itemize}

\subsection{Aspetti positivi}

\subsection{Criticità e fattori di rischio}

\subsection{Conclusioni}