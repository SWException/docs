\section{Customer guide} \label{_cliente}
\subsection{Purpose of the section}
The goal of this section of the document is to describe what a customer can do using EmporioLambda.

\subsection{How to land to the web-application} \label{_homepage}
The only current way to access the application is by direct link. After accessing the the application by link you will land to the homepage.

\subsection{How to Sign in} \label{_signin}
In order to login in the platform you have to pressing the login button from any page of the e-commerce.
\begin{figure}[H]
    \centering
    \includegraphics[width=5em]{res/images/cliente/loginbutton.png}
    \caption{Sign in}
\end{figure}

Remember: if you don't have an account you can always register one following the \hyperref[_signup]{instructions}.
To successfully login please submit the following info: 

\begin{itemize} 
    \item \textbf{Email};
    \item \textbf{Password}.
\end{itemize}

\begin{figure}[H]
    \centering
    \includegraphics[width=30em]{res/images/cliente/signin.png}
    \caption{Sign in}
\end{figure}

\subsection{How to Sign up} \label{_signup}
In order to create a new account you have to press the SignUp button. To successfully register a new you have to submit the following info:
\begin{itemize} 
    \item \textbf{Email};
    \item \textbf{Password};
    \item \textbf{Name};
    \item \textbf{Surname}.
\end{itemize}
To ensure the user to write the email and the password correctly, the application force to re-enter these data twice.

\begin{figure}[H]
    \centering
    \includegraphics[width=30em]{res/images/cliente/signup.png}
    \caption{Sign up}
\end{figure}

After that you will receive an e-mail with the code to insert to verify the account.

\begin{figure}[H]
    \centering
    \includegraphics[width=30em]{res/images/cliente/verify.png}
    \caption{Verify code}
\end{figure}


\subsection{How to Sign out} \label{_signout}
In order to be able to sign out you have to be signed in. Then you will find the logout button in the Navigation Bar (Top right).

\begin{figure}[H]
    \centering
    \includegraphics[width=5em]{res/images/cliente/signout.png}
    \caption{Sign out}
\end{figure}

\subsection{How to restore your password} \label{_passwordrecover}
To restore your password, in the \hyperref[_signup]{sign up page}, you find the recover password button.
\begin{figure}[H]
    \centering
    \includegraphics[width=5em]{res/images/cliente/recover.png}
    \caption{Recover button}
\end{figure}

Then you will be asked to insert the e-mail of the account you want to log in into.
\begin{figure}[H]
    \centering
    \includegraphics[width=30em]{res/images/cliente/recoveremail.png}
    \caption{Code field}
\end{figure}

You will receive an e-mail with a code. Insert it and set the new password.
\begin{figure}[H]
    \centering
    \includegraphics[width=30em]{res/images/cliente/newpassword.png}
    \caption{New password fields}
\end{figure}


\subsection{How to look for a product} \label{_lookforproduct}
There are several ways to find a product in the website. One way is to find the desired product in the \textbf{Best products} section that is in the \hyperref[_homepage]{homepage}. Another way is to be in a Product Listing Page. You can reach this page by:
\begin{itemize} 
    \item \textbf{Typing in the search bar};
    \item \textbf{Clicking the category name}.
\end{itemize}

\begin{figure}[H]
    \centering
    \includegraphics[width=22em]{res/images/cliente/searchingbar.png}
    \caption{Searching Bar}
\end{figure}

\begin{figure}[H]
    \centering
    \includegraphics[width=18em]{res/images/cliente/categories.png}
    \caption{Categories}
\end{figure}

\begin{figure}[H]
    \centering
    \includegraphics[width=\linewidth]{res/images/cliente/plp.png}
    \caption{Product Listing Page}
\end{figure}

\subsubsection{How to refine your search list}
There are 2 more options to find the best product for you:
\begin{enumerate} 
    \item \textbf{Filter between costs}: it allows to filter the list of products between 2 costs (minimum and maximum);
    \item \textbf{Order by cost}: it allows to order the list from the cheapest to the more expensive and vice versa.
\end{enumerate}

\begin{figure}[H]
    \centering
    \includegraphics[width=35
    em]{res/images/cliente/refinementtools.png}
    \caption{Refinement tools}
\end{figure}

\subsection{How to buy a product} \label{_buyproduct}
In order to buy a product, you have to follow these steps:
\begin{itemize} 
    \item \textbf{Find the desired product}: this action can be done via these \hyperref[_lookforproduct]{instructions};
    \item \textbf{Add the product to the cart}: this action can be done via these \hyperref[_addproduct]{instructions};
    \item \textbf{Check the cart}: this action can be done via these \hyperref[_checkcart]{instructions};
    \item \textbf{Proceed to the checkout}: this action can be done via these \hyperref[_checkout]{instructions}.  
\end{itemize}

\subsubsection{How to add a product to the cart} \label{_addproduct}
In order to add a product to your cart first you have to find the desired product/s. You can find here the \hyperref[_lookforproduct]{instructions}.
Open the details page by clicking the desired product.
\begin{figure}[H]
    \centering
    \includegraphics[width=40em]{res/images/cliente/pdp.png}
    \caption{Details page of product}
\end{figure}
Then you just simply select the number of items, using the + or - buttons. Then click the "Add to cart" button.
\begin{figure}[H]
    \centering
    \includegraphics[width=10em]{res/images/cliente/addtocart.png}
    \caption{Add to cart}
\end{figure}

\subsubsection{How to check your cart} \label{_checkcart}
To check your cart you can simply press the cart icon in the top right corner of the page.
\begin{figure}[H]
    \centering
    \includegraphics[width=5em]{res/images/cliente/carticon.png}
    \caption{Cart Icon}
\end{figure}
Here you can check your cart. If the cart is fine, you can proceed to the checkout by pressing the checkout button.
\begin{figure}[H]
    \centering
    \includegraphics[width=7em]{res/images/cliente/checkoutbutton.png}
    \caption{Checkout Button}
\end{figure}

\subsubsection{How to proceed to the checkout} \label{_checkout}
After the click on "Checkout" button in the cart, you will be redirected to the wizard to enter information regarding the shipping address, billing address and payment details.
You have only complete one page an click on "Next" button.

In the first part you can manually type all the shipping informations or just use a previously stored address.
\begin{figure}[H]
    \centering
    \includegraphics[width=30em]{res/images/cliente/checkoutshipping.png}
    \caption{Checkout shipping info}
\end{figure}
Once you have complete the shipping page, you have to insert the informations about billing.
You can use the shipping address as billing or choose another
\begin{figure}[H]
    \centering
    \includegraphics[width=30em]{res/images/cliente/checkoutbilling.png}
    \caption{Checkout billing info}
\end{figure}

\begin{figure}[H]
    \centering
    \includegraphics[width=20em]{res/images/cliente/billingbutton.png}
    \caption{Checkout billing info select}
\end{figure}


Once you have insert info about billing and shipping, you have to insert the informations about shipping.
Remember: this platform will never handle payment data. Your data are securely used only by \href{https://stripe.com}{Stripe}.
\begin{figure}[H]
    \centering
    \includegraphics[width=30em]{res/images/cliente/checkoutpay.png}
    \caption{Payment}
\end{figure}

\subsection{How to manage your cart} \label{_cart}
In order to manage your cart you have to access to it. This action can be done via these \hyperref[_checkcart]{instructions}.
Here you can:
\begin{itemize} 
    \item \textbf{Edit item quantity}: by pressing the + or - buttons;
    \item \textbf{Remove completely the product}: by pressing the X button of the no more desired product;
    \item \textbf{Remove completely all products}: by pressing the Remove all button;  
\end{itemize}
\begin{figure}[H]
    \centering
    \includegraphics[width=\linewidth]{res/images/cliente/cart.png}
    \caption{Cart}
\end{figure}

\subsection{How to manage your orders} \label{_orders}
In order to be able to see your orders you have to be signed in. This action can be done via these \hyperref[_signin]{instructions}.
To manage your orders you have to go to click on the button of the profile and select "My orders".
\begin{figure}[H]
    \centering
    \includegraphics[width=10em]{res/images/cliente/profileorder.png}
    \caption{Profile button - My orders}
\end{figure}
Here you will find the list of your past orders.
\begin{figure}[H]
    \centering
    \includegraphics[width=\linewidth]{res/images/cliente/orders.png}
    \caption{Orders list}
\end{figure}
Each order has a detail page where you can:
\begin{itemize} 
    \item \textbf{Check all the info about the selected order};
    \item \textbf{Ask assistance for the order}; 
    \item \textbf{Ask to return a product of the order}; 
    \item \textbf{Ask to cancel the order}.
\end{itemize}
The last 3 options are implemented via e-mail. This allows you to directly message with the seller.
\begin{figure}[H]
    \centering
    \includegraphics[width=30em]{res/images/cliente/order.png}
    \caption{Order detail}
\end{figure}

\subsection{How to manage your credentials and personal informations} \label{_credentials}
In order to be able to manage your credentials you have to be signed in. This action can be done via these \hyperref[_signin]{instructions}.
Then you have to access the profile page by clicking the profile button.
\begin{figure}[H]
    \centering
    \includegraphics[width=10em]{res/images/cliente/profileaccount.png}
    \caption{Profile Button}
\end{figure}

Here you can edit your current password and your current e-mail.
To edit the password:
\begin{itemize} 
    \item \textbf{Old password}: insert the old password;
    \item \textbf{New password}: insert the new password; 
    \item \textbf{Confirm new password}: re-insert the new password. This step ensures the user not to  mistype the password.
\end{itemize}

\begin{figure}[H]
    \centering
    \includegraphics[width=30em]{res/images/cliente/credentialpwd.png}
    \caption{Change password form}
\end{figure}

To edit the e-mail:
\begin{itemize} 
    \item \textbf{New password}: insert the new e-mail; 
    \item \textbf{Confirm new password}: re-insert the new e-mail. This step ensures the user not to  mistype the e-mail.
\end{itemize}

\begin{figure}[H]
    \centering
    \includegraphics[width=30em]{res/images/cliente/credentialemail.png}
    \caption{Change email form}
\end{figure}

To edit your personal informations:
\begin{itemize} 
    \item \textbf{New Name}: insert the new name; 
    \item \textbf{New Surname}:insert the new surname.
\end{itemize}

\begin{figure}[H]
    \centering
    \includegraphics[width=30em]{res/images/cliente/personalinfo.png}
    \caption{Change personal informations form}
\end{figure}

\subsection{How to delete your account} \label{_delete}
In order to be able to request an account deletion you have to be signed in. This action can be done via these \hyperref[_signin]{instructions}.
Then you have to access the profile page by clicking the profile button.
When you click the "Request account deletion" button, you will be delete from site.
\begin{figure}[H]
    \centering
    \includegraphics[width=30em]{res/images/cliente/delete.png}
    \caption{Delete Account Request}
\end{figure}

\subsection{How to manage your device} \label{_device}
In order to be able to request an account deletion you have to be signed in. This action can be done via these \hyperref[_signin]{instructions}.
Then you have to access the profile page by clicking the profile button.
Here you can:
\begin{itemize} 
    \item \textbf{Forget a device}: you have to click on the "Delete" button next to the selected device; 
    \item \textbf{Forget all devices}:click on the "Froget all devices" button;
    \item \textbf{logout from all devices}:click on the "Logout from all devices" button.
\end{itemize}

\begin{figure}[H]
    \centering
    \includegraphics[width=30em]{res/images/cliente/device.png}
    \caption{Device Management}
\end{figure}


\subsection{How to contact the seller} \label{_contacts}
To contact the seller you can find the link of the e-mail in the footer of the \hyperref[_homepage]{homepage}.

