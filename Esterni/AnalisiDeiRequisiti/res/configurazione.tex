\usepackage{geometry}
\usepackage{graphicx} 
\usepackage[T1]{fontenc}
\usepackage[utf8]{inputenc}
\usepackage{charter}
\usepackage{eurosym}
\usepackage[italian]{babel}
\usepackage{float}
\usepackage{subcaption}
\usepackage{wrapfig}
\usepackage{fancyhdr} 
\usepackage{lastpage}
\usepackage{amsfonts}
\usepackage{fancyvrb}
\usepackage{xcolor}
\usepackage{hyperref}   
\usepackage{listings}
\usepackage{longtable}
\usepackage{colortbl}
\usepackage{tikz}
\usepackage{titlesec}


% Impostazione sottotitolo di quarto livello e quinto livello

\setcounter{secnumdepth}{4}
\setcounter{tocdepth}{4}

\titleformat{\paragraph}
{\normalfont\normalsize\bfseries}{\theparagraph}{1em}{}
\titlespacing*{\paragraph}{0pt}{2.25ex plus 1ex minus .2ex}{1.5ex plus .2ex}

\titleformat{\subparagraph}
{\normalfont\normalsize\bfseries}{\thesubparagraph}{1em}{}
\titlespacing*{\subparagraph}{0pt}{1.75ex plus 1ex minus .2ex}{.75ex plus .1ex}


% Impostazioni pagina e margini

\geometry{
    margin=1.0in,
    top=19.2mm, % NON TOCCARE
    bottom=30mm,
    left=20mm,
    right=20mm
}

% Definizione colori

\definecolor{footer-gray}{HTML}{808080}
\definecolor{light-gray}{gray}{0.6} 
\definecolor{light-grayer}{gray}{0.75} 
\definecolor{lighter-grayer}{gray}{0.85} 
\definecolor{lightest-grayest}{gray}{0.94} 
\definecolor{codegreen}{rgb}{0,0.4,0.2}
\definecolor{codegray}{rgb}{0.5,0.5,0.5}
\definecolor{codepurple}{rgb}{0.58,0,0.82}
\definecolor{backcolour}{rgb}{0.95,0.95,0.96}


% Impostazione header e footer

\pagestyle{fancy}
\setlength\headheight{33pt}
\renewcommand{\headrulewidth}{0pt}
\fancyhead{}

\lhead{ \textcolor{footer-gray}{\docNome - v\docVersione} }
\rhead{ \textcolor{footer-gray}{\leftmark}}
\renewcommand{\footrulewidth}{0.1pt}
\fancyfoot{}

\renewcommand{\footrule}{\hbox to\headwidth{\color{light-grayer}\leaders\hrule height \footrulewidth\hfill}}
\cfoot{ \textcolor{footer-gray}{Pagina \thepage \hspace{1pt} di \pageref*{LastPage}} }

% Grandezza paragrafi e spaziatura frasi

\setlength{\parindent}{1.7em}
\setlength{\parskip}{1.1em}
\renewcommand{\baselinestretch}{1.05}

% Colori link

\hypersetup{
    colorlinks,
    linkcolor=[HTML]{404040},
    citecolor={blue!50!black},
    urlcolor={blue!50!black}
}
\PassOptionsToPackage{hyphens}{url}\usepackage{hyperref}

% Equivalente a <hr>

\newcommand{\hr}{\par\vspace{-.1\ht\strutbox}\noindent\hrulefill\par}

% Tabelle e tabulazione

\setlength{\tabcolsep}{10pt}
\renewcommand{\arraystretch}{1.4}

% Unicode per simbolo euro

\DeclareUnicodeCharacter{20AC}{\euro}

% Codice e snippet

\renewcommand{\lstlistingname}{Snippet}
\renewcommand{\lstlistlistingname}{Lista di \lstlistingname s}    
 

\lstdefinestyle{chungusHighlight}{
    frame=tb,
    backgroundcolor=\color{backcolour},   
    commentstyle=\color{codegreen},
    keywordstyle=\color{magenta}\textbf,
    numberstyle=\color{codegray},
    stringstyle=\color{codepurple},
    basicstyle={\ttfamily},
    breakatwhitespace=false,         
    breaklines=true,                 
    captionpos=b,                    
    keepspaces=true,                 
    numbers=left,                    
    numbersep=5pt,                  
    showspaces=false,                
    showstringspaces=false,
    showtabs=false,
    numbers=none,                  
    tabsize=2
}

\lstset{style=chungusHighlight}


% Comando per aggiungere le pagine di ogni sezione

\newcommand{\yetAnotherSectionNamed}[1]{%
    \newpage
    \input{res/sections/#1}
}%


% Comando per i documenti esterni e il glossario

\newcommand{\dext}[1]{\textit{#1}\textsubscript{\textit{D}}}

\newcommand{\glock}[1]{textit{#1}\textsubscript{\textit{G}}}
% Aggiungere o modificare di seguito eventuali configuraioni valide solo per questo documento

%contatore dei requisiti
\newcounter{CR} % Contatore Requisiti 
\setcounter{CR}{0}
\newcounter{CSR} % Contatore Sotto-Requisiti
\setcounter{CSR}{0}
\newcommand{\stepCR}[0]{\stepcounter{CR}\setcounter{CSR}{0}} % incrementa il contatore CR
\newcommand{\valueCR}[0]{\arabic{CR}} % ritorna il valore del contatore CR
\newcommand{\stepsubCR}{\stepcounter{CSR}} % incrementa il contatore CSR
\newcommand{\valuesubCR}[0]{\arabic{CR}.\arabic{CSR}} % ritorna il valore del contatore CSR
\newcommand{\resetCR}{\setcounter{CR}{0}\setcounter{CSR}{0}}  % resetta il contatore CR e CSR

% comandi per formattare il codice dei requisiti
\newcommand{\creazioneCodiceRequisito}[3]{\textbf{R#1#2#3}}
\newcommand{\creazioneCodiceSottoRequisito}[3]{\textcolor{black!75}{R#1#2#3}}

%comando per creare i codici dei requisiti e una lable per poi riferirsi ad esso
\makeatletter
\newcommand{\creazioneCodiceRequisitoConLabel}[3]{%
  \phantomsection
  \creazioneCodiceRequisito{#1}{#2}{#3}\def\@currentlabel{\creazioneCodiceRequisito{#1}{#2}{#3}}\label{Req#2#3}%
}
\makeatother

% comando per creare i codici dei sottorequisiti e una lable per poi riferirsi ad esso
\makeatletter
\newcommand{\creazioneCodiceSottoRequisitoConLabel}[3]{%
  \phantomsection
  \creazioneCodiceSottoRequisito{#1}{#2}{#3}\def\@currentlabel{\creazioneCodiceSottoRequisito{#1}{#2}{#3}}\label{Req#2#3}%
}
\makeatother

% comandi che generano automaticamente i codici dei requisiti e le relative lable
% PARAMETRO 1: Importanza
% PARAMETRO 2: Tipologia
\newcommand{\req}[2]{\stepCR\creazioneCodiceRequisitoConLabel{#1}{#2}{\valueCR}}
\newcommand{\sreq}[2]{\stepsubCR\creazioneCodiceSottoRequisitoConLabel{#1}{#2}{\valuesubCR}}

% comando per riferirsi ad un requisito riportando anche il codice completo di esso
% PARAMETRO 1: Tipologia
% PARAMETRO 2: Codice numerico (Es: 1.2, 5, 7.1, ...)
\newcommand{\refreqID}[2]{\ref{Req#1#2}}

% comandi che generano automaticamente i riferimenti ai requisiti con il contatore
% PARAMETRO 1: Tipologia
\newcommand{\refreq}[1]{\stepCR\refreqID{#1}{\valueCR}}
\newcommand{\refsreq}[1]{\stepsubCR\refreqID{#1}{\valuesubCR}}

%comando per riferirsi ad uno o più User Cases che hanno la lable uguale al codice (Es: UC2.2)
% PARAMETRO 1: Lista di User Cases separati da una virgola
\newcommand{\refUserCase}[1]{\foreach [count=\i] \ucref in {#1}{\ifnum\i=1\hyperref[\ucref]{\ucref}\else, \hyperref[\ucref]{\ucref}\fi}}

% comando per riferirsi ad una lista di requisito riportando anche i codici completi di essi
% PARAMETRO 1: Tipologia dei requisiti riportati nella lista al parametro 2
% PARAMETRO 2: lista dei Codici numerici dei requisiti separati da una virgola (Es: 1.2, 5, 7.1, ...)
\newcommand{\refRequisiti}[2]{\foreach [count=\i] \reqID in {#2}{\ifnum\i=1\refreqID{#1}{\reqID}\else, \refreqID{#1}{\reqID}\fi}}

%contatore dei UserCases
\newcounter{CUC} % Contatore UserCases 
\newcounter{CSUC} % Contatore Sotto-UserCases
\newcounter{CSSUC} % Contatore Sotto-Sotto-UserCases
\newcommand{\stepUserCase}[0]{\stepcounter{CUC}\setcounter{CSUC}{0}\setcounter{CSSUC}{0}} % incrementa il contatore CUC
\newcommand{\stepsubUserCase}[0]{\stepcounter{CSUC}\setcounter{CSSUC}{0}} % incrementa il contatore CSUC
\newcommand{\stepsubsubUserCase}[0]{\stepcounter{CSSUC}} % incrementa il contatore CSSUC
\newcommand{\valueUserCase}[0]{UC\arabic{CUC} } % ritorna il valore del contatore CUC
\newcommand{\valuesubUserCase}[0]{UC\arabic{CUC}.\arabic{CSUC} } % ritorna il valore del contatore CSUC
\newcommand{\valuesubsubUserCase}[0]{UC\arabic{CUC}.\arabic{CSUC}.\arabic{CSSUC} } % ritorna il valore del contatore CSSUC
\newcommand{\valueUC}[0]{UC\arabic{CUC}} % ritorna il valore del contatore CUC senza spazio finale
\newcommand{\valuesubUC}[0]{UC\arabic{CUC}.\arabic{CSUC}} % ritorna il valore del contatore CSUC senza spazio finale
\newcommand{\valuesubsubUC}[0]{UC\arabic{CUC}.\arabic{CSUC}.\arabic{CSSUC}} % ritorna il valore del contatore CSSUC senza spazio finale
\newcommand{\resetCUC}[0]{\setcounter{CUC}{0}\setcounter{CSUC}{0}\setcounter{CSSUC}{0}}  % resetta il contatore CUC e CSUC
\newcommand{\labelUserCase}[0]{\label{UC\arabic{CUC}}}
\newcommand{\labelsubUserCase}[0]{\label{UC\arabic{CUC}.\arabic{CSUC}}}
\newcommand{\labelsubsubUserCase}[0]{\label{UC\arabic{CUC}.\arabic{CSUC}.\arabic{CSSUC}}}
