\input{\pathToTemplate/res/configurazione}
% Aggiungere o modificare di seguito eventuali configuraioni valide solo per questo documento

%contatore dei requisiti
\newcounter{CR} % Contatore Requisiti 
\setcounter{CR}{0}
\newcounter{CSR} % Contatore Sotto-Requisiti
\setcounter{CSR}{0}
\newcommand{\stepCR}[0]{\stepcounter{CR}\setcounter{CSR}{0}} % incrementa il contatore CR
\newcommand{\valueCR}[0]{\arabic{CR}} % ritorna il valore del contatore CR
\newcommand{\stepsubCR}{\stepcounter{CSR}} % incrementa il contatore CSR
\newcommand{\valuesubCR}[0]{\arabic{CR}.\arabic{CSR}} % ritorna il valore del contatore CSR
\newcommand{\resetCR}{\setcounter{CR}{0}\setcounter{CSR}{0}}  % resetta il contatore CR e CSR

% comandi per formattare il codice dei requisiti
\newcommand{\creazioneCodiceRequisito}[3]{\textbf{R#1#2#3}}
\newcommand{\creazioneCodiceSottoRequisito}[3]{\textcolor{black!75}{R#1#2#3}}

%comando per creare i codici dei requisiti e una lable per poi riferirsi ad esso
\makeatletter
\newcommand{\creazioneCodiceRequisitoConLabel}[3]{%
  \phantomsection
  \creazioneCodiceRequisito{#1}{#2}{#3}\def\@currentlabel{\creazioneCodiceRequisito{#1}{#2}{#3}}\label{Req#2#3}%
}
\makeatother

% comando per creare i codici dei sottorequisiti e una lable per poi riferirsi ad esso
\makeatletter
\newcommand{\creazioneCodiceSottoRequisitoConLabel}[3]{%
  \phantomsection
  \creazioneCodiceSottoRequisito{#1}{#2}{#3}\def\@currentlabel{\creazioneCodiceSottoRequisito{#1}{#2}{#3}}\label{Req#2#3}%
}
\makeatother

% comandi che generano automaticamente i codici dei requisiti e le relative lable
% PARAMETRO 1: Importanza
% PARAMETRO 2: Tipologia
\newcommand{\req}[2]{\stepCR\creazioneCodiceRequisitoConLabel{#1}{#2}{\valueCR}}
\newcommand{\sreq}[2]{\stepsubCR\creazioneCodiceSottoRequisitoConLabel{#1}{#2}{\valuesubCR}}

% comando per riferirsi ad un requisito riportando anche il codice completo di esso
% PARAMETRO 1: Tipologia
% PARAMETRO 2: Codice numerico (Es: 1.2, 5, 7.1, ...)
\newcommand{\refreqID}[2]{\ref{Req#1#2}}

% comandi che generano automaticamente i riferimenti ai requisiti con il contatore
% PARAMETRO 1: Tipologia
\newcommand{\refreq}[1]{\stepCR\refreqID{#1}{\valueCR}}
\newcommand{\refsreq}[1]{\stepsubCR\refreqID{#1}{\valuesubCR}}

%comando per riferirsi ad uno o più User Cases che hanno la lable uguale al codice (Es: UC2.2)
% PARAMETRO 1: Lista di User Cases separati da una virgola
\newcommand{\refUserCase}[1]{\foreach [count=\i] \ucref in {#1}{\ifnum\i=1\hyperref[\ucref]{\ucref}\else, \hyperref[\ucref]{\ucref}\fi}}

% comando per riferirsi ad una lista di requisito riportando anche i codici completi di essi
% PARAMETRO 1: Tipologia dei requisiti riportati nella lista al parametro 2
% PARAMETRO 2: lista dei Codici numerici dei requisiti separati da una virgola (Es: 1.2, 5, 7.1, ...)
\newcommand{\refRequisiti}[2]{\foreach [count=\i] \reqID in {#2}{\ifnum\i=1\refreqID{#1}{\reqID}\else, \refreqID{#1}{\reqID}\fi}}

% Aggiungere o modificare di seguito eventuali configuraioni valide solo per questo documento

%contatore dei requisiti
\newcounter{CR} % Contatore Requisiti 
\setcounter{CR}{0}
\newcounter{CSR} % Contatore Sotto-Requisiti
\setcounter{CSR}{0}
\newcommand{\stepCR}[0]{\stepcounter{CR}\setcounter{CSR}{0}} % incrementa il contatore CR
\newcommand{\valueCR}[0]{\arabic{CR}} % ritorna il valore del contatore CR
\newcommand{\stepsubCR}{\stepcounter{CSR}} % incrementa il contatore CSR
\newcommand{\valuesubCR}[0]{\arabic{CR}.\arabic{CSR}} % ritorna il valore del contatore CSR
\newcommand{\resetCR}{\setcounter{CR}{0}\setcounter{CSR}{0}}  % resetta il contatore CR e CSR

% comandi per formattare il codice dei requisiti
\newcommand{\creazioneCodiceRequisito}[3]{\textbf{R#1#2#3}}
\newcommand{\creazioneCodiceSottoRequisito}[3]{\textcolor{black!75}{R#1#2#3}}

%comando per creare i codici dei requisiti e una lable per poi riferirsi ad esso
\makeatletter
\newcommand{\creazioneCodiceRequisitoConLabel}[3]{%
  \phantomsection
  \creazioneCodiceRequisito{#1}{#2}{#3}\def\@currentlabel{\creazioneCodiceRequisito{#1}{#2}{#3}}\label{Req#2#3}%
}
\makeatother

% comando per creare i codici dei sottorequisiti e una lable per poi riferirsi ad esso
\makeatletter
\newcommand{\creazioneCodiceSottoRequisitoConLabel}[3]{%
  \phantomsection
  \creazioneCodiceSottoRequisito{#1}{#2}{#3}\def\@currentlabel{\creazioneCodiceSottoRequisito{#1}{#2}{#3}}\label{Req#2#3}%
}
\makeatother

% comandi che generano automaticamente i codici dei requisiti e le relative lable
% PARAMETRO 1: Importanza
% PARAMETRO 2: Tipologia
\newcommand{\req}[2]{\stepCR\creazioneCodiceRequisitoConLabel{#1}{#2}{\valueCR}}
\newcommand{\sreq}[2]{\stepsubCR\creazioneCodiceSottoRequisitoConLabel{#1}{#2}{\valuesubCR}}

% comando per riferirsi ad un requisito riportando anche il codice completo di esso
% PARAMETRO 1: Tipologia
% PARAMETRO 2: Codice numerico (Es: 1.2, 5, 7.1, ...)
\newcommand{\refreqID}[2]{\ref{Req#1#2}}

% comandi che generano automaticamente i riferimenti ai requisiti con il contatore
% PARAMETRO 1: Tipologia
\newcommand{\refreq}[1]{\stepCR\refreqID{#1}{\valueCR}}
\newcommand{\refsreq}[1]{\stepsubCR\refreqID{#1}{\valuesubCR}}

%comando per riferirsi ad uno o più User Cases che hanno la lable uguale al codice (Es: UC2.2)
% PARAMETRO 1: Lista di User Cases separati da una virgola
\newcommand{\refUserCase}[1]{\foreach [count=\i] \ucref in {#1}{\ifnum\i=1\hyperref[\ucref]{\ucref}\else, \hyperref[\ucref]{\ucref}\fi}}

% comando per riferirsi ad una lista di requisito riportando anche i codici completi di essi
% PARAMETRO 1: Tipologia dei requisiti riportati nella lista al parametro 2
% PARAMETRO 2: lista dei Codici numerici dei requisiti separati da una virgola (Es: 1.2, 5, 7.1, ...)
\newcommand{\refRequisiti}[2]{\foreach [count=\i] \reqID in {#2}{\ifnum\i=1\refreqID{#1}{\reqID}\else, \refreqID{#1}{\reqID}\fi}}

% Aggiungere o modificare di seguito eventuali configuraioni valide solo per questo documento

%contatore dei requisiti
\newcounter{CR} % Contatore Requisiti 
\setcounter{CR}{0}
\newcounter{CSR} % Contatore Sotto-Requisiti
\setcounter{CSR}{0}
\newcommand{\stepCR}[0]{\stepcounter{CR}\setcounter{CSR}{0}} % incrementa il contatore CR
\newcommand{\valueCR}[0]{\arabic{CR}} % ritorna il valore del contatore CR
\newcommand{\stepsubCR}{\stepcounter{CSR}} % incrementa il contatore CSR
\newcommand{\valuesubCR}[0]{\arabic{CR}.\arabic{CSR}} % ritorna il valore del contatore CSR
\newcommand{\resetCR}{\setcounter{CR}{0}\setcounter{CSR}{0}}  % resetta il contatore CR e CSR

% comandi per formattare il codice dei requisiti
\newcommand{\creazioneCodiceRequisito}[3]{\textbf{R#1#2#3}}
\newcommand{\creazioneCodiceSottoRequisito}[3]{\textcolor{black!75}{R#1#2#3}}

%comando per creare i codici dei requisiti e una lable per poi riferirsi ad esso
\makeatletter
\newcommand{\creazioneCodiceRequisitoConLabel}[3]{%
  \phantomsection
  \creazioneCodiceRequisito{#1}{#2}{#3}\def\@currentlabel{\creazioneCodiceRequisito{#1}{#2}{#3}}\label{Req#2#3}%
}
\makeatother

% comando per creare i codici dei sottorequisiti e una lable per poi riferirsi ad esso
\makeatletter
\newcommand{\creazioneCodiceSottoRequisitoConLabel}[3]{%
  \phantomsection
  \creazioneCodiceSottoRequisito{#1}{#2}{#3}\def\@currentlabel{\creazioneCodiceSottoRequisito{#1}{#2}{#3}}\label{Req#2#3}%
}
\makeatother

% comandi che generano automaticamente i codici dei requisiti e le relative lable
% PARAMETRO 1: Importanza
% PARAMETRO 2: Tipologia
\newcommand{\req}[2]{\stepCR\creazioneCodiceRequisitoConLabel{#1}{#2}{\valueCR}}
\newcommand{\sreq}[2]{\stepsubCR\creazioneCodiceSottoRequisitoConLabel{#1}{#2}{\valuesubCR}}

% comando per riferirsi ad un requisito riportando anche il codice completo di esso
% PARAMETRO 1: Tipologia
% PARAMETRO 2: Codice numerico (Es: 1.2, 5, 7.1, ...)
\newcommand{\refreqID}[2]{\ref{Req#1#2}}

% comandi che generano automaticamente i riferimenti ai requisiti con il contatore
% PARAMETRO 1: Tipologia
\newcommand{\refreq}[1]{\stepCR\refreqID{#1}{\valueCR}}
\newcommand{\refsreq}[1]{\stepsubCR\refreqID{#1}{\valuesubCR}}

%comando per riferirsi ad uno o più User Cases che hanno la lable uguale al codice (Es: UC2.2)
% PARAMETRO 1: Lista di User Cases separati da una virgola
\newcommand{\refUserCase}[1]{\foreach [count=\i] \ucref in {#1}{\ifnum\i=1\hyperref[\ucref]{\ucref}\else, \hyperref[\ucref]{\ucref}\fi}}

% comando per riferirsi ad una lista di requisito riportando anche i codici completi di essi
% PARAMETRO 1: Tipologia dei requisiti riportati nella lista al parametro 2
% PARAMETRO 2: lista dei Codici numerici dei requisiti separati da una virgola (Es: 1.2, 5, 7.1, ...)
\newcommand{\refRequisiti}[2]{\foreach [count=\i] \reqID in {#2}{\ifnum\i=1\refreqID{#1}{\reqID}\else, \refreqID{#1}{\reqID}\fi}}

% Aggiungere o modificare di seguito eventuali configuraioni valide solo per questo documento

%contatore dei requisiti
\newcounter{CR} % Contatore Requisiti 
\setcounter{CR}{0}
\newcounter{CSR} % Contatore Sotto-Requisiti
\setcounter{CSR}{0}
\newcommand{\stepCR}[0]{\stepcounter{CR}\setcounter{CSR}{0}} % incrementa il contatore CR
\newcommand{\valueCR}[0]{\arabic{CR}} % ritorna il valore del contatore CR
\newcommand{\stepsubCR}{\stepcounter{CSR}} % incrementa il contatore CSR
\newcommand{\valuesubCR}[0]{\arabic{CR}.\arabic{CSR}} % ritorna il valore del contatore CSR
\newcommand{\resetCR}{\setcounter{CR}{0}\setcounter{CSR}{0}}  % resetta il contatore CR e CSR

% comandi per formattare il codice dei requisiti
\newcommand{\creazioneCodiceRequisito}[3]{\textbf{R#1#2#3}}
\newcommand{\creazioneCodiceSottoRequisito}[3]{\textcolor{black!75}{R#1#2#3}}

%comando per creare i codici dei requisiti e una lable per poi riferirsi ad esso
\makeatletter
\newcommand{\creazioneCodiceRequisitoConLabel}[3]{%
  \phantomsection
  \creazioneCodiceRequisito{#1}{#2}{#3}\def\@currentlabel{\creazioneCodiceRequisito{#1}{#2}{#3}}\label{Req#2#3}%
}
\makeatother

% comando per creare i codici dei sottorequisiti e una lable per poi riferirsi ad esso
\makeatletter
\newcommand{\creazioneCodiceSottoRequisitoConLabel}[3]{%
  \phantomsection
  \creazioneCodiceSottoRequisito{#1}{#2}{#3}\def\@currentlabel{\creazioneCodiceSottoRequisito{#1}{#2}{#3}}\label{Req#2#3}%
}
\makeatother

% comandi che generano automaticamente i codici dei requisiti e le relative lable
% PARAMETRO 1: Importanza
% PARAMETRO 2: Tipologia
\newcommand{\req}[2]{\stepCR\creazioneCodiceRequisitoConLabel{#1}{#2}{\valueCR}}
\newcommand{\sreq}[2]{\stepsubCR\creazioneCodiceSottoRequisitoConLabel{#1}{#2}{\valuesubCR}}

% comando per riferirsi ad un requisito riportando anche il codice completo di esso
% PARAMETRO 1: Tipologia
% PARAMETRO 2: Codice numerico (Es: 1.2, 5, 7.1, ...)
\newcommand{\refreqID}[2]{\ref{Req#1#2}}

% comandi che generano automaticamente i riferimenti ai requisiti con il contatore
% PARAMETRO 1: Tipologia
\newcommand{\refreq}[1]{\stepCR\refreqID{#1}{\valueCR}}
\newcommand{\refsreq}[1]{\stepsubCR\refreqID{#1}{\valuesubCR}}

%comando per riferirsi ad uno o più User Cases che hanno la lable uguale al codice (Es: UC2.2)
% PARAMETRO 1: Lista di User Cases separati da una virgola
\newcommand{\refUserCase}[1]{\foreach [count=\i] \ucref in {#1}{\ifnum\i=1\hyperref[\ucref]{\ucref}\else, \hyperref[\ucref]{\ucref}\fi}}

% comando per riferirsi ad una lista di requisito riportando anche i codici completi di essi
% PARAMETRO 1: Tipologia dei requisiti riportati nella lista al parametro 2
% PARAMETRO 2: lista dei Codici numerici dei requisiti separati da una virgola (Es: 1.2, 5, 7.1, ...)
\newcommand{\refRequisiti}[2]{\foreach [count=\i] \reqID in {#2}{\ifnum\i=1\refreqID{#1}{\reqID}\else, \refreqID{#1}{\reqID}\fi}}
