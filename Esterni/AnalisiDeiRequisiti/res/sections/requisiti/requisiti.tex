%  CODICE DEI REQUISITI (da NdP)
%          
%             R[Importanza][Tipologia][Codice]
%       
%       Importanza
%               1: requisito obbligatorio;
%               2: requisito desiderabile;
%               3: requisito opzionale.
%       Tipologia
%               V: requisito di vincolo;
%               F: requisito funzionale;
%               P: requisito prestazionale;
%               Q: requisito di qualità.
%       Codice
%   	        [CodiceBase].[CodiceSottoRequisito]

\section{Requisiti}

%contatore dei requisiti
\newcounter{CR} % Contatore Requisiti 
\setcounter{CR}{0}
\newcounter{CSR} % Contatore Sotto-Requisiti
\setcounter{CSR}{0}
\newcommand{\stepCR}[0]{\stepcounter{CR}\setcounter{CSR}{0}\arabic{CR}}
\newcommand{\valueCR}[0]{\arabic{CR}}
\newcommand{\stepsubCR}{\stepcounter{CSR}\arabic{CR}.\arabic{CSR}}
\newcommand{\valuesubCR}[0]{\arabic{CR}.\arabic{CSR}}
\newcommand{\resetCR}{\setcounter{CR}{0}\setcounter{CSR}{0}}

% comandi per le tabelle
\newcommand{\creazioneCodiceRequisito}[3]{\textbf{R#1#2#3}}
\newcommand{\creazioneCodiceSottoRequisito}[3]{{\color{gray} R#1#2#3}}

\newcommand{\req}[2]{\creazioneCodiceRequisito{#1}{#2}{\stepCR}}
\newcommand{\sreq}[2]{\creazioneCodiceSottoRequisito{#1}{#2}{\stepsubCR}}

\newcommand{\row}{ \\ \hline}

\subsection{Introduzione}
In seguito viene riportato, organizzato in forma tabellare, l'elenco dei requisiti del progetto suddivisi per tipologia. Ciascun requisito possiede un codice identificativo stabilito in accordo con quanto scritto nelle \dext{Norme di Progetto v1.0.0}.

\subsection{Requisiti funzionali}
\begin{center}
    \rowcolors{1}{lightest-grayest}{blue!20}
    \begin{longtable}{|p{3cm}|p{9.85cm}|p{2cm}|}
        \hline
        \rowcolor{lighter-grayer}
        \textbf{ID Requisito} & \textbf{Descrizione} & \textbf{Fonti} \\
        \hline
        \endhead
        \hline
        \multicolumn{3}{|c|}{\textit{Continua nella pagina successiva...}} \\
        \hline
        \endfoot
        \endlastfoot

        %requisiti ivan
        \req{1}{F} & L'utente deve poter visualizzare le informazione sul venditore & \hyperref[UC1]{UC1} \row
        \sreq{1}{F} & L'utente deve poter visualizzare le informazioni di contatto del venditore & \hyperref[UC1]{UC1}  \row
        \req{1}{F} & L'utente deve poter visualizzare i prodotti di una singola categoria & \hyperref[UC2]{UC2} \row
        \sreq{1}{F} & L'utente deve poter  & \hyperref[UC2]{UC2} \row
        \sreq{1}{F} & L'utente deve poter  & \hyperref[UC2]{UC2} \row

        %requisiti gianmarco

        %requisiti francesco
        \req{1}{F} & Il venditore deve poter visualizzare e modificare la lista degli ordini effettuati dai clienti & \hyperref[UC15]{UC15} \row
        \sreq{2}{F} & Il venditore deve poter cercare un ordine presente nel sistema & \hyperref[UC15]{UC15} \row
        \sreq{2}{F} & Il venditore deve avere a disposizione dei filtri per ordinare gli ordini  & \hyperref[UC15]{UC15} \row
        \sreq{1}{F} & Il venditore deve poter modificare lo stato di un ordine  & \hyperref[UC15.1]{UC15.1} \row
        \sreq{1}{F} & Il venditore deve poter stampare la bolla per un ordine  & \hyperref[UC15.2]{UC15.2} \row
        \sreq{1}{F} & Il venditore deve poter visualizzare i dettagli di un determinato ordine  & \hyperref[UC15.3]{UC15.3} \row
        \sreq{2}{F} & Il venditore deve avere a disposizione dei filtri per ordinare gli ordini  & \hyperref[UC15]{UC15} \row
        \req{1}{F} & Il venditore deve poter visualizzare la lista dei clienti del sito & \hyperref[UC16]{UC16} \row
        \sreq{2}{F} & Il venditore deve poter cercare un cliente presente nel sistema & \hyperref[UC16]{UC16} \row
        \sreq{2}{F} & Il venditore deve avere a disposizione dei filtri per ordinare i clienti  & \hyperref[UC16]{UC16} \row
        \sreq{1}{F} & Il venditore deve potere contattare il cliente tramite un form  & \hyperref[UC21]{UC21} \row

        \req{1}{F} & Il venditore deve poter gestire le aliquote IVA per la tassazione dei prodotti & \hyperref[UC17]{UC17} \row
        \sreq{1}{F} & Il venditore deve poter visualizzare le aliquote IVA già presenti nel sistema & \hyperref[UC17]{UC17} \row
        \sreq{1}{F} & Il venditore deve poter aggiungere un'aliquota IVA impostandone la percentuale e una breve descrizione & \hyperref[UC17.1]{UC17.1} \row
        \sreq{1}{F} & Il venditore deve poter modificare percentuale e descrizione delle aliquote IVA già presenti nel sistema & \hyperref[UC17.2]{UC17.2} \row
        \sreq{1}{F} & Il venditore deve poter eliminare un'aliquota IVA presente nel sistema & \hyperref[UC17.3]{UC17.3} \row
        \req{1}{F} & Il venditore deve poter gestire i prodotti del sistema & \hyperref[UC18]{UC18} \row
        \sreq{2}{F} & Il venditore deve poter cercare un prodotto presente nel sistema & \hyperref[UC18]{UC18} \row
        \sreq{2}{F} & Il venditore deve avere a disposizione dei filtri per ordinare i prodotti del negozio & \hyperref[UC18]{UC18} \row
        \sreq{3}{F} & Un prodotto ha alcuni campi obbligatori e altri opzionali (che non devono essere necessariamente impostati) & \hyperref[UC18]{UC18} \row
        \sreq{1}{F} & Il venditore deve poter aggiungere un prodotto al sistema & \hyperref[UC18.1]{UC18.1} \row
        \sreq{1}{F} & Il venditore deve compilare tutti i campi obbligatori del form per aggiungere un prodotto al negozio & \hyperref[UC18.1]{UC18.1} \row
        \sreq{1}{F} & Il venditore deve poter modificare un prodotto presente nel sistema, aggiornando i campi già presenti e/o inserendone di nuovi & \hyperref[UC18.2]{UC18.2} \row
        \sreq{1}{F} & Il venditore deve poter eliminare un prodotto presente nel sistema & \hyperref[UC18.3]{UC18.3} \row
        \req{1}{F} & Il venditore deve poter gestire le categorie di prodotti & \hyperref[UC19]{UC19} \row
        \sreq{1}{F} & Il venditore deve poter aggiungere una categoria di prodotti impostandone il nome & \hyperref[UC19.1]{UC19.1} \row
        \sreq{1}{F} & Il venditore deve poter rimuovere una categoria di prodotti & \hyperref[UC19.2]{UC19.2} \row
        \sreq{1}{F} & Il venditore deve poter modificare il nome di una categoria di prodotti & \hyperref[UC19.3]{UC19.3} \row
        \req{1}{F} & Il venditore deve poter effettuare il login & \hyperref[UC22]{UC22} \row
        \req{1}{F} & Il venditore deve poter effettuare il logout & \hyperref[UC20]{UC20} \row
 



        \rowcolor{white}
        \caption{Requisiti funzionali con rispettiva descrizione e fonte}
    \end{longtable}
\end{center}

\resetCR
\subsection{Requisiti di qualità}
\begin{center}
    \rowcolors{1}{lightest-grayest}{blue!20}
    \begin{longtable}{|p{3cm}|p{9.85cm}|p{2cm}|}
        \hline
        \rowcolor{lighter-grayer}
        \textbf{ID Requisito} & \textbf{Descrizione} & \textbf{Fonti} \\
        \hline
        \endhead
        \hline
        \multicolumn{3}{|c|}{\textit{Continua nella pagina successiva...}} \\
        \hline
        \endfoot
        \endlastfoot

        \req{1}{Q} & L'utente deve potersi autenticare per accedere alla web app & \hyperref[UC1]{UC1} \row
        \sreq{1}{Q} & L'utente deve poter usufruire dell'autenticazione a due fattori & \hyperref[UC1]{UC1} \row
        %requisiti ivan

        %requisiti gianmarco

        %requisiti francesco
        \req{1}{Q} & Il prodotto deve essere sviluppato rispettando quanto stabilito nelle \dext{Norme di Progetto v1.0.0} & Decisione interna \row
        \rowcolor{white}
        \caption{Requisiti di qualità con rispettiva descrizione e fonte}
    \end{longtable}
\end{center}

\resetCR
\subsection{Requisiti di vincolo}
\begin{center}
    \rowcolors{1}{lightest-grayest}{blue!20}
    \begin{longtable}{|p{3cm}|p{9.85cm}|p{2cm}|}
        \hline
        \rowcolor{lighter-grayer}
        \textbf{ID Requisito} & \textbf{Descrizione} & \textbf{Fonti} \\
        \hline
        \endhead
        \hline
        \multicolumn{3}{|c|}{\textit{Continua nella pagina successiva...}} \\
        \hline
        \endfoot
        \endlastfoot

        \req{1}{V} & L'utente deve potersi autenticare per accedere alla web app & \hyperref[UC1]{UC1} \row
        \sreq{1}{V} & L'utente deve poter usufruire dell'autenticazione a due fattori & \hyperref[UC1]{UC1} \row
        %requisiti ivan

        %requisiti gianmarco

        %requisiti francesco

        \rowcolor{white}
        \caption{Requisiti di vincolo con rispettiva descrizione e fonte}
    \end{longtable}
\end{center}

\resetCR
\subsection{Requisiti prestazionali}
\begin{center}
    \rowcolors{1}{lightest-grayest}{blue!20}
    \begin{longtable}{|p{3cm}|p{9.85cm}|p{2cm}|}
        \hline
        \rowcolor{lighter-grayer}
        \textbf{ID Requisito} & \textbf{Descrizione} & \textbf{Fonti} \\
        \hline
        \endhead
        \hline
        \multicolumn{3}{|c|}{\textit{Continua nella pagina successiva...}} \\
        \hline
        \endfoot
        \endlastfoot

        \req{1}{P} & L'utente deve potersi autenticare per accedere alla web app & \hyperref[UC1]{UC1} \row
        \sreq{1}{P} & L'utente deve poter usufruire dell'autenticazione a due fattori & \hyperref[UC1]{UC1} \row
        %requisiti ivan

        %requisiti gianmarco

        %requisiti francesco

        \rowcolor{white}
        \caption{Requisiti prestazionali con rispettiva descrizione e fonte}
    \end{longtable}
\end{center}
