%  CODICE DEI REQUISITI (da NdP)
%          
%             R[Importanza][Tipologia][Codice]
%       
%       Importanza
%               1: requisito obbligatorio;
%               2: requisito desiderabile;
%               3: requisito opzionale.
%       Tipologia
%               V: requisito di vincolo;
%               F: requisito funzionale;
%               P: requisito prestazionale;
%               Q: requisito di qualità.
%       Codice
%   	        [CodiceBase].[CodiceSottoRequisito]

\section{Requisiti}

% comandi per le tabelle
\newcommand{\req}[3]{\textbf{R#1-#2-#3}}
\newcommand{\sreq}[3]{{\color{gray} R#1-#2-#3}}
\newcommand{\row}{ \\ \hline} 

\subsection{Introduzione}
In seguito viene riportato, organizzato in forma tabellare, l'elenco dei requisiti del progetto suddivisi per tipologia. Ciascun requisito possiede un codice identificativo stabilito in accordo con quanto scritto nelle \dext{Norme di Progetto v1.0.0}.

\subsection{Requisiti funzionali}
\begin{center}
    \rowcolors{1}{lightest-grayest}{blue!20}
    \begin{longtable}{|p{3cm}|p{9.85cm}|p{2cm}|}
    \hline
    \rowcolor{lighter-grayer}
    \textbf{ID Requisito} & \textbf{Descrizione} & \textbf{Fonti} \\
    \hline
    \endhead
    \hline
    \multicolumn{3}{|c|}{\textit{Continua nella pagina successiva...}}\\
    \hline
    \endfoot
    \endlastfoot

    \req{A}{F}{1} 		& L'utente deve potersi autenticare per accedere alla web app & UC 1 \row
        \sreq{A}{F}{1.1} 	& L'utente deve poter usufruire dell'autenticazione a due fattori & UC 26, UC 14, Interna \row
    
    %requisiti ivan

    %requisiti gianmarco

    %requisiti francesco

    \rowcolor{white}
    \caption{Requisiti funzionali con rispettiva descrizione e fonte}
	\end{longtable}
    \end{center}
    
\subsection{Requisiti di qualità}
\begin{center}
    \rowcolors{1}{lightest-grayest}{blue!20}
    \begin{longtable}{|p{3cm}|p{9.85cm}|p{2cm}|}
    \hline
    \rowcolor{lighter-grayer}
    \textbf{ID Requisito} & \textbf{Descrizione} & \textbf{Fonti} \\
    \hline
    \endhead
    \hline
    \multicolumn{3}{|c|}{\textit{Continua nella pagina successiva...}}\\
    \hline
    \endfoot
    \endlastfoot

    \req{A}{F}{1} 		& L'utente deve potersi autenticare per accedere alla web app & UC 1 \row
        \sreq{A}{F}{1.1} 	& L'utente deve poter usufruire dell'autenticazione a due fattori & UC 26, UC 14, Interna \row
    %requisiti ivan

    %requisiti gianmarco

    %requisiti francesco

    \rowcolor{white}
    \caption{Requisiti di qualità con rispettiva descrizione e fonte}
	\end{longtable}
	\end{center}

\subsection{Requisiti di vincolo}
\begin{center}
    \rowcolors{1}{lightest-grayest}{blue!20}
    \begin{longtable}{|p{3cm}|p{9.85cm}|p{2cm}|}
    \hline
    \rowcolor{lighter-grayer}
    \textbf{ID Requisito} & \textbf{Descrizione} & \textbf{Fonti} \\
    \hline
    \endhead
    \hline
    \multicolumn{3}{|c|}{\textit{Continua nella pagina successiva...}}\\
    \hline
    \endfoot
    \endlastfoot

    \req{A}{F}{1} 		& L'utente deve potersi autenticare per accedere alla web app & UC 1 \row
        \sreq{A}{F}{1.1} 	& L'utente deve poter usufruire dell'autenticazione a due fattori & UC 26, UC 14, Interna \row
    %requisiti ivan

    %requisiti gianmarco

    %requisiti francesco

    \rowcolor{white}
    \caption{Requisiti di vincolo con rispettiva descrizione e fonte}
	\end{longtable}
    \end{center}
    
\subsection{Requisiti prestazionali}
\begin{center}
    \rowcolors{1}{lightest-grayest}{blue!20}
    \begin{longtable}{|p{3cm}|p{9.85cm}|p{2cm}|}
    \hline
    \rowcolor{lighter-grayer}
    \textbf{ID Requisito} & \textbf{Descrizione} & \textbf{Fonti} \\
    \hline
    \endhead
    \hline
    \multicolumn{3}{|c|}{\textit{Continua nella pagina successiva...}}\\
    \hline
    \endfoot
    \endlastfoot

    \req{A}{F}{1} 		& L'utente deve potersi autenticare per accedere alla web app & UC 1 \row
        \sreq{A}{F}{1.1} 	& L'utente deve poter usufruire dell'autenticazione a due fattori & UC 26, UC 14, Interna \row
    %requisiti ivan

    %requisiti gianmarco

    %requisiti francesco

    \rowcolor{white}
    \caption{Requisiti prestazionali con rispettiva descrizione e fonte}
	\end{longtable}
    \end{center}
    