
%16
\stepUserCase
\subsection{\valueUserCase - Contatta cliente}
\labelUserCase
\begin{itemize}
    \item \textbf{Attore Primario:} venditore autenticato;
    \item \textbf{Descrizione:} il venditore deve contattare il cliente;
    \item \textbf{Precondizione:} il venditore deve trovarsi nella lista degli ordini \hyperref[UC17]{UC17} o nella lista clienti\hyperref[UC18]{UC18};
    \item \textbf{Postcondizione:} il venditore deve aver contattato il cliente con successo;
    \item \textbf{Scenario principale:}
    \begin{enumerate}
        \item il venditore ha aperto il form di contatto;
        \item compila il form;
        \item il venditore contatta con successo il cliente;
    \end{enumerate}
\end{itemize}

%17
\stepUserCase
\subsection{\valueUserCase - Lista ordini}
\labelUserCase
\begin{itemize}
    \item \textbf{Attore Primario:} venditore autenticato;
    \item \textbf{Descrizione:} il venditore vuole visualizzare tutta la lista degli ordini effettuati dai clienti;
    \item \textbf{Precondizione:} il venditore deve essere loggato;
    \item \textbf{Postcondizione:} il venditore riesce a visualizzare la lista;
    \item \textbf{Scenario principale:}
    \begin{enumerate}
        \item il venditore visualizza tutta la lista degli ordini effettuati dai clienti;
         \item il venditore può modificare lo stato di un ordine (\hyperref[UC17.1]{UC17.1});
        \item il venditore può contattare il cliente che ha effettuato un determinato ordine (\hyperref[UC16]{UC16});
        \item può stampare la bolla di un ordine (\hyperref[UC17.2]{UC17.2});
        \item può visualizzare i dettagli di un determinato ordine (\hyperref[UC17.3]{UC17.3}); 
    \end{enumerate}
\end{itemize}

%17.1
\stepsubUserCase
\subsubsection{\valuesubUserCase- Modifica stato ordine}
\labelsubUserCase
\begin{itemize}
    \item \textbf{Attore Primario:} venditore autenticato;
    \item \textbf{Descrizione:} il venditore vuole modificare lo stato dell'ordine;
    \item \textbf{Precondizione:} il venditore deve trovarsi nella sezione di gestione degli ordini;
    \item \textbf{Postcondizione:} il venditore modifica lo stato di un ordine
    \item \textbf{Scenario principale:}
    \begin{enumerate}
        \item il venditore seleziona un ordine dalla lista;
        \item il venditore modifica lo stato dell'ordine;
    \end{enumerate}
\end{itemize}

%17.2
\stepsubUserCase
\subsubsection{\valuesubUserCase- Stampa bolla}
\labelsubUserCase
\begin{itemize}
    \item \textbf{Attore Primario:} venditore autenticato;
    \item \textbf{Descrizione:}il venditore vuole stampare la bolla per l'ordine;
    \item \textbf{Precondizione:} il venditore deve trovarsi nella lista degli ordini;
    \item \textbf{Postcondizione:} il venditore stampa la bolla;
    \item \textbf{Scenario principale:}
    \begin{enumerate}
        \item il venditore seleziona un determinato ordine;
        \item il venditore stampa la bolla;
    \end{enumerate}
\end{itemize}

%17.3
\stepsubUserCase
\subsubsection{\valuesubUserCase- Visualizzazione dettagli ordine}
\labelsubUserCase
\begin{itemize}
    \item \textbf{Attore Primario:} venditore autenticato;
    \item \textbf{Descrizione:} il venditore vuole visualizzare i dettagli di un determinato ordine;
    \item \textbf{Precondizione:} il venditore deve trovarsi nella lista degli ordini e selezionarne uno;
    \item \textbf{Postcondizione:} il venditore visualizza tutti i dettagli dell'ordine;
    \item \textbf{Scenario principale:}
    \begin{enumerate}
        \item il venditore seleziona un determinato ordine;
        \item il venditore apre la pagina con i dettagli dell'ordine selezionato;
    \end{enumerate}
\end{itemize}

%18
\stepUserCase
\subsection{\valueUserCase - Lista clienti}
\labelUserCase
\begin{itemize}
    \item \textbf{Attore Primario:} venditore autenticato;
    \item \textbf{Descrizione:} il venditore vuole visualizzare la lista dei clienti del sito;
    \item \textbf{Precondizione:} il venditore deve essere loggato;
    \item \textbf{Postcondizione:} il venditore visualizza tutta la lista clienti;
    \item \textbf{Scenario principale:}
    \begin{enumerate}
        \item il venditore visualizza la lista;
         \item il venditore può contattare un cliente (\hyperref[UC16]{UC16});
    \end{enumerate}
\end{itemize}

%19
\stepUserCase
\subsection{\valueUserCase - Amministrazione tasse}
\labelUserCase
\begin{itemize}
    \item \textbf{Attore Primario:} venditore autenticato
    \item \textbf{Descrizione:} Il venditore ha possibilità di gestire le operazioni relative all’amministrazione della tassazione dei prodotti 
    \item \textbf{Precondizione:} il venditore ha cliccato nel bottone relativo alla gestione delle tasse all’interno della dashboard venditore.
    \item \textbf{Input:} il venditore preme su un pulsante dedicato alla gestione delle aliquote IVA
    \item \textbf{Postcondizione:} il venditore ha visione della tabella relativa alle aliquote IVA già presenti nel sistema e può intraprendere una delle azioni disponibili.
    \item \textbf{Scenario principale:} Il venditore ha cliccato nel bottone relativo alla gestione delle tasse e ha visione della tabella riepilogativa delle varie aliquote IVA presenti nel sistema, può scegliere di effettuare una delle seguenti azioni 
    \begin{enumerate}
        \item Aggiunta aliquote IVA;
        \item Rimozione aliquote IVA;
        \item Modifica aliquota IVA.
    \end{enumerate}
\end{itemize}

%19.1
\stepsubUserCase
\subsubsection{\valuesubUserCase- Aggiunta aliquota IVA}
\labelsubUserCase
\begin{itemize}
    \item \textbf{Attore Primario:}  venditore autenticato
    \item \textbf{Descrizione:} Il venditore aggiunge una nuova aliquota IVA  
    \item \textbf{Precondizione:} il venditore si trova sulla pagina per la gestione delle aliquote IVA
    \item \textbf{Input:} il venditore preme su un pulsante dedicato all'aggiunta di un'aliquota IVA
    \item \textbf{Postcondizione:} è stata aggiunta una nuova aliquota IVA nel sistema
    \item \textbf{Scenario principale:} Il venditore ha cliccato nel bottone di aggiunta aliquota, posto nella pagina di amministrazione della tassazione e gli viene richiesto di inserire:
    \begin{enumerate}
        \item Inserimento percentuale IVA;
        \item Inserimento descrizione aliquota.
    \end{enumerate}
\end{itemize}

%19.2
\stepsubUserCase
\subsubsection{\valuesubUserCase- Modifica aliquota IVA}
\labelsubUserCase
\begin{itemize}
    \item \textbf{Attore Primario:}  venditore autenticato
    \item \textbf{Descrizione:} Il venditore effettua la modifica di una aliquota IVA già presente nel sistema;
    \item \textbf{Precondizione:} Pre-condizione: il venditore ha cliccato nel bottone per effettuare la modifica di un’aliquota IVA;
    \item \textbf{Input:} il venditore preme su un pulsante dedicato alla modifica di un'aliquota IVA
    \item \textbf{Postcondizione:} è stata modificata l'aliquota IVA selezionata
    \item \textbf{Scenario principale:} Il venditore ha cliccato nel bottone di modifica di un’aliquota IVA già presente nel sistema per apportare delle modifiche, può quindi effettuare le seguenti azioni
    \begin{enumerate}
        \item Modificare percentuale IVA;
        \item Modificare descrizione aliquota.
    \end{enumerate}
\end{itemize}

%19.3
\stepsubUserCase
\subsubsection{\valuesubUserCase- Eliminazione aliquota IVA}
\labelsubUserCase
\begin{itemize}
    \item \textbf{Attore Primario:}  venditore autenticato
    \item \textbf{Descrizione:} Il venditore effettua l'eliminazione di una aliquota IVA già presente nel sistema;
    \item \textbf{Precondizione:} Pre-condizione: il venditore ha cliccato nel bottone per effettuare la modifica di un’aliquota IVA;
    \item \textbf{Input:} il venditore preme su un pulsante dedicato all'eliminazione di un'aliquota IVA
    \item \textbf{Postcondizione:} l'aliquota IVA selezionata è stata modificata
    \item \textbf{Scenario principale:} Il venditore ha cliccato nel bottone di modifica di un’aliquota IVA già presente nel sistema per apportare delle modifiche, può quindi effettuare le seguenti azioni
    \begin{enumerate}
        \item Modificare percentuale IVA;
        \item Modificare descrizione aliquota.
    \end{enumerate}
\end{itemize}

%20
\stepUserCase
\subsection{\valueUserCase - Amministrazione prodotti}
\labelUserCase
\begin{itemize}
    \item \textbf{Attore Primario:}  venditore autenticato
    \item \textbf{Descrizione:} Il venditore ha possibilità di visualizzare l’elenco prodotti inseriti nel sistema e di effettuare una delle azioni rese disponibili.
    \item \textbf{Precondizione:} il venditore si trova nella dashboard di amministrazione.
    \item \textbf{Input:} il venditore preme sul pulsante per l'accesso alla pagina di amministrazione prodotti
    \item \textbf{Postcondizione:} il venditore ha visione della tabella relativa dei prodotti già presenti nel sistema e può intraprendere una delle azioni disponibili.
    \item \textbf{Scenario principale:} Il venditore ha cliccato nel bottone relativo alla gestione dei prodotti e ha visione della tabella riepilogativa dei vari prodotti presenti nel sistema, può scegliere di effettuare una delle seguenti azioni 
    \begin{enumerate}
        \item Aggiunta prodotto;
        \item Modifica prodotto;
        \item Rimuovi prodotto;
        \item Aggiorna giacenza.
    \end{enumerate}
\end{itemize}

%20.1
\stepsubUserCase
\subsubsection{\valuesubUserCase- Aggiunta prodotto}
\labelsubUserCase
\begin{itemize}
    \item \textbf{Attore Primario:}  venditore autenticato.
    \item \textbf{Descrizione:} Il venditore ha possibilità di aggiungere un nuovo prodotto all’interno del portale.
    \item \textbf{Precondizione:} il venditore si trova nella pagina di gestione prodotti
    \item \textbf{Input:} click sul bottone per inserire un nuovo articolo
    \item \textbf{Postcondizione:} il nuovo prodotto è stato inserito nel sistema
    \item \textbf{Scenario principale:} Il venditore ha cliccato nel bottone per inserire un nuovo prodotto all’interno del catalogo prodotti gestito dal sistema, le azioni da completare sono: 
    \begin{enumerate}
        \item Inserimento titolo prodotto;
        \item Inserimento descrizione prodotto;
        \item Inserimento immagine prodotto;
        \item Selezione categorie prodotto;
        \item Inserimento prezzo netto prodotto;
        \item Selezione aliquota IVA;
        \item Selezione stato visibilità prodotto; 
        \item Aggiungi attributi SEO prodotto;
        \item Inserimento dati giacenza magazzino del prodotto;
        \item Visualizzazione in home (dato che indica se visualizzare o meno il prodotto in vetrina);
        \item Salvataggio.
    \end{enumerate}
    \item \textbf{Estensione:}
    \begin{itemize}
        \item L'inserimento fallisce perché il titolo è vuoto (campo obbligatorio) 
        \item L'inserimento fallisce perché la quantità inserita non è valida (es. negativa o valore testuale)
    \end{itemize}
\end{itemize}

%20.2
\stepsubUserCase
\subsubsection{\valuesubUserCase- Modifica  prodotto}
\labelsubUserCase
\begin{itemize}
    \item \textbf{Attore Primario:}  venditore autenticato.
    \item \textbf{Descrizione:} Il venditore ha possibilità di modificare un prodotto già inserito nel sistema.
    \item \textbf{Precondizione:} il venditore si trova nella pagina di gestione prodotti
    \item \textbf{Input:} click sul bottone per la modifica di un determinato prodotto
    \item \textbf{Postcondizione:} il prodotto selezionato è stato modificato
    \item \textbf{Scenario principale:} Il venditore ha cliccato nel bottone per modificare un prodotto all’interno del catalogo prodotti gestito dal sistema, le azioni che potrebbe attuare sono: 
    \begin{enumerate}
        \item Modifica titolo prodotto;
        \item Modifica descrizione prodotto;
        \item Modifica immagine prodotto;
        \item Modifica categorie prodotto;
        \item Modifica prezzo netto prodotto;
        \item Modifica aliquota IVA;
        \item Modifica stato visibilità prodotto; 
        \item Modifica attributi SEO prodotto;
        \item Modifica visualizzazione in home (togliere o aggiungere il prodotto dalla home);
        \item Salvataggio.
    \end{enumerate}
    \item \textbf{Estensione:}
    \begin{itemize}
        \item La modifica fallisce perché il titolo è vuoto (campo obbligatorio) 
        \item La modifica fallisce perché la quantità inserita non è valida (es. negativa o valore testuale)
    \end{itemize}
\end{itemize}

%20.3
\stepsubUserCase
\subsubsection{\valuesubUserCase- Rimozione prodotto}
\labelsubUserCase
\begin{itemize}
    \item \textbf{Attore Primario:}  venditore autenticato.
    \item \textbf{Descrizione:} Il venditore può rimuovere un prodotto
    \item \textbf{Precondizione:} il venditore si trova nella pagina di amministrazione prodotti
    \item \textbf{Input:} il venditore clicca sul pulsante di eliminazione per un determinato prodotto
    \item \textbf{Postcondizione:} il prodotto selezionato non è più presente nel sistema 
\end{itemize}

%20.4
\stepsubUserCase
\subsubsection{\valuesubUserCase- Aggiorna giacenza}
\labelsubUserCase
\begin{itemize}
    \item \textbf{Attore Primario:}  venditore autenticato.
    \item \textbf{Descrizione:}Il venditore può aggiornare la giacenza del prodotto nel magazzino
    \item \textbf{Precondizione:} il venditore si trova sulla pagina di amministrazione prodotti
    \item \textbf{Input:} inserimento nuovo dato quantità per un determinato prodotto
    \item \textbf{Postcondizione:} il prodotto è stato con il nuovo dato di giacenza
    \item \textbf{Scenario principale:} l’utente visualizza la tabella riepilogativa dei prodotti e aggiorna il dato relativo alla giacenza di un prodotto
    \item \textbf{Estensione:}
    \begin{itemize}
        \item la quantità inserita non è valida (es. negativa o valore testuale), la modifica fallisce
    \end{itemize}
\end{itemize}

%21
\stepUserCase
\subsection{\valueUserCase - Amministrazione categorie}
\labelUserCase
\begin{itemize}
    \item \textbf{Attore Primario:}  venditore autenticato.
    \item \textbf{Descrizione:}  Il venditore può visualizzare e gestire le categorie di prodotto.
    \item \textbf{Precondizione:}  il venditore si trova nella dashboard di amministrazione
    \item \textbf{Input:} il venditore seleziona il pulsante per l'amministrazione delle categorie
    \item \textbf{Postcondizione:} l’utente visualizza la tabella riepilogativa delle categorie prodotto presenti nel sistema.
    \item \textbf{Scenario principale:} visualizza la tabella riepilogativa delle categorie prodotto già presenti nel sistema e può intraprendere una delle seguenti azioni
    \begin{enumerate}
        \item Aggiungi categoria;
        \item Modifica categoria;
        \item Rimuovi categoria;
    \end{enumerate}
\end{itemize}

%21.1
\stepsubUserCase
\subsubsection{\valuesubUserCase- Aggiungi categoria}
\labelsubUserCase
\begin{itemize}
    \item \textbf{Attore Primario:}  venditore autenticato.
    \item \textbf{Descrizione:}  Il venditore può aggiungere una categoria.
    \item \textbf{Precondizione:} il venditore si trova sulla pagina di amministrazione categorie
    \item \textbf{Input:} il venditore ha cliccato nel pulsante per aggiungere una nuova categoria di prodotto
    \item \textbf{Postcondizione:} la nuova categoria è stata inserita nel sistema
    \item \textbf{Scenario principale:} il venditore ha cliccato sul pulsante di inserimento nuova categoria disponibile nella pagina di gestione categorie.  Le azioni che dovrà compiere sono
    \begin{enumerate}
        \item Inserimento nome categoria;
        \item Salvataggio;
    \end{enumerate}
    \item \textbf{Estensioni:} L'inserimento fallisce se:
    \begin{itemize}
        \item la categoria è già esistente
        \item il nome categoria è vuoto
    \end{itemize}
\end{itemize}

%21.2
\stepsubUserCase
\subsubsection{\valuesubUserCase- Modifica categoria}
\labelsubUserCase
\begin{itemize}
    \item \textbf{Attore Primario:}  venditore autenticato.
    \item \textbf{Descrizione:}  Il venditore può modificare il nome di una categoria.
    \item \textbf{Precondizione:} il venditore si trova sulla pagina di amministrazione categorie
    \item \textbf{Input:} il venditore ha cliccato nel pulsante per modificare una categoria di prodotto
    \item \textbf{Postcondizione:} la categoria selezionata è stata modificata
    \item \textbf{Scenario principale:}
    \begin{itemize}
        \item il venditore inserisce il nuovo nome per la categoria
        \item preme quindi su conferma
    \end{itemize}
    \item \textbf{Estensioni:} La modifica fallisce se:
    \begin{itemize}
        \item il nuovo nome inserito è già utilizzato per un'altra categoria
        \item il nome categoria inserito è vuoto
    \end{itemize}
\end{itemize}

%21.3
\stepsubUserCase
\subsubsection{\valuesubUserCase- Rimozione categoria}
\labelsubUserCase
\begin{itemize}
    \item \textbf{Attore Primario:}  venditore autenticato.
    \item \textbf{Descrizione:}  Il venditore può rimuovere una categoria.
    \item \textbf{Precondizione:} il venditore si trova sulla pagina di amministrazione categorie
    \item \textbf{Input:} il venditore ha cliccato nel pulsante per rimuovere una nuova categoria di prodotto
    \item \textbf{Postcondizione:} Il sistema ha rimosso la categoria selezionata dal venditore
    \item \textbf{Scenario principale:} il venditore ha cliccato sul pulsante di rimozione categoria disponibile nella pagina di gestione categorie. 
    \item \textbf{Estensioni:} la rimozione di una categoria fallisce se vi sono prodotti ancora collegati a quella categoria. In quel caso viene richiesto
                                di inserire a quale categoria dirottare i prodotti coinvolti o se lasciarli senza categoria.
\end{itemize}

%22
\stepUserCase
\subsection{\valueUserCase - Logout venditore}
\labelUserCase
\begin{itemize}
    \item \textbf{Attore Primario:} venditore autenticato;
    \item \textbf{Descrizione:} logout venditore dal backoffice;
    \item \textbf{Precondizione:} il venditore si trova sulla dashboard di amministrazione
    \item \textbf{Input:} il venditore preme sul pulsante per eseguire il logout
    \item \textbf{Postcondizione:} il venditore non è più autenticato nel backoffice e si trova sulla pagina per login del venditore;
\end{itemize}

%23
\stepUserCase
\subsection{\valueUserCase - Login venditore}
\labelUserCase
\begin{itemize}
    \item \textbf{Attore Primario:} venditore non autenticato;
    \item \textbf{Descrizione:} autenticazione del venditore nel backoffice;
    \item \textbf{Precondizione:} il venditore non è autenticato
    \item \textbf{Input:} il venditore chiede di accedere al backoffice inserendo il link al portale di amministrazione
    \item \textbf{Postcondizione:} il venditore è autenticato
    \item \textbf{Scenario principale:}
    \begin{enumerate}
        \item il venditore inserisci i dati di login (username e password)
        \item conferma il login;
    \end{enumerate}
    \item \textbf{Estensioni:} i dati di login sono errati, il login fallisce e si chiede il reinserimento dei dati
\end{itemize}