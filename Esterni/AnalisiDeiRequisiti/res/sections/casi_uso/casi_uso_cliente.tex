\subsection{UC1 - Visualizzazione contenuti informativi}
\label{UC1}
\begin{itemize}
    \item \textbf{Attore Primario:} cliente generico;
    \item \textbf{Descrizione:} il cliente vuole accedere alle informazioni riguardanti la piattaforma e il venditore;
    \item \textbf{Precondizione:} il cliente non sta visualizzando alcuna informazione della piattaforma;
    \item \textbf{Postcondizione:} il cliente visualizza le informazioni relative alla pagina scelta;
    \item \textbf{Scenario principale:}
    \begin{enumerate}
        \item l'utente si collega alla piattaforma e naviga fino alla pagina scelta;
    \end{enumerate}
\end{itemize}

\subsection{UC2 - Scelta categoria}
\label{UC2}
\begin{itemize}
    \item \textbf{Attore Primario:} cliente generico;
    \item \textbf{Descrizione:} il cliente vuole visualizzare tutti i prodotti riguardanti una determinata categoria;
    \item \textbf{Precondizione:} il cliente si trova in una pagina dove la scelta di categoria è permessa;
    \item \textbf{Input:} seleziona una categoria;
    \item \textbf{Postcondizione:} il cliente visualizza tutti i prodotti relativi alla categoria selezionata;
    \item \textbf{Scenario principale:}
    \begin{enumerate}
        \item il cliente entra in una pagina dove è presente la scelta di una categoria;
        \item seleziona una categoria;
        \item viene reindirizzato ad una pagina di visualizzazione prodotto (PLP) e visualizza i prodotti desiderati (\hyperref[UC4]{UC4});
    \end{enumerate}
\end{itemize}

\subsection{UC3 - Ricerca}
\label{UC3}
\begin{itemize}
    \item \textbf{Attore Primario:} cliente generico;
    \item \textbf{Descrizione:} il cliente vuole visualizzare i prodotti che contengono nel nome prodotto una determinata parola
    \item \textbf{Precondizione:} il cliente si trova in una pagina dove funzione di ricerca è permessa;
    \item \textbf{Input:} una stringa;
    \item \textbf{Postcondizione:} il cliente visualizza tutti i prodotti che contengono la determinata stringa nel nome;
    \item \textbf{Scenario principale:}
    \begin{enumerate}
        \item il cliente entra in una pagina dove è permessa la ricerca di un prodotto, inserisce la parola da cercare e viene reindirizzato ad una pagina di visualizzazione prodotto (PLP) e visualizza i prodotti desiderati (\hyperref[UC4]{UC4});
    \end{enumerate}
\end{itemize}

\subsection{UC4 - Visualizzazione lista prodotti (PLP)}
\label{UC4}
\begin{itemize}
    \item \textbf{Attore Primario:} cliente generico;
    \item \textbf{Descrizione:} questa pagina visualizza tutti i prodotti che corrispondono ad una categoria o che corrispondono ad una parola cercata;
    \item \textbf{Precondizione:} il cliente sceglie uno dei due modi per accedere alla PLP (\hyperref[UC2]{UC2} e \hyperref[UC3]{UC3});
    \item \textbf{Postcondizione:} il cliente visualizza i prodotti che corrispondono alla scelta;
    \item \textbf{Scenario principale:}
    \begin{enumerate}
        \item il cliente può visualizzare tutti i prodotti che corrispondono alle politiche di visualizzazione;
    \end{enumerate}
    \item \textbf{Estensioni:}
    \begin{itemize}
        \item la pagina visualizza una PLP vuota perché nessun prodotto corrisponde alle politiche di visualizzazione;
    \end{itemize}
\end{itemize}

\subsubsection{UC4.1 - Filtri}
\label{UC4.1}
\begin{itemize}
    \item \textbf{Attore Primario:} cliente generico;
    \item \textbf{Descrizione:} il cliente può inserire ulteriori filtri per raffinare la visualizzazione dei prodotti;
    \item \textbf{Precondizione:} il cliente si deve trovare in una PLP (UC4);
    \item \textbf{Postcondizione:} il cliente visualizza i prodotti con i filtri aggiornati;
    \item \textbf{Scenario principale:}
    \begin{enumerate}
        \item il cliente seleziona il filtro;
        \item il cliente modifica il valore del filtro;
        \item il cliente visualizza i prodotti secondo le nuove politiche; 
    \end{enumerate}
    \item \textbf{Specializzazioni: }
    \begin{itemize}
        \item il cliente filtra i prodotti per prezzo (\hyperref[UC4.1.1]{UC4.1.1});
        \item il cliente filtra i prodotti per categoria (\hyperref[UC4.1.2]{UC4.1.2});
    \end{itemize}
\end{itemize}

\paragraph{UC4.1.1 - Inserimento filtro prezzo}
\label{UC4.1.1}
\begin{itemize}
    \item \textbf{Attore Primario:} cliente generico;
    \item \textbf{Descrizione:} questa funzionalità permette al cliente di filtrare i prodotti visualizzati per prezzo;
    \item \textbf{Precondizione:} il cliente si deve trovare in una PLP (\hyperref[UC4]{UC4});
    \item \textbf{Postcondizione:} il cliente visualizza i prodotti che corrispondono alla scelta
    \item \textbf{Scenario principale:}
    \begin{enumerate}
        \item il cliente inserisce l'intervallo di prezzo desiderato;
        \item il cliente visualizza i prodotti che corrispondono alla scelta;
    \end{enumerate}
    \item \textbf{Estensioni:}
    \begin{itemize}
        \item la pagina visualizza una PLP vuota perché nessun prodotto corrisponde alle politiche di visualizzazione;
    \end{itemize}
\end{itemize}

\paragraph{UC4.1.2 - Inserimento filtro categoria}
\label{UC4.1.2}
\begin{itemize}
    \item \textbf{Attore Primario:} cliente generico;
    \item \textbf{Descrizione:} questa pagina visualizza tutti i prodotti che corrispondono ad una categoria o che corrispondono ad una parola cercata;
    \item \textbf{Precondizione:} il cliente sceglie uno dei due modi per accedere alla PLP (\hyperref[UC2]{UC2} e \hyperref[UC3]{UC3});
    \item \textbf{Postcondizione:} il cliente visualizza i prodotti che corrispondono alla scelta
    \item \textbf{Scenario principale:}
    \begin{enumerate}
        \item il cliente inserisce una nuova categoria o modifica una nuova esistente;
        \item il cliente visualizza i prodotti che corrispondono alla scelta;
    \end{enumerate}
    \item \textbf{Estensioni:}
    \begin{itemize}
        \item la pagina visualizza una PLP vuota perché nessun prodotto corrisponde alle politiche di visualizzazione;
    \end{itemize}
\end{itemize}

\subsection{UC5 - Apertura dettagli prodotto}
\label{UC5}
\begin{itemize}
    \item \textbf{Attore Primario:} cliente generico;
    \item \textbf{Descrizione:} questa pagina visualizza tutti i dettagli del prodotto;
    \item \textbf{Precondizione:} il cliente deve trovarsi in una pagina dove la visualizzazione dei dettagli del prodotto sia permessa;
    \item \textbf{Postcondizione:} il cliente visualizza tutti i dettagli del prodotto;
    \item \textbf{Scenario principale:}
    \begin{enumerate}
        \item il cliente visualizza tutti i dettagli del prodotto;
    \end{enumerate}
\end{itemize}

\subsection{UC6 - Aggiunta prodotto al carrello}
\label{UC6}
\begin{itemize}
    \item \textbf{Attore Primario:} Cliente generico;
    \item \textbf{Descrizione:} questa azione serve ad aggiungere un prodotto in vendita nel sito
                                all'interno del carrello personale utente, specificando in quale quantità;
    \item \textbf{Precondizione:} il cliente si trova nella PDP di un prodotto;
    \item \textbf{Input:} il cliente clicca sul pulsante "aggiungi al carrello" dopo aver selezionato la quantità (di default a 1)
    \item \textbf{Postcondizione:} l'oggetto è stato inserito nel carrello nella quantità desiderata (se disponibile in magazzino)
    \item \textbf{Scenario principale:}
    \begin{enumerate}
        \item il cliente seleziona la quantità desiderata o lascia quella di default;
        \item il cliente clicca sul pulsante "aggiungi al carrello" o simile;
    \end{enumerate}
    \item \textbf{Inclusioni:}
    \begin{itemize}
        \item viene eseguito un controllo sulla quantità disponibile di quel determinato prodotto;
    \end{itemize}
    \item \textbf{Estensioni:}
    \begin{itemize}
        \item in caso la quantità disponibile sia insufficiente a soddisfare la richiesta del cliente viene visualizzato un errore;
    \end{itemize}
\end{itemize}

\subsection{UC7 - Gestione carrello}
\label{UC7}
\begin{itemize}
    \item \textbf{Attore Primario:} cliente generico;
    \item \textbf{Descrizione:} insieme di azioni atte alla gestione degli articoli già presenti nel carrello;
    \item \textbf{Precondizione:} il cliente sta navigando il sito web;
    \item \textbf{Input:} il cliente esegue un'operazione gestionale sul carrello;
    \item \textbf{Postcondizione:} il cliente ha eseguito operazioni gestionali sul carrello;
\end{itemize}

\subsubsection{UC7.1 - Visualizzazione articoli}
\label{UC7.1}
\begin{itemize}
    \item \textbf{Attore Primario:} cliente generico;
    \item \textbf{Descrizione:} scenario per la visualizzazione degli articoli nel carrello;
    \item \textbf{Precondizione:} il cliente si trova su una qualunque pagina del sito web;
    \item \textbf{Input:} il cliente preme su un pulsante dedicato all'accesso al carrello;
    \item \textbf{Postcondizione:} il cliente visualizza un resoconto di tutti i prodotti inseriti nel carrello durante
                                   la sessione di utilizzo corrente (se non loggato), altrimenti il carrello mantiene la consistenza
                                   su ogni dispositivo fino al checkout.
\end{itemize}

\subsubsection{UC7.2 - Variazione quantità articolo}
\label{UC7.2}
\begin{itemize}
    \item \textbf{Attore primario:} cliente generico;
    \item \textbf{Descrizione:} scenario per la variazione della quantità di un singolo articolo presente nel carrello;
    \item \textbf{Precondizione:} il cliente ha eseguito \hyperref[UC7.1]{UC7.1};
    \item \textbf{Input:} il cliente inserisce la nuova quantità desiderata per quel determinato articolo;
    \item \textbf{Postcondizione:} se la quantità desiderata è disponibile, viene aggiornato l'articolo nel carrello;
    \item \textbf{Scenario principale:}
    \begin{itemize}
        \item il cliente seleziona la nuova quantità desiderata;
        \item il cliente conferma la scelta
    \end{itemize}
    \item \textbf{Inclusioni:}
    \begin{itemize}
        \item viene eseguito un controllo sulla quantità disponibile di quel determinato prodotto;
    \end{itemize}
    \item \textbf{Estensioni:}
    \begin{itemize}
        \item in caso la quantità disponibile sia insufficiente a soddisfare la richiesta del cliente viene visualizzato un errore;
    \end{itemize}
\end{itemize}

\subsubsection{UC7.3 - Rimozione articolo}
\label{UC7.3}
\begin{itemize}
    \item \textbf{Attore Primario:} cliente generico;
    \item \textbf{Descrizione:} scenario per la rimozione di un articolo dal carrello;
    \item \textbf{Precondizione:} il cliente ha eseguito \hyperref[UC7.1]{UC7.1};
    \item \textbf{Input:} il cliente preme su un pulsante dedicato alla rimozione di un determinato articolo dal carrello;
    \item \textbf{Postcondizione:} l'articolo selezionato non è più presente nel carrello.
\end{itemize}

\subsection{UC8 - Checkout}
\label{UC8}
\begin{figure}[H]
    \centering
    \includegraphics[width=\linewidth]{res/images/UC/UC8.png}
    \caption{Diagramma che descrive UC8 - checkout} 
\end{figure}
\begin{itemize}
    \item \textbf{Attore Primario:} cliente generico;
    \item \textbf{Descrizione:} caso d'uso per l'acquisto dei prodotti inseriti nel carrello;
    \item \textbf{Precondizione:} il cliente ha eseguito \hyperref[UC7.1]{UC7.1} e il suo carrello contiene almeno un articolo;
    \item \textbf{Input:} il cliente preme sul pulsante per avviare il checkout;
    \item \textbf{Postcondizione:} viene emesso un ordine contenente gli articoli precedentemente inseriti nel carrello. Vengono quindi rimossi dal magazzino;
    \item \textbf{Scenario principale:}
    \begin{enumerate}
        \item il cliente preme sul pulsante per avviare il checkout
        \item inserisce gli indirizzi di fatturazione e spedizione (UC 8.1 - 8.2)
        \item vengono aggiunti i costi della spedizione all'importo totale(UC 8.3), definiti come importo fisso;
        \item si effettua il pagamento (UC 8.4)
        \item l'ordine è stato emesso e contrassegnato come pagato
    \end{enumerate}
    \item \textbf{Estensioni:}
    \begin{itemize}
        \item in caso di fallimento del pagamento l'ordine non viene emesso, è necessario quindi riprovare il pagamento e viene visualizzato un errore (UC 8.6);
        \item l'utente decide di annullare il processo di checkout premendo l'apposito pulsante (UC 8.5);
    \end{itemize}
\end{itemize}

\subsubsection{UC8.1 - Inserimento indirizzo fatturazione}
\label{UC8.1}
\begin{itemize}
    \item \textbf{Attore Primario:} cliente generico;
    \item \textbf{Descrizione:} scenario per l'inserimento dell'indirizzo di fatturazione in fase di checkout;
    \item \textbf{Precondizione:} il cliente ha avviato il processo di checkout;
    \item \textbf{Input:} il cliente inserisce i dati tramite l'apposito form oppure sceglie tra quelli personali già presenti (solo se autenticato);
    \item \textbf{Postcondizione:} l'utente procede con la fase successiva del checkout;
    \item \textbf{Scenario principale:}
    \begin{itemize}
        \item l'utente ha 2 modi per inserire l'informazione:
        \begin{itemize}
            \item compilare il relativo form;
            \item scegliere tra gli indirizzi salvati (solo se autenticato).
        \end{itemize}
    \end{itemize}
\end{itemize}

\subsubsection{UC8.2 - Inserimento indirizzo spedizione}
\label{UC8.2}
\begin{itemize}
    \item \textbf{Attore Primario:} cliente generico;
    \item \textbf{Descrizione:} scenario per l'inserimento dell'indirizzo di spedizione in fase di checkout;
    \item \textbf{Precondizione:} il cliente ha avviato il checkout;
    \item \textbf{Input:} il cliente inserisce i dati tramite l'apposito form oppure sceglie tra quelli personali già presenti (solo se autenticato);
    \item \textbf{Postcondizione:} l'utente procede con la fase successiva del checkout;
    \item \textbf{Scenario principale:}
    \begin{itemize}
        \item l'utente ha 3 modi per inserire l'informazione:
        \begin{itemize}
            \item compilare il relativo form;
            \item scegliere tra gli indirizzi salvati (solo se autenticato);
            \item riutilizzare l'indirizzo di fatturazione precedentemente inserito.
        \end{itemize}
    \end{itemize}
\end{itemize}

\subsubsection{UC8.4 - Pagamento con servizio di terze parti}
\label{UC8.4}
\begin{itemize}
    \item \textbf{Attore primario:} cliente generico;
    \item \textbf{Attore secondario:} gestore dei pagamenti esterno;
    \item \textbf{Descrizione:} scenario per il pagamento del totale dell'ordine mediante un gestore di pagamenti esterno a EmporioLambda;
    \item \textbf{Precondizione:} sono stati eseguiti tutti i passi precedenti del checkout, ovvero sono stati applicati i costi della spedizione al totale;
    \item \textbf{Input:} il cliente preme sul pulsante per pagare;
    \item \textbf{Postcondizione:} il pagamento è stato eseguito e l'ordine viene emesso. Vengono quindi rimosse dal magazzino le merci acquistate;
    \item \textbf{Scenario principale:}
    \begin{itemize}
        \item il cliente premendo sul pulsante per eseguire il pagamento viene rimandato alla piattaforma esterna;
        \item esegue il pagamento interagendo con il servizio esterno;
        \item in caso di pagamento riuscito l'ordine viene emesso e le merci acquistate vengono rimosse dal magazzino (UC8.6);
        \item in caso di pagamento fallito viene visualizzato un errore (UC8.5) e si offre la possibilità di riprovare premendo di nuovo sul tasto per eseguire il pagamento.
    \end{itemize}
\end{itemize}

\subsubsection{UC8.5 - Visualizzazione errore pagamento}
\label{UC8.5}
\begin{itemize}
    \item \textbf{Attore primario:} sito EmporioLambda;
    \item \textbf{Descrizione:} scenario per la visualizzazione di un errore dovuto al fallimento di un pagamento in fase di checkout dell'ordine;
    \item \textbf{Precondizione:} il cliente sta eseguendo il checkout dell'ordine;
    \item \textbf{Input:} il gestore esterno dei pagamenti restituisce lo stato di fallimento del pagamento;
    \item \textbf{Postcondizione:} viene visualizzato a video un messaggio che indica il fallimento del pagamento e si ritorna in condizioni di poter ritentare il pagamento.
\end{itemize}

\subsubsection{UC8.6 - Rimozione merci dal magazzino}
\label{UC8.6}
\begin{itemize}
    \item \textbf{Attore primario:} sito EmporioLambda;
    \item \textbf{Descrizione:} scenario per la rimozione delle merci acquistate dal magazzino;
    \item \textbf{Precondizione:} il cliente esegue sta eseguendo il checkout dell'ordine;
    \item \textbf{Input:} il gestore dei pagamenti esterno indica un successo del pagamento;
    \item \textbf{Postcondizione:} le merci acquistate vengono rimosse dal magazzino secondo i quantitativi specificati nell'ordine.
\end{itemize}

\subsection{UC9 - Login cliente}
\label{UC9}
\begin{itemize}
    \item \textbf{Attore primario:} cliente non autenticato;
    \item \textbf{Attore secondario:} gestore delle credenziali esterno;
    \item \textbf{Descrizione:} caso d'uso per l'autenticazione del cliente;
    \item \textbf{Precondizione:} il cliente non si è ancora autenticato nell'applicazione;
    \item \textbf{Input:} il cliente inserisce invia i dati per il login;
    \item \textbf{Postcondizione:} il cliente è autenticato;
    \item \textbf{Scenario principale:}
    \begin{enumerate}
        \item il cliente inserisce i dati di login (es. username e password);
        \item invia i dati inseriti;
        \item i dati vengono verificati dal gestire esterno;
        \item se i dati sono corretti e identificato un profilo utente il cliente è autenticato con questo profilo.
    \end{enumerate}
    \item \textbf{Estensioni:}
    \begin{itemize}
        \item il gestore delle credenziali restituisce un errore indicante che i dati inseriti sono errati;
        \item viene quindi chiesto di reinserire le credenziali, contestualmente si visualizza un messaggio di errore.
    \end{itemize}
\end{itemize}

\subsection{UC10 - Amministrazione account}
\label{UC10}
\begin{figure}[H]
    \centering
    \includegraphics[width=\linewidth]{res/images/UC/UC10.png}
    \caption{Diagramma che descrive UC10 - Amministrazione account} 
\end{figure}
\begin{itemize}
    \item \textbf{Attore Primario:} cliente autenticato;
    \item \textbf{Descrizione:} caso d'uso per la gestione del profilo utente di un cliente;
    \item \textbf{Precondizione:} il cliente ha eseguito il login (\hyperref[UC9]{UC9}) e si trova sulla pagina per la gestione del profilo;
    \item \textbf{Input:} il cliente avvia un'operazione di gestione profilo;
    \item \textbf{Postcondizione:} il cliente ha modificato il suo profilo o richiesto la modifica al venditore;
\end{itemize}

\subsubsection{UC10.1 - Inserimento indirizzo}
\label{UC10.1}
\begin{itemize}
    \item \textbf{Attore primario:} cliente autenticato;
    \item \textbf{Descrizione:} scenario per l'inserimento di un nuovo indirizzo (utilizzabile per spedizione e/o fatturazione in fase di checkout);
    \item \textbf{Precondizione:} il cliente ha eseguito il login (\hyperref[UC9]{UC9}) e si trova sulla pagina per la gestione del profilo;
    \item \textbf{Input:} il cliente seleziona l'opzione per inserire un nuovo indirizzo;
    \item \textbf{Postcondizione:} nel sistema esiste l'indirizzo inserito dal cliente;
    \item \textbf{Scenario principale:}
    \begin{itemize}
        \item il cliente compila il form per l'inserimento dell'indirizzo;
        \item invia quindi i dati.
    \end{itemize}
\end{itemize}

\subsubsection{UC10.2 - Eliminazione indirizzo}
\label{UC10.2}
\begin{itemize}
    \item \textbf{Attore primario:} cliente autenticato;
    \item \textbf{Descrizione:} scenario per l'eliminazione di un indirizzo precedentemente inserito;
    \item \textbf{Precondizione:} il cliente ha eseguito il login (\hyperref[UC9]{UC9}) e si trova sulla pagina per la gestione del profilo;
    \item \textbf{Input:} il cliente seleziona l'opzione per eliminare un indirizzo;
    \item \textbf{Postcondizione:} nel sistema non esiste più l'indirizzo rimosso.
\end{itemize}

\subsubsection{UC10.3 - Richiesta cancellazione account}
\label{UC10.3}
\begin{itemize}
    \item \textbf{Attore primario:} cliente autenticato;
    \item \textbf{Descrizione:} scenario per la richiesta di cancellazione del profilo cliente al venditore;
    \item \textbf{Precondizione:} il cliente ha eseguito il login (\hyperref[UC9]{UC9}) e si trova sulla pagina per la gestione del profilo;
    \item \textbf{Input:} il cliente sceglie l'opzione per aprire un ticket di assistenza e richiedere la cancellazione dell'account;
    \item \textbf{Postcondizione:} è stato aperto un ticket di assistenza nella piattaforma, il venditore quindi provvederà a gestirlo opportunamente;
\end{itemize}

\subsubsection{UC10.4 - Modifica email}
\label{UC10.4}
\begin{itemize}
    \item \textbf{Attore primario:} cliente autenticato;
    \item \textbf{Descrizione:} scenario per la modifica dell'email (username) dell'utente;
    \item \textbf{Precondizione:} il cliente ha eseguito il login (\hyperref[UC9]{UC9}) e si trova sulla pagina per la gestione del profilo;
    \item \textbf{Input:} il cliente sceglie l'opzione per modificare l'email che lo identifica univocamente all'interno del sito;
    \item \textbf{Postcondizione:} l'email è stata modificata con quella nuova inserita;
    \item \textbf{Scenario principale:}
    \begin{itemize}
        \item il cliente inserisce l'email nel campo apposito;
        \item il cliente reinserisce la stessa email in un altro campo per la verifica;
        \item invia quindi i dati inseriti.
    \end{itemize}
    \item \textbf{Estensioni:}
    \begin{itemize}
        \item nel caso in cui le due email inserite non corrispondano si chiede di reinserirle, visualizzando l'errore.
    \end{itemize}
\end{itemize}

\subsubsection{UC10.5 - Modifica password}
\label{UC10.5}
\begin{itemize}
    \item \textbf{Attore primario:} cliente autenticato;
    \item \textbf{Descrizione:} scenario per la modifica della password del cliente;
    \item \textbf{Precondizione:} il cliente ha eseguito il login (\hyperref[UC9]{UC9}) e si trova sulla pagina per la gestione del profilo;
    \item \textbf{Input:} il cliente seleziona l'opzione per modificare la password di accesso;
    \item \textbf{Postcondizione:} la password di accesso al sistema per quel cliente è stata modificata;
    \item \textbf{Scenario principale:}
    \begin{itemize}
        \item il cliente inserisce la vecchia password;
        \item il cliente inserisce la nuova password;
        \item il cliente reinserisce la nuova password;
        \item invia quindi i dati inseriti.
    \end{itemize}
    \item \textbf{Estensioni:}
    \begin{itemize}
        \item nel caso in cui le due password inserite non corrispondano o la vecchia password sia errata
        si chiede di ripetere il procedimento, visualizzando l'errore e quindi palesando il fallimento, specificando il motivo.
    \end{itemize}
\end{itemize}

\subsection{UC11 - Gestione ordini utente}
\label{UC11}
\begin{itemize}
    \item \textbf{Attore Primario:} 
    \item \textbf{Descrizione:}
    \item \textbf{Precondizione:}
    \item \textbf{Input:}
    \item \textbf{Postcondizione:}
    \item \textbf{Scenario principale:}
    \begin{enumerate}
        \item 
    \end{enumerate}
    \item \textbf{Inclusioni:}
    \begin{itemize}
        \item
    \end{itemize}
\end{itemize}

\subsection{UC12 - Logout cliente}
\label{UC12}
\begin{itemize}
    \item \textbf{Attore Primario:} 
    \item \textbf{Descrizione:}
    \item \textbf{Precondizione:}
    \item \textbf{Input:}
    \item \textbf{Postcondizione:}
    \item \textbf{Scenario principale:}
    \begin{enumerate}
        \item 
    \end{enumerate}
    \item \textbf{Inclusioni:}
    \begin{itemize}
        \item
    \end{itemize}
\end{itemize}