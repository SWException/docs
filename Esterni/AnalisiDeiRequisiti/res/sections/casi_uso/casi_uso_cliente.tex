\subsection{UC1 - Visualizzazione contenuti informativi}
\label{UC1}
\begin{itemize}
    \item \textbf{Attore Primario:} cliente generico;
    \item \textbf{Descrizione:} il cliente vuole accedere alle informazioni riguardanti la piattaforma e il venditore;
    \item \textbf{Precondizione:} il cliente non sta visualizzando alcuna informazione della piattaforma;
    \item \textbf{Postcondizione:} il cliente visualizza le informazioni relative alla pagina scelta;
    \item \textbf{Scenario principale:}
    \begin{enumerate}
        \item l'utente si collega alla piattaforma e naviga fino alla pagina scelta;
    \end{enumerate}
\end{itemize}

\subsection{UC2 - Scelta categoria}
\label{UC2}
\begin{itemize}
    \item \textbf{Attore Primario:} cliente generico;
    \item \textbf{Descrizione:} il cliente vuole visualizzare tutti i prodotti riguardanti una determinata categoria;
    \item \textbf{Precondizione:} il cliente si trova in una pagina dove la scelta di categoria è permessa;
    \item \textbf{Input:} seleziona una categoria;
    \item \textbf{Postcondizione:} il cliente visualizza tutti i prodotti relativi alla categoria selezionata;
    \item \textbf{Scenario principale:}
    \begin{enumerate}
        \item il cliente entra in una pagina dove è presente la scelta di una categoria;
        \item seleziona una categoria;
        \item viene reindirizzato ad una pagina di visualizzazione prodotto (PLP) e visualizza i prodotti desiderati (\hyperref[UC4]{UC4});
    \end{enumerate}
\end{itemize}

\subsection{UC3 - Ricerca}
\label{UC3}
\begin{itemize}
    \item \textbf{Attore Primario:} cliente generico;
    \item \textbf{Descrizione:} il cliente vuole visualizzare i prodotti che contengono nel nome prodotto una determinata parola
    \item \textbf{Precondizione:} il cliente si trova in una pagina dove funzione di ricerca è permessa;
    \item \textbf{Input:} una stringa;
    \item \textbf{Postcondizione:} il cliente visualizza tutti i prodotti che contengono la determinata stringa nel nome;
    \item \textbf{Scenario principale:}
    \begin{enumerate}
        \item il cliente entra in una pagina dove è permessa la ricerca di un prodotto, inserisce la parola da cercare e viene reindirizzato ad una pagina di visualizzazione prodotto (PLP) e visualizza i prodotti desiderati (\hyperref[UC4]{UC4});
    \end{enumerate}
\end{itemize}

\subsection{UC4 - Visualizzazione lista prodotti (PLP)}
\label{UC4}
\begin{itemize}
    \item \textbf{Attore Primario:} cliente generico;
    \item \textbf{Descrizione:} questa pagina visualizza tutti i prodotti che corrispondono ad una categoria o che corrispondono ad una parola cercata;
    \item \textbf{Precondizione:} il cliente sceglie uno dei due modi per accedere alla PLP (\hyperref[UC2]{UC2} e \hyperref[UC3]{UC3});
    \item \textbf{Postcondizione:} il cliente visualizza i prodotti che corrispondono alla scelta;
    \item \textbf{Scenario principale:}
    \begin{enumerate}
        \item il cliente può visualizzare tutti i prodotti che corrispondono alle politiche di visualizzazione;
    \end{enumerate}
    \item \textbf{Estensioni:}
    \begin{itemize}
        \item la pagina visualizza una PLP vuota perché nessun prodotto corrisponde alle politiche di visualizzazione;
    \end{itemize}
\end{itemize}

\subsubsection{UC4.1 - Filtri}
\label{UC4.1}
\begin{itemize}
    \item \textbf{Attore Primario:} cliente generico;
    \item \textbf{Descrizione:} il cliente può inserire ulteriori filtri per raffinare la visualizzazione dei prodotti;
    \item \textbf{Precondizione:} il cliente si deve trovare in una PLP (UC4);
    \item \textbf{Postcondizione:} il cliente visualizza i prodotti con i filtri aggiornati;
    \item \textbf{Scenario principale:}
    \begin{enumerate}
        \item il cliente seleziona il filtro;
        \item il cliente modifica il valore del filtro;
        \item il cliente visualizza i prodotti secondo le nuove politiche; 
    \end{enumerate}
    \item \textbf{Specializzazioni: }
    \begin{itemize}
        \item il cliente filtra i prodotti per prezzo (\hyperref[UC4.1.1]{UC4.1.1});
        \item il cliente filtra i prodotti per categoria (\hyperref[UC4.1.2]{UC4.1.2});
    \end{itemize}
\end{itemize}

\paragraph{UC4.1.1 - Inserimento filtro prezzo}
\label{UC4.1.1}
\begin{itemize}
    \item \textbf{Attore Primario:} cliente generico;
    \item \textbf{Descrizione:} questa funzionalità permette al cliente di filtrare i prodotti visualizzati per prezzo;
    \item \textbf{Precondizione:} il cliente si deve trovare in una PLP (\hyperref[UC4]{UC4});
    \item \textbf{Postcondizione:} il cliente visualizza i prodotti che corrispondono alla scelta
    \item \textbf{Scenario principale:}
    \begin{enumerate}
        \item il cliente inserisce l'intervallo di prezzo desiderato;
        \item il cliente visualizza i prodotti che corrispondono alla scelta;
    \end{enumerate}
    \item \textbf{Estensioni:}
    \begin{itemize}
        \item la pagina visualizza una PLP vuota perché nessun prodotto corrisponde alle politiche di visualizzazione;
    \end{itemize}
\end{itemize}

\paragraph{UC4.1.2 - Inserimento filtro categoria}
\label{UC4.1.2}
\begin{itemize}
    \item \textbf{Attore Primario:} cliente generico;
    \item \textbf{Descrizione:} questa pagina visualizza tutti i prodotti che corrispondono ad una categoria o che corrispondono ad una parola cercata;
    \item \textbf{Precondizione:} il cliente sceglie uno dei due modi per accedere alla PLP (\hyperref[UC2]{UC2} e \hyperref[UC3]{UC3});
    \item \textbf{Postcondizione:} il cliente visualizza i prodotti che corrispondono alla scelta
    \item \textbf{Scenario principale:}
    \begin{enumerate}
        \item il cliente inserisce una nuova categoria o modifica una nuova esistente;
        \item il cliente visualizza i prodotti che corrispondono alla scelta;
    \end{enumerate}
    \item \textbf{Estensioni:}
    \begin{itemize}
        \item la pagina visualizza una PLP vuota perché nessun prodotto corrisponde alle politiche di visualizzazione;
    \end{itemize}
\end{itemize}

\subsection{UC5 - Apertura dettagli prodotto}
\label{UC5}
\begin{itemize}
    \item \textbf{Attore Primario:} cliente generico;
    \item \textbf{Descrizione:} questa pagina visualizza tutti i dettagli del prodotto;
    \item \textbf{Precondizione:} il cliente deve trovarsi in una pagina dove la visualizzazione dei dettagli del prodotto sia permessa;
    \item \textbf{Postcondizione:} il cliente visualizza tutti i dettagli del prodotto;
    \item \textbf{Scenario principale:}
    \begin{enumerate}
        \item il cliente visualizza tutti i dettagli del prodotto;
    \end{enumerate}
\end{itemize}

\subsection{UC6 - Aggiunta prodotto al carrello}
\label{UC6}
\begin{itemize}
    \item \textbf{Attore Primario:} Cliente generico;
    \item \textbf{Descrizione:} questa azione serve ad aggiungere un prodotto in vendita nel sito
                                all'interno del carrello personale utente, specificando in quale quantità;
    \item \textbf{Precondizione:} il cliente si trova nella PDP di un prodotto;
    \item \textbf{Input:} il cliente clicca sul pulsante "aggiungi al carrello" dopo aver selezionato la quantità (di default a 1)
    \item \textbf{Postcondizione:} l'oggetto è stato inserito nel carrello nella quantità desiderata (se disponibile in magazzino)
    \item \textbf{Scenario principale:}
    \begin{enumerate}
        \item il cliente seleziona la quantità desiderata o lascia quella di default;
        \item il cliente clicca sul pulsante "aggiungi al carrello" o simile;
    \end{enumerate}
    \item \textbf{Inclusioni:}
    \begin{itemize}
        \item viene eseguito un controllo sulla quantità disponibile di quel determinato prodotto;
    \end{itemize}
    \item \textbf{Estensioni:}
    \begin{itemize}
        \item in caso la quantità disponibile sia insufficiente a soddisfare la richiesta del cliente viene visualizzato un errore;
    \end{itemize}
\end{itemize}

\subsection{UC7 - Gestione carrello}
\label{UC7}
\begin{itemize}
    \item \textbf{Attore Primario:} cliente generico;
    \item \textbf{Descrizione:} insieme di azioni atte alla gestione degli articoli già presenti nel carrello;
    \item \textbf{Precondizione:} il cliente sta navigando il sito web;
    \item \textbf{Input:} il cliente esegue un'operazione gestionale sul carrello;
    \item \textbf{Postcondizione:} il cliente ha eseguito operazioni gestionali sul carrello;
\end{itemize}

\subsubsection{UC7.1 - Visualizzazione articoli}
\label{UC7.1}
\begin{itemize}
    \item \textbf{Attore Primario:} cliente generico;
    \item \textbf{Descrizione:} scenario per la visualizzazione degli articoli nel carrello;
    \item \textbf{Precondizione:} il cliente si trova su una qualunque pagina del sito web;
    \item \textbf{Input:} il cliente preme su un pulsante dedicato all'accesso al carrello;
    \item \textbf{Postcondizione:} il cliente visualizza un resoconto di tutti i prodotti inseriti nel carrello durante
                                   la sessione di utilizzo corrente (se non loggato), altrimenti il carrello mantiene la consistenza
                                   su ogni dispositivo fino al checkout.
\end{itemize}

\subsubsection{UC7.2 - Variazione quantità articolo}
\label{UC7.2}
\begin{itemize}
    \item \textbf{Attore primario:} cliente generico;
    \item \textbf{Descrizione:} scenario per la variazione della quantità di un singolo articolo presente nel carrello;
    \item \textbf{Precondizione:} il cliente ha eseguito \hyperref[UC7.1]{UC7.1};
    \item \textbf{Input:} il cliente inserisce la nuova quantità desiderata per quel determinato articolo;
    \item \textbf{Postcondizione:} se la quantità desiderata è disponibile, viene aggiornato l'articolo nel carrello;
    \item \textbf{Scenario principale:}
    \begin{itemize}
        \item il cliente seleziona la nuova quantità desiderata;
        \item il cliente conferma la scelta
    \end{itemize}
    \item \textbf{Inclusioni:}
    \begin{itemize}
        \item viene eseguito un controllo sulla quantità disponibile di quel determinato prodotto;
    \end{itemize}
    \item \textbf{Estensioni:}
    \begin{itemize}
        \item in caso la quantità disponibile sia insufficiente a soddisfare la richiesta del cliente viene visualizzato un errore;
    \end{itemize}
\end{itemize}

\subsubsection{UC7.3 - Rimozione articolo}
\label{UC7.3}
\begin{itemize}
    \item \textbf{Attore Primario:} cliente generico;
    \item \textbf{Descrizione:} scenario per la rimozione di un articolo dal carrello;
    \item \textbf{Precondizione:} il cliente ha eseguito \hyperref[UC7.1]{UC7.1};
    \item \textbf{Input:} il cliente preme su un pulsante dedicato alla rimozione di un determinato articolo dal carrello;
    \item \textbf{Postcondizione:} l'articolo selezionato non è più presente nel carrello.
\end{itemize}

\subsection{UC8 - Checkout}
\label{UC8}
\begin{itemize}
    \item \textbf{Attore Primario:} cliente generico;
    \item \textbf{Descrizione:} caso d'uso per l'acquisto dei prodotti inseriti nel carrello;
    \item \textbf{Precondizione:} il cliente ha eseguito \hyperref[UC7.1]{UC7.1} e il suo carrello contiene almeno un articolo;
    \item \textbf{Input:} il cliente preme sul pulsante per avviare il checkout;
    \item \textbf{Postcondizione:} viene emesso un ordine contenente gli articoli precedentemente inseriti nel carrello. Vengono quindi rimossi dal magazzino;
    \item \textbf{Scenario principale:}
    \begin{enumerate}
        \item il cliente preme sul pulsante per avviare il checkout
        \item inserisce gli indirizzi di spedizione o fatturazione (UC 8.1 - 8.2)
        \item vengono aggiunti i costi della spedizione all'importo totale(UC 8.3)
        \item si effettua il pagamento (UC 8.4)
        \item l'ordine è stato emesso e contrassegnato come pagato
    \end{enumerate}
    \item \textbf{Estensioni:}
    \begin{itemize}
        \item in caso di fallimento del pagamento l'ordine non viene emesso, è necessario quindi riprovare il pagamento e viene visualizzato un errore (UC 8.6);
        \item l'utente decide di annullare il processo di checkout premendo l'apposito pulsante (UC 8.5);
    \end{itemize}
\end{itemize}

\subsubsection{UC8.1 - Inserimento indirizzo spedizione}
\label{UC8.1}
\begin{itemize}
    \item \textbf{Attore Primario:} cliente generico;
    \item \textbf{Descrizione:} scenario per l'inserimento dell'indirizzo di spedizione in fase di checkout;
    \item \textbf{Precondizione:} il cliente ha avviato il checkout;
    \item \textbf{Input:} il cliente inserisce i dati tramite l'apposito form oppure sceglie tra quelli personali già presenti (solo se autenticato);
    \item \textbf{Postcondizione:} l'utente procede con la fase successiva del checkout;
    \item \textbf{Scenario principale:}
    \begin{itemize}
        \item l'utente compila il form con i dati dell'indirizzo oppure ne sceglie uno tra quelli salvati (se autenticato)
    \end{itemize}
\end{itemize}

\subsubsection{UC8.2 - Inserimento indirizzo fatturazione}
\label{UC8.1}
\begin{itemize}
    \item \textbf{Attore Primario:} cliente generico;
    \item \textbf{Descrizione:} scenario per l'inserimento dell'indirizzo di fatturazione in fase di checkout;
    \item \textbf{Precondizione:} il cliente ha inserito l'indirizzo di spedizione;
    \item \textbf{Input:} il cliente inserisce i dati tramite l'apposito form oppure sceglie tra quelli personali già presenti (solo se autenticato);
    \item \textbf{Postcondizione:} l'utente procede con la fase successiva del checkout;
    \item \textbf{Scenario principale:}
    \begin{itemize}
        \item l'utente compila il form con i dati dell'indirizzo oppure ne sceglie uno tra quelli salvati (se autenticato)
    \end{itemize}
\end{itemize}


\subsection{UC9 - Login}
\label{UC9}
\begin{itemize}
    \item \textbf{Attore Primario:} 
    \item \textbf{Descrizione:}
    \item \textbf{Precondizione:}
    \item \textbf{Input:}
    \item \textbf{Postcondizione:}
    \item \textbf{Scenario principale:}
    \begin{enumerate}
        \item 
    \end{enumerate}
    \item \textbf{Inclusioni:}
    \begin{itemize}
        \item
    \end{itemize}
\end{itemize}

\subsection{UC10 - Amministrazione account}
\label{UC10}
\begin{itemize}
    \item \textbf{Attore Primario:} 
    \item \textbf{Descrizione:}
    \item \textbf{Precondizione:}
    \item \textbf{Input:}
    \item \textbf{Postcondizione:}
    \item \textbf{Scenario principale:}
    \begin{enumerate}
        \item 
    \end{enumerate}
    \item \textbf{Inclusioni:}
    \begin{itemize}
        \item
    \end{itemize}
\end{itemize}

\subsection{UC11 - Gestione ordini cliente}
\label{UC11}
\begin{itemize}
    \item \textbf{Attore Primario:} cliente autenticato;
    \item \textbf{Descrizione:} il cliente vuole accedere all'area di gestione degli ordini;
    \item \textbf{Precondizione:} il cliente è loggato nel sito;
    \item \textbf{Postcondizione:} il cliente risulta nella pagina dove è presente la lista di tutti gli ordini effettuati;
    \item \textbf{Scenario principale:}
    \begin{enumerate}
        \item il cliente entra nella pagina con la lista degli ordini;
        \item il cliente può annullare l'ordine \hyperref[UC11.1]{UC11.1};
        \item il cliente richiedere assistenza per un ordine \hyperref[UC11.2]{UC11.2};
        \item il cliente può richiedere un reso \hyperref[UC11.3]{UC11.3};
        \item il cliente può visualizzare il riepilogo di un determinato ordine \hyperref[UC11.4]{UC11.4};
    \end{enumerate}
\end{itemize}

\subsubsection{UC11.1 - Annullamento ordine}
\label{UC11.1}
\begin{itemize}
\item \textbf{Attore Primario:} cliente autenticato;
\item \textbf{Descrizione:} il cliente vuole annullare un ordine effettuato;
\item \textbf{Precondizione:} il cliente deve aver effettuato un ordine e trovarsi nella sezione gestione degli ordini;
\item \textbf{Postcondizione:} il cliente deve aver contatto il venditore per l'annullamento;
\item \textbf{Scenario principale:}
\begin{enumerate}
    \item cliente sceglie ordine da annullare;
    \item cliente viene portato al form di contatto;
\end{enumerate}
\item \textbf{Inclusioni:}
\begin{itemize}
    \item per annullare un ordine viene aperto il form di contatto che permette al cliente di contattare il venditore \hyperref[UC13]{UC13};
\end{itemize}
\end{itemize}

\subsubsection{UC11.2 - Assistenza cliente}
\label{UC11.2}
\begin{itemize}
\item \textbf{Attore Primario:} cliente autenticato;
\item \textbf{Descrizione:} il cliente vuole contattare il venditore per un problema (errore indirizzo, domande, etc.);
\item \textbf{Precondizione:} il cliente deve trovarsi nella pagina della gestione degli ordini e quindi aver effettuato l'ordine;
\item \textbf{Postcondizione:}il cliente deve aver contatto il venditore per richiedere assistenza;
\item \textbf{Scenario principale:}
\begin{enumerate}
    \item cliente decide di contattare l'assistenza;
    \item cliente viene portato al form di contatto;
\end{enumerate}
\item \textbf{Inclusioni:}
\begin{itemize}
    \item per l'assistenza su un ordine viene aperto il form di contatto che permette al cliente di contattare il venditore \hyperref[UC13]{UC13};
\end{itemize}
\end{itemize}

\subsubsection{UC11.3 - Richiesta reso}
\label{UC11.3}
\begin{itemize}
\item \textbf{Attore Primario:} cliente autenticato;
\item \textbf{Descrizione:} il cliente vuole effettuare un reso di un ordine ricevuto;
\item \textbf{Precondizione:} il cliente deve aver effettuato un ordine, averlo ricevuto, ed essere nella pagina di gestione degli ordini
\item \textbf{Postcondizione:} il cliente deve aver contatto il venditore per il reso;
\item \textbf{Scenario principale:}
\begin{enumerate}
    \item cliente decide di effettuare un reso;
    \item cliente viene portato al form di contatto;
\end{enumerate}
\item \textbf{Inclusioni:}
\begin{itemize}
    \item per il reso di un ordine viene aperto il form di contatto che permette al cliente di contattare il venditore \hyperref[UC13]{UC13};
\end{itemize}
\end{itemize}

\subsubsection{UC11.4 - Riepilogo ordine}
\label{UC11.4}
\begin{itemize}
\item \textbf{Attore Primario:} cliente autenticato;
\item \textbf{Descrizione:} il cliente vuole visualizzare il riepilogo dell'ordine effettuato;
\item \textbf{Precondizione:} il cliente deve aver effettuato l'ordine e trovarsi nella pagina di gestione degli ordini;
\item \textbf{Postcondizione:} il cliente visualizza il riepilogo;
\item \textbf{Scenario principale:}
\begin{enumerate}
    \item cliente decide di aprire il riepilogo dell'ordine;
    \item si apre la pagina dove è presentato il riepilogo;
\end{enumerate}
\end{itemize}

\subsection{UC12 - Logout cliente}
\label{UC12}
\begin{itemize}
    \item \textbf{Attore Primario:} cliente autenticato;
    \item \textbf{Descrizione:} il cliente vuole uscire dal sito;
    \item \textbf{Precondizione:} il cliente è loggato al sito;
    \item \textbf{Postcondizione:} il cliente risulta non autenticato nel sito;
    \item \textbf{Scenario principale:}
    \begin{enumerate}
        \item il cliente effettua il logout;
        \item il cliente si ritrova nella pagina principale del sito;
    \end{enumerate}
\end{itemize}

\subsection{UC13 - Form di contatto cliente} 
\label{UC13}
\begin{itemize}
    \item \textbf{Attore Primario:} cliente autenticato;
    \item \textbf{Descrizione:} il cliente vuole/deve contattare il venditore attraverso il form di contatto;
    \item \textbf{Precondizione:} il cliente deve trovarsi nella gestione degli ordini \hyperref[UC11]{UC11};
    \item \textbf{Postcondizione:} il cliente riesce a contattare il venditore;
    \item \textbf{Scenario principale:}
    \begin{enumerate}
        \item il cliente ha aperto il form di contatto;
        \item compila il form;
        \item il cliente contatta con successo il venditore;
    \end{enumerate}
\end{itemize}