\section{Casi d'uso}
\label{_casi_uso}
\resetCUC

%1
\stepUserCase
\subsection{\valueUserCase - Visualizzazione informazioni azienda}
\labelUserCase
\begin{figure}[H]
    \centering
    \includegraphics[width=30em]{res/images/UC/\valueUC.jpg}
    \caption{Diagramma \valueUC}
\end{figure}
\begin{itemize}
    \item \textbf{Attore primario:} cliente generico;
    \item \textbf{Descrizione:} il cliente apre il sito nella pagina principale;
    \item \textbf{Precondizione:} nessuna;
    \item \textbf{Input:} il cliente accede all'e-commerce tramite link diretto;
    \item \textbf{Postcondizione:} il cliente si trova nella pagina principale;
    \item \textbf{Scenario principale:} il cliente visualizza le informazioni sull'azienda venditrice (partita iva, indirizzo, ragione sociale, numero di telefono, etc.) tra cui l'email cliccabile.
\end{itemize}


%2
\stepUserCase
\subsection{\valueUserCase - Scelta categoria}
\labelUserCase
\begin{itemize}
    \item \textbf{Attore primario:} cliente generico;
    \item \textbf{Descrizione:} il cliente vuole visualizzare tutti i prodotti riguardanti una determinata categoria;
    \item \textbf{Precondizione:} il cliente si trova in una pagina dove la scelta di categoria è permessa;
    \item \textbf{Input:} seleziona una categoria;
    \item \textbf{Postcondizione:} il cliente visualizza tutti i prodotti relativi alla categoria selezionata;
    \item \textbf{Scenario principale:}
          \begin{enumerate}
              \item il cliente entra in una pagina dove è presente la scelta di una categoria;
              \item seleziona una categoria;
              \item viene reindirizzato ad una pagina di visualizzazione prodotto (PLP) e visualizza i prodotti desiderati (\hyperref[UC4]{UC4}).
          \end{enumerate}
\end{itemize}

%3
\stepUserCase
\subsection{\valueUserCase - Ricerca}
\labelUserCase
\begin{itemize}
    \item \textbf{Attore primario:} cliente generico;
    \item \textbf{Descrizione:} il cliente vuole visualizzare i prodotti che contengono nel nome prodotto una determinata parola;
    \item \textbf{Precondizione:} il cliente si trova in una pagina dove funzione di ricerca è permessa;
    \item \textbf{Input:} una stringa;
    \item \textbf{Postcondizione:} il cliente visualizza tutti i prodotti che contengono la determinata stringa nel nome;
    \item \textbf{Scenario principale:} il cliente entra in una pagina dove è permessa la ricerca di un prodotto, inserisce la parola da cercare e viene reindirizzato ad una pagina di visualizzazione prodotto (PLP) e visualizza i prodotti desiderati (\hyperref[UC4]{UC4}).
\end{itemize}

%4
\stepUserCase
\subsection{\valueUserCase - Visualizzazione lista prodotti (PLP)}
\labelUserCase
\begin{figure}[H]
    \centering
    \includegraphics[width=30em]{res/images/UC/\valueUC.jpg}
    \caption{Diagramma \valueUC}
\end{figure}
\begin{itemize}
    \item \textbf{Attore primario:} cliente generico;
    \item \textbf{Descrizione:} questa pagina visualizza tutti i prodotti che corrispondono ad una categoria o che corrispondono ad una parola cercata;
    \item \textbf{Precondizione:} il cliente sceglie uno dei due modi per accedere alla PLP (\hyperref[UC2]{UC2} e \hyperref[UC3]{UC3});
    \item \textbf{Postcondizione:} il cliente visualizza i prodotti che corrispondono alla scelta;
    \item \textbf{Scenario principale:}
          \begin{enumerate}
              \item il cliente può visualizzare tutti i prodotti che corrispondono alle politiche di visualizzazione;
              \item per ogni prodotto vengono visualizzati codice identificativo, titolo, foto principale e prezzo.
          \end{enumerate}
    \item \textbf{Estensioni:}
          \begin{itemize}
              \item la pagina visualizza una PLP vuota perché nessun prodotto corrisponde alle politiche di visualizzazione.
          \end{itemize}
\end{itemize}

%4.1
\stepsubUserCase
\subsubsection{\valuesubUserCase - Visualizzazione singolo prodotto lista}
\labelsubUserCase
\begin{figure}[H]
    \centering
    \includegraphics[width=30em]{res/images/UC/\valuesubUC.jpg}
    \caption{Diagramma \valuesubUC}
\end{figure}
\begin{itemize}
    \item \textbf{Attore primario:} cliente generico;
    \item \textbf{Descrizione:} il cliente visualizza il singolo prodotto all'interno della lista;
    \item \textbf{Precondizione:} il cliente deve trovarsi nella PLP;
    \item \textbf{Postcondizione:} il cliente visualizza caratteristiche del prodotto;
    \item \textbf{Scenario principale:} per ogni articolo il cliente visualizza titolo, foto principale e prezzo.
\end{itemize}


%5
\stepUserCase
\subsubsection{\valueUserCase - Filtro prezzo}
\labelUserCase
\begin{itemize}
    \item \textbf{Attore primario:} cliente generico;
    \item \textbf{Descrizione:} questa funzionalità permette al cliente di filtrare i prodotti visualizzati per prezzo;
    \item \textbf{Precondizione:} il cliente si deve trovare in una PLP (\hyperref[UC4]{UC4});
    \item \textbf{Postcondizione:} il cliente visualizza i prodotti che corrispondono alla scelta;
    \item \textbf{Scenario principale:}
          \begin{enumerate}
              \item il cliente inserisce l'intervallo di prezzo desiderato;
              \item il cliente visualizza i prodotti che corrispondono alla scelta;
          \end{enumerate}
    \item \textbf{Estensioni:}
          \begin{itemize}
              \item la pagina visualizza una PLP vuota perché nessun prodotto corrisponde alle politiche di visualizzazione.
          \end{itemize}
\end{itemize}

%6
\stepUserCase
\subsection{\valueUserCase - Ordinamento prodotti per prezzo}
\labelUserCase
\begin{figure}[H]
    \centering
    \includegraphics[width=30em]{res/images/UC/\valueUC.jpg}
    \caption{Diagramma \valueUC}
\end{figure}
\begin{itemize}
    \item \textbf{Attore primario:} cliente generico;
    \item \textbf{Descrizione:} il cliente può ordinare i prodotti per prezzo crescente o decrescente;
    \item \textbf{Precondizione:} il cliente si deve trovare in una PLP (\hyperref[UC4]{UC4});
    \item \textbf{Postcondizione:} il cliente visualizza i prodotti ordinati;
    \item \textbf{Scenario principale:}
          \begin{enumerate}
              \item il cliente seleziona se ordinare per prezzo crescente (\hyperref[UC6.1]{UC6.1}) o decrescente (\hyperref[UC6.2]{UC6.2});
              \item il cliente visualizza la lista prodotti ordinata.
          \end{enumerate}
\end{itemize}

%6.1
\stepsubUserCase
\subsubsection{\valuesubUserCase - Ordinamento prezzo crescente}
\labelsubUserCase
\begin{itemize}
    \item \textbf{Attore primario:} cliente generico;
    \item \textbf{Descrizione:} il cliente vuole ordinare la lista per prezzo crescente;
    \item \textbf{Precondizione:} il cliente deve trovarsi nella PLP;
    \item \textbf{Postcondizione:} il cliente visualizza i prodotti in ordine crescente;
    \item \textbf{Scenario principale:}
          \begin{enumerate}
              \item il cliente seleziona l'ordinamento per prezzo crescente;
              \item il cliente visualizza i prodotti in ordine crescente;
          \end{enumerate}
\end{itemize}

%6.2
\stepsubUserCase
\subsubsection{\valuesubUserCase - Ordinamento prezzo decrescente}
\labelsubUserCase
\begin{itemize}
    \item \textbf{Attore primario:} cliente generico;
    \item \textbf{Descrizione:} il cliente vuole ordinare la lista per prezzo decrescente;
    \item \textbf{Precondizione:} il cliente deve trovarsi nella PLP;
    \item \textbf{Postcondizione:} il cliente visualizza i prodotti in ordine decrescente;
    \item \textbf{Scenario principale:}
          \begin{enumerate}
              \item il cliente seleziona l'ordinamento per prezzo decrescente;
              \item il cliente visualizza i prodotti in ordine decrescente;
          \end{enumerate}
\end{itemize}

%7
\stepUserCase
\subsection{\valueUserCase - Visualizzazione pagina dettagli prodotto (PDP)}
\labelUserCase
\begin{figure}[H]
    \centering
    \includegraphics[width=30em]{res/images/UC/\valueUC.jpg}
    \caption{Diagramma \valueUC}
\end{figure}8
\begin{itemize}
    \item \textbf{Attore primario:} cliente generico;
    \item \textbf{Descrizione:} questa pagina visualizza tutti i dettagli del prodotto;
    \item \textbf{Precondizione:} il cliente deve trovarsi in una pagina dove la visualizzazione dei dettagli del prodotto sia permessa;
    \item \textbf{Postcondizione:} il cliente visualizza tutti i dettagli del prodotto;
    \item \textbf{Scenario principale:}
          \begin{enumerate}
              \item il cliente visualizza il titolo, le foto, il prezzo e la descrizione del prodotto.
          \end{enumerate}
\end{itemize}


%8
\stepUserCase
\subsection{\valueUserCase - Aggiunta prodotto al carrello}
\labelUserCase
\begin{figure}[H]
    \centering
    \includegraphics[width=30em]{res/images/UC/\valueUC.jpg}
    \caption{Diagramma \valueUC}
\end{figure}
\begin{itemize}
    \item \textbf{Attore primario:} cliente generico;
    \item \textbf{Descrizione:} questa azione serve ad aggiungere un prodotto in vendita nel sito
          all'interno del carrello personale del cliente, specificando in quale quantità;
    \item \textbf{Precondizione:} il cliente si trova nella PDP di un prodotto (\hyperref[UC7]{UC7});
    \item \textbf{Input:} il cliente aggiunge al carrello dopo aver selezionato la quantità (di default a 1);
    \item \textbf{Postcondizione:} l'oggetto è stato inserito nel carrello nella quantità desiderata (se disponibile in magazzino);
    \item \textbf{Scenario principale:}
          \begin{enumerate}
              \item il cliente seleziona la quantità desiderata o lascia quella di default (1);
              \item il cliente clicca sul pulsante "aggiungi al carrello".
          \end{enumerate}
    \item \textbf{Inclusioni:}
          \begin{itemize}
              \item viene eseguito un controllo sulla quantità disponibile di quel determinato prodotto;
          \end{itemize}
    \item \textbf{Estensioni:}
          \begin{itemize}
              \item in caso la quantità disponibile sia insufficiente a soddisfare la richiesta del cliente viene visualizzato un errore.
          \end{itemize}
\end{itemize}

%9
\stepUserCase
\subsection{\valueUserCase - Gestione carrello}
\labelUserCase
\begin{figure}[H]
    \centering
    \includegraphics[width=30em]{res/images/UC/\valueUC.jpg}
    \caption{Diagramma \valueUC}
\end{figure}
\begin{itemize}
    \item \textbf{Attore primario:} cliente generico;
    \item \textbf{Descrizione:} insieme di azioni per la gestione degli articoli già presenti nel carrello;
    \item \textbf{Precondizione:} il cliente sta navigando il sito web;
    \item \textbf{Input:} il cliente esegue un'operazione gestionale sul carrello;
    \item \textbf{Postcondizione:} il cliente ha eseguito operazioni gestionali sul carrello;
    \item \textbf{Scenario principale:}
          \begin{enumerate}
              \item il cliente visualizza tutti gli articoli del carrello (\hyperref[UC9.1]{UC9.1});
              \item il cliente può variare la quantità di ogni articolo (\hyperref[UC9.2]{UC9.2});
              \item il cliente può rimuovere un articolo dal carrello (\hyperref[UC9.3]{UC9.3}).
          \end{enumerate}
    
\end{itemize}

%9.1
\stepsubUserCase
\subsubsection{\valuesubUserCase - Visualizzazione articoli carrello}
\labelsubUserCase
\begin{figure}[H]
    \centering
    \includegraphics[width=30em]{res/images/UC/\valuesubUC.jpg}
    \caption{Diagramma \valuesubUC}
\end{figure}
\begin{itemize}
    \item \textbf{Attore primario:} cliente generico;
    \item \textbf{Descrizione:} scenario per la visualizzazione degli articoli nel carrello;
    \item \textbf{Precondizione:} il cliente si trova su una qualunque pagina del sito web;
    \item \textbf{Input:} il cliente preme su un pulsante dedicato all'accesso al carrello;
    \item \textbf{Postcondizione:} il cliente visualizza un resoconto di tutti i prodotti inseriti nel carrello durante la sessione di utilizzo corrente (se non autenticato), altrimenti il carrello mantiene la consistenza su ogni dispositivo fino al checkout.
    \item \textbf{Scenario principale:} il cliente visualizza la lista dei prodotti con le loro caratteristiche.
\end{itemize}

%9.1.1
\stepsubsubUserCase
\paragraph{\valuesubsubUserCase - Visualizzazione singolo articolo carrello}
\labelsubsubUserCase
\begin{figure}[H]
    \centering
    \includegraphics[width=30em]{res/images/UC/\valuesubsubUC.jpg}
    \caption{Diagramma \valuesubsubUC}
\end{figure}
\begin{itemize}
    \item \textbf{Attore primario:} cliente generico;
    \item \textbf{Descrizione:} scenario per la visualizzazione singolo articolo nel carrello;
    \item \textbf{Precondizione:} il cliente si trova nel carrello;
    \item \textbf{Postcondizione:} il cliente visualizza le caratteristiche del singolo prodotto.
    \item \textbf{Scenario principale:} zil cliente visualizza per ogni articolo codice, titolo, foto, prezzo e quantità e totale parziale.
\end{itemize}


%9.2
\stepsubUserCase
\subsubsection{\valuesubUserCase - Variazione quantità articolo}
\labelsubUserCase
\begin{figure}[H]
    \centering
    \includegraphics[width=30em]{res/images/UC/\valuesubUC.jpg}
    \caption{Diagramma \valuesubUC}
\end{figure}
\begin{itemize}
    \item \textbf{Attore primario:} cliente generico;
    \item \textbf{Descrizione:} scenario per la variazione della quantità di un singolo articolo presente nel carrello;
    \item \textbf{Precondizione:} il cliente si trova nel carrello;
    \item \textbf{Input:} il cliente inserisce la nuova quantità desiderata per quel determinato articolo;
    \item \textbf{Postcondizione:} se la quantità desiderata è disponibile, viene aggiornato l'articolo nel carrello;
    \item \textbf{Scenario principale:}
          \begin{enumerate}
              \item il cliente seleziona la nuova quantità desiderata;
              \item il cliente conferma la scelta;
          \end{enumerate}
    \item \textbf{Inclusioni:}
          \begin{itemize}
              \item viene eseguito un controllo sulla quantità disponibile di quel determinato prodotto;
          \end{itemize}
    \item \textbf{Estensioni:}
          \begin{itemize}
              \item in caso la quantità disponibile sia insufficiente a soddisfare la richiesta del cliente viene visualizzato un errore.
          \end{itemize}
\end{itemize}

%9.3
\stepsubUserCase
\subsubsection{\valuesubUserCase - Rimozione articolo}
\labelsubUserCase
\begin{itemize}
    \item \textbf{Attore primario:} cliente generico;
    \item \textbf{Descrizione:} scenario per la rimozione di un articolo dal carrello;
    \item \textbf{Precondizione:} il cliente ha eseguito (\hyperref[UC9.1]{UC9.1});
    \item \textbf{Input:} il cliente preme su un pulsante dedicato alla rimozione di un determinato articolo dal carrello;
    \item \textbf{Postcondizione:} l'articolo selezionato non è più presente nel carrello.
\end{itemize}

%10
\stepUserCase
\subsection{\valueUserCase - Checkout}
\labelUserCase
\begin{figure}[H]
    \centering
    \includegraphics[width=30em]{res/images/UC/\valueUC.jpg}
    \caption{Diagramma \valueUC}
\end{figure}
\begin{itemize}
    \item \textbf{Attore primario:} cliente autenticato;
    \item \textbf{Descrizione:} caso d'uso per l'acquisto dei prodotti inseriti nel carrello;
    \item \textbf{Precondizione:} il cliente si trova nel carrello che contiene almeno un articolo;
    \item \textbf{Input:} il cliente preme sul pulsante per avviare il checkout;
    \item \textbf{Postcondizione:} viene emesso un ordine contenente gli articoli precedentemente inseriti nel carrello. Vengono quindi rimossi dal magazzino;
    \item \textbf{Scenario principale:}
          \begin{enumerate}
              \item il cliente preme sul pulsante per avviare il checkout;
              \item inserisce gli indirizzi di fatturazione e spedizione;
              \item vengono aggiunti i costi della spedizione all'importo totale, definiti come importo fisso;
              \item si effettua il pagamento;
              \item l'ordine è stato emesso e contrassegnato come pagato;
          \end{enumerate}
    \item \textbf{Estensioni:}
          \begin{itemize}
              \item in caso di fallimento del pagamento l'ordine non viene emesso, è necessario quindi riprovare il pagamento e viene visualizzato un errore;
              \item il cliente decide di annullare il processo di checkout premendo l'apposito pulsante.
          \end{itemize}
\end{itemize}

%10.1
\stepsubUserCase
\subsubsection{\valuesubUserCase - Inserimento indirizzo fatturazione}
\labelsubUserCase
\begin{figure}[H]
    \centering
    \includegraphics[width=30em]{res/images/UC/\valuesubUC.jpg}
    \caption{Diagramma \valuesubUC}
\end{figure}
\begin{itemize}
    \item \textbf{Attore primario:} cliente autenticato;
    \item \textbf{Descrizione:} scenario per l'inserimento dell'indirizzo di fatturazione in fase di checkout;
    \item \textbf{Precondizione:} il cliente ha avviato il processo di checkout;
    \item \textbf{Input:} il cliente inserisce i dati tramite l'apposito form oppure sceglie tra quelli personali già presenti;
    \item \textbf{Postcondizione:} il cliente procede con la fase successiva del checkout;
    \item \textbf{Scenario principale:}
          \begin{enumerate}
              \item il cliente ha 2 modi per inserire l'informazione:
                    \begin{itemize}
                        \item compilare il relativo form: nome, cognome, via/piazza, numero civico, codice avviamento postale, città, provincia, stato;
                        \item scegliere tra gli indirizzi salvati.
                    \end{itemize}
          \end{enumerate}
\end{itemize}

%10.1.1
\stepsubsubUserCase
\paragraph{\valuesubsubUserCase - Scelta tra indirizzi salvati}
\labelsubsubUserCase
\begin{itemize}
    \item \textbf{Attore primario:} cliente autenticato;
    \item \textbf{Descrizione:} il cliente può scegliere tra i vari indirizzi salvati;
    \item \textbf{Precondizione:} il cliente ha avviato il processo di checkout;
    \item \textbf{Input:} sceglie tra gli indirizzi personali già presenti;
    \item \textbf{Postcondizione:} il cliente ha scelto un indirizzo;
    \item \textbf{Scenario principale:}  il cliente seleziona dalla lista l'indirizzo di interesse.
\end{itemize}

%10.2
\stepsubUserCase
\subsubsection{\valuesubUserCase - Inserimento indirizzo spedizione}
\labelsubUserCase
\begin{figure}[H]
    \centering
    \includegraphics[width=30em]{res/images/UC/\valuesubUC.jpg}
    \caption{Diagramma \valuesubUC}
\end{figure}
\begin{itemize}
    \item \textbf{Attore primario:} cliente autenticato;
    \item \textbf{Descrizione:} scenario per l'inserimento dell'indirizzo di spedizione in fase di checkout;
    \item \textbf{Precondizione:} il cliente ha avviato il checkout;
    \item \textbf{Input:} il cliente inserisce i dati tramite l'apposito form oppure sceglie tra quelli personali già presenti;
    \item \textbf{Postcondizione:} il cliente procede con la fase successiva del checkout;
    \item \textbf{Scenario principale:}
          \begin{enumerate}
              \item il cliente ha 3 modi per inserire l'informazione:
                    \begin{itemize}
                        \item compilare il relativo form: nome, cognome, via/piazza, numero civico, codice avviamento postale, città, provincia, stato;
                        \item scegliere tra gli indirizzi salvati;
                        \item riutilizzare l'indirizzo di fatturazione precedentemente inserito.
                    \end{itemize}
          \end{enumerate}
\end{itemize}

%10.3
\stepsubUserCase
\subsubsection{\valuesubUserCase - Pagamento con servizio di terze parti}
\labelsubUserCase
\begin{itemize}
    \item \textbf{Attore primario:} cliente autenticato;
    \item \textbf{Attore secondario:} gestore dei pagamenti esterno;
    \item \textbf{Descrizione:} scenario per il pagamento del totale dell'ordine mediante un gestore di pagamenti esterno a EmporioLambda;
    \item \textbf{Precondizione:} sono stati eseguiti tutti i passi precedenti del checkout, ovvero sono stati applicati i costi della spedizione al totale;
    \item \textbf{Input:} il cliente preme sul pulsante per pagare;
    \item \textbf{Postcondizione:} il pagamento è stato eseguito e l'ordine viene emesso. Vengono quindi rimosse dal magazzino le merci acquistate;
    \item \textbf{Scenario principale:}
          \begin{enumerate}
              \item il cliente premendo sul pulsante per eseguire il pagamento viene rimandato alla piattaforma esterna;
              \item esegue il pagamento interagendo con il servizio esterno;
              \item in caso di pagamento riuscito l'ordine viene emesso e le merci acquistate vengono rimosse dal magazzino;
          \end{enumerate}
    \item \textbf{Estensioni:}
          \begin{itemize}
              \item in caso di pagamento fallito viene visualizzato un errore e si offre la possibilità di riprovare premendo di nuovo sul tasto per eseguire il pagamento.
          \end{itemize}
\end{itemize}

%11
\stepUserCase
\subsection{\valueUserCase - Login cliente}
\labelUserCase
\begin{figure}[H]
    \centering
    \includegraphics[width=30em]{res/images/UC/\valueUC.jpg}
    \caption{Diagramma \valueUC}
\end{figure}
\begin{itemize}
    \item \textbf{Attore primario:} cliente non autenticato;
    \item \textbf{Attore secondario:} gestore delle credenziali esterno;
    \item \textbf{Descrizione:} caso d'uso per l'autenticazione del cliente;
    \item \textbf{Precondizione:} il cliente non si è ancora autenticato nell'applicazione;
    \item \textbf{Input:} il cliente inserisce ed invia i dati per il login;
    \item \textbf{Postcondizione:} il cliente è autenticato;
    \item \textbf{Scenario principale:}
          \begin{enumerate}
              \item il cliente inserisce username e password;
              \item invia i dati inseriti;
              \item i dati vengono verificati dal gestire esterno;
              \item se i dati sono corretti e identificato un profilo utente il cliente è autenticato con questo profilo;
          \end{enumerate}
    \item \textbf{Estensioni:}
          \begin{enumerate}
              \item il gestore delle credenziali restituisce un errore indicante che i dati inseriti sono errati;
              \item viene quindi chiesto di reinserire le credenziali, contestualmente si visualizza un messaggio di errore.
          \end{enumerate}
\end{itemize}

%12
\stepUserCase
\subsection{\valueUserCase - Registrazione cliente}
\labelUserCase
\begin{figure}[H]
    \centering
    \includegraphics[width=30em]{res/images/UC/\valueUC.jpg}
    \caption{Diagramma \valueUC}
\end{figure}
\begin{itemize}
    \item \textbf{Attore primario:} cliente non autenticato;
    \item \textbf{Attore secondario:} gestore delle credenziali esterno;
    \item \textbf{Descrizione:} caso d'uso per la registrazione di un nuovo cliente;
    \item \textbf{Precondizione:} il cliente non si è ancora autenticato nell'applicazione;
    \item \textbf{Input:} il cliente inserisce ed invia i dati per la registrazione;
    \item \textbf{Postcondizione:} il cliente è autenticato con il nuovo profilo appena inserito nel sistema;
    \item \textbf{Scenario principale:}
          \begin{enumerate}
              \item il cliente inserisce l'email, la quale verrà utilizzata per contattarlo e per identificare univocamente l'utente nel sistema;
              \item il cliente reinserisce l'email per conferma;
              \item il cliente inserisce una password che rispetti i requisiti minimi di sicurezza;
              \item il cliente reinserisce la stessa password per conferma.
              \item i dati vengono inviati al servizio di terze parti per la gestione dei dati di login;
              \item arriva un'email al cliente contenente un link per la verifica all'indirizzo indicato;
              \item il cliente deve premere su quel link per attivare l'account entro una finestra di tempo limitata.
          \end{enumerate}
    \item \textbf{Estensioni:} la registrazione fallisce se:
          \begin{itemize}
              \item nel sistema esiste già un cliente con la stessa email;
              \item le due email inserite non corrispondono;
              \item le due password inserite non corrispondono;
              \item il cliente non preme il link di verifica inviato alla sua casella di posta.
          \end{itemize}
\end{itemize}

%13
\stepUserCase
\subsection{\valueUserCase - Reimpostazione password cliente}
\labelUserCase
\begin{figure}[H]
    \centering
    \includegraphics[width=30em]{res/images/UC/\valueUC.jpg}
    \caption{Diagramma \valueUC}
\end{figure}
\begin{itemize}
    \item \textbf{Attore primario:} cliente non autenticato;
    \item \textbf{Attore secondario:} gestore delle credenziali esterno;
    \item \textbf{Descrizione:} caso d'uso per la reimpostazione della password, utile quando il cliente la dimentica;
    \item \textbf{Precondizione:} il cliente non si è ancora autenticato nell'applicazione e si trova nella pagina di login;
    \item \textbf{Input:} il cliente preme sul pulsante per la reimpostazione della password;
    \item \textbf{Postcondizione:} il cliente ha modificato la sua password e può effettuare il login;
    \item \textbf{Scenario principale:}
          \begin{enumerate}
              \item il cliente chiede la reimpostazione della password;
              \item riceve tramite email un link per la reimpostazione valido entro un certo limite di tempo;
              \item preme sul link e viene portato su una pagina per l'inserimento della password;
              \item inserisce la nuova password;
              \item reinserisce la password per conferma;
              \item invia i dati inseriti al gestore delle credenziali.
          \end{enumerate}
    \item \textbf{Estensioni:} la reimpostazione della password fallisce se:
          \begin{itemize}
              \item le password inserite non corrispondono;
              \item la nuova password non rispetta i requisiti minimi di sicurezza.
          \end{itemize}
\end{itemize}

%14
\stepUserCase
\subsection{\valueUserCase - Gestione indirizzi account}
\labelUserCase
\begin{figure}[H]
    \centering
    \includegraphics[width=30em]{res/images/UC/\valueUC.jpg}
    \caption{Diagramma \valueUC}
\end{figure}
\begin{itemize}
    \item \textbf{Attore primario:} cliente autenticato;
    \item \textbf{Descrizione:} il cliente vuole gestire i suoi indirizzi;
    \item \textbf{Precondizione:} il cliente ha eseguito il login (\hyperref[UC11]{UC11}) e si trova sulla pagina per la gestione del profilo;
    \item \textbf{Input:} il cliente ha cliccato sul pulsante per l'amministrazione dell'account;
    \item \textbf{Postcondizione:} il cliente ha inserito o eliminato un indirizzo.
    \item \textbf{Scenario principale:}
          \begin{enumerate}
              \item il cliente può inserire un nuovo indirizzo (\hyperref[UC14.1]{UC14.1});
              \item il cliente può eliminare un indirizzo presente (\hyperref[UC14.2]{UC14.2});
          \end{enumerate}
\end{itemize}

%14.1
\stepsubUserCase
\subsubsection{\valuesubUserCase - Inserimento indirizzo}
\labelsubUserCase
\begin{figure}[H]
    \centering
    \includegraphics[width=30em]{res/images/UC/\valuesubUC.jpg}
    \caption{Diagramma \valuesubUC}
\end{figure}
\begin{itemize}
    \item \textbf{Attore primario:} cliente autenticato;
    \item \textbf{Descrizione:} scenario per l'inserimento di un nuovo indirizzo (utilizzabile per spedizione e/o fatturazione in fase di checkout);
    \item \textbf{Precondizione:} il cliente ha eseguito il login (\hyperref[UC11]{UC11}) e si trova sulla pagina per la gestione del profilo;
    \item \textbf{Input:} il cliente seleziona l'opzione per inserire un nuovo indirizzo;
    \item \textbf{Postcondizione:} nel sistema esiste l'indirizzo inserito dal cliente;
    \item \textbf{Scenario principale:}
          \begin{enumerate}
              \item il cliente compila il form per l'inserimento dell'indirizzo: nome, cognome, via/piazza, numero civico, codice avviamento postale, città, provincia, stato;
              \item il cliente invia quindi i dati.
          \end{enumerate}
\end{itemize}

%14.2
\stepsubUserCase
\subsubsection{\valuesubUserCase - Eliminazione indirizzo}
\labelsubUserCase
\begin{itemize}
    \item \textbf{Attore primario:} cliente autenticato;
    \item \textbf{Descrizione:} scenario per l'eliminazione di un indirizzo precedentemente inserito;
    \item \textbf{Precondizione:} il cliente ha eseguito il login (\hyperref[UC11]{UC11}) e si trova sulla pagina per la gestione del profilo;
    \item \textbf{Input:} il cliente seleziona l'opzione per eliminare un indirizzo;
    \item \textbf{Postcondizione:} nel sistema non esiste più l'indirizzo rimosso.
\end{itemize}

%15
\stepUserCase
\subsection{\valueUserCase - Visualizzazione lista ordini}
\labelUserCase
\begin{figure}[H]
    \centering
    \includegraphics[width=30em]{res/images/UC/\valueUC.jpg}
    \caption{Diagramma \valuesubUC}
\end{figure}
\begin{itemize}
    \item \textbf{Attore primario:} cliente autenticato;
    \item \textbf{Descrizione:} il cliente vuole visualizzare tutti gli ordini da lui effettuati;
    \item \textbf{Precondizione:} il cliente ha effettuato il login;
    \item \textbf{Input:} il cliente clicca sul pulsante per visualizzare gli ordini;
    \item \textbf{Postcondizione:} il cliente visualizza tutti gli ordini effettuati in maniera sintetica;
    \item \textbf{Scenario principale:}
          \begin{enumerate}
              \item il cliente decide di aprire la lista ordini;
              \item si apre la pagina contenente la lista degli ordini effettuati.
            \item il cliente puo' aprire il riepilogo di un ordine;
          \end{enumerate}
\end{itemize}

%15.1
\stepsubUserCase
\subsubsection{\valuesubUserCase - Visualizzazione singolo ordine lista }
\labelsubUserCase
\begin{figure}[H]
    \centering
    \includegraphics[width=30em]{res/images/UC/\valuesubUC.jpg}
    \caption{Diagramma \valuesubUC}
\end{figure}
\begin{itemize}
    \item \textbf{Attore primario:} cliente autenticato;
    \item \textbf{Descrizione:} il cliente vuole visualizzare le caratteristiche di ogni ordine nella lista;
    \item \textbf{Precondizione:} il cliente ha effettuato l'ordine e deve trovarsi nella pagina lista degli ordini;
    \item \textbf{Postcondizione:} il cliente visualizza le informazioni;
    \item \textbf{Scenario principale:}
          \begin{enumerate}
              \item per ogni ordini viene visualizzato il codice identificativo, la data, lo stato, il numero totale di articoli ed il totale.
          \end{enumerate}
\end{itemize}


%15.2
\stepsubUserCase
\subsubsection{\valuesubUserCase - Riepilogo ordine}
\labelsubUserCase
\begin{figure}[H]
    \centering
    \includegraphics[width=30em]{res/images/UC/\valuesubUC.jpg}
    \caption{Diagramma \valuesubUC}
\end{figure}
\begin{itemize}
    \item \textbf{Attore primario:} cliente autenticato;
    \item \textbf{Descrizione:} il cliente vuole visualizzare il riepilogo dell'ordine effettuato;
    \item \textbf{Precondizione:} il cliente ha effettuato l'ordine e deve trovarsi nella pagina lista degli ordini;
    \item \textbf{Input:} il cliente clicca sul pulsante per la visualizzazione del riepilogo dell'ordine;
    \item \textbf{Postcondizione:} il cliente visualizza il riepilogo;
    \item \textbf{Scenario principale:}
          \begin{enumerate}
              \item il cliente decide di aprire il riepilogo dell'ordine;
              \item si apre la pagina dove è presentato il riepilogo: numero identificativo, data, importo totale, prodotti acquistati (per ognuno codice, titolo, prezzo e quantità), indirizzo di fatturazione e spedizione, stato dell'ordine.
          \end{enumerate}
\end{itemize}

%16
\stepUserCase
\subsection{\valueUserCase - Logout cliente}
\labelUserCase
\begin{itemize}
    \item \textbf{Attore primario:} cliente autenticato;
    \item \textbf{Descrizione:} il cliente vuole effettuare il logout;
    \item \textbf{Precondizione:} il cliente è autenticato al sito;
    \item \textbf{Input:} il cliente clicca sul pulsante per il logout;
    \item \textbf{Postcondizione:} il cliente risulta non autenticato nel sito;
    \item \textbf{Scenario principale:}
          \begin{enumerate}
              \item il cliente effettua il logout;
              \item il cliente si ritrova nella pagina principale del sito.
          \end{enumerate}
\end{itemize}

%17
\stepUserCase
\subsection{\valueUserCase - Contatta venditore}
\labelUserCase
\begin{figure}[H]
    \centering
    \includegraphics[width=30em]{res/images/UC/\valueUC.jpg}
    \caption{Diagramma \valueUC}
\end{figure}
\begin{itemize}
    \item \textbf{Attore primario:} cliente generico;
    \item \textbf{Descrizione:} il cliente vuole contattare il venditore attraverso il form di contatto;
    \item \textbf{Precondizione:} il cliente sta navigando il sito, o arriva ha richiesto assistenza o un reso o un annullamento;
    \item \textbf{Postcondizione:} il cliente riesce a contattare il venditore;
    \item \textbf{Scenario principale:}
          \begin{enumerate}
              \item il cliente ha aperto il form di contatto;
              \item il cliente compila il form: oggetto della richiesta, email a cui rispondere, testo del messaggio;
              \item eventualmente il cliente inserisce il numero dell'ordine a cui si vuole riferire (se non ha richiesto assistenza o un reso o l'annullamento di un ordine, in quanto viene inserito in automatico.) ;
              \item il cliente contatta con successo il venditore.
          \end{enumerate}
\end{itemize}


%18
\stepUserCase
\subsection{\valueUserCase - Modifica credenziali}
\labelUserCase
\begin{figure}[H]
    \centering
    \includegraphics[width=30em]{res/images/UC/\valueUC.jpg}
    \caption{Diagramma \valueUC}
\end{figure}
\begin{itemize}
    \item \textbf{Attore primario:} cliente autenticato;
    \item \textbf{Descrizione:} il cliente vuole modificare le proprie credenziali;
    \item \textbf{Precondizione:} il cliente ha eseguito il login (\hyperref[UC11]{UC11}) e si trova sulla pagina per la gestione del profilo;
    \item \textbf{Input:} il cliente inserisce la nuova password e/o email nell'apposito spazio;
    \item \textbf{Postcondizione:} il cliente ha modificato le credenziali.
    \item \textbf{Scenario principale:}
    \begin{enumerate}
        \item il cliente puo' modificare l'email;
        \item il cliente puo' modificare la password;
    \end{enumerate}
\end{itemize}

%18.1
\stepsubUserCase
\subsubsection{\valuesubUserCase - Modifica email}
\labelsubUserCase
\begin{figure}[H]
    \centering
    \includegraphics[width=30em]{res/images/UC/\valuesubUC.jpg}
    \caption{Diagramma \valuesubUC}
\end{figure}
\begin{itemize}
    \item \textbf{Attore primario:} cliente autenticato;
    \item \textbf{Descrizione:} scenario per la modifica dell'email (username) dell'utente;
    \item \textbf{Precondizione:} il cliente ha eseguito il login (\hyperref[UC11]{UC11}) e si trova sulla pagina per la gestione del profilo;
    \item \textbf{Input:} il cliente sceglie l'opzione per modificare l'email che lo identifica univocamente all'interno del sito;
    \item \textbf{Postcondizione:} l'email è stata modificata con quella nuova inserita;
    \item \textbf{Scenario principale:}
          \begin{enumerate}
              \item il cliente inserisce l'email nel campo apposito;
              \item il cliente reinserisce la stessa email in un altro campo per la verifica;
              \item il cliente invia quindi i dati inseriti;
              \item si ripete il processo di verifica email spiegato nel caso d'uso della registrazione.
          \end{enumerate}
    \item \textbf{Estensioni:} la modifica dell'email fallisce se:
          \begin{itemize}
              \item i due indirizzi email non corrispondono;
              \item il nuovo indirizzo non viene verificato.
          \end{itemize}
\end{itemize}

%18.2
\stepsubUserCase
\subsubsection{\valuesubUserCase - Modifica password}
\labelsubUserCase
\begin{figure}[H]
    \centering
    \includegraphics[width=30em]{res/images/UC/\valuesubUC.jpg}
    \caption{Diagramma \valuesubUC}
\end{figure}
\begin{itemize}
    \item \textbf{Attore primario:} cliente autenticato;
    \item \textbf{Descrizione:} scenario per la modifica della password del cliente;
    \item \textbf{Precondizione:} il cliente ha eseguito il login (\hyperref[UC11]{UC11}) e si trova sulla pagina per la gestione del profilo;
    \item \textbf{Input:} il cliente seleziona l'opzione per modificare la password di accesso;
    \item \textbf{Postcondizione:} la password di accesso al sistema per quel cliente è stata modificata;
    \item \textbf{Scenario principale:}
          \begin{enumerate}
              \item il cliente inserisce la vecchia password;
              \item il cliente inserisce la nuova password;
              \item il cliente reinserisce la nuova password;
              \item il cliente invia quindi i dati inseriti.
          \end{enumerate}
    \item \textbf{Estensioni:} la modifica della password fallisce se:
          \begin{itemize}
              \item le password inserite non corrispondono;
              \item la nuova password non rispetta i requisiti minimi di sicurezza.
          \end{itemize}
\end{itemize}

%19
\stepUserCase
\subsection{\valueUserCase - Richiesta cancellazione account}
\labelUserCase
\begin{itemize}
    \item \textbf{Attore primario:} cliente autenticato;
    \item \textbf{Descrizione:} scenario per la richiesta di cancellazione del profilo cliente al venditore;
    \item \textbf{Precondizione:} il cliente ha eseguito il login (\hyperref[UC11]{UC11}) e si trova sulla pagina per la gestione del profilo;
    \item \textbf{Input:} il cliente preme il pulsante per richiedere la cancellazione dell'account;
    \item \textbf{Postcondizione:} il sistema invia una email predefinita al venditore contenente la richiesta di cancellazione del relativo account.
\end{itemize}


%20
\stepUserCase
\subsection{\valueUserCase - Annullamento ordine}
\labelUserCase
\begin{figure}[H]
    \centering
    \includegraphics[width=30em]{res/images/UC/\valueUC.jpg}
    \caption{Diagramma \valueUC}
\end{figure}
\begin{itemize}
    \item \textbf{Attore primario:} cliente autenticato;
    \item \textbf{Descrizione:} il cliente vuole annullare un ordine effettuato;
    \item \textbf{Precondizione:} il cliente deve aver effettuato un ordine e trovarsi nella sezione riepilogo ordine;
    \item \textbf{Input:} il cliente clicca sul pulsante per l'annullamento dell'ordine;
    \item \textbf{Postcondizione:} il cliente ha contatto il venditore per l'annullamento;
    \item \textbf{Scenario principale:}
          \begin{enumerate}
              \item il cliente sceglie ordine da annullare;
              \item il cliente viene portato al form di contatto dove sarà già precompilato il numero dell'ordine che si desidera annullare;
              \item il cliente seleziona il motivo dell'annullamento;
              \item il cliente scrive un messaggio per il venditore.
          \end{enumerate}
    \item \textbf{Inclusioni:}
          \begin{itemize}
              \item per annullare un ordine viene aperto il form di contatto che permette al cliente di contattare il venditore (\hyperref[UC17]{UC17}).
          \end{itemize}
\end{itemize}

%21
\stepUserCase
\subsection{\valueUserCase - Assistenza cliente}
\labelUserCase
\begin{figure}[H]
    \centering
    \includegraphics[width=30em]{res/images/UC/\valueUC.jpg}
    \caption{Diagramma \valueUC}
\end{figure}
\begin{itemize}
    \item \textbf{Attore primario:} cliente autenticato;
    \item \textbf{Descrizione:} il cliente vuole contattare il venditore per un problema (errore indirizzo, domande, etc.);
    \item \textbf{Precondizione:}  il cliente deve aver effettuato un ordine e trovarsi nella sezione riepilogo ordine;
    \item \textbf{Input:} il cliente clicca sul pulsante per l'assistenza;
    \item \textbf{Postcondizione:} il cliente ha contatto il venditore per richiedere assistenza;
    \item \textbf{Scenario principale:}
          \begin{enumerate}
              \item il cliente decide di contattare l'assistenza;
              \item il cliente viene portato al form di contatto dove sarà già precompilato il numero dell'ordine al quale si desidera chiedere assistenza;
          \end{enumerate}
    \item \textbf{Inclusioni:}
          \begin{itemize}
              \item per l'assistenza su un ordine viene aperto il form di contatto che permette al cliente di contattare il venditore (\hyperref[UC17]{UC17}).
          \end{itemize}
\end{itemize}

%22
\stepUserCase
\subsection{\valueUserCase - Richiesta reso}
\labelUserCase
\begin{figure}[H]
    \centering
    \includegraphics[width=30em]{res/images/UC/\valueUC.jpg}
    \caption{Diagramma \valueUC}
\end{figure}
\begin{itemize}
    \item \textbf{Attore primario:} cliente autenticato;
    \item \textbf{Descrizione:} il cliente vuole effettuare un reso di un ordine ricevuto;
    \item \textbf{Precondizione:}  il cliente deve aver effettuato un ordine e trovarsi nella sezione riepilogo ordine;
    \item \textbf{Input:} il cliente clicca sul pulsante per il reso dell'ordine;
    \item \textbf{Postcondizione:} il cliente ha contatto il venditore per il reso;
    \item \textbf{Scenario principale:}
          \begin{enumerate}
              \item il cliente decide di effettuare un reso;
              \item il cliente viene portato al form di contatto dove sarà già precompilato il numero dell'ordine di cui si desidera effettuare il reso;
          \end{enumerate}
    \item \textbf{Inclusioni:}
          \begin{itemize}
              \item per il reso di un ordine viene aperto il form di contatto che permette al cliente di contattare il venditore (\hyperref[UC17]{UC17}).
          \end{itemize}
\end{itemize}t

%23
\stepUserCase
\subsection{\valueUserCase - Contatta cliente}
\labelUserCase
\begin{figure}[H]
    \centering
    \includegraphics[width=30em]{res/images/UC/\valueUC.jpg}
    \caption{Diagramma \valueUC}
\end{figure}
\begin{itemize}
    \item \textbf{Attore primario:} venditore autenticato;
    \item \textbf{Descrizione:} il venditore deve poter contattare il cliente;
    \item \textbf{Precondizione:} il venditore deve trovarsi nella lista degli ordini o nella lista clienti;
    \item \textbf{Postcondizione:} il venditore ha contattato il cliente con successo;
    \item \textbf{Scenario principale:}
          \begin{enumerate}
              \item il venditore ha aperto il form di contatto;
              \item compila il form: oggetto del messaggio, email a cui rispondere, testo del messaggio;
              \item il venditore contatta con successo il cliente;
          \end{enumerate}
\end{itemize}

%24
\stepUserCase
\subsection{\valueUserCase - Visualizzazione lista ordini dei clienti}
\labelUserCase
\begin{figure}[H]
    \centering
    \includegraphics[width=30em]{res/images/UC/\valueUC.jpg}
    \caption{Diagramma \valueUC}
\end{figure}
\begin{itemize}
    \item \textbf{Attore primario:} venditore autenticato;
    \item \textbf{Descrizione:} il venditore visualizza la lista di tutti gli ordini ricevuti;
    \item \textbf{Precondizione:} il venditore deve essere autenticato;
    \item \textbf{Input:} il venditore esegue un'operazione sugli ordini;
    \item \textbf{Postcondizione:} il venditore visualizza la lista;
    \item \textbf{Scenario principale:}
          \begin{enumerate}
              \item il venditore visualizzare tutta la lista degli ordini ricevuti dai clienti in ordine cronologico (dal più recente al più vecchio).
              \item il venditore visualizza le informazioni generali per ogni ordine
              \item il venditore puo' visualizzare nel dettaglio un ordine
            \end{enumerate}
\end{itemize}

%24.1
\stepsubUserCase
\subsubsection{\valuesubUserCase - Visualizzazione singolo ordine lista}
\labelsubUserCase
\begin{figure}[H]
    \centering
    \includegraphics[width=30em]{res/images/UC/\valuesubUC.jpg}
    \caption{Diagramma \valuesubUC}
\end{figure}
\begin{itemize}
    \item \textbf{Attore primario:} venditore autenticato;
    \item \textbf{Descrizione:} il venditore vuole visualizzare le informazioni per ogni ordine;
    \item \textbf{Precondizione:} il venditore deve essere autenticato e trovarsi nella lista degli ordini;
    \item \textbf{Input:} il venditore clicca sul pulsante per la lista degli ordini;
    \item \textbf{Postcondizione:} il venditore visualizza le informazioni di ogni ordine.
    \item \textbf{Scenario principale:}
        \begin{enumerate}
            \item il venditore visualizza per ogni ordine codice, email cliente, stato, data e totale.
        \end{enumerate}
\end{itemize}

%24.2
\stepsubUserCase
\subsubsection{\valuesubUserCase- Visualizzazione dettagli ordine}
\labelsubUserCase
\begin{figure}[H]
    \centering
    \includegraphics[width=30em]{res/images/UC/\valuesubUC.jpg}
    \caption{Diagramma \valuesubUC}
\end{figure}
\begin{itemize}
    \item \textbf{Attore primario:} venditore autenticato;
    \item \textbf{Descrizione:} il venditore vuole visualizzare tutti i dettagli di un determinato ordine;
    \item \textbf{Precondizione:} il venditore deve trovarsi nella lista degli ordini e selezionarne uno;
    \item \textbf{Input:} il venditore sceglie un ordine di cui visualizzarne i dettagli;
    \item \textbf{Postcondizione:} il venditore visualizza tutti i dettagli dell'ordine;
    \item \textbf{Scenario principale:}
          \begin{enumerate}
              \item il venditore seleziona un determinato ordine;
              \item il venditore apre la pagina con i dettagli dell'ordine selezionato: stato dell'ordine, numero identificativo, email del cliente, articoli (titolo articolo, quantità, codice articolo), totale.
          \end{enumerate}
\end{itemize}


%25
\stepUserCase
\subsection{\valueUserCase - Lista clienti}
\labelUserCase
\begin{figure}[H]
    \centering
    \includegraphics[width=30em]{res/images/UC/\valueUC.jpg}
    \caption{Diagramma \valueUC}
\end{figure}
\begin{itemize}
    \item \textbf{Attore primario:} venditore autenticato;
    \item \textbf{Descrizione:} il venditore vuole visualizzare la lista dei clienti del sito;
    \item \textbf{Precondizione:} il venditore deve essere autenticato;
    \item \textbf{Input:} il venditore clicca sul pulsante per accedere alla lista clienti;
    \item \textbf{Postcondizione:} il venditore visualizza tutta la lista clienti;
    \item \textbf{Scenario principale:}
          \begin{enumerate}
              \item il venditore può contattare un cliente (\hyperref[UC18]{UC18}):
          \end{enumerate}
\end{itemize}

%25.1
\stepsubUserCase
\subsubsection{\valuesubUserCase - Visualizzazione singolo cliente lista}
\labelsubUserCase
\begin{figure}[H]
    \centering
    \includegraphics[width=30em]{res/images/UC/\valuesubUC.jpg}
    \caption{Diagramma \valuesubUC}
\end{figure}
\begin{itemize}
    \item \textbf{Attore primario:} venditore autenticato;
    \item \textbf{Descrizione:} il venditore vuole visualizzare le informazioni del clienti;
    \item \textbf{Precondizione:} il venditore deve essere autenticato e trovarsi nella lista clienti;
    \item \textbf{Postcondizione:} il venditore visualizza le informazioni del cliente.
    \item \textbf{Scenario principale:}
        \begin{enumerate}
            \item il venditore visualizza per ogni cliente della lista il nome, il cognome e l'email;
        \end{enumerate}
\end{itemize}

%26
\stepUserCase
\subsection{\valueUserCase- Ricerca cliente tramite email}
\labelUserCase
\begin{figure}[H]
    \centering
    \includegraphics[width=30em]{res/images/UC/\valueUC.jpg}
    \caption{Diagramma \valueUC}
\end{figure}
\begin{itemize}
    \item \textbf{Attore primario:} venditore autenticato;
    \item \textbf{Descrizione:} il venditore ricerca un cliente iscritto alla piattaforma;
    \item \textbf{Precondizione:} il venditore deve trovarsi nella lista clienti (\hyperref[UC20]{UC20});
    \item \textbf{Input:} inserisce l'email da ricercare;
    \item \textbf{Postcondizione:} il venditore visualizza il risultato della ricerca;
    \item \textbf{Scenario principale:}
          \begin{enumerate}
              \item il venditore inserisce l'email del cliente nell'apposito campo;
              \item il venditore visualizza il cliente.
          \end{enumerate}
    \item \textbf{Estensioni:} nel caso in cui l'email inserita non rappresenti nessun cliente, dev'essere visualizzato il relativo messaggio d'errore.
\end{itemize}

%27
\stepUserCase
\subsection{\valueUserCase - Amministrazione tasse}
\labelUserCase
\begin{figure}[H]
    \centering
    \includegraphics[width=30em]{res/images/UC/\valueUC.jpg}
    \caption{Diagramma \valueUC}
\end{figure}
\begin{itemize}
    \item \textbf{Attore primario:} venditore autenticato;
    \item \textbf{Descrizione:} il venditore ha possibilità di gestire le operazioni relative all'amministrazione della tassazione dei prodotti;
    \item \textbf{Precondizione:} il venditore ha cliccato nel bottone relativo alla gestione delle tasse all'interno della dashboard venditore;
    \item \textbf{Input:} il venditore preme su un pulsante dedicato alla gestione delle aliquote IVA;
    \item \textbf{Postcondizione:} il venditore ha visione della tabella relativa alle aliquote IVA già presenti nel sistema e può intraprendere una delle azioni disponibili;
    \item \textbf{Scenario principale:} il venditore ha cliccato nel bottone relativo alla gestione delle tasse e ha visione della tabella riepilogativa delle varie aliquote IVA presenti nel sistema, può scegliere di effettuare una delle seguenti azioni:
          \begin{enumerate}
              \item aggiunta aliquote IVA;
              \item rimozione aliquote IVA;
              \item modifica aliquota IVA.
          \end{enumerate}
\end{itemize}

%27.1
\stepsubUserCase
\subsubsection{\valuesubUserCase- Aggiunta aliquota IVA}
\labelsubUserCase
\begin{figure}[H]
    \centering
    \includegraphics[width=30em]{res/images/UC/\valuesubUC.jpg}
    \caption{Diagramma \valuesubUC}
\end{figure}
\begin{itemize}
    \item \textbf{Attore primario:} venditore autenticato;
    \item \textbf{Descrizione:} il venditore aggiunge una nuova aliquota IVA;
    \item \textbf{Precondizione:} il venditore si trova sulla pagina per la gestione delle aliquote IVA;
    \item \textbf{Input:} il venditore preme su un pulsante dedicato all'aggiunta di un'aliquota IVA;
    \item \textbf{Postcondizione:} è stata aggiunta una nuova aliquota IVA nel sistema;
    \item \textbf{Scenario principale:} il venditore ha cliccato nel bottone di aggiunta aliquota, posto nella pagina di amministrazione della tassazione e gli viene richiesto di inserire:
          \begin{enumerate}
              \item inserimento percentuale IVA;
              \item inserimento descrizione aliquota.
          \end{enumerate}
\end{itemize}

%27.2
\stepsubUserCase
\subsubsection{\valuesubUserCase- Modifica aliquota IVA}
\labelsubUserCase
\begin{itemize}
    \item \textbf{Attore primario:} venditore autenticato;
    \item \textbf{Descrizione:} il venditore effettua la modifica di una aliquota IVA già presente nel sistema;
    \item \textbf{Precondizione:} il venditore si trova sulla pagina per la gestione delle aliquote IVA ed è presente almeno un'aliquota nel sistema;
    \item \textbf{Input:} il venditore preme su un pulsante dedicato alla modifica di un'aliquota IVA;
    \item \textbf{Postcondizione:} è stata modificata l'aliquota IVA selezionata;
    \item \textbf{Scenario principale:} il venditore ha cliccato nel bottone di modifica di un'aliquota IVA già presente nel sistema per apportare delle modifiche, può quindi effettuare le seguenti azioni:
          \begin{enumerate}
              \item modificare percentuale IVA;
              \item modificare descrizione aliquota.
          \end{enumerate}
\end{itemize}

%27.3
\stepsubUserCase
\subsubsection{\valuesubUserCase- Eliminazione aliquota IVA}
\labelsubUserCase
\begin{itemize}
    \item \textbf{Attore primario:} venditore autenticato;
    \item \textbf{Descrizione:} il venditore effettua l'eliminazione di una aliquota IVA già presente nel sistema;
    \item \textbf{Precondizione:} il venditore si trova sulla pagina per la gestione delle aliquote IVA ed è presente almeno un'aliquota nel sistema;
    \item \textbf{Input:} il venditore preme su un pulsante dedicato all'eliminazione di un'aliquota IVA;
    \item \textbf{Postcondizione:} l'aliquota IVA selezionata è stata eliminata;
    \item \textbf{Scenario principale:} il venditore ha cliccato nel bottone per eliminare un'aliquota IVA già presente nel sistema ed essa viene eliminata.
\end{itemize}

%28
\stepUserCase
\subsection{\valueUserCase- Visualizzazione lista prodotti}
\labelUserCase\begin{figure}[H]
    \centering
    \includegraphics[width=30em]{res/images/UC/\valueUC.jpg}
    \caption{Diagramma \valueUC}
\end{figure}
\begin{itemize}
    \item \textbf{Attore primario:} venditore autenticato;
    \item \textbf{Descrizione:} il venditore può visualizzare la tabella dei prodotti presenti nel sistema;
    \item \textbf{Precondizione:} il venditore deve essere autenticato;
    \item \textbf{Input:} il venditore clicca sul pulsante di visualizzazione dei prodotti;
    \item \textbf{Postcondizione:} il venditore riesce a visualizzare la tabella dei prodotti;
    \item \textbf{Scenario principale:}
        \begin{enumerate}
            \item il venditore visualizza la tabella dei prodotti al momento inseriti.
        \end{enumerate}
\end{itemize}

%28.1
\stepsubUserCase
\subsubsection{\valuesubUserCase - Visualizzazione singolo prodotto lista}
\labelsubUserCase
\begin{figure}[H]
    \centering
    \includegraphics[width=30em]{res/images/UC/\valuesubUC.jpg}
    \caption{Diagramma \valuesubUC}
\end{figure}
\begin{itemize}
    \item \textbf{Attore primario:} venditore autenticato;
    \item \textbf{Descrizione:} il venditore vuole visualizzare le informazioni di un prodotto;
    \item \textbf{Precondizione:} il venditore deve essere autenticato e trovarsi nella lista dei prodotti;
    \item \textbf{Postcondizione:} il venditore visualizza le informazioni del prodotto.
    \item \textbf{Scenario principale:}
        \begin{enumerate}
            \item il venditore visualizza per ogni prodotto: codice, nome, prezzo, categoria.
        \end{enumerate}
\end{itemize}

%29
\stepUserCase
\subsection{\valueUserCase - Amministrazione categorie}
\labelUserCase
\begin{figure}[H]
    \centering
    \includegraphics[width=30em]{res/images/UC/\valueUC.jpg}
    \caption{Diagramma \valueUC}
\end{figure}
\begin{itemize}
    \item \textbf{Attore primario:} venditore autenticato;
    \item \textbf{Descrizione:}  il venditore può visualizzare e gestire le categorie di prodotto;
    \item \textbf{Precondizione:}  il venditore si trova nella dashboard di amministrazione;
    \item \textbf{Input:} il venditore seleziona il pulsante per l'amministrazione delle categorie;
    \item \textbf{Postcondizione:} il venditore si trova nella pagina per gestire le categorie;
    \item \textbf{Scenario principale:} il venditore visualizza può intraprendere una delle seguenti azioni:
          \begin{enumerate}
              \item aggiungi categoria (\hyperref[UC29.1]{UC29.1});
              \item modifica categoria (\hyperref[UC29.2]{UC29.2});
              \item rimuovi categoria (\hyperref[UC29.3]{UC29.3});
              \item visualizza categorie (\hyperref[UC29.4]{UC29.4}).
          \end{enumerate}
\end{itemize}

%29.1
\stepsubUserCase
\subsubsection{\valuesubUserCase- Aggiungi categoria}
\labelsubUserCase
\begin{figure}[H]
    \centering
    \includegraphics[width=30em]{res/images/UC/\valuesubUC.jpg}
    \caption{Diagramma \valuesubUC}
\end{figure}
\begin{itemize}
    \item \textbf{Attore primario:} venditore autenticato.
    \item \textbf{Descrizione:} il venditore può aggiungere una categoria;
    \item \textbf{Precondizione:} il venditore si trova sulla pagina di amministrazione categorie;
    \item \textbf{Input:} il venditore ha cliccato nel pulsante per aggiungere una nuova categoria di prodotto;
    \item \textbf{Postcondizione:} la nuova categoria è stata inserita nel sistema;
    \item \textbf{Scenario principale:} il venditore ha cliccato sul pulsante di inserimento nuova categoria disponibile nella pagina di gestione categorie. Le azioni che dovrà compiere sono:
          \begin{enumerate}
              \item inserimento nome categoria;
              \item salvataggio;
          \end{enumerate}
    \item \textbf{Estensioni:} l'inserimento fallisce se:
          \begin{itemize}
              \item la categoria è già esistente;
              \item il nome categoria è vuoto.
          \end{itemize}
\end{itemize}

%29.2
\stepsubUserCase
\subsubsection{\valuesubUserCase- Modifica categoria}
\labelsubUserCase
\begin{figure}[H]
    \centering
    \includegraphics[width=30em]{res/images/UC/\valuesubUC.jpg}
    \caption{Diagramma \valuesubUC}
\end{figure}
\begin{itemize}
    \item \textbf{Attore primario:} venditore autenticato;
    \item \textbf{Descrizione:} il venditore può modificare il nome di una categoria;
    \item \textbf{Precondizione:} il venditore si trova sulla pagina di amministrazione categorie;
    \item \textbf{Input:} il venditore ha cliccato nel pulsante per modificare una categoria di prodotto;
    \item \textbf{Postcondizione:} la categoria selezionata è stata modificata;
    \item \textbf{Scenario principale:}
          \begin{enumerate}
              \item il venditore inserisce il nuovo nome per la categoria;
              \item preme quindi su conferma;
          \end{enumerate}
    \item \textbf{Estensioni:} La modifica fallisce se:
          \begin{itemize}
              \item il nuovo nome inserito è già utilizzato per un'altra categoria;
              \item il nome categoria inserito è vuoto.
          \end{itemize}
\end{itemize}

%29.3
\stepsubUserCase
\subsubsection{\valuesubUserCase- Rimozione categoria}
\labelsubUserCase
\begin{itemize}
    \item \textbf{Attore primario:} venditore autenticato;
    \item \textbf{Descrizione:} il venditore può rimuovere una categoria;
    \item \textbf{Precondizione:} il venditore si trova sulla pagina di amministrazione categorie;
    \item \textbf{Input:} il venditore ha cliccato nel pulsante per rimuovere una nuova categoria di prodotto;
    \item \textbf{Postcondizione:} il sistema ha rimosso la categoria selezionata dal venditore;
    \item \textbf{Scenario principale:} il venditore ha cliccato sul pulsante di rimozione categoria disponibile nella pagina di gestione categorie;
    \item \textbf{Estensioni:} la rimozione di una categoria fallisce se vi sono prodotti ancora collegati a quella categoria. In quel caso viene richiesto
          di inserire a quale categoria dirottare i prodotti coinvolti o se lasciarli senza categoria.
\end{itemize}

%29.4
\stepsubUserCase
\subsubsection{\valuesubUserCase- Visualizzazione categorie}
\labelsubUserCase
\begin{itemize}
    \item \textbf{Attore primario:} venditore autenticato;
    \item \textbf{Descrizione:} il venditore può visualizzare tutte le categorie;
    \item \textbf{Precondizione:} il venditore si trova sulla pagina di amministrazione categorie;
    \item \textbf{Input:} il venditore ha cliccato nel pulsante per visualizzare le categorie;
    \item \textbf{Postcondizione:} si visualizzano tutte le categorie esistenti;
    \item \textbf{Scenario principale:} il venditore ha cliccato sul pulsante di visualizzazione categorie disponibile nella pagina di gestione categorie. Per ogni categoria si visualizza il nome;
    \item \textbf{Estensioni:} se non vi sono categorie presenti a sistema deve comparire il relativo messaggio.
\end{itemize}

%30
\stepUserCase
\subsection{\valueUserCase - Logout venditore}
\labelUserCase
\begin{itemize}
    \item \textbf{Attore primario:} venditore autenticato;
    \item \textbf{Descrizione:} logout venditore dal backoffice;
    \item \textbf{Precondizione:} il venditore si trova sulla dashboard di amministrazione;
    \item \textbf{Input:} il venditore preme sul pulsante per eseguire il logout;
    \item \textbf{Postcondizione:} il venditore non è più autenticato nel backoffice e si trova sulla pagina per login del venditore.
\end{itemize}

%31
\stepUserCase
\subsection{\valueUserCase - Login venditore}
\labelUserCase
\begin{figure}[H]
    \centering
    \includegraphics[width=30em]{res/images/UC/\valueUC.jpg}
    \caption{Diagramma \valueUC}
\end{figure}
\begin{itemize}
    \item \textbf{Attore primario:} venditore non autenticato;
    \item \textbf{Descrizione:} autenticazione del venditore nel backoffice;
    \item \textbf{Precondizione:} il venditore non è autenticato;
    \item \textbf{Input:} il venditore inserisce le credenziali;
    \item \textbf{Postcondizione:} il venditore è autenticato;
    \item \textbf{Scenario principale:}
          \begin{enumerate}
              \item il venditore inserisce username e password);
              \item conferma il login;
          \end{enumerate}
    \item \textbf{Estensioni:} i dati di login sono errati, il login fallisce e si chiede il reinserimento dei dati.
\end{itemize}

%32
\stepUserCase
\subsection{\valueUserCase - Aggiunta prodotto}
\labelUserCase
\begin{figure}[H]
    \centering
    \includegraphics[width=30em]{res/images/UC/\valueUC.jpg}
    \caption{Diagramma \valueUC}
\end{figure}
\begin{itemize}
    \item \textbf{Attore primario:} venditore autenticato;
    \item \textbf{Descrizione:} il venditore ha possibilità di aggiungere un nuovo prodotto all'interno del portale;
    \item \textbf{Precondizione:} il venditore si trova nella pagina di gestione prodotti;
    \item \textbf{Input:} click sul bottone per inserire un nuovo articolo;
    \item \textbf{Postcondizione:} il nuovo prodotto è stato inserito nel sistema;
    \item \textbf{Scenario principale:} il venditore ha cliccato nel bottone per inserire un nuovo prodotto all'interno del catalogo prodotti gestito dal sistema, le azioni da completare sono:
          \begin{enumerate}
              \item inserimento titolo prodotto;
              \item inserimento descrizione prodotto;
              \item inserimento immagine principale;
              \item inserimento altre immagini (massimo 4);
              \item selezione categorie prodotto;
              \item inserimento prezzo netto prodotto;
              \item selezione aliquota IVA;
              \item selezione stato visibilità prodotto;
              \item inserimento dati giacenza magazzino del prodotto;
              \item visualizzazione in home (dato che indica se visualizzare o meno il prodotto in vetrina);
              \item inserimento giacenza prodotto;
              \item salvataggio.
          \end{enumerate}
    \item \textbf{Estensione:}
          \begin{itemize}
              \item l'inserimento fallisce perché il titolo è vuoto (campo obbligatorio) ;
              \item l'inserimento fallisce perché la quantità inserita non è valida (es. negativa o valore testuale).
          \end{itemize}
\end{itemize}

%33
\stepUserCase
\subsection{\valueUserCase- Modifica prodotto}
\labelUserCase
\begin{figure}[H]
    \centering
    \includegraphics[width=30em]{res/images/UC/\valueUC.jpg}
    \caption{Diagramma \valueUC}
\end{figure}
\begin{itemize}
    \item \textbf{Attore primario:} venditore autenticato;
    \item \textbf{Descrizione:} il venditore ha possibilità di modificare un prodotto già inserito nel sistema;
    \item \textbf{Precondizione:} il venditore si trova nella pagina di gestione prodotti;
    \item \textbf{Input:} click sul bottone per la modifica di un determinato prodotto;
    \item \textbf{Postcondizione:} il prodotto selezionato è stato modificato;
    \item \textbf{Scenario principale:} il venditore ha cliccato nel bottone per modificare un prodotto all'interno del catalogo prodotti gestito dal sistema, le azioni che potrebbe attuare sono:
          \begin{enumerate}
              \item modifica titolo prodotto;
              \item modifica descrizione prodotto;
              \item modifica immagine principale;
              \item modifica immagini secondarie;
              \item modifica categorie prodotto;
              \item modifica prezzo netto prodotto;
              \item modifica aliquota IVA;
              \item modifica stato visibilità prodotto;
              \item modifica visualizzazione in home (togliere o aggiungere il prodotto dalla home);
              \item modifica giacenza prodotto.
              \item salvataggio.
          \end{enumerate}
    \item \textbf{Estensione:}
          \begin{itemize}
              \item la modifica fallisce perché il titolo è vuoto (campo obbligatorio);
              \item la modifica fallisce perché la quantità inserita non è valida (es. negativa o valore testuale).
          \end{itemize}
\end{itemize}

%34
\stepUserCase
\subsection{\valueUserCase- Rimozione prodotto}
\labelUserCase
\begin{itemize}
    \item \textbf{Attore primario:} venditore autenticato;
    \item \textbf{Descrizione:} il venditore può rimuovere un prodotto;
    \item \textbf{Precondizione:} il venditore si trova nella pagina di amministrazione prodotti;
    \item \textbf{Input:} il venditore clicca sul pulsante di eliminazione per un determinato prodotto;
    \item \textbf{Postcondizione:} il prodotto selezionato non è più presente nel sistema.
\end{itemize}

%35
\stepUserCase
\subsection{\valueUserCase- Ricerca prodotto per nome}
\labelUserCase
\begin{figure}[H]
    \centering
    \includegraphics[width=30em]{res/images/UC/\valueUC.jpg}
    \caption{Diagramma \valueUC}
\end{figure}
\begin{itemize}
    \item \textbf{Attore primario:} venditore autenticato;
    \item \textbf{Descrizione:} il venditore può ricercare un prodotto per nome;
    \item \textbf{Precondizione:} il venditore si trova nella pagina di amministrazione prodotti;
    \item \textbf{Input:} il venditore clicca sul pulsante di ricerca prodotti;
    \item \textbf{Postcondizione:} si visualizza il prodotto ricercato;
    \item \textbf{Estensioni:} il prodotto ricercato non è presente a sistema, si visualizza quindi il relativo messaggio d'errore.
\end{itemize}

%36
\stepUserCase
\subsection{\valueUserCase- Filtri prodotti}
\labelUserCase
\begin{figure}[H]
    \centering
    \includegraphics[width=30em]{res/images/UC/\valueUC.jpg}
    \caption{Diagramma \valueUC}
\end{figure}
\begin{itemize}
    \item \textbf{Attore primario:} venditore autenticato;
    \item \textbf{Descrizione:} il venditore può filtrare i prodotti;
    \item \textbf{Precondizione:} il venditore si trova nella pagina di amministrazione prodotti;
    \item \textbf{Input:} il venditore seleziona un filtro;
    \item \textbf{Postcondizione:} si visualizzano i prodotti che rispettano le condizioni;
    \item \textbf{Scenario principale:}
    \begin{itemize}
        \item i prodotti possono essere filtrati per categoria.
    \end{itemize}
    \item \textbf{Estensioni:} il prodotto ricercato non è presente a sistema, si visualizza quindi il relativo messaggio d'errore.
\end{itemize}

%37
\stepUserCase
\subsection{\valueUserCase- Ricerca nella lista ordini}
\labelUserCase
\begin{figure}[H]
    \centering
    \includegraphics[width=30em]{res/images/UC/\valueUC.jpg}
    \caption{Diagramma \valueUC}
\end{figure}
\begin{itemize}
    \item \textbf{Attore primario:} venditore autenticato;
    \item \textbf{Descrizione}: il venditore vuole cercare un determinato ordine all'interno della lista;
    \item \textbf{Precondizione:} il venditore deve trovarsi nella pagina della lista degli ordini;
    \item \textbf{Input:} il venditore inserisce il numero dell'ordine da cercare;
    \item \textbf{Postcondizione:} il venditore visualizza l'ordine richiesto;
    \item \textbf{Scenario principale:}
        \begin{enumerate}
            \item il venditore inserisce nel campo per la ricerca il codice dell'ordine;
            \item il venditore visualizza l'ordine richiesto.
        \end{enumerate}
    \item \textbf{Estensione:}
    \begin{itemize}
        \item il codice d'ordine inserito non esiste, quindi si visualizza il relativo messaggio di errore.
    \end{itemize}
\end{itemize}

%38
\stepUserCase
\subsection{\valueUserCase- Modifica stato ordine}
\labelUserCase
\begin{itemize}
    \item \textbf{Attore primario:} venditore autenticato;
    \item \textbf{Descrizione:} il venditore vuole modificare lo stato di un ordine;
    \item \textbf{Precondizione:} il venditore deve trovarsi nella pagina con la lista degli ordini;
    \item \textbf{Input:} il venditore sceglie un ordine a cui modifica lo stato;
    \item \textbf{Postcondizione:} il venditore ha modificato lo stato dell'ordine scelto;
    \item \textbf{Scenario principale:}
          \begin{enumerate}
              \item il venditore seleziona un ordine dalla lista;
              \item il venditore modifica lo stato dell'ordine scegliendo tra: accettato, in elaborazione, spedito, consegnato, cancellato.
          \end{enumerate}
\end{itemize}

%39
\stepUserCase
\subsection{\valueUserCase- Stampa bolla}
\labelUserCase
\begin{itemize}
    \item \textbf{Attore primario:} venditore autenticato;
    \item \textbf{Descrizione:} il venditore vuole stampare la bolla per l'ordine;
    \item \textbf{Precondizione:} il venditore deve trovarsi nella pagina con la lista degli ordini;
    \item \textbf{Input:} il venditore sceglie un ordine di cui stampare la bolla;
    \item \textbf{Postcondizione:} il venditore ha stampato la bolla;
    \item \textbf{Scenario principale:}
          \begin{enumerate}
              \item il venditore seleziona un determinato ordine;
              \item il venditore stampa la bolla.
          \end{enumerate}
\end{itemize}

%40
\stepUserCase
\subsection{\valueUserCase - Visualizzazione errore quantità disponibile}
\labelUserCase
\begin{itemize}
    \item \textbf{Attore primario:} cliente autenticato;
    \item \textbf{Descrizione:} il cliente ha inserito una quantità non disponibile che deve essere notificata;
    \item \textbf{Precondizione:} il cliente deve aver selezionato una quantità;
    \item \textbf{Postcondizione:} il cliente visualizza messaggio d'errore per la quantità disponibile.
    \item \textbf{Scenario principale:} il cliente ha selezionato 
\end{itemize}

%41
\stepUserCase
\subsection{\valueUserCase - Visualizzazione errore pagamento}
\labelUserCase
\begin{itemize}
    \item \textbf{Attore primario:} cliente autenticato;
    \item \textbf{Descrizione:} il cliente vuole procedere con il pagamento ma si verifica un errore;
    \item \textbf{Precondizione:} il cliente ha avviato il checkout e vuole procedere al pagamento;
    \item \textbf{Input:} il cliente ha cliccato sul pulsante per pagare;
    \item \textbf{Postcondizione:} il cliente visualizza un messaggio d'errore del pagamento.
    \item \textbf{Scenario principale:} il cliente vuole procedere con il pagamento ma si verifica un problema e viene visualizzato un messaggio d'errore.
\end{itemize}

%42
\stepUserCase
\subsection{\valueUserCase - Visualizzazione errore email}
\labelUserCase
\begin{itemize}
    \item \textbf{Attore primario:} cliente autenticato o venditore autenticato;
    \item \textbf{Descrizione:} il cliente/venditore visualizza un messaggio di errore per l'email;
    \item \textbf{Precondizione:} il cliente/venditore ha inserito l'email;
    \item \textbf{Postcondizione:} il cliente/ venditore visualizza un messaggio d'errore del'email. 
\end{itemize}

%43
\stepUserCase
\subsection{\valueUserCase - Visualizzazione errore password}
\labelUserCase
\begin{itemize}
    \item \textbf{Attore primario:} cliente non autenticato o venditore non autenticato;
    \item \textbf{Descrizione:} il cliente/venditore visualizza un messaggio di errore per la password;
    \item \textbf{Precondizione:} il cliente/venditore ha inserito la password;
    \item \textbf{Postcondizione:} il cliente/ venditore visualizza un messaggio d'errore la password. 
\end{itemize}

%44
\stepUserCase
\subsection{\valueUserCase - Visualizzazione errore cambio password}
\labelUserCase
\begin{itemize}
    \item \textbf{Attore primario:} cliente autenticato;
    \item \textbf{Descrizione:} il cliente vuole cambiare password ma si verifica un errore;
    \item \textbf{Precondizione:} il cliente deve aver inserito i dati per la nuova password;
    \item \textbf{Postcondizione:} il cliente visualizza un messaggio d'errore per il cambio password.
    \item \textbf{Scenario principale:} il cliente inserisce la password nuova ma si verifica un errore che porta alla visualizzazione del messaggio.
\end{itemize}

%45
\stepUserCase
\subsection{\valueUserCase - Visualizzazione errore "nessun utente trovato"}
\labelUserCase
\begin{itemize}
    \item \textbf{Attore primario:} venditore autenticato;
    \item \textbf{Descrizione:} il venditore vuole cercare un cliente ma la ricerca non va a buon fine;
    \item \textbf{Precondizione:} il venditore si deve trovare nella pagina della lista dei clienti ed aver avviato una ricerca;
    \item \textbf{Postcondizione:} il venditore visualizza messaggio d'errore "nessun utente trovato".
    \item \textbf{Scenario principale:} il venditore avvia una ricerca ma non viene trovato nessun cliente relativo ad essa, quindi viene visualizzato l'errore
\end{itemize}

%46
\stepUserCase
\subsection{\valueUserCase - Visualizzazione messaggio "alcuni prodotti hanno quell'aliquota" }
\labelUserCase
\begin{itemize}
    \item \textbf{Attore primario:} venditore autenticato;
    \item \textbf{Descrizione:} il venditore vuole rimuovere un'aliquota ma sono presenti dei prodotti associati ad essa;
    \item \textbf{Precondizione:} il venditore deve trovarsi nella pagina delle tasse.
    \item \textbf{Input:} il venditore ha cliccato sul pulsante di rimozione di un'aliquota;
    \item \textbf{Postcondizione:} il venditore visualizza il messaggio "alcuni prodotti hanno quell'aliquota".
    \item \textbf{Scenario principale:} il venditore decide quale aliquota eliminare, viene visualizzato messaggio d'errore se sono presenti prodotti associati ad essa.
\end{itemize}

%47
\stepUserCase
\subsection{\valueUserCase - Visualizzazione messaggio rimozione fallita}
\labelUserCase
\begin{itemize}
    \item \textbf{Attore primario:} venditore autenticato;
    \item \textbf{Descrizione:} il venditore vuole gestire rimuovere una categoria ma l'azione fallisce;
    \item \textbf{Precondizione:} il venditore deve trovarsi nella pagina delle categorie e aver selezionato la categoria da eliminare;
    \item \textbf{Input:} il venditore ha cliccato sul pulsante rimuovi;
    \item \textbf{Postcondizione:} il venditore visualizza il messaggio del fallimento.
    \item \textbf{Scenario principale:} il venditore clicca per eliminare una categoria ma l'azione fallisce, viene quindi visualizzato il relativo messaggio.
\end{itemize}

%48
\stepUserCase
\subsection{\valueUserCase - Visualizzazione messaggio inserimento fallita}
\labelUserCase
\begin{itemize}
    \item \textbf{Attore primario:} venditore autenticato;
    \item \textbf{Descrizione:} il venditore vuole gestire inserire una categoria ma l'azione fallisce;
    \item \textbf{Precondizione:} il venditore deve trovarsi nella pagina per l'aggiunta di una categoria;
    \item \textbf{Input:} il venditore ha cliccato sul pulsante per salvare l'inserimento;
    \item \textbf{Postcondizione:} il venditore visualizza il messaggio del fallimento.
    \item \textbf{Scenario principale:} il venditore clicca per salvare l'inserimento di una categoria ma l'azione fallisce, viene quindi visualizzato il relativo messaggio.
\end{itemize}

%49
\stepUserCase
\subsection{\valueUserCase - Visualizzazione messaggio modifica fallita}
\labelUserCase
\begin{itemize}
    \item \textbf{Attore primario:} venditore autenticato;
    \item \textbf{Descrizione:} il venditore vuole gestire modificare una categoria ma l'azione fallisce;
    \item \textbf{Precondizione:} il venditore deve trovarsi nella pagina per la modifica di una categoria;
    \item \textbf{Input:} il venditore ha cliccato sul pulsante per salvare le modifiche;
    \item \textbf{Postcondizione:} il venditore visualizza il messaggio del fallimento.
    \item \textbf{Scenario principale:} il venditore clicca per salvare una modifica di una categoria ma l'azione fallisce, viene quindi visualizzato il relativo messaggio.
\end{itemize}

%50
\stepUserCase
\subsection{\valueUserCase - Visualizzazione errore titolo mancante}
\labelUserCase
\begin{itemize}
    \item \textbf{Attore primario:} venditore autenticato;
    \item \textbf{Descrizione:} il venditore vuole aggiungere o modificare un prodotto ma non inserisce il nome del prodotto, così viene visualizzato il relativo messaggio;
    \item \textbf{Precondizione:} il venditore deve trovarsi nella pagina per l'inserimento o la modifica di un prodotto;
    \item \textbf{Input:} il venditore ha cliccato sul pulsante per salvare l'inserimento o la modifica;
    \item \textbf{Postcondizione:} il venditore visualizza il messaggio per il titolo mancante.
    \item \textbf{Scenario principale:} il venditore clicca per salvare l'inserimento o la modifica di un prodotto ma non ha inserito il nome, viene quindi visualizzato il relativo messaggio.
\end{itemize}

%51
\stepUserCase
\subsection{\valueUserCase - Visualizzazione messaggio prodotto assente}
\labelUserCase
\begin{itemize}
    \item \textbf{Attore primario:} venditore autenticato;
    \item \textbf{Descrizione:} il venditore vuole filtrare i prodotti o cercare un prodotto in particolare, il quale però non è presente nel sistema;
    \item \textbf{Precondizione:} il venditore deve trovasi nella pagina dei prodotti;
    \item \textbf{Input:} il venditore ha cliccato per avviare una ricerca o selezionato un filtro;
    \item \textbf{Postcondizione:} il venditore visualizza il messaggio per il prodotto assente.
    \item \textbf{Scenario principale:} il venditore avvia la ricerca o il filtraggio, ma il prodotto non è presente nel sistema. Viene visualizzato il messaggio relativo.
\end{itemize}

%52
\stepUserCase
\subsection{\valueUserCase - Visualizzazione messaggio codice inesistente}
\labelUserCase
\begin{itemize}
    \item \textbf{Attore primario:} venditore autenticato;
    \item \textbf{Descrizione:} il venditore vuole filtrarevuole cercare un determinato ordine, il quale però non è presente nel sistema;
    \item \textbf{Precondizione:} il venditore deve trovasi nella pagina degli ordini;
    \item \textbf{Input:} il venditore ha cliccato per avviare una ricerca;
    \item \textbf{Postcondizione:} il venditore visualizza il messaggio per il codice inesistente.
    \item \textbf{Scenario principale:} il venditore avvia la ricerca, ma l'ordine non è presente nel sistema. Viene visualizzato il messaggio relativo.
\end{itemize}

%53
\stepUserCase
\subsection{\valueUserCase - Visualizzazione errore quantità non valida}
\labelUserCase
\begin{itemize}
    \item \textbf{Attore primario:} venditore autenticato;
    \item \textbf{Descrizione:} il venditore vuole aggiungere o modificare la quantità di un prodotto ma ha inserito una quantità non valida, così viene visualizzato il relativo messaggio;
    \item \textbf{Precondizione:} il venditore deve trovarsi nella pagina per l'inserimento o la modifica di un prodotto;
    \item \textbf{Input:} il venditore ha cliccato sul pulsante per salvare l'inserimento o la modifica;
    \item \textbf{Postcondizione:} il venditore visualizza il messaggio per la quantità non valida.
    \item \textbf{Scenario principale:} il venditore clicca per salvare l'inserimento o la modifica di un prodotto ma ha inserito una quantità non valida, viene quindi visualizzato il relativo messaggio.
\end{itemize}