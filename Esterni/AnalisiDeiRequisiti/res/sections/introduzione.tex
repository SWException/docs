\section{Introduzione} \label{_introduzione}
\subsection{Scopo del documento}
Lo scopo del documento è quello di fornire una lista completa e dettagliata dei \glock{casi d'uso} identificati
da un'attenta lettura del capitolato, seguita da un ragionamento collettivo con tutti i membri del gruppo.
Sono inoltre stati coinvolti i proponenti tramite un meeting, di cui sono riportate le informazioni nel \dext{Verbale Esterno 2020-12-28}.

\subsection{Riferimenti a glossario e documenti esterni}
Nella lettura di questo documento si incontreranno dei termini che possono risultare ambigui estratti dal contesto
del progetto, alla loro prima occorrenza saranno quindi contrassegnati con il pedice \textit{G} e riportati con la relativa definizione nel \dext{Glossario v1.0.0}. \\
I documenti che vengono menzionati sono seguiti dalla lettera \textit{D} a pedice.

\subsection{Riferimenti}
\subsubsection{Riferimenti normativi}
\begin{itemize}
    \item \dext{Norme di Progetto v.1.0.0}
    \item \href{https://www.math.unipd.it/~tullio/IS-1/2020/Progetto/C2.pdf}{Capitolato d'appalto C2}
    \item \dext{Verbale Esterno 2020-12-28}
    \item \dext{Verbale Esterno 2021-01-08}
\end{itemize}

\subsubsection{Riferimenti informativi}
\begin{itemize}
    \item \href{https://www.math.unipd.it/~tullio/IS-1/2020/Progetto/C2.pdf}{Capitolato d'appalto C2}
    \item \href{https://www.math.unipd.it/~tullio/IS-1/2020/Dispense/L07.pdf}{Analisi dei Requisiti - Slide del corso di Ingegneria del Software, A.A 2020/21}
    \item \href{https://www.math.unipd.it/~rcardin/swea/2021/Diagrammi%20Use%20Case_4x4.pdf}{Diagrammi dei casi d'uso - Slide del corso di Ingegneria del Software, A.A 2020/21}
    
\end{itemize}
