\appendix
\section{Riscontro dei rischi} \label{_riscontroDeiRischi}
\begin{center}
			\rowcolors{1}{lightest-grayest}{blue!20}
			\begin{longtable}{p{2cm}|p{3cm}|p{4cm}|p{4cm}|}
			\hline
			\rowcolor{lighter-grayer}
			\textbf{Codice} & \textbf{Periodo} & \textbf{Problema} & \textbf{Soluzione} \\
			\hline
			\endfirsthead

			\hline
			RST1 & Analisi dei requisiti & Per molti componenti del gruppo le tecnologie e i linguaggi da utilizzare per il progetto sono completamente nuovi. Nello specifico l'utilizzo di \LaTeX ha richiesto ad alcuni componenti uno studio da zero. & Tempo permettendo, è stato assegnato uno studio preliminare delle tecnologie attraverso la documentazione fornita dalle stesse in modo da arrivare all'inizio dello sviluppo con delle basi e dei concetti già chiari. L'inizializzazione dei primi documenti Latex è stata svolta dai componenti con più esperienza in modo da permettere agli altri membri di recuperare il gap. \\
			\hline
			RSO1 & Analisi dei requisiti & Alcuni membri del gruppo devono svolgere oltre al progetto di Ingegneria del Software anche altri esami e scadenze in altri corsi. & I compiti sono stati riassegnati per agevolare i membri del gruppo con difficoltà o con più esami da portare in parallelo. \\
			\hline
			RSO3 & Analisi dei requisiti & Alcune scadenze durante il periodo di Analisi dei Requisiti sono state stringenti. & Le scadenze successive hanno tenuto conto della mole di lavoro di ogni componente permettendo ad ognuno di avere delle tempistiche adeguate. \\
			\hline
			RSR1 & Analisi dei requisiti & Dopo la prima Analisi dei Requisiti il gruppo ha trovato alcune difficoltà nella ricerca dei casi d'uso e nelle feature da inserire all'interno del progetto proposto. & Dopo le difficoltà è stata richiesta una chiamata con il proponente che ha permesso al gruppo di chiarirsi le idee e di avere una visione chiara nel complesso del progetto. \\
			\hline
			RSR1 & Progettazione della \glock{Technology Baseline} & In seguito alle correzioni ricevute circa il materiale consegnato per la Revisione dei Requisiti il gruppo ha incontrato qualche difficoltà nella sistemazione dei casi d'uso & Un colloquio con il professore per approfondire le correzioni e una concentrazione da parte dei membri sul problema hanno condotto alla risoluzione dello stesso. \\
			\hline
			RSO1 & Progettazione della Technology Baseline & La mancata esperienza in merito ha causato dei leggeri ritardi nella definizione della technology baseline. & Il leggero ritardo è stato facilmente recuperato dividendo i membri in sottogruppi e assegnando a ciascun gruppo una tecnologia sulla quale concentrarsi.  \\
			\hline
			RST1 & Progettazione e codifica del \glock{Proof of Concept} & La mancata familiarità dei componenti con le tecnologie ha portato a delle difficoltà nell'implementazione del Proof of Concept & La formazione individuale e a gruppi, utilizzando Zoom come metodo di comunicazione, ha permesso a tutti di ottenere familiarità con le tecnologie . \\
			\hline
			RSP1 & Progettazione e codifica del Proof of Concept & Gli impegni accademici di alcuni componenti del gruppo, uniti ad alcuni malintesi nella comunicazione hanno portato ad alcuni attriti & Con una chiamata ognuno ha comunicato i propri impegni del prossimo futuro, in modo da pianificare al meglio gli incrementi successivi e non incorrere in ritardi o battibecchi. \\
			\rowcolor{white} 
			\caption{Tabella contenente i rischi incontrati}
	\end{longtable}
\end{center}