\section{Modello di sviluppo} \label{_modelloDiSviluppo}
\subsection{Motivazioni}
	Per poter produrre un software che rispetti le richieste del proponente bisogna adottare un modello di sviluppo in grado di effettuare la pianificazione in base alle caratteristiche richieste imponendo dei vincoli alla pianificazione stessa. La scelta del modello di sviluppo è un'attività molto importante perché pone dei vincoli ai componenti del gruppo, aiutandoli nella sviluppo e fornendo delle direttive precise per ogni periodo temporale. L'obbiettivo principale del gruppo è fornire un software di qualità, il quale deve riflettere l'organizzazione data dal modello scelto e il raggiungimento degli obbiettivi fissati dal modello stesso.
	\newline
	Partendo da queste considerazioni e valutando la natura del progetto è stato adottato il \textbf{modello di sviluppo incrementale}, il quale prevede lo sviluppo del prodotto tramite incrementi multipli e successivi. I rilasci hanno l’obbiettivo di
aggiungere funzionalità separate e accessorie a un sistema funzionante in cui sono presenti requisiti di base.
	\newline
	Nel modello di sviluppo incrementale i requisiti vengono classificati in base alla loro importanza strategica a livello di sistema. Viene fatta inizialmente una analisi preliminare per identificare i requisiti fondamentali rispetto a quelli opzionali. Il modello scelto prende in considerazione prima i requisiti più importanti, che sono da trattare fin dai primi incrementi, in modo da renderli chiari nel minor tempo possibile per creare un'infrastruttura stabile su cui sviluppare caratteristiche del sistema successive. Gli incrementi successivi coprono, quindi, requisiti meno importanti che hanno necessariamente più tempo per integrarsi col sistema.
\subsection{Vantaggi modello di sviluppo Incrementale}
	I vantaggi del modello di sviluppo incrementale sono i seguenti:
	\begin{itemize}
		\item ogni incremento produce valore aggiunto, rendendo disponibili un'insieme di funzionalità subito disponibili per l'utilizzo e chiarendo meglio i requisiti per gli incrementi successivi;
		\item le funzionalità principali vengono sviluppate all'inizio con i primi incrementi in modo da fornire una base solida su cui adottare i successivi incrementi;
		\item l'Analisi dei Requisiti e la Progettazione Architetturale vengono svolte una volta sola per stabilire fin da subito i requisiti fondamentali;
		\item vengono decisi preventivamente il numero di incrementi che il gruppo si pone come punto di riferimento in modo da assegnare ad ogni componente un obbiettivo specifico;
		\item gli errori in caso di un rilascio fallace sono facilmente individuabili e risolvibili in tempi stretti fornendo anche la possibilità di ritornare allo stato funzionale precedente;
	\end{itemize}

\subsection{Incrementi pianificati}

Si prevede di andare a svolgere \textbf{12 incrementi}, raggruppati in \textbf{3 fasi} in base agli obiettivi chiave del prodotto, col fine di poter integrare tutte le funzionalità richieste dal progetto. Di seguito si riporta una tabella riassuntiva degli incrementi con una breve descrizione sul relativo svolgimento.

\begin{center}
	\rowcolors{1}{blue!20}{lightest-grayest}
	\begin{longtable}{|p{2.5cm}|p{6.5cm}|p{6cm}|}
	\hline
	\rowcolor{blue!20}
	\textbf{Nome} & \textbf{Breve descrizione} & \textbf{Requisiti} \\
	\hline
	\endfirsthead
	\hline
    \multicolumn{3}{|c|}{\textit{Continua nella pagina successiva...}}\\
    \hline
    \endfoot
    \endlastfoot
	\hline
	\rowcolor{lighter-grayer} \multicolumn{3}{|c|}{\textbf{Progettazione e codifica del Proof of Concept e funzionalità essenziali}} \\ \hline 
	Incremento I	& Configurazione di \glock{Apache Kafka}. & RA-F-72 \\\hline
	\hline
	\rowcolor{lighter-grayer} \multicolumn{3}{|c|}{\textbf{Progettazione completa dell'architettura e implementazione delle funzionalità}} \\ \hline 
	Incremento V	& Implementazione della configurazione dinamica per un \glock{gateway}. & --- \\\hline
	\hline
	\rowcolor{lighter-grayer} \multicolumn{3}{|c|}{\textbf{Completamento dell'implementazione e raffinamento delle funzionalità}} \\ \hline 
	Incremento IX	& Implementazione della parte utente per la web app. & --- \\ \hline
	\caption{Riassunto degli incrementi pianificati}
	\end{longtable}
\end{center}
 %%La suddivisione degli incrementi si basa sulle \textit{funzionalità} che si vogliono portare a termine e che possono essere di interesse per il proponente. La lunghezza dei periodi è stata scelta in base al quantitativo di requisiti da convalidare e in base alle attività che devono essere svolte. Si fa notare che molti dei requisiti riportati possono essere soddisfatti pienamente solamente quando altre funzionalità vengono effettivamente implementate, poiché da esse sono \textit{dipendenti}. 
%%\newline 
%%Se necessario, saranno eseguiti degli aggiustamenti in base agli eventuali accorgimenti avanzati dal proponente o nel caso in cui alcune funzionalità richiedano più tempo di quello preventivato.