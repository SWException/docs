% !TeX root = ../../../main.tex
\section{Consuntivi di periodo} \label{_consuntivo}
In questa sezione viene indicato il consumo effettivo di ore per ogni periodo di lavoro svolto. Questo numero viene confrontato con quanto preventivato, ottenendo un bilancio che può essere:
\begin{itemize}
	\item \textbf{positivo}: se la spesa effettiva è minore di quella preventivata;
	\item \textbf{pari}: se la spesa effettiva e quella preventivata sono uguali;
	\item \textbf{negativo}:  se la spesa effettiva è maggiore di quella preventivata.
\end{itemize}

\subsection{Fase di analisi dei requisiti} \label{_consuntivoAnalisiDeiRequisiti}
Le ore di lavoro di questa fase sono considerate ore di investimento quindi non sono rendicontate.

\rowcolors{1}{lightest-grayest}{blue!20}
\begin{longtable}{|l|c|c|c|c|c|c|c|}
	\hline
	\rowcolor{lighter-grayer}
	\textbf{Ruolo}             & \textbf{Ore} & \textbf{Costo in €} \\
	\hline
	\endfirsthead

	\hline
	Responsabile               & 27           & 810,00              \\
	\hline
	\hline
	Amministratore             & 35(+2)       & 700,00(+40,00)      \\
	\hline
	\hline
	Analista                   & 70(+5)       & 1.750,00(+125,00)   \\
	\hline
	\hline
	Progettista                & -            & -                   \\
	\hline
	\hline
	Programmatore              & -            & -                   \\
	\hline
	\hline
	Verificatore               & 78(+2)       & 1.170,00(+30,00)    \\
	\hline
	\textbf{Totale Preventivo} & 210          & 4.430,00            \\
	\hline
	\hline
	\textbf{Totale Consuntivo} & 219          & 4.625,00            \\
	\hline
	\hline
	\textbf{Differenza}        & +9           & +195,00             \\
	\hline
	\rowcolor{white}
	\caption{Tabella contenente il consuntivo della fase di analisi dei requisiti}
\end{longtable}
\subsubsection{Conclusioni}
Come si evince dalla tabella sono state impiegate più ore di quanto preventivato nei ruoli di \textit{Amministratore}, \textit{Analista} e \textit{Verificatore} per i seguenti motivi:
\begin{itemize}
	\item \textbf{Amministratore di Progetto}: le \dext{NormeDiProgetto\_3.0.0} hanno subito durante questo periodo numerosi aggiornamenti che hanno fatto si che le ore necessarie a ricoprire questo ruolo aumentassero;
	\item \textbf{Analista}: la redazione dell'\dext{AnalisiDeiRequisiti\_3.0.0} ha richiesto più tempo del previsto;
	\item \textbf{Verificatore}: le ripetute modifiche subite dai documenti nominati ai punti precedenti hanno avuto come conseguenza anche l'aumento del tempo richiesto per questo ruolo.
\end{itemize}

\subsubsection{Preventivo a finire}
Il leggero aumento di risorse impiegate rispetto al consuntivo non è ritenuto un problema in quanto la fase di analisi non è rendicontata. Inoltre il surplus di tempo utilizzato in questo periodo ha permesso di gettare delle solide fondamenta per le fasi successive del progetto.


\subsection{Fase di consolidamento dei requisiti} \label{_consuntivoConsolidamentoDeiRequisiti}
Le ore di lavoro di questa fase sono considerate ore di investimento quindi non sono rendicontate.

\rowcolors{1}{lightest-grayest}{blue!20}
\begin{longtable}{|l|c|c|c|c|c|c|c|}
	\hline
	\rowcolor{lighter-grayer}
	\textbf{Ruolo}             & \textbf{Ore} & \textbf{Costo in €} \\
	\hline
	\endfirsthead

	\hline
	Responsabile               & 4           & 120,00              \\
	\hline
	\hline
	Amministratore             & 5       & 100,00      \\
	\hline
	\hline
	Analista                   & 13       & 325,00   \\
	\hline
	\hline
	Progettista                & -            & -                   \\
	\hline
	\hline
	Programmatore              & -            & -                   \\
	\hline
	\hline
	Verificatore               & 13       & 195,00    \\
	\hline
	\textbf{Totale Preventivo} & 35          & 740,00            \\
	\hline
	\hline
	\textbf{Totale Consuntivo} & 35          & 740,00            \\
	\hline
	\hline
	\textbf{Differenza}        & -           & -            \\
	\hline
	\rowcolor{white}
	\caption{Tabella contenente il consuntivo della fase di consolidamento dei requisiti}
\end{longtable}
\subsubsection{Conclusioni}
Come si evince dalla tabella sono state impiegate esattamente lo stesso numero di ore preventivate. Ciò è stato possibile grazie alla cura nell'analisi e la verifica adottate nella fase precedente. Il piccolo surplus di risorse impiegate nella fase di analisi ha permesso di lavorare più agevolmente e velocemente in questa fase.

\subsubsection{Preventivo a finire}
Poiché le risorse utilizzate in questa fase corrispondono esattamente a quanto preventivato e visto che le due fasi concluse non verranno rendicontate, non si ritiene necessario effettuare alcuna modifica al preventivo a finire.


\subsection{Fase di progettazione della technology baseline} \label{_consuntivoTB}
Le ore di lavoro di questa fase sono relative ai due periodi descritti in \S\ref{_pianificazioneProgettazioneTechnologyBaseline}.

\rowcolors{1}{lightest-grayest}{blue!20}
\begin{longtable}{|l|c|c|c|c|c|c|c|}
	\hline
	\rowcolor{lighter-grayer}
	\textbf{Ruolo}             & \textbf{Ore} & \textbf{Costo in €} \\
	\hline
	\endfirsthead

	\hline
	Responsabile               & 6(-1)           & 180,00(-30)              \\
	\hline
	\hline
	Amministratore             & 15       & 300,00      \\
	\hline
	\hline
	Analista                   & 25       & 625,00   \\
	\hline
	\hline
	Progettista                & 22            & 484,00              \\
	\hline
	\hline
	Programmatore              & -            & -                   \\
	\hline
	\hline
	Verificatore               & 16(+1)       & 240,00(+15,00)    \\
	\hline
	\textbf{Totale Preventivo} & 84          & 1.844,00            \\
	\hline
	\hline
	\textbf{Totale Consuntivo} & 84          & 1.829,00            \\
	\hline
	\hline
	\textbf{Differenza}        & -           & -15,00           \\
	\hline
	\rowcolor{white}
	\caption{Tabella contenente il consuntivo della fase di progettazione della technology baseline}
\end{longtable}
\subsubsection{Conclusioni}
Come si evince dalla tabella sono state impiegate esattamente lo stesso numero di ore preventivate. Tuttavia è stato necessario effettuare una modifica sulla distribuzione delle ore tra i ruoli. In particolare:
\begin{itemize}
	\item \textbf{Responsabile}: i compiti destinati a questa figura  sono stati svolti più velocemente di quanto preventivato. Pertanto si è scelto di utilizzare l'ora avanzata dall'impiego di questo ruolo in altre attività;
	\item  \textbf{Verificatore}: il tempo necessario a validare le numerose modifiche effettuate durante la progettazione, causate dalla scarsa esperienza dei membri del gruppo con le tecnologie utilizzate, si è rivelato insufficiente. Pertanto l'ora di avanzo ottenuta per quanto spiegato al punto sopra è stata assegnata per attività di verifica.
\end{itemize}

\subsubsection{Preventivo a finire}
Poiché il monte ore utilizzato in questa fase corrisponde a quanto preventivato, vista la scarsa differenza tra preventivo e consuntivo (di € 15,00), che evidenzia in ogni caso un risparmio del gruppo, ritenuto che le risorse risparmiate potranno essere investite nella fase successiva, non si ritiene necessario effettuare alcuna modifica sostanziale al preventivo.



\subsection{Fase di Progettazione e codifica del Proof of Concept} \label{_consuntivoPoC}
Le ore di lavoro di questa fase sono relative ai tre incrementi descritti in \S\ref{_pianificazioneCodificaPoC}.
Le sezioni seguenti riportano i consuntivi di periodo relativi ai diversi incrementi di questa fase.

\subsubsection{I incremento} \label{_consuntivoPoC1}
\rowcolors{1}{lightest-grayest}{blue!20}
\begin{longtable}{|l|c|c|c|c|c|c|c|}
	\hline
	\rowcolor{lighter-grayer}
	\textbf{Ruolo}             & \textbf{Ore} & \textbf{Costo in €} \\
	\hline
	\endfirsthead

	\hline
	Responsabile               & 7           & 210,00              \\
	\hline
	\hline
	Amministratore             & 11(+1)       & 220,00(+20,00)      \\
	\hline
	\hline
	Analista                   & -       & -   \\
	\hline
	\hline
	Progettista                & 16(-1)            & 352,00(-22,00)              \\
	\hline
	\hline
	Programmatore              & 10            & 150,00                   \\
	\hline
	\hline
	Verificatore               & 12       & 180,00    \\
	\hline
	\textbf{Totale Preventivo} & 56          & 1.114,00            \\
	\hline
	\hline
	\textbf{Totale Consuntivo} & 56          & 1.112,00            \\
	\hline
	\hline
	\textbf{Differenza}        & -           & -2,00           \\
	\hline
	\rowcolor{white}
	\caption{Tabella contenente il consuntivo del I incremento}
\end{longtable}
\paragraph{Conclusioni}
Come si evince dalla tabella sono state impiegate esattamente lo stesso numero di ore preventivate. Tuttavia è stato necessario effettuare una modifica sulla distribuzione delle ore tra i ruoli. In particolare:
begin\begin{itemize}
	\item \textbf{Progettista}: è stata sottratta un'ora a questo ruolo in quanto il lavoro di progettazione svolto nella fase di progettazione della technology baseline si è rivelato utile a ottenere un risparmio orario in questo incremento;
	\item \textbf{Amministratore}: si è scelto di investire le ore avanzate dalla progettazione nel ruolo di amministratore. Questo ha permesso di aggiornare efficacemente le \dext{NormeDiProgetto\_3.0.0} in modo da normare rigidamente il processo di codifica appena iniziato.
\end{itemize} 

\paragraph{Preventivo a finire}
Poiché il monte ore utilizzato in questa fase corrisponde a quanto preventivato, vista la scarsa differenza tra preventivo e consuntivo (di € 2,00), che evidenzia in ogni caso un risparmio del gruppo rispetto al preventivo, ritenuto che le risorse risparmiate potranno essere investite nei successivi incrementi di questa fase, non si ritiene necessario effettuare alcuna modifica sostanziale al preventivo.

\subsubsection{II incremento}\label{_consuntivoPoC2}
\rowcolors{1}{lightest-grayest}{blue!20}
\begin{longtable}{|l|c|c|c|c|c|c|c|}
	\hline
	\rowcolor{lighter-grayer}
	\textbf{Ruolo}             & \textbf{Ore} & \textbf{Costo in €} \\
	\hline
	\endfirsthead

	\hline
	Responsabile               & 6           & 180,00              \\
	\hline
	\hline
	Amministratore             & 6       & 120,00      \\
	\hline
	\hline
	Analista                   & 3       & 75,00   \\
	\hline
	\hline
	Progettista                & 9(-1)            & 198,00(-22,00)              \\
	\hline
	\hline
	Programmatore              & 22(+1)            & 330,00(+15,00)                   \\
	\hline
	\hline
	Verificatore               & 17       & 255,00    \\
	\hline
	\textbf{Totale Preventivo} & 63          & 1.165,00            \\
	\hline
	\hline
	\textbf{Totale Consuntivo} & 63          & 1.158,00            \\
	\hline
	\hline
	\textbf{Differenza}        & -           & -7,00           \\
	\hline
	\rowcolor{white}
	\caption{Tabella contenente il consuntivo del II incremento}
\end{longtable}
\paragraph{Conclusioni}
Come si evince dalla tabella sono state impiegate esattamente lo stesso numero di ore preventivate. Tuttavia è stato necessario effettuare una modifica sulla distribuzione delle ore tra i ruoli. In particolare:
begin\begin{itemize}
	\item \textbf{Progettista}: è stata sottratta un'ora a questo ruolo in quanto il lavoro di progettazione svolto nella fase di progettazione della technology baseline si è rivelato utile a ottenere un risparmio orario in questo incremento;
	\item \textbf{Programmatore}: si è scelto di investire le ore avanzate dalla progettazione nel ruolo di programmatore. L'implementazione delle tecnologie utilizzate per il backend, infatti, ha richiesto più tempo di quanto preventivato.
\end{itemize} 

\paragraph{Preventivo a finire}
Poiché il monte ore utilizzato in questa fase corrisponde a quanto preventivato, vista la scarsa differenza tra preventivo e consuntivo (di € 7,00), che evidenzia in ogni caso un risparmio del gruppo rispetto al preventivo, ritenuto che le risorse risparmiate potranno essere investite nei successivi incrementi di questa fase, non si ritiene necessario effettuare alcuna modifica sostanziale al preventivo.
\newpage
\subsubsection{III incremento}\label{_consuntivoPoC3}
\rowcolors{1}{lightest-grayest}{blue!20}
\begin{longtable}{|l|c|c|c|c|c|c|c|}
	\hline
	\rowcolor{lighter-grayer}
	\textbf{Ruolo}             & \textbf{Ore} & \textbf{Costo in €} \\
	\hline
	\endfirsthead

	\hline
	Responsabile               & 4(+1)           & 120,00(+30,00)              \\
	\hline
	\hline
	Amministratore             & 1       & 20,00      \\
	\hline
	\hline
	Analista                   & 5       & 125,00   \\
	\hline
	\hline
	Progettista                & 15            & 330,00              \\
	\hline
	\hline
	Programmatore              & 14(-2)            & 210,00(-30,00)                   \\
	\hline
	\hline
	Verificatore               & 10(+1)       & 150,00(+15,00)    \\
	\hline
	\textbf{Totale Preventivo} & 49          & 940,00            \\
	\hline
	\hline
	\textbf{Totale Consuntivo} & 49          & 955,00            \\
	\hline
	\hline
	\textbf{Differenza}        & -           & +15,00           \\
	\hline
	\rowcolor{white}
	\caption{Tabella contenente il consuntivo del III incremento}
\end{longtable}
\paragraph{Conclusioni}
Come si evince dalla tabella sono state impiegate esattamente lo stesso numero di ore preventivate. Tuttavia è stato necessario effettuare una modifica sulla distribuzione delle ore tra i ruoli. In particolare:
begin\begin{itemize}
	\item \textbf{Progettista}: sono state sottratte due ore a questo ruolo in quanto il lavoro di progettazione svolto nella fase di progettazione della technology baseline e nei due incrementi precedenti si è rivelato utile a ottenere un risparmio orario in questo incremento;
	\item \textbf{Programmatore}: si è scelto di investire le ore avanzate dalla progettazione nel ruolo di programmatore. L'ultimazione del codice del frontend, infatti, ha richiesto più tempo di quanto preventivato;
	\item \textbf{Responsabile}: è stata necessaria un'ora in più per aggiustare la pianificazione delle fasi future del progetto, visto quando riscontrato fino a questo momento.
\end{itemize} 

\paragraph{Preventivo a finire}
Poiché il monte ore utilizzato in questa fase corrisponde a quanto preventivato, vista la scarsa differenza tra preventivo e consuntivo (di € 15,00), pur costituendo questa uno sforamento rispetto al preventivo, non si ritiene necessario effettuare alcuna modifica sostanziale al preventivo. Inoltre, l'eccesso nel consumo di risorse in questo incremento è totalmente ammortizzato dal risparmio accumulato dagli incrementi precedenti, come è possibile evincere da \S\ref{riepilogoPAF}.

\subsection{Fase di progettazione completa e prima implementazione di base} \label{_consuntivoPoC}
Le ore di lavoro di questa fase sono relative ai quattro incrementi descritti in \S\ref{_pianificazioneProgettazioneCompletaImplementazione}.
Le sezioni seguenti riportano i consuntivi di periodo relativi ai diversi incrementi di questa fase.


%% IV incremento
\subsubsection{IV incremento}\label{_consuntivoImp1}
\rowcolors{1}{lightest-grayest}{blue!20}
\begin{longtable}{|l|c|c|c|c|c|c|c|}
	\hline
	\rowcolor{lighter-grayer}
	\textbf{Ruolo}             & \textbf{Ore} & \textbf{Costo in €} \\
	\hline
	\endfirsthead

	\hline
	Responsabile               & 5           & 150,00 \\
	\hline
	\hline
	Amministratore             & 5       & 100,00      \\
	\hline
	\hline
	Analista                   & 2 (+2)       & 50,00 (+50,00)   \\
	\hline
	\hline
	Progettista                & 13 (-2)            & 286,00 (-44,00)              \\
	\hline
	\hline
	Programmatore              & 11            & 165,00                   \\
	\hline
	\hline
	Verificatore               & 13       & 195,00    \\
	\hline
	\textbf{Totale Preventivo} & 49          & 940,00           \\
	\hline
	\hline
	\textbf{Totale Consuntivo} & 49          & 946,00            \\
	\hline
	\hline
	\textbf{Differenza}        & -           & +6,00           \\
	\hline
	\rowcolor{white}
	\caption{Tabella contenente il consuntivo del IV incremento}
\end{longtable}
\paragraph{Conclusioni}
Come si evince dalla tabella sono state impiegate esattamente lo stesso numero di ore preventivate. Tuttavia è stato necessario effettuare una modifica sulla distribuzione delle ore tra i ruoli. In particolare:
begin\begin{itemize}
	\item \textbf{Progettista}: sono state sottratte due ore a questo ruolo in quanto il lavoro di progettazione svolto nella fase di progettazione della technology baseline e negli incrementi precedenti si è rivelato utile a ottenere un risparmio orario in questo incremento;
	\item \textbf{Analista}: si è scelto di investire le ore avanzate dalla progettazione nel ruolo di analista, fondamentale ad ultimare l'attività di analisi dei requisiti e a definire gli ultimi dettagli delle norme di progetto del gruppo.
\end{itemize} 

\paragraph{Preventivo a finire}
Poiché il monte ore utilizzato in questa fase corrisponde a quanto preventivato, vista la scarsa differenza tra preventivo e consuntivo (di € 6,00), pur costituendo questa uno sforamento rispetto al preventivo, non si ritiene necessario effettuare alcuna modifica sostanziale al preventivo.Tale scelta è motivata anche dal fatto che tutti i requisiti obbligatori pianificati per questo incremento (\S\ref{_incrementi}) sono stati soddisfatti. Inoltre, l'eccesso nel consumo di risorse in questo incremento è totalmente ammortizzato dal risparmio accumulato dagli incrementi precedenti, come è possibile evincere da \S\ref{riepilogoPAF}.

%% V incremento
\subsubsection{V incremento}\label{_consuntivoImp2}
\rowcolors{1}{lightest-grayest}{blue!20}
\begin{longtable}{|l|c|c|c|c|c|c|c|}
	\hline
	\rowcolor{lighter-grayer}
	\textbf{Ruolo}             & \textbf{Ore} & \textbf{Costo in €} \\
	\hline
	\endfirsthead

	\hline
	Responsabile               & 1           & 30,00    \\
	\hline
	\hline
	Amministratore             & 0       & -      \\
	\hline
	\hline
	Analista                   & 0       & -   \\
	\hline
	\hline
	Progettista                & 6            & 132,00              \\
	\hline
	\hline
	Programmatore              & 11            & 165,00                  \\
	\hline
	\hline
	Verificatore               & 11       & 165,00    \\
	\hline
	\textbf{Totale Preventivo} & 28          & 462,00            \\
	\hline
	\hline
	\textbf{Totale Consuntivo} & 28          & 462,00            \\
	\hline
	\hline
	\textbf{Differenza}        & -           & -           \\
	\hline
	\rowcolor{white}
	\caption{Tabella contenente il consuntivo del V incremento}
\end{longtable}
\paragraph{Conclusioni}
Come si evince dalla tabella sono state impiegate esattamente lo stesso numero di ore preventivate e anche la pianificazione dei ruoli è rimasta invariata. Si ritiene che questa perfetta aderenza al preventivo sia dovuta, oltre che alla scarsa lunghezza temporale dell'incremento, anche all'acquisizione di esperienza da parte dei membri sulle tecnologie e sull'architettura del sistema da realizzare, acquisita negli incrementi precedenti.

\paragraph{Preventivo a finire}
Poiché il monte ore utilizzato e la pianificazione dei ruoli in questo periodo corrisponde a quanto preventivato e, vista la scarsa differenza tra preventivo e consuntivo accumulata finora, non si ritiene necessario effettuare alcuna modifica sostanziale al preventivo.Tale scelta è motivata anche dal fatto che tutti i requisiti obbligatori pianificati per questo incremento (\S\ref{_incrementi}) sono stati soddisfatti. Inoltre, come è possibile evincere da \S\ref{riepilogoPAF}, il preventivo a finire prevede un risparmio rispetto a quanto preventivato inizialmente.

%% VI incremento
\subsubsection{VI incremento}\label{_consuntivoImp3}
\rowcolors{1}{lightest-grayest}{blue!20}
\begin{longtable}{|l|c|c|c|c|c|c|c|}
	\hline
	\rowcolor{lighter-grayer}
	\textbf{Ruolo}             & \textbf{Ore} & \textbf{Costo in €} \\
	\hline
	\endfirsthead

	\hline
	Responsabile               & 1           & 30,00              \\
	\hline
	\hline
	Amministratore             & 2       & 40,00      \\
	\hline
	\hline
	Analista                   & 4       & 100,00   \\
	\hline
	\hline
	Progettista                & 9            & 198,00              \\
	\hline
	\hline
	Programmatore              & 8            & 120,00                   \\
	\hline
	\hline
	Verificatore               & 4       & 60,00    \\
	\hline
	\textbf{Totale Preventivo} & 28          & 548,00            \\
	\hline
	\hline
	\textbf{Totale Consuntivo} & 28          & 548,00            \\
	\hline
	\hline
	\textbf{Differenza}        & -           & -           \\
	\hline
	\rowcolor{white}
	\caption{Tabella contenente il consuntivo del VI incremento}
\end{longtable}
\paragraph{Conclusioni}
Come si evince dalla tabella sono state impiegate esattamente lo stesso numero di ore preventivate e anche la pianificazione dei ruoli è rimasta invariata. Si ritiene che questa perfetta aderenza al preventivo sia dovuta come per l'incremento precedente, oltre che alla scarsa lunghezza temporale dell'incremento, anche all'acquisizione di esperienza da parte dei membri sulle tecnologie e sull'architettura del sistema da realizzare, acquisita negli incrementi precedenti che ha permesso di procedere in maniera lineare.

\paragraph{Preventivo a finire}
Poiché il monte ore utilizzato e la pianificazione dei ruoli in questo periodo corrisponde a quanto preventivato e, vista la scarsa differenza tra preventivo e consuntivo accumulata finora, non si ritiene necessario effettuare alcuna modifica sostanziale al preventivo.Tale scelta è motivata anche dal fatto che tutti i requisiti obbligatori pianificati per questo incremento (\S\ref{_incrementi}) sono stati soddisfatti. Inoltre, come è possibile evincere da \S\ref{riepilogoPAF}, il preventivo a finire prevede un risparmio rispetto a quanto preventivato inizialmente.

%% VII incremento
\subsubsection{VII incremento}\label{_consuntivoImp4}
\rowcolors{1}{lightest-grayest}{blue!20}
\begin{longtable}{|l|c|c|c|c|c|c|c|}
	\hline
	\rowcolor{lighter-grayer}
	\textbf{Ruolo}             & \textbf{Ore} & \textbf{Costo in €} \\
	\hline
	\endfirsthead

	\hline
	Responsabile               & 3(+1)           & 90,00(+30,00)              \\
	\hline
	\hline
	Amministratore             & 4       & 80,00      \\
	\hline
	\hline
	Analista                   & 2       & 50,00   \\
	\hline
	\hline
	Progettista                & 3(-2)            & 66,00(-44,00)              \\
	\hline
	\hline
	Programmatore              & 19         & 285,00                   \\
	\hline
	\hline
	Verificatore               & 18(+1)       & 270,00(+15,00)    \\
	\hline
	\textbf{Totale Preventivo} & 49          & 840,00            \\
	\hline
	\hline
	\textbf{Totale Consuntivo} & 49          & 841,00            \\
	\hline
	\hline
	\textbf{Differenza}        & -           & +1,00           \\
	\hline
	\rowcolor{white}
	\caption{Tabella contenente il consuntivo del VII incremento}
\end{longtable}
\paragraph{Conclusioni}
Come si evince dalla tabella sono state impiegate esattamente lo stesso numero di ore preventivate. Tuttavia è stato necessario effettuare una modifica sulla distribuzione delle ore tra i ruoli. In particolare:
begin\begin{itemize}
	\item \textbf{Progettista}: sono state sottratte due ore a questo ruolo in quanto il lavoro di progettazione svolto nella fase di progettazione della technology baseline e negli incrementi precedenti si è rivelato utile a ottenere un risparmio orario in questo incremento;
	\item \textbf{Verificatore}: si è scelto di investire un'ora avanzate dalla progettazione nel ruolo di verificatore. Si è ritenuto fondamentale impiegare il tempo risparmiato in attività di verifica, sia di prodotti documentali che software;
	\item \textbf{Responsabile}: è stata necessaria un'ora in più per aggiustare la pianificazione delle fasi future del progetto, visto il leggero ritardo causato dalla durata di questo incremento, conclusosi qualche giorno dopo rispetto a quanto pianificato inizialmente.
\end{itemize} 

\paragraph{Preventivo a finire}
Poiché il monte ore utilizzato in questa fase corrisponde a quanto preventivato, vista la minima differenza tra preventivo e consuntivo (un risparmio di € 1,00), non si ritiene necessario effettuare alcuna modifica sostanziale al preventivo. Inoltre, l'eccesso nel consumo di risorse in questo incremento è totalmente ammortizzato dal risparmio accumulato dagli incrementi precedenti, come è possibile evincere da \S\ref{riepilogoPAF}.

%% PaF riepilogo
\subsubsection{Riepilogo del preventivo a finire}\label{riepilogoPAF}

La tabella seguente riporta il preventivo a finire, alla luce dei periodi consuntivati.

\rowcolors{1}{lightest-grayest}{blue!20}
\begin{longtable}{|p{6cm}|c|c|c|}
	\hline
	\rowcolor{lighter-grayer}
	\textbf{Periodo}             & \textbf{Preventivo in €} & \textbf{Consuntivo in €} & \textbf{Differenza in €} \\
	\hline
	\endfirsthead

	\hline
	Analisi dei requisiti            & 4.430,00        & 4.625,00  & +195,00   \\ 
	\hline
	\hline
	Consolidamento dei requisiti     & 740,00        & 740,00    & -  \\ 
	\hline
	\hline
	Progettazione Technology Baseline      & 1.844,00       & 1.829,00  & -15,00  \\
	\hline
	\hline
	Progettazione e codifica Proof of Concept - I incremento      & 1.114,00            & 1.112,00   & -2,00          \\
	\hline
	\hline
	Progettazione e codifica Proof of Concept - II incremento     & 1.165,00            & 1.158,00 & -7,00           \\
	\hline
	\hline
	Progettazione e codifica Proof of Concept - III incremento    & 940,00       & 955,00  & +15,00   \\
	\hline
	\hline
	Progettazione completa e prima implementazione - IV incremento    & 940,00       & 946,00  & +6,00   \\
	\hline
	\hline
	Progettazione completa e prima implementazione - V incremento    & 462,00       & 462,00  & -   \\
	\hline
	\hline
	Progettazione completa e prima implementazione - VI incremento    & 548,00       & 548,00  & -   \\
	\hline
	\hline
	Progettazione completa e prima implementazione - VII incremento    & 840,00       & 841,00  & +1,00   \\
	\hline
	\textbf{Totale } & 18.469,00     & 18.662,00   &  +193,00       \\
	\hline
	\hline
	\textbf{Totale rendicontato} & 13.299,00       & 13.297,00     & -2,00     \\
	\hline
	\hline
	\rowcolor{white}
	\caption{Tabella contenente il preventivo a finire}
\end{longtable}

Come si può notare, nonostante il totale non rendicontato si discosti di € 193 da quanto preventivato, il totale rendicontato si riporta molto vicino alla stima iniziale del preventivo. Lo scarto ammonta ad € 2,00 rispetto al preventivo, cifra che risulta costituire un risparmio del gruppo rispetto alle previsioni. \\
Dunque:
\begin{itemize}
	\item dato che in seguito ai periodi non rendicontati, in cui, come spiegato in \S\ref{_consuntivoAnalisiDeiRequisiti}, le attività che hanno portato allo sforamento hanno contribuito a gettare delle solide basi per il progetto, non vi sono stati altri sostanziali scostamenti dal punto di vista economico rispetto al preventivo;
	\item visto che lo scostamento temporale di pochi giorni rispetto alla data di consegna originariamente pianificata per il materiale in ingresso alla Revisione di Qualifica è dovuto più alla necessità di tutti i membri del gruppo di investire più tempo nello studio in vista del primo appello scritto di Ingegneria del Software piuttosto che da difficoltà intrinseche al progetto;
	\item vista la natura "a sportello" assunta quest'anno delle revisioni di avanzamento, grazie alla quale un leggero ritardo nella consegna del materiale, previo accordo con il docente Vardanega, non porta ad alcuna penalizzazione;  
	\item visto il totale soddisfacimento dei requisiti pianificati per gli incrementi trascorsi (\S\ref{_incrementi}).
\end{itemize}
  non si ritiene necessario apportare modifiche al preventivo in \S\ref{_preventivo} per quanto riguarda il monte ore e la distribuzione dei ruoli. \\
  La pianificazione temporale degli incrementi futuri è stata invece modificata, traslando in avanti ogni incremento di otto giorni, ovvero il ritardo accumulato rispetto alla data pianificata per la consegna del materiale in ingresso alla Revisione di Qualifica. Infatti, poiché le attività stanno procedendo senza intoppi il gruppo non ha ragione di pensare che negli incrementi futuri verrà accumulato ulteriore ritardo. 