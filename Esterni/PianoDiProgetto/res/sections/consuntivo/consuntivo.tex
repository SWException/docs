% !TeX root = ../../../main.tex
\section{Consuntivi di periodo} \label{_consuntivo}
In questa sezione viene indicato il consumo effettivo di ore per ogni periodo di lavoro svolto. Questo numero viene confrontato con quanto preventivato, ottenendo un bilancio che può essere:
\begin{itemize}
	\item \textbf{positivo}: se la spesa effettiva è minore di quella preventivata;
	\item \textbf{pari}: se la spesa effettiva e quella preventivata sono uguali;
	\item \textbf{negativo}:  se la spesa effettiva è maggiore di quella preventivata.
\end{itemize}

\subsection{Periodo di analisi dei requisiti} \label{_consuntivoAnalisiDeiRequisiti}
Le ore di lavoro di questa fase sono considerate ore di investimento quindi non sono rendicontate.

\rowcolors{1}{lightest-grayest}{blue!20}
\begin{longtable}{|l|c|c|c|c|c|c|c|}
	\hline
	\rowcolor{lighter-grayer}
	\textbf{Ruolo}             & \textbf{Ore} & \textbf{Costo in €} \\
	\hline
	\endfirsthead

	\hline
	Responsabile               & 27           & 810,00              \\
	\hline
	\hline
	Amministratore             & 35(+2)       & 700,00(+40,00)      \\
	\hline
	\hline
	Analista                   & 70(+5)       & 1.750,00(+125,00)   \\
	\hline
	\hline
	Progettista                & -            & -                   \\
	\hline
	\hline
	Programmatore              & -            & -                   \\
	\hline
	\hline
	Verificatore               & 78(+2)       & 1.170,00(+30,00)    \\
	\hline
	\textbf{Totale Preventivo} & 210          & 4.430,00            \\
	\hline
	\hline
	\textbf{Totale Consuntivo} & 219          & 4.625,00            \\
	\hline
	\hline
	\textbf{Differenza}        & +9           & +195,00             \\
	\hline
	\rowcolor{white}
	\caption{Tabella contenente il consuntivo della fase di analisi dei requisiti}
\end{longtable}
\subsubsection{Conclusioni}
Come si evince dalla tabella sono state impiegate più ore di quanto preventivato nei ruoli di \textit{Amministratore}, \textit{Analista} e \textit{Verificatore} per i seguenti motivi:
\begin{itemize}
	\item \textbf{Amministratore di Progetto}: le \dext{NormeDiProgetto\_1.0.0} hanno subito durante questo periodo numerosi aggiornamenti che hanno fatto si che le ore necessarie a ricoprire questo ruolo aumentassero;
	\item \textbf{Analista}: la redazione dell'\dext{AnalisiDeiRequisiti\_1.0.0} ha richiesto più tempo del previsto;
	\item \textbf{Verificatore}: le ripetute modifiche subite dai documenti nominati ai punti precedenti hanno avuto come conseguenza anche l'aumento del tempo richiesto per questo ruolo.
\end{itemize}

\subsubsection{Preventivo a finire}
Il leggero aumento di risorse impiegate rispetto al consuntivo non è ritenuto un problema in quanto la fase di analisi non è rendicontata. Inoltre il surplus di tempo utilizzato in questo periodo ha permesso di gettare delle solide fondamenta per le fasi successive del progetto.
