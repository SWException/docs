\section{Preventivo}
In questa sezione viene riportato il preventivo per il progetto, suddividendo le ore di lavoro e i costi per le varie fasi previste da %ref{_pianificazione}.
Le fasi di analisi e di consolidamento dei requisiti, tuttavia, non vengono tenute in considerazione nel calcolo del preventivo finale.
La suddivisione oraria dei ruoli per ogni membro del gruppo dovrà rispettare le seguenti regole:
		\begin{itemize}
		\item ogni componente dovrà ricoprire almeno una volta ogni ruolo per almeno otto ore, in modo che tutti i componenti apprendano le attività e le responsabilità legate ai diversi ruoli;
		\item le ore lavorative per ogni fase dovranno essere le stesse per ogni componente, per far si che tutti apportino lo stesso contributo al progetto, senza alcuna differenza.
	\end{itemize}
	 Per riportare i ruoli nelle tabelle, essi sono abbreviati con le seguenti sigle:
			\begin{itemize}
			\item\textbf{Re:} responsabile;
			\item\textbf{Am:} amministratore;
			\item\textbf{An:} analista;
			\item\textbf{Pg:} progettista;
			\item\textbf{Pr:} programmatore;
			\item\textbf{Ve:} verificatore.
		\end{itemize}
	
	\subsection{Fase di analisi dei requisiti}
		\subsubsection{Prospetto orario}
			Durante la fase di analisi dei requisiti la distribuzione oraria sarà la seguente:
			
			\rowcolors{1}{lightest-grayest}{blue!20}
			\begin{longtable}{|l|c|c|c|c|c|c|c|}
				\hline
				\rowcolor{lighter-grayer}
				\textbf{Nome} & \textbf{Re} & \textbf{Am} & \textbf{An} & \textbf{Pg}  & \textbf{Pr}   & \textbf{Ve} & \textbf{Totale} \\
				\hline
				\endfirsthead
				
				\hline
				Marco Canovese & 0 & 10 & 8 & 0 & 0 & 12 & 30\\
				\hline
				\hline
				Nicole Davanzo & 8 & 1 & 12 & 0 & 0 & 9 & 30\\
				\hline
				\hline
				Ivan Furlan & 0 & 7 & 11 & 0 & 0 & 12 & 30\\
				\hline
				\hline
				Gianmarco Guazzo & 4 & 6 & 11 & 0 & 0 & 9 & 30\\
				\hline
				\hline
				Stefano Lazzaroni & 8 & 0 & 12 & 0 & 0 & 10 & 30\\
				\hline
				\hline
				Francesco Trolese & 7 & 5 & 7 & 0 & 0 & 11 & 30\\
				\hline
				\hline
				Michele Veronesi & 0 & 6 & 9 & 0 & 0 & 15 & 30\\
				\hline 
				\hline
				Totale & 27 & 35 & 70 & 0 & 0 & 78 & 210\\
				\hline 
				\caption{Tabella contenente il prospetto orario preventivato per la fase di analisi dei requisiti}
			\end{longtable}

		
			La tabella può essere riassunta nel seguente istogramma:
		
			% \begin{figure}[H]
			% 	\centering
			% 	\includegraphics[width=0.8\linewidth]{./images/preventivo/analisi1.png}
			% 	\caption{Diagramma ore/ruolo componenti nella fase di analisi dei requisiti}
			% 	\label{fig:diagramma suddivione ruoli fase analisi dei requisiti}
			% \end{figure}
		
			\subsubsection{Prospetto economico}
			In base al prospetto orario, quello economico sarà il seguente: 
			
			\rowcolors{1}{white}{blue!20}
			\begin{longtable}{|l|c|c|c|c|c|c|c|}
				\hline
				\rowcolor{lighter-grayer}
				\textbf{Ruolo} & \textbf{Ore} & \textbf{Costo in €} \\
				\hline
				\endfirsthead
				
				\hline
				Responsabile & 27 & 810,00\\
				\hline
				\hline
				Amministratore & 35 & 700,00\\
				\hline
				\hline
				Analista & 70 & 1.750,00\\
				\hline
				\hline
				Progettista & - & -\\
				\hline
				\hline
				Programmatore & - & -\\
				\hline
				\hline
				Verificatore & 78 & 1.170,00\\
				\hline
				\textbf{Totale} & 210 & 4.430,00\\
				\hline
				\caption{Tabella contenente il prospetto economico in riferimento al prospetto orario}
			\end{longtable}
			\pagebreak
		
			La tabella può essere riassunta nel seguente areogramma:
			% \begin{figure}[H]
			% 	\centering
			% 	\includegraphics[width=0.8\linewidth]{./images/preventivo/analisi2.png}
			% 	\caption{Diagramma percentuale ore/ruolo nella fase di analisi dei requisiti}
			% 	\label{fig:diagramma costi ruolo fase analisi dei requisiti}
            % \end{figure}

            \subsection{Fase di consolidamento dei requisiti}
            \subsubsection{Prospetto orario}
			Durante la fase di consolidamento dei requisiti la distribuzione oraria sarà la seguente:
			
			\rowcolors{1}{lightest-grayest}{blue!20}
			\begin{longtable}{|l|c|c|c|c|c|c|c|}
				\hline
				\rowcolor{lighter-grayer}
				\textbf{Nome} & \textbf{Re} & \textbf{Am} & \textbf{An} & \textbf{Pg}  & \textbf{Pr}   & \textbf{Ve} & \textbf{Totale} \\
				\hline
				\endfirsthead
				
				\hline
				Marco Canovese & 2 & 0 & 3 & 0 & 0 & 0 & 5\\
				\hline
				\hline
				Nicole Davanzo & 0 & 0 & 3 & 0 & 0 & 2 & 5\\
				\hline
				\hline
				Ivan Furlan & 0 & 0 & 3 & 0 & 0 & 2 & 5\\
				\hline
				\hline
				Gianmarco Guazzo & 0 & 0 & 3 & 0 & 0 & 2 & 5\\
				\hline
				\hline
				Stefano Lazzaroni & 0 & 5 & 0 & 0 & 0 & 0 & 5\\
				\hline
				\hline
				Francesco Trolese & 0 & 0 & 1 & 0 & 0 & 4 & 5\\
				\hline
				\hline
				Michele Veronesi & 2 & 0 & 0 & 0 & 0 & 3 & 5\\
				\hline 
				\hline
				Totale & 4 & 5 & 13 & 0 & 0 & 13 & 35\\
				\hline 
				\caption{Tabella contenente il prospetto orario preventivato per la fase di consolidamento dei requisiti}
			\end{longtable}

		
			La tabella può essere riassunta nel seguente istogramma:
		
			% \begin{figure}[H]
			% 	\centering
			% 	\includegraphics[width=0.8\linewidth]{./images/preventivo/analisi1.png}
			% 	\caption{Diagramma ore/ruolo componenti nella fase di analisi dei requisiti}
			% 	\label{fig:diagramma suddivione ruoli fase analisi dei requisiti}
			% \end{figure}
		
			\subsubsection{Prospetto economico}
			In base al prospetto orario, quello economico sarà il seguente: 
			
			\rowcolors{1}{white}{blue!20}
			\begin{longtable}{|l|c|c|c|c|c|c|c|}
				\hline
				\rowcolor{lighter-grayer}
				\textbf{Ruolo} & \textbf{Ore} & \textbf{Costo in €} \\
				\hline
				\endfirsthead
				
				\hline
				Responsabile & 4 & 120,00\\
				\hline
				\hline
				Amministratore & 5 & 100,00\\
				\hline
				\hline
				Analista & 13 & 325,00\\
				\hline
				\hline
				Progettista & - & -\\
				\hline
				\hline
				Programmatore & - & -\\
				\hline
				\hline
				Verificatore & 13 & 195,00\\
				\hline
				\textbf{Totale} & 35 & 740,00\\
				\hline
				\caption{Tabella contenente il prospetto economico in riferimento al prospetto orario}
			\end{longtable}
			\pagebreak
		
			La tabella può essere riassunta nel seguente areogramma:
			% \begin{figure}[H]
			% 	\centering
			% 	\includegraphics[width=0.8\linewidth]{./images/preventivo/analisi2.png}
			% 	\caption{Diagramma percentuale ore/ruolo nella fase di analisi dei requisiti}
			% 	\label{fig:diagramma costi ruolo fase analisi dei requisiti}
            % \end{figure}

			\subsection{Fase di progettazione della technology baseline}
            \subsubsection{Prospetto orario}
			Durante la fase di progettazione della technology baseline la distribuzione oraria sarà la seguente:
			
			\rowcolors{1}{lightest-grayest}{blue!20}
			\begin{longtable}{|l|c|c|c|c|c|c|c|}
				\hline
				\rowcolor{lighter-grayer}
				\textbf{Nome} & \textbf{Re} & \textbf{Am} & \textbf{An} & \textbf{Pg}  & \textbf{Pr}   & \textbf{Ve} & \textbf{Totale} \\
				\hline
				\endfirsthead
				
				\hline
				Marco Canovese & 0 & 5 & 3 & 0 & 0 & 4 & 12\\
				\hline
				\hline
				Nicole Davanzo & 0 & 5 & 0 & 6 & 0 & 1 & 12\\
				\hline
				\hline
				Ivan Furlan & 7 & 0 & 5 & 0 & 0 & 0 & 12\\
				\hline
				\hline
				Gianmarco Guazzo & 0 & 3 & 3 & 3 & 0 & 3 & 12\\
				\hline
				\hline
				Stefano Lazzaroni & 0 & 0 & 3 & 9 & 0 & 0 & 12\\
				\hline
				\hline
				Francesco Trolese & 0 & 2 & 6 & 0 & 0 & 4 & 12\\
				\hline
				\hline
				Michele Veronesi & 0 & 0 & 5 & 4 & 0 & 3 & 12\\
				\hline 
				\hline
				Totale & 7 & 15 & 25 & 22 & 0 & 15 & 84\\
				\hline 
				\caption{Tabella contenente il prospetto orario preventivato per la fase di progettazione della technology baseline.}
			\end{longtable}

		
			La tabella può essere riassunta nel seguente istogramma:
		
			% \begin{figure}[H]
			% 	\centering
			% 	\includegraphics[width=0.8\linewidth]{./images/preventivo/analisi1.png}
			% 	\caption{Diagramma ore/ruolo componenti nella fase di analisi dei requisiti}
			% 	\label{fig:diagramma suddivione ruoli fase analisi dei requisiti}
			% \end{figure}
		
			\subsubsection{Prospetto economico}
			In base al prospetto orario, quello economico sarà il seguente: 
			
			\rowcolors{1}{white}{blue!20}
			\begin{longtable}{|l|c|c|c|c|c|c|c|}
				\hline
				\rowcolor{lighter-grayer}
				\textbf{Ruolo} & \textbf{Ore} & \textbf{Costo in €} \\
				\hline
				\endfirsthead
				
				\hline
				Responsabile & 7 & 210,00\\
				\hline
				\hline
				Amministratore & 15 & 300,00\\
				\hline
				\hline
				Analista & 25 & 625,00\\
				\hline
				\hline
				Progettista & 22 & 484,00\\
				\hline
				\hline
				Programmatore & - & -\\
				\hline
				\hline
				Verificatore & 15 & 225,00\\
				\hline
				\textbf{Totale} & 84 & 1.844,00\\
				\hline
				\caption{Tabella contenente il prospetto economico in riferimento al prospetto orario}
			\end{longtable}
			\pagebreak
		
			La tabella può essere riassunta nel seguente areogramma:
			% \begin{figure}[H]
			% 	\centering
			% 	\includegraphics[width=0.8\linewidth]{./images/preventivo/analisi2.png}
			% 	\caption{Diagramma percentuale ore/ruolo nella fase di analisi dei requisiti}
			% 	\label{fig:diagramma costi ruolo fase analisi dei requisiti}
            % \end{figure}

			\subsection{Fase di progettazione e codifica del Proof of Concept}
            \subsubsection{Prospetto orario}
			Durante la fase di progettazione della proof of concept la distribuzione oraria sarà la seguente:
			
			\rowcolors{1}{lightest-grayest}{blue!20}
			\begin{longtable}{|l|c|c|c|c|c|c|c|}
				\hline
				\rowcolor{lighter-grayer}
				\textbf{Nome} & \textbf{Re} & \textbf{Am} & \textbf{An} & \textbf{Pg}  & \textbf{Pr}   & \textbf{Ve} & \textbf{Totale} \\
				\hline
				\endfirsthead
				
				\hline
				Marco Canovese & 4& 0 & 0 & 7 & 9 & 4 & 24\\
				\hline
				\hline
				Nicole Davanzo & 0 & 5 & 0 & 8 & 4 & 7 & 24\\
				\hline
				\hline
				Ivan Furlan & 4 & 3 & 0 & 0 & 8 & 9 & 24\\
				\hline
				\hline
				Gianmarco Guazzo & 3 & 5 & 0 & 8 & 5 & 3 & 24\\
				\hline
				\hline
				Stefano Lazzaroni & 0 & 2 & 0 & 9 & 6 & 7 & 24\\
				\hline
				\hline
				Francesco Trolese & 0 & 2 & 3 & 6 & 9 & 4 & 24\\
				\hline
				\hline
				Michele Veronesi & 5 & 0 & 5 & 4 & 6 & 4 & 24\\
				\hline 
				\hline
				Totale & 16 & 17 & 8 & 42 & 47 & 38 & 168\\
				\hline 
				\caption{Tabella contenente il prospetto orario preventivato per la fase di progettazione della proof of concept.}
			\end{longtable}

		
			La tabella può essere riassunta nel seguente istogramma:
		
			% \begin{figure}[H]
			% 	\centering
			% 	\includegraphics[width=0.8\linewidth]{./images/preventivo/analisi1.png}
			% 	\caption{Diagramma ore/ruolo componenti nella fase di analisi dei requisiti}
			% 	\label{fig:diagramma suddivione ruoli fase analisi dei requisiti}
			% \end{figure}
		
			\subsubsection{Prospetto economico}
			In base al prospetto orario, quello economico sarà il seguente: 
			
			\rowcolors{1}{white}{blue!20}
			\begin{longtable}{|l|c|c|c|c|c|c|c|}
				\hline
				\rowcolor{lighter-grayer}
				\textbf{Ruolo} & \textbf{Ore} & \textbf{Costo in €} \\
				\hline
				\endfirsthead
				
				\hline
				Responsabile & 16 & 480,00\\
				\hline
				\hline
				Amministratore & 17 & 340,00\\
				\hline
				\hline
				Analista & 8 & 200,00\\
				\hline
				\hline
				Progettista & 42 & 924,00\\
				\hline
				\hline
				Programmatore & 47 & 705,00\\
				\hline
				\hline
				Verificatore & 38 & 570,00\\
				\hline
				\textbf{Totale} & 168 & 3.219,00\\
				\hline
				\caption{Tabella contenente il prospetto economico in riferimento al prospetto orario}
			\end{longtable}
			\pagebreak
		
			La tabella può essere riassunta nel seguente areogramma:
			% \begin{figure}[H]
			% 	\centering
			% 	\includegraphics[width=0.8\linewidth]{./images/preventivo/analisi2.png}
			% 	\caption{Diagramma percentuale ore/ruolo nella fase di analisi dei requisiti}
			% 	\label{fig:diagramma costi ruolo fase analisi dei requisiti}
            % \end{figure}

			\subsection{Fase di progettazione completa e prima implementazione di base}
            \subsubsection{Prospetto orario}
			Durante la fase di progettazione completa e prima implementazione di base la distribuzione oraria sarà la seguente:
			
			\rowcolors{1}{lightest-grayest}{blue!20}
			\begin{longtable}{|l|c|c|c|c|c|c|c|}
				\hline
				\rowcolor{lighter-grayer}
				\textbf{Nome} & \textbf{Re} & \textbf{Am} & \textbf{An} & \textbf{Pg}  & \textbf{Pr}   & \textbf{Ve} & \textbf{Totale} \\
				\hline
				\endfirsthead
				
				\hline
				Marco Canovese & 2 & 0 & 2 & 5 & 9 & 4 & 22\\
				\hline
				\hline
				Nicole Davanzo & 0 & 0 & 0 & 0 & 12 & 10 & 22\\
				\hline
				\hline
				Ivan Furlan & 0 & 0 & 0 & 7 & 6 & 9 & 22\\
				\hline
				\hline
				Gianmarco Guazzo & 0 & 0 & 0 & 9 & 8 & 5 & 22\\
				\hline
				\hline
				Stefano Lazzaroni & 2 & 2 & 0 & 5 & 3 & 10 & 22\\
				\hline
				\hline
				Francesco Trolese & 4 & 2 & 0 & 6 & 3 & 7 & 22\\
				\hline
				\hline
				Michele Veronesi & 0 & 7 & 0 & 3 & 12 & 0 & 22\\
				\hline 
				\hline
				Totale & 8 & 11 & 2 & 35 & 53 & 45 & 154\\
				\hline 
				\caption{Tabella contenente il prospetto orario preventivato per la fase di progettazione completa e prima implementazione di base}
			\end{longtable}

		
			La tabella può essere riassunta nel seguente istogramma:
		
			% \begin{figure}[H]
			% 	\centering
			% 	\includegraphics[width=0.8\linewidth]{./images/preventivo/analisi1.png}
			% 	\caption{Diagramma ore/ruolo componenti nella fase di analisi dei requisiti}
			% 	\label{fig:diagramma suddivione ruoli fase analisi dei requisiti}
			% \end{figure}
		
			\subsubsection{Prospetto economico}
			In base al prospetto orario, quello economico sarà il seguente: 
			
			\rowcolors{1}{white}{blue!20}
			\begin{longtable}{|l|c|c|c|c|c|c|c|}
				\hline
				\rowcolor{lighter-grayer}
				\textbf{Ruolo} & \textbf{Ore} & \textbf{Costo in €} \\
				\hline
				\endfirsthead
				
				\hline
				Responsabile & 8 & 240,00\\
				\hline
				\hline
				Amministratore & 11 & 220,00\\
				\hline
				\hline
				Analista & 2 & 50,00\\
				\hline
				\hline
				Progettista & 35 & 770,00\\
				\hline
				\hline
				Programmatore & 53 & 795,00\\
				\hline
				\hline
				Verificatore & 45 & 675,00\\
				\hline
				\textbf{Totale} & 154 & 2.750,00\\
				\hline
				\caption{Tabella contenente il prospetto economico in riferimento al prospetto orario}
			\end{longtable}
			\pagebreak
		
			La tabella può essere riassunta nel seguente areogramma:
			% \begin{figure}[H]
			% 	\centering
			% 	\includegraphics[width=0.8\linewidth]{./images/preventivo/analisi2.png}
			% 	\caption{Diagramma percentuale ore/ruolo nella fase di analisi dei requisiti}
			% 	\label{fig:diagramma costi ruolo fase analisi dei requisiti}
            % \end{figure}

			\subsection{Termine implementazione e raffinamento generale}
            \subsubsection{Prospetto orario}
			Durante la fase di termine dell'implementazione e raffinamento generale la distribuzione oraria sarà la seguente:
			
			\rowcolors{1}{lightest-grayest}{blue!20}
			\begin{longtable}{|l|c|c|c|c|c|c|c|}
				\hline
				\rowcolor{lighter-grayer}
				\textbf{Nome} & \textbf{Re} & \textbf{Am} & \textbf{An} & \textbf{Pg}  & \textbf{Pr}   & \textbf{Ve} & \textbf{Totale} \\
				\hline
				\endfirsthead
				
				\hline
				Marco Canovese & 0 & 0 & 0 & 8 & 8 & 12 & 28\\
				\hline
				\hline
				Nicole Davanzo & 1 & 5 & 0 & 8 & 4 & 10 & 28\\
				\hline
				\hline
				Ivan Furlan & 0 & 3 & 0 & 8 & 11 & 6 & 28\\
				\hline
				\hline
				Gianmarco Guazzo & 4 & 0 & 0 & 6 & 9 & 9 & 28\\
				\hline
				\hline
				Stefano Lazzaroni & 0 & 0 & 0 & 9 & 6 & 10 & 28\\
				\hline
				\hline
				Francesco Trolese & 0 & 0 & 0 & 7 & 11 & 10 & 28\\
				\hline
				\hline
				Michele Veronesi & 3 & 0 & 0 & 11 & 3 & 11 & 28\\
				\hline 
				\hline
				Totale & 8 & 8 & 0 & 57 & 55 & 68 & 196\\
				\hline 
				\caption{Tabella contenente il prospetto orario preventivato per la fase di termine dell'implementazione e raffinamento generale}
			\end{longtable}

		
			La tabella può essere riassunta nel seguente istogramma:
		
			% \begin{figure}[H]
			% 	\centering
			% 	\includegraphics[width=0.8\linewidth]{./images/preventivo/analisi1.png}
			% 	\caption{Diagramma ore/ruolo componenti nella fase di analisi dei requisiti}
			% 	\label{fig:diagramma suddivione ruoli fase analisi dei requisiti}
			% \end{figure}
		
			\subsubsection{Prospetto economico}
			In base al prospetto orario, quello economico sarà il seguente: 
			
			\rowcolors{1}{white}{blue!20}
			\begin{longtable}{|l|c|c|c|c|c|c|c|}
				\hline
				\rowcolor{lighter-grayer}
				\textbf{Ruolo} & \textbf{Ore} & \textbf{Costo in €} \\
				\hline
				\endfirsthead
				
				\hline
				Responsabile & 8 & 240,00\\
				\hline
				\hline
				Amministratore & 8 & 160,00\\
				\hline
				\hline
				Analista & - & -\\
				\hline
				\hline
				Progettista & 57 & 1.254,00\\
				\hline
				\hline
				Programmatore & 55 & 825,00\\
				\hline
				\hline
				Verificatore & 68 & 1.020,00\\
				\hline
				\textbf{Totale} & 196 & 3.499,00\\
				\hline
				\caption{Tabella contenente il prospetto economico in riferimento al prospetto orario}
			\end{longtable}
			\pagebreak
		
			La tabella può essere riassunta nel seguente areogramma:
			% \begin{figure}[H]
			% 	\centering
			% 	\includegraphics[width=0.8\linewidth]{./images/preventivo/analisi2.png}
			% 	\caption{Diagramma percentuale ore/ruolo nella fase di analisi dei requisiti}
			% 	\label{fig:diagramma costi ruolo fase analisi dei requisiti}
            % \end{figure}

			\subsection{Validazione e collaudo}
            \subsubsection{Prospetto orario}
			Durante la fase di validazione e collaudo la distribuzione oraria sarà la seguente:
			
			\rowcolors{1}{lightest-grayest}{blue!20}
			\begin{longtable}{|l|c|c|c|c|c|c|c|}
				\hline
				\rowcolor{lighter-grayer}
				\textbf{Nome} & \textbf{Re} & \textbf{Am} & \textbf{An} & \textbf{Pg}  & \textbf{Pr}   & \textbf{Ve} & \textbf{Totale} \\
				\hline
				\endfirsthead
				
				\hline
				Marco Canovese & 5 & 2 & 0 & 0 & 2 & 5 & 14\\
				\hline
				\hline
				Nicole Davanzo & 3 & 0 & 0 & 3 & 4 & 4 & 14\\
				\hline
				\hline
				Ivan Furlan & 0 & 4 & 4 & 3 & 0 & 3 & 14\\
				\hline
				\hline
				Gianmarco Guazzo & 3 & 3 & 0 & 2 & 0 & 6 & 14\\
				\hline
				\hline
				Stefano Lazzaroni & 0 & 4 & 2 & 2 & 6 & 0 & 14\\
				\hline
				\hline
				Francesco Trolese & 0 & 3 & 5 & 0 & 3 & 3 & 14\\
				\hline
				\hline
				Michele Veronesi & 3 & 0 & 4 & 1 & 2 & 4 & 14\\
				\hline 
				\hline
				Totale & 14 & 16 & 15 & 11 & 17 & 25 & 98\\
				\hline 
				\caption{Tabella contenente il prospetto orario preventivato per la fase di validazione e collaudo.}
			\end{longtable}

		
			La tabella può essere riassunta nel seguente istogramma:
		
			% \begin{figure}[H]
			% 	\centering
			% 	\includegraphics[width=0.8\linewidth]{./images/preventivo/analisi1.png}
			% 	\caption{Diagramma ore/ruolo componenti nella fase di analisi dei requisiti}
			% 	\label{fig:diagramma suddivione ruoli fase analisi dei requisiti}
			% \end{figure}
		
			\subsubsection{Prospetto economico}
			In base al prospetto orario, quello economico sarà il seguente: 
			
			\rowcolors{1}{white}{blue!20}
			\begin{longtable}{|l|c|c|c|c|c|c|c|}
				\hline
				\rowcolor{lighter-grayer}
				\textbf{Ruolo} & \textbf{Ore} & \textbf{Costo in €} \\
				\hline
				\endfirsthead
				
				\hline
				Responsabile & 14 & 420,00\\
				\hline
				\hline
				Amministratore & 16 & 320,00\\
				\hline
				\hline
				Analista & 15 & 375,00\\
				\hline
				\hline
				Progettista & 11 & 242,00\\
				\hline
				\hline
				Programmatore & 17 & 255,00\\
				\hline
				\hline
				Verificatore & 25 & 375,00\\
				\hline
				\textbf{Totale} & 98 & 1.987,00\\
				\hline
				\caption{Tabella contenente il prospetto economico in riferimento al prospetto orario}
			\end{longtable}
			\pagebreak
		
			La tabella può essere riassunta nel seguente areogramma:
			% \begin{figure}[H]
			% 	\centering
			% 	\includegraphics[width=0.8\linewidth]{./images/preventivo/analisi2.png}
			% 	\caption{Diagramma percentuale ore/ruolo nella fase di analisi dei requisiti}
			% 	\label{fig:diagramma costi ruolo fase analisi dei requisiti}
            % \end{figure}
			
			\subsection{Riepilogo ore totali}
			\subsubsection{Totale orario complessivo}
			\subsubsection{Prospetto orario complessivo}
			Riepilogo della distribuzione oraria di tutte le fasi:
			
			\rowcolors{1}{lightest-grayest}{blue!20}
			\begin{longtable}{|l|c|c|c|c|c|c|c|}
				\hline
				\rowcolor{lighter-grayer}
				\textbf{Nome} & \textbf{Re} & \textbf{Am} & \textbf{An} & \textbf{Pg}  & \textbf{Pr}   & \textbf{Ve} & \textbf{Totale} \\
				\hline
				\endfirsthead
				
				\hline
				Marco Canovese & 13 & 17 & 16 & 20 & 28 & 41 & 135\\
				\hline
				\hline
				Nicole Davanzo & 12 & 16 & 15 & 25 & 24 & 43 & 135\\
				\hline
				\hline
				Ivan Furlan & 11 & 17 & 23 & 18 & 25 & 41 & 135\\
				\hline
				\hline
				Gianmarco Guazzo & 14 & 17 & 17 & 28 & 22 & 37 & 135\\
				\hline
				\hline
				Stefano Lazzaroni & 10 & 13 & 17 & 34 & 24 & 37 & 135\\
				\hline
				\hline
				Francesco Trolese & 11 & 14 & 22 & 19 & 26 & 43 & 135\\
				\hline
				\hline
				Michele Veronesi & 13 & 13 & 23 & 23 & 23 & 40 & 135\\
				\hline 
				\hline
				Totale & 84 & 107 & 133 & 167 & 172 & 282 & 945\\
				\hline 
				\caption{Tabella contenente il prospetto orario preventivato tenendo conto di tutte le fasi}
			\end{longtable}

		
			La tabella può essere riassunta nel seguente istogramma:
		
			% \begin{figure}[H]
			% 	\centering
			% 	\includegraphics[width=0.8\linewidth]{./images/preventivo/analisi1.png}
			% 	\caption{Diagramma ore/ruolo componenti nella fase di analisi dei requisiti}
			% 	\label{fig:diagramma suddivione ruoli fase analisi dei requisiti}
			% \end{figure}
		
			\subsubsection{Prospetto economico complessivo}
			In base al prospetto orario, quello economico sarà il seguente: 
			
			\rowcolors{1}{white}{blue!20}
			\begin{longtable}{|l|c|c|c|c|c|c|c|}
				\hline
				\rowcolor{lighter-grayer}
				\textbf{Ruolo} & \textbf{Ore} & \textbf{Costo in €} \\
				\hline
				\endfirsthead
				
				\hline
				Responsabile & 84 & 2.520,00\\
				\hline
				\hline
				Amministratore & 107 & 2.140,00\\
				\hline
				\hline
				Analista & 133 & 3.325,00\\
				\hline
				\hline
				Progettista & 167 & 3.674,00\\
				\hline
				\hline
				Programmatore & 172 & 2.580,00\\
				\hline
				\hline
				Verificatore & 282 & 4.230,00\\
				\hline
				\textbf{Totale} & 945 & 18.469,00\\
				\hline
				\caption{Tabella contenente il prospetto economico in riferimento al prospetto orario}
			\end{longtable}
			\pagebreak
		
			La tabella può essere riassunta nel seguente areogramma:
			% \begin{figure}[H]
			% 	\centering
			% 	\includegraphics[width=0.8\linewidth]{./images/preventivo/analisi2.png}
			% 	\caption{Diagramma percentuale ore/ruolo nella fase di analisi dei requisiti}
			% 	\label{fig:diagramma costi ruolo fase analisi dei requisiti}
            % \end{figure}

			\subsubsection{Totale orario rendicontato}
			\subsubsection{Prospetto orario rendicontato}
			Riepilogo della distribuzione oraria delle fasi rendicontate, escludendo quindi le fasi di
analisi e consolidamento dei requisiti:
			
			\rowcolors{1}{lightest-grayest}{blue!20}
			\begin{longtable}{|l|c|c|c|c|c|c|c|}
				\hline
				\rowcolor{lighter-grayer}
				\textbf{Nome} & \textbf{Re} & \textbf{Am} & \textbf{An} & \textbf{Pg}  & \textbf{Pr}   & \textbf{Ve} & \textbf{Totale} \\
				\hline
				\endfirsthead
				
				\hline
				Marco Canovese & 11 & 7 & 5 & 20 & 28 & 29 & 100\\
				\hline
				\hline
				Nicole Davanzo & 4 & 15 & 0 & 25 & 24 & 32 & 100\\
				\hline
				\hline
				Ivan Furlan & 11 & 10 & 9 & 18 & 25 & 27 & 100\\
				\hline
				\hline
				Gianmarco Guazzo & 10 & 11 & 3 & 28 & 22 & 26 & 100\\
				\hline
				\hline
				Stefano Lazzaroni & 2 & 8 & 5 & 34 & 24 & 27 & 100\\
				\hline
				\hline
				Francesco Trolese & 4 & 9 & 14 & 19 & 26 & 28 & 100\\
				\hline
				\hline
				Michele Veronesi & 11 & 7 & 14 & 23 & 23 & 22 & 100\\
				\hline 
				\hline
				Totale & 53 & 67 & 50 & 167 & 172 & 191 & 700\\
				\hline 
				\caption{Tabella contenente il prospetto orario preventivato tenendo conto delle fasi rendicontate}
			\end{longtable}

		
			La tabella può essere riassunta nel seguente istogramma:
		
			% \begin{figure}[H]
			% 	\centering
			% 	\includegraphics[width=0.8\linewidth]{./images/preventivo/analisi1.png}
			% 	\caption{Diagramma ore/ruolo componenti nella fase di analisi dei requisiti}
			% 	\label{fig:diagramma suddivione ruoli fase analisi dei requisiti}
			% \end{figure}
		
			\subsubsection{Prospetto economico rendicontato}
			In base al prospetto orario, quello economico sarà il seguente: 
			
			\rowcolors{1}{white}{blue!20}
			\begin{longtable}{|l|c|c|c|c|c|c|c|}
				\hline
				\rowcolor{lighter-grayer}
				\textbf{Ruolo} & \textbf{Ore} & \textbf{Costo in €} \\
				\hline
				\endfirsthead
				
				\hline
				Responsabile & 53 & 1.590,00\\
				\hline
				\hline
				Amministratore & 67 & 1.340,00\\
				\hline
				\hline
				Analista & 50 & 1.250,00\\
				\hline
				\hline
				Progettista & 167 & 3.674,00\\
				\hline
				\hline
				Programmatore & 172 & 2.580,00\\
				\hline
				\hline
				Verificatore & 191 & 2.865,00\\
				\hline
				\textbf{Totale} & 700 & 13.299,00\\
				\hline
				\caption{Tabella contenente il prospetto economico in riferimento al prospetto orario}
			\end{longtable}
			\pagebreak
		
			La tabella può essere riassunta nel seguente areogramma:
			% \begin{figure}[H]
			% 	\centering
			% 	\includegraphics[width=0.8\linewidth]{./images/preventivo/analisi2.png}
			% 	\caption{Diagramma percentuale ore/ruolo nella fase di analisi dei requisiti}
			% 	\label{fig:diagramma costi ruolo fase analisi dei requisiti}
            % \end{figure}

		