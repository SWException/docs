% !TeX root = ../../../main.tex
\section{Analisi dei rischi} \label{_analisiDeiRischi}
	La gestione del rischio collegato all'attività di sviluppo del progetto è di vitale importanza per la consegna secondo le direttive date dal proponente. La corretta gestione dei rischi è un processo in cui si cerca di prevedere eventuali problemi che si potrebbero incontrare durante l'intero corso del progetto. L'analisi preventiva dei rischi e la definizione di strategie per la risoluzione degli stessi porta ad un'organizzazione interna che è definita in diverse attività.
	Un piano per la gestione dei rischi si articola generalmente in quattro punti fondamentali: 
\begin{itemize}
	\item \textbf{Identificazione dei rischi}: attività iniziale che ha lo scopo di individuare eventuali problematiche nello sviluppo del progetto che nella sua attuazione possono diminuire la qualità del prodotto fornito;
	\item \textbf{Analisi dei rischi}: attività che scende nello specifico di ogni rischio individuato e che si articola nella individuazione della probabilità di occorrenza, nella definizione dell'indice di gravità e nelle conseguenze al progetto stesso nel caso queste evenienze si verificassero;
	\item \textbf{Pianificazione}: attività che ha lo scopo sia di evitare l'incorrere nei rischi da parte del gruppo, sia di definire comportamenti nel caso in cui questi rischi si presentassero;
	\item \textbf{Controllo dei rischi}: attività parallela durante tutto il percorso di sviluppo del progetto per tenere sotto controllo i rischi analizzati ed eventualmente poterli rilevare in maniera repentina in modo da minimizzare i danni prodotti.
\end{itemize}

\subsection{Tipologia del rischio}
L'attività di Analisi dei rischi si sviluppa verticalmente su quattro macro categorie relative al progetto:
\begin{itemize}
	\item Rischi personali;
	\item Rischi tecnologici;
	\item Rischi organizzativi;
	\item Rischi dei Requisiti.
\end{itemize}

\subsection{Tabella dei rischi}
Di seguito viene presentata la tabella dei rischi, in cui i codici vengono definiti secondo lo standard definito in \dext{Norme di Progetto}.
Per ognuno dei rischi individuati viene fornita una descrizione, una Probabilità di Occorrenza(PrO), Pericolosità dell'occorrenza (PeO), il metodo di rilevamento e la proposta di mitigazione. C'è da sottolineare che l'analisi dei rischi non è un'attività statica ma richiede una vigilanza costante durante tutta la durata del progetto. La mitigazione proposta potrebbe non essere sufficiente o non risolutiva e quindi l'analisi va raffinata nel tempo secondo le evenienze che potrebbero formarsi.

\newpage
		
\subsection{Elenco dei rischi preventivati}

\subsubsection{Rischi Personali}
\paragraph{Cooperazione e Problemi decisionali}
\begin{center}
	\rowcolors{2}{blue!20}{lightest-grayest}
	\begin{longtable}{p{1cm}|p{4cm}|p{0.7cm}|p{0.7cm}|p{3cm}|p{4cm}}
		\arrayrulecolor{lightest-grayest}
		\rowcolor{blue!20}
		\textbf{Codice} & 
		\textbf{Descrizione} &
		\textbf{PrO}  &
		\textbf{PeO}  &				        
		\textbf{Rilevamento} &
		\textbf{Mitigazione} \\	
		RSP1 & La cooperazione all'interno del gruppo è attività fondamentale per la buona riuscita del progetto. Gli impegni personali, la non conoscenza pregressa delle persone con cui si lavora e la coordinazione necessaria possono essere un fattore di rischio & Bassa & Media & La rilevazione del rischio è a carico di ogni componente. & Eventuali tensioni fra i componenti del gruppo sono da discutere inizialmente in autonomia e nel caso in cui siano gravi saranno discusse da tutti i componenti. Il Responsabile di Progetto cercherà di limitare l'interazione tra i membri del gruppo interessati cercando quindi di sanare le discordie. \\
		\end{longtable}
\end{center}

\subsubsection{Rischi Tecnologici}
\paragraph{Inesperienza tecnologica}		
		\begin{center}
	\rowcolors{2}{blue!20}{lightest-grayest}
	\begin{longtable}{p{1cm}|p{4cm}|p{0.7cm}|p{0.7cm}|p{3cm}|p{4cm}}
		\arrayrulecolor{white}
		\hline
		\rowcolor{blue!20}
		\textbf{Codice} & 
		\textbf{Descrizione} &
		\textbf{PrO}  &
		\textbf{PeO}  &				        
		\textbf{Rilevamento} &
		\textbf{Mitigazione} \\
		\hline
		RST1 & Il gruppo dovrà interfacciarsi con tecnologie e linguaggi di programmazione non conosciuti e/o utilizzati in precedenza. Sarà quindi necessario del tempo per poter metabolizzare il loro funzionamento e adattarlo alle richieste del proponente. & Medio & Media & La formazione dei componenti è un'attività personale e quindi ognuno dovrà gestire in autonomia la propria istruzione. Nel caso in cui ci siano delle carenze ogni componente è tenuto ad avvisare il gruppo. & Per mitigare questo rischio le soluzioni sono due: i componenti più esperti in determinate tecnologie, tempo permettendo, aiuteranno i membri meno formati nelle stesse e la suddivisione del lavoro terrà in considerazione i livelli di conoscenza personale dei linguaggi e delle tecnologie da utilizzare. \\
		\end{longtable}
\end{center}		
		
\subsubsection{Rischi Organizzativi}
\paragraph{Calcolo delle tempistiche}
\begin{center}

	\rowcolors{2}{blue!20}{lightest-grayest}
	\begin{longtable}{p{1cm}|p{4cm}|p{0.7cm}|p{0.7cm}|p{3cm}|p{4cm}}
		\arrayrulecolor{white}
		\hline
		\rowcolor{blue!20}
		\textbf{Codice} & 
		\textbf{Descrizione} &
		\textbf{PrO}  &
		\textbf{PeO}  &				        
		\textbf{Rilevamento} &
		\textbf{Mitigazione} \\
		\hline			
		RSO1 & Gli strumenti e le tecnologie utilizzate necessitano di tempo da parte dei componenti per assimilare concetti nuovi. Essendo le tempistiche e le milestone stringenti, questo può essere un fattore di rischio. & Media & Alta & Durante lo svolgimento del progetto e durante le videoconferenze il gruppo è tenuto a discutere di eventuali problematiche con l'apprendimento o con la elevata mole di lavoro & Nel caso in cui siano riscontrate delle scadenze difficili da rispettare il Responsabile è tenuto a gestirle in maniera adeguata tenendo in considerazione le risorse umane e di tempo. \\
		\end{longtable}
\end{center}
\paragraph{Comunicazione gruppo - proponente}
\begin{center}

	\rowcolors{2}{blue!20}{lightest-grayest}
	\begin{longtable}{p{1cm}|p{4cm}|p{0.7cm}|p{0.7cm}|p{3cm}|p{4cm}}
		\arrayrulecolor{white}
		\hline
		\rowcolor{blue!20}
		\textbf{Codice} & 
		\textbf{Descrizione} &
		\textbf{PrO}  &
		\textbf{PeO}  &				        
		\textbf{Rilevamento} &
		\textbf{Mitigazione} \\
		\hline	
		RSO2 & Le comunicazioni tra il gruppo e il proponente può essere un fattore di rischio nel caso in cui vengono trascurate. & Bassa & Media & Ogni componente del gruppo è tenuto a far presente eventuali dubbi riguardo il capitolato e di annotarsi eventuali domande da porre al proponente. & Per mitigare questo rischio il gruppo si pone l'obbiettivo di contattare il proponente con una cadenza prestabilita e ogni qualvolta sorgano dei dubbi che possano minare lo sviluppo del progetto.  \\
		\end{longtable}
\end{center}
\newpage
\paragraph{Assegnazione scadenze}
		\begin{center}

	\rowcolors{2}{blue!20}{lightest-grayest}
	\begin{longtable}{p{1cm}|p{4cm}|p{0.7cm}|p{0.7cm}|p{3cm}|p{4cm}}
		\arrayrulecolor{white}
		\hline
		\rowcolor{blue!20}
		\textbf{Codice} & 
		\textbf{Descrizione} &
		\textbf{PrO}  &
		\textbf{PeO}  &				        
		\textbf{Rilevamento} &
		\textbf{Mitigazione} \\
		\hline	
		RSO3 & La corretta esecuzione di un progetto richiede la divisione del lavoro da parte del gruppo e di conseguenza la necessità di avere delle scadenze per permettere ad ognuno di svolgere il proprio lavoro. L'assegnazione di scadenze troppo stringenti può aumentare il rischio di insuccesso mentre delle scadenze troppo lasche rallenterebbero notevolmente il lavoro del gruppo. & Media & Media & La rilevazione di questo rischio avviene grazie all'analisi dei compiti completati alla scadenza concordata. Se alcuni componenti non riescono a consegnare in tempo è molto probabile che ci siano dei problemi. & Per mitigare questo rischio ogni componente del gruppo è tenuto a comunicare eventuali difficoltà nel portare a termini i compiti personali. Alcuni componenti più liberi possono aiutare quello in difficoltà e nelle videoconferenze successive all'ordine del giorno ci sarà una discussione sulla definizione delle scadenze. \\
		\end{longtable}
\end{center}
\subsubsection{Rischi Requisiti}
\paragraph{Problemi Relativi all'Analisi dei Requisiti}
\begin{center}

	\rowcolors{2}{blue!20}{lightest-grayest}
	\begin{longtable}{p{1cm}|p{4cm}|p{0.7cm}|p{0.7cm}|p{3cm}|p{4cm}}
		\arrayrulecolor{white}
		\hline
		\rowcolor{blue!20}
		\textbf{Codice} & 
		\textbf{Descrizione} &
		\textbf{PrO}  &
		\textbf{PeO}  &				        
		\textbf{Rilevamento} &
		\textbf{Mitigazione} \\
		\hline	
		RSR1 & L'analisi dei requisiti è un'attività fondamentale per la riuscita del progetto.  Errori nella definizione dei casi d'uso da attuare o nell' analisi dei requisiti obbligatori potrebbe portare a dei rischi nella riuscita del progetto. & Bassa & Alta & Per rilevare eventuali problematiche relative ai requisiti è necessario che il gruppo svolga un'attenta analisi dei casi d'uso e nel caso sorgano dei dubbi fissare una video chiamata con il proponente. & Per mitigare questo problema il Responsabile di progetto dovrà raccogliere eventuali dubbi da parte dei membri del gruppo e richiedere in maniera repentina un confronto col l'azienda proponente. \\
	\end{longtable}
\end{center}