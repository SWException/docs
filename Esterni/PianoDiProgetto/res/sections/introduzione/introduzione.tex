% !TeX root = ../../../main.tex
\section{Introduzione} \label{_introduzione}

\subsection{Scopo del documento}
Lo scopo di questo documento è analizzare il progetto che il gruppo ha scelto di produrre in modo da:
\begin{itemize}
    \item Individuare i rischi e trovare delle soluzioni che li minimizzino;
    \item Descrivere il modello di sviluppo adottato;
    \item Pianificare le attività necessarie al suo compimento e suddividerle tra i membri del gruppo;
    \item Stimare i costi (in termini di tempo e di denaro).
\end{itemize}
Fare ciò è necessario per portare a termine il progetto nel modo più efficace ed efficiente possibile.

\subsection{Scopo del prodotto}
Il \glock{capitolato} C6 prevede lo sviluppo di un'applicazione in una piattaforma serverless composta da microservizi, la quale implementa un template per siti e-commerce basata sui servizi AWS.
%da ampliare

\subsection{Glossario  e documenti esterni}
Per evitare possibili ambiguità nella lettura del documento da parte di soggetti estranei all'organizzazione \textit{SWException} vengono utilizzate due notazioni particolari:
\begin{itemize}
    \item Una \textit{D} usata come pedice di una parola o un gruppo di parole in \textit{corsivo} indica il nome di un documento;
    \item Una \textit{G} a pedice di una parola o un gruppo di parole in \textit{corsivo} indica la presenza del termine all'interno del \dext{glossario}.
\end{itemize}

\subsection{Riferimenti}

\subsubsection{Normativi}
\begin{itemize}
    \item \textbf{Norme di progetto}: \dext{Norme di Progetto};
    \item \href{https://www.math.unipd.it/~tullio/IS-1/2020/Progetto/RO.html}{regolamento per l'organigramma e l'offerta tecnico-economica;}
    \item \href{https://www.math.unipd.it/~tullio/IS-1/2020/Progetto/C2.pdf}{capitolato C6: EmporioLambda.}
\end{itemize}

\subsubsection{Informativi}
\begin{itemize}
    \item \textbf{Piano di qualifica}: \dext{piano di qualifica};
    \item \href{https://www.math.unipd.it/~tullio/IS-1/2020/Dispense/L05.pdf}{ciclo di vita del software - dispense del corso di Ingegneria del Software;}
    \item \href{https://www.math.unipd.it/~tullio/IS-1/2020/Dispense/L05.pdf}{gestione di progetto - dispense del corso di Ingegneria del Software.}
\end{itemize}

\subsection{Scadenze}
Il gruppo \textit{SWException} prevede di partecipare alle revisioni committenti nelle date:
\begin{itemize}
    \item \textbf{2021-01-18}: Revisione dei Requisiti;
    \item \textbf{2021-03-08}: Revisione di Progettazione;
    \item \textbf{2021-04-09}: Revisione di Qualifica;
    \item \textbf{2021-05-10}: Revisione di Accettazione.
\end{itemize}