\section*{A}
\subsection*{Applicazione mobile}
Programma che specializza il funzionamento di un computer in una determinata attività,
pensato per dispositivi adatti alla mobilità, quali smartphone e tablet.

\subsection*{Amministratore}
Tipologia di utente dell'applicazione EmporioLambda con la possibilità di
\begin{itemize}
    \item rilasciare l'applicazione nell'infrastruttura cloud AWS;
    \item gestire la configurazione dei servizi di terze parti integrati.
\end{itemize}
Nello svolgimento di questo progetto è rappresentato dal gruppo SWException.

\section*{B}
Empty

\section*{C}
\subsection*{Cliente}
Tipologia di utente dell'applicazione EmporioLambda che rappresenta l'utente finale, che usa
il prodotto per acquistare beni o servizi dal commerciante.

\subsection*{Commerciante}
Tipologia di utente dell'applicazione EmporioLambda che acquista il prodotto per specializzarlo
alle sue necessità, mettendo in vendita i propri beni o servizi offerti ai clienti.

\subsection*{Committente}
Persone o gruppo di persone incaricato ad individuare il/i capitolato/i da presentare al fornitore.
Chi commette, cioè ordina ad altri l'esecuzione di un lavoro o di una prestazione


\section*{D}
Empty

\section*{E}
\subsection*{EML-FE}
\textit{EmporioLambda Front-End:} servizio per il front-end menzionato nel capitolato d'appalto C2.

\subsection*{EML-BE}
\textit{EmporioLambda Back-End:} servizio per la parte di back-end del capitolato C2.

\subsection*{EML-I}
\textit{EmporioLambda Integration:} Servizi di terze parti integrati nel modulo di back-end.

\subsection*{EML-MON}
\textit{EmporioLambda Monitoring:} insieme di strumenti usati dall'amministratore dell'applicazione EmporioLambda
per monitorare lo stato della stessa.

\section*{F}
Empty

\section*{G}
\subsection*{Git}
Version Control Manager usato per il versionamento di codice sorgente e documenti latex.

\subsection*{GitHub}
Piattaforma che offre il software Git as-a-service, insieme a molte altre funzionalità.

\section*{H}
Empty

\section*{I}
Empty

\section*{J}
Empty

\section*{K}
Empty

\section*{L}
Empty

\section*{M}
\subsection*{Microservizi}
I microservizi sono un approccio architetturale alla realizzazione di applicazioni. 
Quello che distingue l'architettura basata su microservizi dagli approcci monolitici tradizionali 
è la suddivisione del prodotto nelle sue funzioni di base. Ciascuna funzione, denominata servizio, può essere 
compilata e implementata in modo indipendente. Pertanto, i singoli servizi possono funzionare, o meno, senza compromettere gli altri.

\subsection*{MIT}
Licenza di software libero creata dal Massachusetts Institute of Technology (MIT).
E' una licenza permissiva, cioè permette il riutilizzo nel software proprietario sotto la condizione che la licenza sia distribuita con tale software.

\section*{N}
Empty

\section*{O}
Empty

\section*{P}
\subsection*{Proponente}
Ente o azienda che propone un capitolato d'appalto.

\section*{Q}
Empty

\section*{R}
Empty

\section*{S}
\subsection*{Server}
In una rete, qualunque computer che offre un servizio agli altri calcolatori (client), come l'accesso a risorse condivise (dischi, stampanti ecc.), la ricerca su basi di dati o altre funzioni applicative.

\subsection*{Servizio}
Singolo componente in un'architettura a microservizi.

\subsection*{Slack}
Applicazione di messaggistica per ambienti enterprise.

\section*{T}
Empty

\section*{U}
Empty

\section*{V}
Empty

\section*{W}
Empty

\section*{X}
Empty

\section*{Y}
Empty

\section*{Z}
Empty
