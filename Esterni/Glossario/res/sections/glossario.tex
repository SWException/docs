\section{A}
\subsection*{Amazon Cognito}
Servizio che permette di aggiungere autenticazione, autorizzazione e gestione degli utenti per le applicazioni Web e mobili.

\subsection*{Amministratore}
Tipologia di utente dell'applicazione EmporioLambda con la possibilità di
\begin{itemize}
    \item rilasciare l'applicazione nell'infrastruttura cloud AWS;
    \item gestire la configurazione dei servizi di terze parti integrati.
\end{itemize}
Nello svolgimento di questo progetto è rappresentato dal gruppo SWException.

\subsection*{Anti collision system}
Sistema che anticipa gli ostacoli per evitare collisioni.

\subsection*{Applicazione mobile}
Programma che specializza il funzionamento di un computer in una determinata attività, pensato per dispositivi adatti alla mobilità, quali smartphone e tablet.

\subsection*{Artifact repository}
Repository che include tutti gli artefatti generati durante lo sviluppo del progetto.

\subsection*{AWS Lambda}
Servizio di elaborazione serverless che consente di eseguire codice senza fare il provisioning o gestire i server, creare una logica di scalabilità del cluster in grado di riconoscere il carico di lavoro, mantenere integrazioni di eventi o gestire i runtime.

\subsection*{AWS CloudWatch}
Servizio di monitoraggio e osservabilità che fornisce dati e approfondimenti utilizzabili per monitorare il software, rispondere ai cambiamenti delle prestazioni a livello di sistema, ottimizzare l'utilizzo delle risorse e ottenere una visione unificata dell'integrità operativa.

\subsection*{AWS DynamoDB}
È un database di documenti e valori-chiave durevole, multi-regione e completamente gestito con sicurezza integrata, backup e ripristino e memorizzazione nella cache in memoria per software.

\subsection*{AWS S3}
Amazon Simple Storage Service è un servizio di storage di oggetti che offre scalabilità, disponibilità dei dati, sicurezza e prestazioni leader del settore.

\subsection*{AWS API Gateway}
È un servizio gestito che semplifica agli sviluppatori la creazione, la pubblicazione, la manutenzione, il monitoraggio e la protezione delle API su qualsiasi scala.

\subsection*{Auth0}
Standard open-source per la delega di accesso, comunemente utilizzato come un modo per gli utenti di Internet di concedere a siti Web o applicazioni l'accesso alle proprie informazioni su altri siti Web, ma senza fornire loro le password.

\newpage
\section{B}
\subsection*{Backend For Frontend (BFF)}
Modello di progettazione cloud utile quando si vuole evitare di personalizzare un singolo backend per più interfacce.

\subsection*{Branch}
Indica un ramo di sviluppo in un Version Control System.

\subsection*{Business logic}
La parte del programma che codifica le regole di business del mondo reale che determinano il modo in cui i dati possono essere creati, archiviati e modificati.

\newpage
\section{C}
\subsection*{C++}
Si tratta di un linguaggio di programmazione general-purpose.

\subsection*{Candidate-release}
Nell'ambito della realizzazione di un progetto informatico, indica una particolare versione del software che prelude alla release finale e stabile.

\subsection*{Capitolato}
Documento del committente nel quale sono specificati il progetto e i vincoli da rispettare. 
\subsection*{Carrello}
Pagina dell'e-commerce che contiene tutti i prodotti che l'utente ha aggiunto per acquistare.

\subsection*{Casi d'uso}
Sono le modalità in cui il sistema può essere utilizzato, cioè le funzionalità che il sistema mette a disposizione dei suoi utilizzatori.

\subsection*{Checkout}
Insieme di passaggi che l'utente deve compiere per acquistare i prodotti precedentemente inseriti nel carrello.

\subsection*{Ciclo di Deming}
E’ un metodo di gestione utilizzato per il controllo e miglioramento continuo dei processi e dei prodotti
suddiviso in 4 fasi: Plan, Do, Check, Act.

\subsection*{Cliente}
Tipologia di utente dell'applicazione EmporioLambda che rappresenta l'utente finale, che usa il prodotto per acquistare beni o servizi dal commerciante.

\subsection*{CloudFormation}
Servizio di AWS che offre un modo semplice per modellare una raccolta di risorse AWS e di terze parti, effettuarne il provisioning in modo rapido e coerente e gestirle in tutto il loro ciclo di vita, trattando l'infrastruttura come codice.

\subsection*{CMS}
\textit{Content Management System:} Software usato per gestire la creazione e modifica dei contenuti di un sito.

\subsection*{Code coverage}
Misura utilizzata per descrivere il grado di esecuzione del codice sorgente di un programma quando viene eseguita una particolare suite di test.

\subsection*{Commerciante}
Tipologia di utente dell'applicazione EmporioLambda che acquista il prodotto per specializzarlo alle sue necessità, mettendo in vendita i propri beni o servizi offerti ai clienti.

\subsection*{Committente}
Persone o gruppo di persone incaricato ad individuare il/i capitolato/i da presentare al fornitore. Chi commette, cioè ordina ad altri l'esecuzione di un lavoro o di una prestazione.

\subsection*{Complessità ciclomatica}
Metrica software utilizzata per misurare la complessità di un programma.

\subsection*{Contentful}
È un Content Management System.

\subsection*{Continuous delivery}
Approccio di ingegneria del software in cui i team producono software in cicli brevi, garantendo che il software possa essere rilasciato in modo affidabile in qualsiasi momento.

\subsection*{Continuous Integration}
Pratica che si applica in contesti in cui lo sviluppo del software avviene attraverso un sistema di controllo versione. Consiste nell'allineamento frequente dagli ambienti di lavoro degli sviluppatori verso l'ambiente condiviso.

\subsection*{Cost Variance}
Indice che indica se il valore realmente maturato è maggiore, uguale o minore rispetto al costo effettivo

\subsection*{CSS}
Cascading Style Sheets, è un linguaggio che permette di aggiungere stile ai documenti web.

\subsection*{CSV}
Comma-separeted value è un file di testo delimitato che utilizza una virgola per separare i valori. Ogni riga del file registra un dato.

\newpage
\section{D}
\subsection*{D3.js}
Libreria di javascript utilizzata per produrre visualizzazioni dinamiche nelle pagine web.

\subsection*{Dashboard del commerciante}
Pagina del sito web accessibile solo da utenti con profilo di tipo commerciante. Offre funzionalità di modifica dei dati sui prodotti presenti nell'applicazione e fornisce una visione generale sugli ordini in corso.

\subsection*{Design pattern}
Nell’ambito dell’ingegneria del software è un concetto che può essere definito come una soluzione progettuale generale ad un problema ricorrente.

\subsection*{Diagramma di Gantt}
Rappresentazione grafica di un calendario di attività, utile al fine di pianificare, coordinare e tracciare specifiche attività in un progetto dando una chiara illustrazione dello stato d'avanzamento del progetto rappresentato.

\subsection*{Docker}
Un insieme di prodotti di piattaforma che utilizzano la virtualizzazione a livello di sistema operativo per fornire software in pacchetti chiamati contenitori.

\subsection*{Dropbox}
Spazio di lavoro sviluppato per gestire il carico di lavoro. 

\newpage
\section{E}
\subsection*{EML-FE}
\textit{EmporioLambda Front-End:} servizio per il front-end menzionato nel capitolato d'appalto C2.

\subsection*{EML-BE}
\textit{EmporioLambda Back-End:} servizio per la parte di back-end del capitolato C2.

\subsection*{EML-I}
\textit{EmporioLambda Integration:} Servizi di terze parti integrati nel modulo di back-end.

\subsection*{EML-MON}
\textit{EmporioLambda Monitoring:} insieme di strumenti usati dall'amministratore dell'applicazione EmporioLambda per monitorare lo stato della stessa.

\subsection*{EmporioLambda}
Si tratta del capitolato da sviluppare che consiste nel creare una piattaforma di e-commerce in stile serverless

\subsection*{ESLint}
Strumento di analisi del codice per identificare i modelli problematici trovati nel codice JavaScript.

\newpage
\section{F}


\newpage
\section{G}
\subsection*{GanttProject}
Software di gestione dei progetti basato su Java.

\subsection*{Git}
Version Control Manager usato per il versionamento di codice sorgente e documenti latex.

\subsection*{GitFlow}
Modello ramificato per Git adatto alla collaborazione.

\subsection*{GitHub}
Piattaforma che offre il software Git as-a-service, insieme a molte altre funzionalità.

\subsection*{GitHub Actions}
API che funziona in modalità causa-effetto su GitHub: rende possibile orchestrare un qualsiasi workflow, basato su qualsiasi evento, mentre GitHub ne gestisce l’esecuzione. Ciò rende possibile ad esempio integrare un processo di \glock{Continuous Integration} a \glock{GitHub}.

\subsection*{GitLab}
Strumento per il ciclo di vita DevOps basato sul Web che fornisce un gestore di repository Git che fornisce funzionalità wiki, tracciamento dei problemi e integrazione continua e distribuzione.

\subsection*{Google Drive}
Servizio di cloud storage che permette la memorizzazione e sincronizzazione di file e cartelle. Può essere usato via web o con app desktop e mobile.

\subsection*{Google Meet}
Piattaforma per videocall e conferenze di proprietà di Google.

\newpage
\section{H}
\subsection*{Homepage}
Pagina principale di un sito. Qui devono essere presenti le principali informazioni utili ad una navigazione proficua del sito da parte dell'utente.

\subsection*{HTML}
Hypertext Markup Language è il linguaggio di markup standard per i documenti progettati per essere visualizzati in un browser web.

\newpage
\section{I}
\subsection*{Identity Manager}
Framework di politiche e tecnologie per garantire che le persone appropriate in un'azienda abbiano l'accesso appropriato alle risorse tecnologiche.

\subsection*{Indice di Gulpease}
Questo indice, inventato nel 1988, ha come obiettivo descrivere la leggibilità di un
testo in lingua italiana in maniera numerica, compresa tra 0 e 100, dove 0 è la leggibilità più
bassa e 100 la leggibilità più alta.

\subsection*{Issue Tracking System}
Strumento per tenere traccia di attività, miglioramenti e bug per quanto riguarda il progetto.

\newpage
\section{J}
\subsection*{JavaScript}
Linguaggio di programmazione orientato agli oggetti e agli eventi, comunemente utilizzato per la creazione, in siti web e applicazioni web, di effetti dinamici interattivi.

\newpage
\section{K}


\newpage
\section{L}
\subsection*{Latex}
Sistema di composizione matematico che include funzionalità progettate per la produzione di documentazione tecnica e scientifica.

\subsection*{Linter}
Strumento di analisi del codice utilizzato per segnalare errori di programmazione, bug, errori stilistici e costrutti sospetti.

\newpage
\section{M}
\subsection*{Merge}
Unione di due branch di sviluppo separati.

\subsection*{Metrica}
Insieme di regole o indici che regolarizzano un determinato punto del progetto.

\subsection*{Microservizi}
I microservizi sono un approccio architetturale alla realizzazione di applicazioni. 
Quello che distingue l'architettura basata su microservizi dagli approcci monolitici tradizionali è la suddivisione del prodotto nelle sue funzioni di base. Ciascuna funzione, denominata servizio, può essere compilata e implementata in modo indipendente. Pertanto, i singoli servizi possono funzionare, o meno, senza compromettere gli altri.

\subsection*{MIT}
Licenza di software libero creata dal Massachusetts Institute of Technology (MIT). È una licenza permissiva, cioè permette il riutilizzo nel software proprietario sotto la condizione che la licenza sia distribuita con tale software.

\subsection*{Multiplayer}
Utilizzato generalmente nei videogiochi, indica che più persone possono partecipare contemporaneamente all'attività.

\newpage
\section{N}
\subsection*{Next.js}
Framework React web di sviluppo front-end che abilita funzionalità come il rendering lato server e la generazione di siti web statici per applicazioni web basate su React.

\subsection*{NoSQL}
è un database che fronisce un meccanismo per l'archiviazione e il recupero dei dati modellato in modi diversi dalle relazioni tabulari utilizzate nei database relazionali.
\newpage
\section{O}
\subsection*{Outlook}
Piattaforma di gestione delle informazioni personali di Microsoft composta da servizi di posta Web, calendario, contatti e attività.

\newpage
\section{P}
\subsection*{PDP}
\textit{Product Detail Pages:} insieme di pagine dell'applicazione EmporioLambda, ognuna di esse contiene informazioni riguardo un singolo prodotto/servizio in vendita nell'e-commerce.

\subsection*{PLP}
\textit{Product Listing Page:} pagina del sito che contiene l'elenco di tutti i prodotti/servizi venduti nell'e-commerce, derivato da un'istanza di EmporioLambda.

\subsection*{POI oriented}
Struttura basata sul Point Of Interest ovvero i punti da raggiungere.

\subsection*{Powerup}
Sono dei bonus che permettono di avanzare più velocemente, tipicamente utilizzati nei videogiochi.

\subsection*{Profilo}
Insieme di informazioni che identificano un univoco utente dell'applicazione.

\subsection*{Proponente}
Ente o azienda che propone un capitolato d'appalto.

\newpage
\section{Q}
\subsection*{Query}
Indica l'interrogazione da parte di un utente di un database.

\newpage
\section{R}
\subsection*{Real time}
Attività svolta in tempo reale.

\subsection*{Redattore}
Persona il cui compito è quello di scrivere i documenti.

\subsection*{Requisito}
Ciascuna delle qualità necessarie e richieste per raggiungere uno scopo determinato.

\subsection*{Responsabile di progetto}
Persona il cui compito è quello di approvare, il lavoro verificato, per il rilascio.

\subsection*{Revisione di progetto}
Milestone fissata dai committenti per valutare i progressi del team. Una revisione permette di capire dove si sta sbagliando in tempo per non pregiudicare l'esito finale del progetto.

\newpage
\section{S}
\subsection*{Schedule Variance}
Indice che indica se si è in linea, in anticipo o in ritardo rispetto alla schedulazione delle attività di progetto pianificate nella baseline.

\subsection*{Selenium}
Insieme di strumenti per l'automatizzazione dei test su interfacce web.

\subsection*{SEO}
\textit{Search Engine Optimization:} Insieme delle attività volte a migliorare il posizionamento (ranking) di un sito o di una pagina web per determinati fattori nei risultati forniti
da un motore di ricerca (Search Engine Result Page o SERP)

\subsection*{Server}
In una rete, qualunque computer che offre un servizio agli altri calcolatori (client), come l'accesso a risorse condivise (dischi, stampanti ecc.), la ricerca su basi di dati o altre funzioni applicative.

\subsection*{Serverless}
Framework web gratuito e open source scritto utilizzando Node.js. È il primo framework sviluppato per la creazione di applicazioni su AWS Lambda.

\subsection*{Servizio}
Singolo componente in un'architettura a microservizi.

\subsection*{Slack}
Applicazione di messaggistica per ambienti enterprise.

\subsection*{Snippet}
Frammento di codice inserito all'interno di un documento.

\subsection*{SQL}
Linguaggio specifico del dominio utilizzato nella programmazione e progettato per la gestione dei dati contenuti in un sistema di gestione di database relazionali.

\subsection*{Stripe}
Azienda americana che offre principalmente software per l'elaborazione dei pagamenti e interfacce di programmazione di applicazioni per siti Web di e-commerce e applicazioni mobili.

\newpage
\section{T}
\subsection*{Typescript}
Linguaggio di programmazione per ambiente web sviluppato e mantenuto da Microsoft. È un sovrainsieme della sintassi di JavaScript, a cui aggiunge opzionalmente un sistema di tipi statico. Viene usato sia lato client che server.

\subsection*{Telegram}
Strumento di messaggistica istantanea utilizzato per la comunicazione tra coppie di utenti o in gruppi.

\subsection*{Tex}
Programma e linguaggio di markup per la stesura di testi scientifici e matematici da cui è stato ricavato \glock{Latex}. L'estensione dei file Tex e Latex è \textit{.tex}.

\subsection*{Texmaker}
Un editor LaTeX open source multipiattaforma con un visualizzatore di PDF integrato.

\subsection*{TomCat}
Apache Tomcat è un'implementazione open source delle tecnologie Java Servlet, JavaServer Pages, Java Expression Language e WebSocket. Tomcat fornisce un ambiente server Web HTTP "puro Java" in cui il codice Java può essere eseguito.

\newpage
\section{U}

\subsection*{UML}
United Modelling Language. Linguaggio di modellazione e specifica basato sul paradigma orientato agli oggetti. Con i suoi diagrammi fornisce uno strumento  ricco nella semantica e nella sintassi, per l'architettura, la progettazione e l'implementazione di sistemi software complessi.

\newpage
\section{V}
\subsection*{Verificatore}
Coloro che hanno il compito di verificare il lavoro svolto dagli altri componenti.

\subsection*{Visual Studio Code}
Visual Studio Code è un code editor ridefinito ed ottimizzato per la costruzione e il debugging per le moderne applicazioni web.

\newpage
\section{W}


\newpage
\section{X}


\newpage
\section{Y}


\newpage
\section{Z}
\subsection*{Zimbra}
Suite software collaborativa che include un server di posta elettronica e un client web.

\subsection*{Zoom}
Servizio di videocall e chat online che attraverso una piattaforma software viene utilizzato per conferenze e lavoro.
