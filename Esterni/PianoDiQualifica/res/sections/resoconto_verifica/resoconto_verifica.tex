\section{Resoconto attività di verifica} \label{_resocontoVerifica}

\subsection{Revisione dei requisiti}

\subsubsection{Analisi statica dei documenti}
Tutta la documentazione prodotta in ingresso alla Revisione dei Requisiti è stata sottoposta ad una meticolosa attività di verifica
basata su quanto previsto all'interno del documento delle \dext{NormeDiProgetto\_1.0.0}.
Questa attività è stata espletata dai verificatori.

\subsubsection{Analisi automatizzata dei documenti}
Al fine di verificare la qualità della documentazione prodotta il gruppo ha deciso di adottare come metrica di verifica
l'indice di Gulpease generato grazie ad un processo automatizzato configurato in \textit{\glock{GitHub}}, unitamente alla generazione di un indice di
correttezza grammaticale del testo.
Di seguito sono presentati i due grafici rappresentativi dei valori ottenuti dall'analisi dei documenti sino all'approvazione degli stessi.
Per quanto riguarda i verbali redatti in occasione dei meeting tra i membri del gruppo e con i proponenti si espongono i valori in forma tabellare
\begin{center}
    \begin{figure}[!htb]
        \centering
        \includegraphics[scale=0.80]{res/images/grafico_gulpease.png}
        \caption{Grafico Indice di Gulpease}
    \end{figure}
    \begin{figure}[!htb]
        \centering
        \includegraphics[scale=0.80]{res/images/grafico_correttezza.png}
        \caption{Grafico indice correttezza grammaticale}
    \end{figure}
    \begin{center}
        \rowcolors{2}{white}{blue!20}
        \begin{longtable}{|c|c|}
            \hline
            \rowcolor{lighter-grayer}
            \textbf{Documento}         & \textbf{Indice di Gulpease} \\
            \hline
            \endfirsthead

            \hline
            Verbale Interno 2020-11-24 & 77                          \\
            Verbale Interno 2020-12-05 & 70                          \\              
            Verbale Esterno 2020-12-10 & 73                          \\
            Verbale Interno 2020-12-14 & 72                          \\
            Verbale Interno 2020-12-17 & 75                          \\
            Verbale Interno 2020-12-22 & 70                          \\
            Verbale Interno 2020-12-28 & 73                          \\
            Verbale Esterno 2020-12-28 & 70                         \\
            Verbale Interno 2021-01-02 & 73                          \\
            Verbale Esterno 2021-01-08 & 74                          \\
            Verbale Interno 2021-01-09 & 70                          \\
            
            \hline
            \rowcolor{white}
            \caption{Indice di Gulpease dei verbali}
        \end{longtable}
    \end{center}
\end{center}

\newpage
\subsection{Revisione di progettazione} 

\subsubsection{Analisi statica e automatizzata dei documenti}\label{resocontoProgettazione}
Tutta la documentazione prodotta in ingresso alla Revisione di Progettazione è stata sottoposta ad una meticolosa attività di verifica
basata su quanto previsto all'interno del documento delle \dext{NormeDiProgetto\_2.0.0}.
Questa attività è stata espletata dai verificatori.
Come avvenuto per la Revisione dei Requisiti tutta la documentazione è stata sottoposta ad analisi automatizzata.
La correttezza ortografica si è mantenuta stabile sul valore nullo.

\begin{center}
    \begin{figure}[!htb]
        \centering
        \includegraphics[scale=0.60]{res/images/grafico_gulpease_rp.png}
        \caption{Grafico Indice di Gulpease}
    \end{figure}
    \begin{center}
        \rowcolors{2}{white}{blue!20}
        \begin{longtable}{|c|c|}
            \hline
            \rowcolor{lighter-grayer}
            \textbf{Documento}         & \textbf{Indice di Gulpease} \\
            \hline
            \endfirsthead

            \hline
            Verbale Interno 2021-01-20 & 60                          \\
            Verbale Interno 2021-01-04 & 65                          \\     
            Verbale Esterno 2021-02-15 & 63                          \\           
            \hline
            \rowcolor{white}
            \caption{Indice di Gulpease dei verbali}
        \end{longtable}
    \end{center}
\end{center}

\subsubsection{Dettaglio delle verifiche di processo}

\paragraph{MPR-1: Budgeted Cost of Work Scheduled (BCWS)}\label{_BCWS}
\begin{figure}[!htb]
    \centering
    \includegraphics[width=0.8\textwidth]{res/images/metriche_costi/BCWS.png}
    \caption{Grafico contenente il costo preventivato in euro per periodo}
\end{figure}

\paragraph{MPR-2: Actual Cost of Work Performed (ACWP)}\label{_ACWP}
\begin{figure}[!htb]
    \centering
    \includegraphics[width=0.8\textwidth]{res/images/metriche_costi/ACWP.png}
    \caption{Grafico contenente il costo effettivo in euro per periodo}
\end{figure}

\paragraph{MPR-3: Budgeted Cost of Work Performed (BCWP)}\label{_BCWP}
\begin{figure}[!htb]
    \centering
    \includegraphics[width=0.8\textwidth]{res/images/metriche_costi/BCWP.png}
    \caption{Grafico contenente il valore del prodotto in euro per periodo}
\end{figure}

\paragraph{MPR-4: Cost Variance (CV)}\label{_CV}
\begin{figure}[!htb]
    \centering
    \includegraphics[width=0.8\textwidth]{res/images/metriche_costi/CV.png}
    \caption{Grafico contenente il valore del cost variance per periodo}
\end{figure}

\paragraph{MPR-5: Schedule Variance (SV)}\label{_SV}
\begin{figure}[!htb]
    \centering
    \includegraphics[width=0.8\textwidth]{res/images/metriche_costi/SV.png}
    \caption{Grafico contenente il valore dello schedule variance per periodo}
\end{figure}