\section{Qualità del Processo}
\subsection{Scopo}
Al fine di rispettare quanto preventivato nel Piano di Progetto e per perseguire gli obiettivi di qualità è necessario operare con un ciclo di vita che determina quali processi
devono essere attivati, verificati e valutati. SWException ha deciso di adottare lo standard ISO/IEC 15504 SPICE per garantire la qualità di tutti i processi.
Per garantire il miglioramento continuo della qualità dei processi si adotta il \textit{\glock{Ciclo di Deming}}. 

\subsection{Processo di pianificazione}
Ha lo scopo di pianificare il lavoro da svolgere per consegnare un prodotto che soddisfi i requisiti richiesti. Include la definizione delle attività, dei compiti e da svolgere 
e della loro suddivisione, la pianificazione temporale ed il costo di produzione del lavoro pianificato.
Il processo di pianificazione si basa principalmente sulla definizione del \textit{way of working} del gruppo permettendo di perseguire la qualità di processo.
Ci si pone come riferimento la lista delle seguenti attività, che dovranno essere attuate durante lo sviluppo del progetto:

\begin{itemize}
    \item \textbf{Suddivisione dei compiti:} assegnazione dei compiti da realizzare ai vari componenti del gruppo;
    \item \textbf{Standard:} definire lo standard di processo ogni qualvolta sia possibile al fine di perseguire un approccio incrementale;
    \item \textbf{Formazione:} ciascun componente del gruppo deve possedere un idoneo  livello di preparazione iniziale al fine di minimizzare i ritardi nella produzione;
    \item \textbf{Budget:} è necessario contenere eventuali eccedenze rispetto al costo preventivato;
    \item \textbf{Calenderizzazione:} assicurare una pianificazione adatta ai compiti da svolgere sulla base delle risorse disponibili al fine di massimizzare l'efficienza della produzione.
\end{itemize}

\subsubsection{Metriche}

\begin{itemize}
    \item \glock{Cost Variance};
    \item \glock{Schedule Variance}.
\end{itemize}

\subsubsection{Obiettivi}
\begin{itemize}
    \item \textbf{Rispetto della pianificazione:} ci si pone l'obiettivo di rispettare quanto pianificato all'interno del \textit{Piano di Progetto v.1.0.0} 
    ove sono presenti le date di scadenza delle varie attività;
    \item \textbf{Rispetto del budget:} le variazioni delle riorse a disposizione dovranno essere minimizzate cercando di evitare discostamenti da quanto preventivato;
    \item \textbf{Rispetto del ciclo di vita:} ogni processo deve seguire le fasi del \textit{Ciclo di Deming};
    \item \textbf{Versionamento:} i prodotti devono essere tracciati con un numero di versione al fine di indivudare le cause di malfunzionamenti.
\end{itemize}


\subsection{Processo di analisi}
Questo processo tiene conto di tutte le attività di analisi intraprese nel corso dello sviluppo del prodotto ovvero in primis il documento di \textit{Analisi dei Requisiti}
ma anche le altre attività di identificazione ed analisi:

\begin{itemize}
    \item  \textbf{Individuazione dei requisiti:} sulla base di quanto descritto dal proponente nel capitolato d'appalto oltre che nei successivi colloqui;
    \item \textbf{Preventivo:} sulla base dei requisiti individuati e delle risorse a disposizione è necessario produrre un preventivo del costo finale del progetto;
    \item \textbf{Analisi ed identificazione dei rischi:} è fondamentale identificare ed analizzare i potenziali rischi e prevedere azioni di mitigazione degli stessi.
\end{itemize}

\subsubsection{Metriche}

Si identificano le seguenti metriche per la verifica dei processi di analisi, i risultati potrebbero non essere disponibili al presente stato di avanzamento del progetto
in quanto non valorizzabili o scarsamente quantificabili.



\subsubsection{MPR-DOC4: Cost Variance (CV)} \label{_MPR-DOC4}
\begin{itemize}
    \item \textbf{Formula}: $CV = BCWP - ACWP$;
    \item \textbf{Risultato}: se $CV > 0$ significa che il progetto produce con maggior efficienza (minor costo) rispetto a quanto pianificato, viceversa se negativo.
\end{itemize}

\subsubsection{MPR-DOC5: Schedule Variance (SV)} \label{_MPR-DOC5}
\begin{itemize}
    \item \textbf{Formula}: $SV = BCWP - BCWS$;
    \item \textbf{Risultato}: se $SV > 0$ significa che il progetto sta producendo con maggior velocità a quanto pianificato, viceversa se negativo.
\end{itemize}




\paragraph{Percentuale requisiti obbligatori soddisfatti}
    
 \begin{itemize}
    \item \textbf{Codice:} MPR-DOC6
    \item \textbf{Processo di riferimento:} sviluppo;
    \item \textbf{Sigla:} \textit{PROS}
  
    
    \begin{center}
        \(PROS = \frac{Numero\ requisiti\ obbligatori\ soddisfatti}{Numero\ requisiti\  obbligatori\ individuati}*100\)
    \end{center}

    \item \textbf{Range di valori accettabili:}
    \begin{itemize}
        \item  \textit{PROS} = 100\%
    \end{itemize}
\end{itemize}
    
  
\paragraph{Requisiti opzionali non soddisfatti:}
  \begin{itemize}
    \item \textbf{Codice:} MPR-DOC7
    \item \textbf{Sigla:} \textit{RONS}
    \item \textbf{Formula:}
    \begin{center}
        \(RONS=\frac{Numero\ requisiti\ opzionali\ non\ soddisfatti}{Numero\ requisiti\  opzionali\ individuati}*100\)
    \end{center}
    \item \textbf{Range di valori accettabili:}
    \begin{itemize}
        \item \textit{PROS} $\leq$ 100\%
    
    \end{itemize}
\end{itemize}


\paragraph{Requisiti desiderabili non soddisfatti:}
\begin{itemize}
  \item \textbf{Codice:} MPR-DOC8
  \item \textbf{Sigla:} \textit{RDNS}
  \item \textbf{Formula:}
  \begin{center}
    \(RDNS=\frac{Numero\ requisiti\ desiderabili\ non\ soddisfatti}{Numero\ desiderabili\  obbligatori\ individuati}*100\)
  \end{center}
  \item \textbf{Range di valori accettabili:}
  \begin{itemize}
      \item \textit{RDNS} $\leq$ 90\%
  
  \end{itemize}
\end{itemize}




\paragraph{Rischi non previsti ma avvenuti:}
\begin{itemize}
    \item \textbf{Codice:} MPR-DOC9
    \item \textbf{Sigla:} \textit{RNPA}
    \item \textbf{Range di valori che può assumere:}
    \begin{itemize}
        \item  RNPA >=0
    \end{itemize}
    \item \textbf{Valore ottimale desiderato:} 0
    

    
\end{itemize}












 

\subsubsection{Obiettivi}
\begin{itemize}
    \item \textbf{Soddisfacimento dei requisiti obbigliatori:} tutti i requisiti obbigliatori individuati devono essere soddisfatti a fine progetto;
    \item \textbf{Soddisfacimento dei requisiti opzionali e desiderabili:} tali requisiti dovrebbero essere sviluppati successivamente a quelli obbligatori sulla base delle 
    risorse eventualmente disponibili;
    \item \textbf{Manifestazione di rischi:} si auspica di non dover rilevare l'avvenimento di rischi durante lo sviluppo del progetto i quali potrebbero compromettere
    la pianificazione dello stesso causando ritardi. I rischi già previsti dovrebbero intaccare marginalmente la pianificazione in quanto saranno messe in campo le attività
    di mitigazione previste.
\end{itemize}



\subsection{Produzione di documenti}
Questo processo termina contestualmente alla consegna del prodotto al comittente in quanto ha il compito di produrre la documentazione che riporti le decisioni intraprese,
gli strumenti utilizzati nonchè le variazioni attuate nel corso del progetto. Si deve tener conto di quanto previsto nel documento delle \textit{Norme di Progetto} per quanto 
concerne il ciclo di vita di un documento


\subsubsection{Obiettivi}
\begin{itemize}
    \item \textbf{Rispetto delle fasi del ciclo di vita:} la produzione di documenti deve rispettare il ciclo di vita prestabilito;
  
\end{itemize}


\subsection{Verifica}
Processo attivo per tutta la durata del progetto per la valutazione della correttezza e qualità dei prodotti e per l'individuazione di errori 

\begin{itemize}
    \item  \textbf{Verifica funzionalità:} tutti i prodotti devono soddisfare i requisiti richiesti in modo corretto;
    \item \textbf{Verifica aderenza alle norme di progetto:} è necessario verificare che lo sviluppo del progetto sia conforme alle norme di progetto stabilite.
    
\end{itemize}


\subsubsection{Obiettivi}
\begin{itemize}
    \item \textbf{Verifica continua:} ogni prodotto deve essere controllato e testato costantemente in particolare dopo ogni modifica;

\end{itemize}

\subsection{Ciclo di Deming}
È un metodo di gestione utilizzato per il controllo e miglioramento continuo dei processi e dei prodotti suddiviso in 4 fasi: Plan, Do, Check, Act.
Al fine di garantire la qualità dei processi e l'aderenza agli standard è stato deciso di adottare il ciclo PDCA.
\begin{itemize}
    \item \textbf{Plan:} ovvero pianificare, significa determinare gli obiettivi del sistema e i suoi processi, implica altresì la definizione delle risorse necessarie per fornire il prodotto conforme ai requisiti del cliente nei tempi concordati;
    \item \textbf{Do:} è la fase realizzativa nella quale si mette in pratica ciò che è stato pianificato;
    \item \textbf{Check:} si analizzano i risultati del punto precedente e li si confrontano con gli obiettivi della pianificazione;
    \item \textbf{Act:} è la fase della correzione e del miglioramento atta ad intraprendere azioni dirette a migliorare le prestazioni.
\end{itemize}














