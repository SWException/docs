\section{Specifica dei test}
\begin{itemize}
    \item \textbf{Test di unità:} si effettua per verificare il corretto comportamento di un singolo componente del programma in modo indipendente da altre unità;
    \item \textbf{Test di integrazione:} si effettua per verificare la corretta integrazione tra due o più unità distinte;
    \item \textbf{Test di sistema:} verificano la funzionalità dell'interno sistema verificando il soddisfacimento dei requisiti obbligatori;
    \item \textbf{Test di accettazione:} vengono eseguiti in collaborazione con il comittente prima di procedere al rilascio.
\end{itemize}

Allo stato attuale non è possibile definire quali test saranno realizzati si rimanda ad una versione del presente documento successiva all'attività di progettazione


\subsection{Test di sistema}

I test di sistema devono assicurare il soddisfacimento dei requisiti individuati nel documento Analisi dei erquisiti. 
Di seguito l'elenco dei test di sistema individuati.

\begin{center}
	\rowcolors{1}{white}{blue!20}
	\begin{longtable}{p{1cm}|p{6.85cm}|p{7cm}|}
	\hline
	\rowcolor{lighter-grayer}
	\textbf{Codice} & \textbf{Titolo} & \textbf{Descrizione} \\
	\hline
	\endfirsthead

	% ----- MARCO -----


	\hline
	TS1 & Visualizzazione homepage e carrello & Verificare che un utente generico possa: \begin{enumerate}
		\item  visualizzare la homepage ed i prodotti in evidenza in essa contenuti;
		\item  raggiungere e visualizzare la pagina del carrello da ogni pagina dell'applicativo.
	\end{enumerate} \\

	TS2 & Visualizzazione informazioni venditore & Verificare che un utente generico possa visualizzare la pagina contenente le informazioni relative al venditore che includono la ragione sociale e i contatti. \\

	TS3 & Funzionalità menù categorie & Verificare che un utente generico: \begin{enumerate}
		\item  possa cliccare in una delle voci del menù delle categorie ;
		\item  riceva un messaggio di errore se non vi sono prodotti all'interno della categoria selezionata;
	\end{enumerate} \\

	TS4 & Funzionalità ricerca & Verificare che un utente generico: \begin{enumerate}
		\item  possa cercare un prodotto mediante il proprio nome utilizzando la funzionalità di ricerca;
		\item  riceva un messaggio di errore se non vi sono prodotti che soddisfano la query di ricerca;
	\end{enumerate} \\


	TS5 & Visualizzazione PLP & Verificare che un utente generico possa: \begin{enumerate}
		\item  visualizzare una PLP;
		\item  visualizzare la lista prodotti con le relative immagini, titolo e prezzo IVA inclusa;
		\item  possa visualizzare la disponibilità o meno dei prodotti elencati;
		\item  possa cliccare in un prodotto per accedere alla relativa PDP.
	\end{enumerate} \\

	TS6 & Ordinamento prodotti PLP & Verificare che un utente generico possa: \begin{enumerate}
		\item  ordinare i prodotti della PLP per prezzo crescente o decrescente;
		\item  ordinare i prodotti della PLP per ordine alfabetico.
	\end{enumerate} \\

	TS7 & Filtri PLP & Verificare che un utente generico possa: \begin{enumerate}
		\item  applicare, modificare e rimuovere uno o più filtri di visualizzazione prodotti basati su un prezzo minimo e/o massimo;
		\item  applicare, modificare e rimuovere uno o più filtri basati su una categoria;
		\item  visualizzare un messaggio di errore nel caso nessun prodotto soddisfi i filtri applicati;
	\end{enumerate} \\

	TS8 & Visualizzazione PDP & Verificare che un utente generico possa visualizzare : \begin{enumerate}
		\item   la PDP di un prodotto contenente la descrizione ed una o più foto del prodotto;
		\item   la disponibilità o meno del prodotto;
		\item   il prezzo del prodotto IVA inclusa;
		\item   se ha già aggiunto l'articolo nel carrello.
	\end{enumerate} \\

	TS9 & Aggiunta a carrello & Verificare che un utente generico possa : \begin{enumerate}
		\item   aggiungere un prodotto al carrello potendo specificare la quantità;
		\item   visualizzare un errore nel caso voglia aggiungere al carrello una quantità maggiore di quella disponibile;
	
	\end{enumerate} \\

	TS10 & Sincronizzazione carrello & Verificare che un utente autenticato possa recuperare il proprio carrello da qualsiasi dispositivo esso faccia il login.\\





	\hline

	\end{longtable}
\end{center}

\begin{center}
	\rowcolors{1}{white}{blue!20}
	\begin{longtable}{|c|c|c|}
	\hline
	\rowcolor{lighter-grayer}
	\textbf{Codice} & \textbf{Titolo} & \textbf{Descrizione} \\
	\hline
	\endfirsthead

	% ----- MICHELE -----


	\hline
	TS1 & Autenticazione utente & Verificare che l'utente non autenticato possa  \\
	TS2 & Autenticazione utente & Verificare che l'utente non autenticato possa  \\

	\hline

	\end{longtable}
\end{center}

\begin{center}
	\rowcolors{1}{white}{blue!20}
	\begin{longtable}{|c|c|c|}
	\hline
	\rowcolor{lighter-grayer}
	\textbf{Codice} & \textbf{Titolo} & \textbf{Descrizione} \\
	\hline
	\endfirsthead

	% ----- STEFANO -----


	\hline
	TS1 & Autenticazione utente & Verificare che l'utente non autenticato possa  \\
	TS2 & Autenticazione utente & Verificare che l'utente non autenticato possa  \\

	\hline

	\end{longtable}
\end{center}


\subsection{Tracciamento test di sistema - requisiti}

\begin{center}
	\rowcolors{1}{white}{blue!20}
	\begin{longtable}{|c|c|}
	\hline
	\rowcolor{lighter-grayer}
	\textbf{Test di sistema} & \textbf{Requisiti} \\
	\hline
	\endfirsthead

	% ----- MARCO -----

	
	\hline
	R1F1 & TSA1  \\
	REQ2 & TSA1 \\

	\hline

	\end{longtable}
\end{center}

\begin{center}
	\rowcolors{1}{white}{blue!20}
	\begin{longtable}{|c|c|}
	\hline
	\rowcolor{lighter-grayer}
	\textbf{Test di sistema} & \textbf{Requisiti} \\
	\hline
	\endfirsthead

	% ----- MICHELE -----

	
	\hline
	R1F1 & TSA1  \\
	REQ2 & TSA1 \\

	\hline

	\end{longtable}
\end{center}


\begin{center}
	\rowcolors{1}{white}{blue!20}
	\begin{longtable}{|c|c|}
	\hline
	\rowcolor{lighter-grayer}
	\textbf{Test di sistema} & \textbf{Requisiti} \\
	\hline
	\endfirsthead

	% ----- STEFANO -----

	
	\hline
	R1F1 & TSA1  \\
	REQ2 & TSA1 \\

	\hline

	\end{longtable}
\end{center}



