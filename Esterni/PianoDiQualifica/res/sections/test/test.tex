\section{Specifica dei test}
\begin{itemize}
    \item \textbf{Test di unità:} si effettua per verificare il corretto comportamento di un singolo componente del programma in modo indipendente da altre unità;
    \item \textbf{Test di integrazione:} si effettua per verificare la corretta integrazione tra due o più unità distinte;
    \item \textbf{Test di sistema:} verificano la funzionalità dell'interno sistema verificando il soddisfacimento dei requisiti obbligatori;
    \item \textbf{Test di accettazione:} vengono eseguiti in collaborazione con il comittente prima di procedere al rilascio.
\end{itemize}

Allo stato attuale non è possibile definire quali test saranno realizzati si rimanda ad una versione del presente documento successiva all'attività di progettazione


\subsection{Test di sistema}

I test di sistema devono assicurare il soddisfacimento dei requisiti individuati nel documento Analisi dei erquisiti. 
Di seguito l'elenco dei test di sistema individuati.

\begin{center}
	\rowcolors{1}{white}{blue!20}
	\begin{longtable}{|c|c|c|}
	\hline
	\rowcolor{lighter-grayer}
	\textbf{Codice} & \textbf{Titolo} & \textbf{Descrizione} \\
	\hline
	\endfirsthead

	% ----- Modificare da qui -----


	\hline
	TSA1 & Autenticazione utente & Verificare che l'utente non autenticato possa  \\
	TSA2 & Autenticazione utente & Verificare che l'utente non autenticato possa  \\

	\hline

	\end{longtable}
\end{center}


\subsection{Tracciamento requisiti - Test di sistema}

\begin{center}
	\rowcolors{1}{white}{blue!20}
	\begin{longtable}{|c|c|}
	\hline
	\rowcolor{lighter-grayer}
	\textbf{Requisito} & \textbf{Test di sistema} \\
	\hline
	\endfirsthead

	% ----- Modificare da qui -----

	
	\hline
	REQ1 & TSA1  \\
	REQ2 & TSA1 \\

	\hline

	\end{longtable}
\end{center}


