\section{Specifica dei test}
\begin{itemize}
    \item \textbf{Test di unità:} si effettua per verificare il corretto comportamento di un singolo componente del programma in modo indipendente da altre unità;
    \item \textbf{Test di integrazione:} si effettua per verificare la corretta integrazione tra due o più unità distinte;
    \item \textbf{Test di sistema:} verificano la funzionalità dell'interno sistema verificando il soddisfacimento dei requisiti obbligatori;
    \item \textbf{Test di accettazione:} vengono eseguiti in collaborazione con il comittente prima di procedere al rilascio.
\end{itemize}

Allo stato attuale non è possibile definire quali test saranno realizzati si rimanda ad una versione del presente documento successiva all'attività di progettazione


\subsection{Test di sistema}

I test di sistema devono assicurare il soddisfacimento dei requisiti individuati nel documento Analisi dei erquisiti. 
Di seguito l'elenco dei test di sistema individuati.

\begin{center}
	\rowcolors{1}{white}{blue!20}
	\begin{longtable}{|c|c|c|}
	\hline
	\rowcolor{lighter-grayer}
	\textbf{Codice} & \textbf{Titolo} & \textbf{Descrizione} \\
	\hline
	\endfirsthead

	% ----- MARCO -----


	\hline
	TS1 & Autenticazione utente & Verificare che l'utente non autenticato possa  \\
	TS2 & Autenticazione utente & Verificare che l'utente non autenticato possa  \\

	\hline

	\end{longtable}
\end{center}

\begin{center}
	\rowcolors{1}{white}{blue!20}
	\begin{longtable}{|c|c|c|}
	\hline
	\rowcolor{lighter-grayer}
	\textbf{Codice} & \textbf{Titolo} & \textbf{Descrizione} \\
	\hline
	\endfirsthead

	% ----- MICHELE -----


	\hline
	TS1 & Autenticazione utente & Verificare che l'utente non autenticato possa  \\
	TS2 & Autenticazione utente & Verificare che l'utente non autenticato possa  \\

	\hline

	\end{longtable}
\end{center}

\begin{center}
	\rowcolors{1}{white}{blue!20}
	\begin{longtable}{|c|c|c|}
	\hline
	\rowcolor{lighter-grayer}
	\textbf{Codice} & \textbf{Titolo} & \textbf{Descrizione} \\
	\hline
	\endfirsthead

	% ----- STEFANO -----


	\hline
	TS3.1 & Verifica gestione propri ordini & Si verifichi che il cliente possa gestire i propri ordini  \\
	TS3.2 & Verifica gestione propri ordini & Si verifichi che il cliente possa contattare il venditore tramite i form per annullare l'ordine, segnalare problemi dell'ordine, chiedere il reso \\
	TS3.3 & Verifica logout cliente & Si verifichi che il cliente loggato possa effettuare il logout e che venga riportato alla home page \\
	TS3.4 & Verifica contatto venditore & Si verifichi che il cliente possa contattare il venditore tramite l'apposito form  \\
	TS3.5 & Verifica dashboard ordini venditore & Si verifichi che il venditore possa visualizzare e modificare la lista degli ordini ricevuti \\
	TS3.6 & Verifica dashboard ordini venditore & Si verifichi che il venditore possa cercare un ordine presente nel sistema \\
	TS3.7 & Verifica dashboard ordini venditore & Si verifichi che il venditore possa modoficare lo stato di un ordine  \\
	TS3.8 & Verifica dashboard ordini venditore & Si verifichi che il venditore possa stampare la bolla per un ordine  \\
	TS3.9 & Verifica dashboard ordini venditore & Si verifichi che il venditore possa visualizzare i dettagli di un ordine \\
	TS3.10 & Verifica dashboard clienti venditore & Si verifichi che il venditore possa visualizzare la lista dei clienti \\
	TS3.11 & Verifica dashboard clienti venditore & Si verifichi che il venditore possa cercare un determinato cliente nel sistema \\
	TS3.12 & Verifica dashboard clienti venditore & Si verifichi che il venditore possa ordinare i clienti con determinati filtri \\
	TS3.13 & Verifica dashboard clienti venditore & Si verifichi che il venditore possa contattare il cliente tramite un form \\
	TS3.14 & Verifica dashboard aliquote venditore & Si verifichi che il venditore possa gestire le aliquote IVA per la tassazione dei prodotti \\
	TS3.15 & Verifica dashboard aliquote venditore & Si verifichi che il venditore possa visualizzare le aliquote IVA già presenti nel sistema \\
	TS3.16 & Verifica dashboard aliquote venditore & Si verifichi che il venditore possa aggiungere un'aliquota IVA \\
	TS3.17 & Verifica dashboard aliquote venditore & Si verifichi che il venditore possa modificare un'aliquota IVA \\
	TS3.18 & Verifica dashboard aliquote venditore & Si verifichi che il venditore possa eliminare un'aliquota IVA già presente nel sistema \\
	TS3.19 & Verifica dashboard prodotti venditore & Si verifichi che il venditore possa visualizzare la lista dei prodotti inseriti nel sistema \\
	TS3.20 & Verifica dashboard prodotti venditore & Si verifichi che il venditore possa cercare un prodotto presente nel sistema \\
	TS3.21 & Verifica dashboard prodotti venditore & Si verifichi che il venditore possa filtrare i prodotti \\
	TS3.22 & Verifica dashboard prodotti venditore & Si verifichi che il venditore possa aggiungere un prodotto al sistema dopo aver compilato tutti i campi obbligatori del prodotto \\
	TS3.23 & Verifica dashboard prodotti venditore & Si verifichi che il venditore possa modificare un prodotto aggiornando i campi già presenti\\
	TS3.24 & Verifica dashboard prodotti venditore & Si verifichi che il venditore possa eliminare un prodotto presente nel sistema\\
	TS3.25 & Verifica dashboard prodotti venditore & Si verifichi che il venditore possa selezionare i prodotti da mostrare nella HomePage cliente \\
	TS3.26 & Verifica dashboard categorie venditore & Si verifichi che il venditore possa visualizzare tutte le categorie presenti nel sistema \\
	TS3.27 & Verifica dashboard categorie venditore & Si verifichi che il venditore possa aggiungere una categoria di prodotti impostando il nome \\
	TS3.28 & Verifica dashboard categorie venditore & Si verifichi che il venditore possa rimuovere una categoria di prodottti \\
	TS3.29 & Verifica dashboard categorie venditore & Si verifichi che il venditore possa modificare il nome di una categoria già esistente nel sistema \\
	TS3.30 & Verifica login venditore & Si verifichi che il venditore possa effettuare il login \\
	TS3.31 & Verifica logout venditore & Si verifichi che il venditore possa effettuare il logout \\
	\hline

	\end{longtable}
\end{center}


\subsection{Tracciamento test di sistema - requisiti}

\begin{center}
	\rowcolors{1}{white}{blue!20}
	\begin{longtable}{|c|c|}
	\hline
	\rowcolor{lighter-grayer}
	\textbf{Test di sistema} & \textbf{Requisiti} \\
	\hline
	\endfirsthead

	% ----- MARCO -----

	
	\hline
	TS3.1 &  R1F19 \\
	TS3.2 &  R1F19 \\
	TS3.3 &  R1F20 \\
	TS3.4 &  R1F21 \\
	TS3.5 &  R1F22 \\
	TS3.6 &  R2F22.1 R2F22.2 \\
	TS3.7 &  R1F22.3 \\
	TS3.8 &  R1F22.4 \\
	TS3.9 &  R1F22.5 \\
	TS3.10 & R1F23 \\
	TS3.11 & R2F23.1 \\
	TS3.12 & R2F23.2 \\
	TS3.13 & R1F23.3 \\
	TS3.14 & R1F24 \\
	TS3.15 & R1F24.1 \\
	TS3.16 & R1F24.2 \\
	TS3.17 & R1F24.3 \\
	TS3.18 & R1F24.4 \\
	TS3.19 & R1F25 \\
	TS3.20 & R2F25.1 \\
	TS3.21 & R2F25.2 \\
	TS3.22 & R1F25.4 \\
	TS3.23 & R1F25.6 \\
	TS3.24 & R1F25.7 \\
	TS3.25 & R1F25.8 \\
	TS3.26 & R1F26 \\
	TS3.27 & R1F26.1 \\
	TS3.28 & R1F26.2 \\
	TS3.29 & R1F26.3  \\
	TS3.30 & R1F27 \\
	TS3.31 & R1F28 \\
	

	\hline

	\end{longtable}
\end{center}

\begin{center}
	\rowcolors{1}{white}{blue!20}
	\begin{longtable}{|c|c|}
	\hline
	\rowcolor{lighter-grayer}
	\textbf{Test di sistema} & \textbf{Requisiti} \\
	\hline
	\endfirsthead

	% ----- MICHELE -----

	
	\hline
	R1F1 & TSA1  \\
	REQ2 & TSA1 \\

	\hline

	\end{longtable}
\end{center}


\begin{center}
	\rowcolors{1}{white}{blue!20}
	\begin{longtable}{|c|c|}
	\hline
	\rowcolor{lighter-grayer}
	\textbf{Test di sistema} & \textbf{Requisiti} \\
	\hline
	\endfirsthead

	% ----- STEFANO -----

	
	\hline
	R1F1 & TSA1  \\
	REQ2 & TSA1 \\

	\hline

	\end{longtable}
\end{center}



