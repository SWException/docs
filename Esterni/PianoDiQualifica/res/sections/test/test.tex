
\newcounter{CTS} % Contatore Test
\newcommand{\stepCTS}[0]{\stepcounter{CTS}} % incrementa il contatore CTS
\newcommand{\valueCTS}[0]{\stepCTS TS\arabic{CTS}} % ritorna il valore del contatore CTS incrementato di 1
\newcommand{\resetCTS}[0]{\setcounter{CTS}{0}}
\resetCTS
\section{Specifica dei test}
\begin{itemize}
    \item \textbf{Test di unità:} si effettua per verificare il corretto comportamento di un singolo componente del programma in modo indipendente da altre unità;
    \item \textbf{Test di integrazione:} si effettua per verificare la corretta integrazione tra due o più unità distinte;
    \item \textbf{Test di sistema:} verificano la funzionalità dell'interno sistema verificando il soddisfacimento dei requisiti obbligatori;
    \item \textbf{Test di accettazione:} vengono eseguiti in collaborazione con il committente prima di procedere al rilascio.
\end{itemize}

Allo stato attuale non è possibile definire quali test saranno realizzati si rimanda ad una versione del presente documento successiva all'attività di progettazione


\subsection{Test di sistema}

I test di sistema devono assicurare il soddisfacimento dei requisiti individuati nel documento Analisi dei erquisiti. 
Di seguito l'elenco dei test di sistema individuati.

\begin{center}
	\rowcolors{1}{white}{blue!20}
	\begin{longtable}{p{1cm}|p{6.85cm}|p{7cm}|}
	\hline
	\rowcolor{lighter-grayer}
	\textbf{Codice} & \textbf{Titolo} & \textbf{Descrizione} \\
	\hline
	\endfirsthead

	% ----- MARCO -----


	\hline
	
	\valueCTS & Visualizzazione homepage e carrello & Verificare che un utente generico possa:
	\begin{enumerate}
		\item  visualizzare la homepage ed i prodotti in evidenza in essa contenuti;
		\item  raggiungere e visualizzare la pagina del carrello da ogni pagina dell'applicativo.
	\end{enumerate} \\

	\valueCTS & Visualizzazione informazioni venditore & Verificare che un utente generico possa visualizzare la pagina contenente le informazioni relative al venditore che includono la ragione sociale e i contatti. \\

	\valueCTS & Funzionalità menù categorie & Verificare che un utente generico: 
	\begin{enumerate}
		\item  possa cliccare in una delle voci del menù delle categorie ;
		\item  riceva un messaggio di errore se non vi sono prodotti all'interno della categoria selezionata;
	\end{enumerate} \\

	\valueCTS & Funzionalità ricerca & Verificare che un utente generico: 
	\begin{enumerate}
		\item  possa cercare un prodotto mediante il proprio nome utilizzando la funzionalità di ricerca;
		\item  riceva un messaggio di errore se non vi sono prodotti che soddisfano la query di ricerca;
	\end{enumerate} \\


	\valueCTS & Visualizzazione PLP & Verificare che un utente generico possa: 
	\begin{enumerate}
		\item  visualizzare una PLP;
		\item  visualizzare la lista prodotti con le relative immagini, titolo e prezzo IVA inclusa;
		\item  visualizzare la disponibilità o meno dei prodotti elencati;
		\item  cliccare in un prodotto per accedere alla relativa PDP.
	\end{enumerate} \\

	\valueCTS & Ordinamento prodotti PLP & Verificare che un utente generico possa: 
	\begin{enumerate}
		\item  ordinare i prodotti della PLP per prezzo crescente o decrescente;
		\item  ordinare i prodotti della PLP per ordine alfabetico.
	\end{enumerate} \\

	\valueCTS & Filtri PLP & Verificare che un utente generico possa: 
	\begin{enumerate}
		\item  applicare, modificare e rimuovere uno o più filtri di visualizzazione prodotti basati su un prezzo minimo e/o massimo;
		\item  applicare, modificare e rimuovere uno o più filtri basati su una categoria;
		\item  visualizzare un messaggio di errore nel caso nessun prodotto soddisfi i filtri applicati;
	\end{enumerate} \\

	\valueCTS & Visualizzazione PDP & Verificare che un utente generico possa visualizzare: 
	\begin{enumerate}
		\item   la PDP di un prodotto contenente la descrizione ed una o più foto del prodotto;
		\item   la disponibilità o meno del prodotto;
		\item   il prezzo del prodotto IVA inclusa;
		\item   se ha già aggiunto l'articolo nel carrello.
	\end{enumerate} \\

	\valueCTS & Aggiunta a carrello & Verificare che un utente generico possa : 
	\begin{enumerate}
		\item   aggiungere un prodotto al carrello potendo specificare la quantità;
		\item   visualizzare un errore nel caso voglia aggiungere al carrello una quantità maggiore di quella disponibile;
	\end{enumerate} \\

	\valueCTS & Sincronizzazione carrello & Verificare che un utente autenticato possa recuperare il proprio carrello da qualsiasi dispositivo esso faccia il login.\\

	\valueCTS & Apertura pagina carrello & Verificare che: 
	\begin{enumerate}
		\item i prodotti inseriti precedentemente vengano visualizzati nella giusta quantità;
		\item il totale dell'ordine sia corretto (solo prodotti, senza tasse e spedizione);
		\item il totale dell'ordine con le tasse aggiunte sia corretto;
		\item si possa vedere il costo dei singoli prodotti;
		\item si visualizzi il messaggio di carrello vuoto nel caso in cui lo sia.
	\end{enumerate} \\
	\valueCTS & Modifica quantità prodotti nel carrello & Verificare che:
	\begin{enumerate}
		\item si possa aumentare e diminuire la quantità di un singolo articolo nel carrello;
		\item in caso di aumento, se la quantità richiesta non è disponibile, venga visualizzato il giusto messaggio di errore;
		\item si possa eliminare un singolo articolo.
	\end{enumerate}\\
	\valueCTS & Svuotare il carrello & Verificare che:
	\begin{enumerate}
		\item tutti gli elementi nel carrello possano essere rimossi.
	\end{enumerate} \\
	\valueCTS & Processo di checkout - inserimento indirizzi e calcolo spedizione & Verificare che:
	\begin{enumerate}
		\item il messaggio mostri un errore se si avvia il checkout con carrello vuoto;
		\item si possano inserire indirizzi di fatturazione e spedizione da form e da quelli salvati (se autenticato);
		\item l'indirizzo di spedizione possa essere copiato con un click da quello scelto per la fatturazione;
		\item vengano aggiunti e visualizzati i costi di spedizione al totale dell'ordine, dopo aver scelto l'indirizzo di spedizione. 
	\end{enumerate} \\
	\valueCTS & Processo di checkout - pagamento & Verificare che:
	\begin{enumerate}
		\item il sistema di pagamento con il provider esterno possa essere utilizzato per concludere il checkout;
		\item nel caso di fallimento del pagamento, il sistema mostri in modo chiaro che il pagamento non è andato a buon fine e l'ordine non è stato emesso;
		\item nel caso di fallimento del pagamento, si possa riprovare immediatamente;
		\item in caso di successo del pagamento, si avvisa il cliente dell'avvenuto ordine, vengono scalate le merci acquistate dal magazzino e il venditore possa gestire l'ordine.
	\end{enumerate} \\
	\valueCTS & Login cliente & Verificare che:
	\begin{enumerate}
		\item il sistema di login cliente permetta di autenticarsi all'applicazione con un profilo già registrato, inserendo le corrette credenziali;
		\item il cliente possa navigare tutto il sito mantenendo la sessione di login;
		\item in caso di credenziali errate, venga mostrato il relativo messaggio di errore e si permetta di riprovare nell'immediato.
	\end{enumerate} \\
	\valueCTS & Registrazione cliente & Verificare che:
	\begin{enumerate}
		\item un cliente non autenticato possa avviare il processo di registrazione;
		\item nel caso in cui il doppio inserimento dell'email fallisca (e.g. non corrisponde) si visualizzi il relativo messaggio d'errore;
		\item nel caso in cui il doppio inserimento della password fallisca (e.g. non corrisponde) si visualizzi il relativo messaggio d'errore;
		\item nel caso in cui la password non rispetti i requisiti minimi di sicurezza si visualizzi il relativo messaggio d'errore;
		\item nel caso in cui esista già un utente con la stessa email si visualizzi il relativo messaggio d'errore;
		\item nel caso in cui tutto vada a buon fine, arrivi l'email per la verifica dell'indirizzo inserito;
		\item il click sul link di conferma ricevuto abiliti l'account appena registrato.
	\end{enumerate} \\
	\valueCTS & Recupero password cliente & Verificare che:
	\begin{enumerate}
		\item durante il processo di login si possa avviare quello per il recupero della password;
		\item si riceva il link per la reimpostazione della password al proprio indirizzo email, nel caso si richieda;
		\item cliccando il link ricevuto si possa inserire la password con un doppio inserimento;
		\item nel caso in cui i due inserimenti non corrispondano e/o la password non rispetti i requisiti di sicurezza si mostri il relativo messaggio d'errore;
		\item nel caso in cui si inseriscano dei dati validi, mostrare il messaggio di conferma cambio password.
	\end{enumerate} \\
	\begin{enumerate}
		\item si possa inserire un nuovo indirizzo e che questo sia utilizzabile durante il processo di checkout; 
		\item nel caso in cui si inserisca un indirizzo non valido (e.g. campi obbligatori vuoti) si mostri un messaggio di errore e si permetta la modifica immediata dei dati inseriti;
		\item si possa eliminare un indirizzo precedentemente inserito.
	\end{enumerate} \\
	\valueCTS & Amministrazione account cliente - assistenza & Verificare che:
	\begin{enumerate}
		\item il cliente possa aprire un ticket di assistenza e che questo sia gestibile dal venditore;
		\item il cliente possa avviare la richiesta di cancellazione account tramite un apposito tipo di ticket di assistenza.
	\end{enumerate} \\
	\valueCTS & Amministrazione account cliente - modifica dati login & Verificare che:
	\begin{enumerate}
		\item il cliente possa cambiare indirizzo email;
		\item il sistema si comporti per il cambio email come nel processo di registrazione;
		\item il cliente possa cambiare password inserendo quella vecchia;
		\item il sistema si comporti per il cambio password come nel processo di registrazione. 
	\end{enumerate} \\

	\hline
	\valueCTS & Verifica gestione propri ordini & Si verifichi che il cliente possa gestire i propri ordini  \\
	\valueCTS& Verifica gestione propri ordini & Si verifichi che il cliente possa contattare il venditore tramite i form per annullare l'ordine, segnalare problemi dell'ordine, chiedere il reso \\
	\valueCTS & Verifica logout cliente & Si verifichi che il cliente loggato possa effettuare il logout e che venga riportato alla home page \\
	\valueCTS & Verifica contatto venditore & Si verifichi che il cliente possa contattare il venditore tramite l'apposito form  \\
	\valueCTS& Verifica dashboard ordini venditore & Si verifichi che il venditore possa visualizzare e modificare la lista degli ordini ricevuti \\
	\valueCTS & Verifica dashboard ordini venditore & Si verifichi che il venditore possa cercare un ordine presente nel sistema \\
	\valueCTS & Verifica dashboard ordini venditore & Si verifichi che il venditore possa modificare lo stato di un ordine  \\
	\valueCTS & Verifica dashboard ordini venditore & Si verifichi che il venditore possa stampare la bolla per un ordine  \\
	\valueCTS & Verifica dashboard ordini venditore & Si verifichi che il venditore possa visualizzare i dettagli di un ordine \\
	\valueCTS & Verifica dashboard clienti venditore & Si verifichi che il venditore possa visualizzare la lista dei clienti \\
	\valueCTS & Verifica dashboard clienti venditore & Si verifichi che il venditore possa cercare un determinato cliente nel sistema \\
	\valueCTS & Verifica dashboard clienti venditore & Si verifichi che il venditore possa ordinare i clienti con determinati filtri \\
	\valueCTS & Verifica dashboard clienti venditore & Si verifichi che il venditore possa contattare il cliente tramite un form \\
	\valueCTS & Verifica dashboard aliquote venditore & Si verifichi che il venditore possa gestire le aliquote IVA per la tassazione dei prodotti \\
	\valueCTS & Verifica dashboard aliquote venditore & Si verifichi che il venditore possa visualizzare le aliquote IVA già presenti nel sistema \\
	\valueCTS & Verifica dashboard aliquote venditore & Si verifichi che il venditore possa aggiungere un'aliquota IVA \\
	\valueCTS & Verifica dashboard aliquote venditore & Si verifichi che il venditore possa modificare un'aliquota IVA \\
	\valueCTS & Verifica dashboard aliquote venditore & Si verifichi che il venditore possa eliminare un'aliquota IVA già presente nel sistema \\
	\valueCTS & Verifica dashboard prodotti venditore & Si verifichi che il venditore possa visualizzare la lista dei prodotti inseriti nel sistema \\
	\valueCTS& Verifica dashboard prodotti venditore & Si verifichi che il venditore possa cercare un prodotto presente nel sistema \\
	\valueCTS & Verifica dashboard prodotti venditore & Si verifichi che il venditore possa filtrare i prodotti \\
	\valueCTS & Verifica dashboard prodotti venditore & Si verifichi che il venditore possa aggiungere un prodotto al sistema dopo aver compilato tutti i campi obbligatori del prodotto \\
	\valueCTS & Verifica dashboard prodotti venditore & Si verifichi che il venditore possa modificare un prodotto aggiornando i campi già presenti\\
	\valueCTS & Verifica dashboard prodotti venditore & Si verifichi che il venditore possa eliminare un prodotto presente nel sistema\\
	\valueCTS & Verifica dashboard prodotti venditore & Si verifichi che il venditore possa selezionare i prodotti da mostrare nella HomePage cliente \\
	\valueCTS & Verifica dashboard categorie venditore & Si verifichi che il venditore possa visualizzare tutte le categorie presenti nel sistema \\
	\valueCTS & Verifica dashboard categorie venditore & Si verifichi che il venditore possa aggiungere una categoria di prodotti impostando il nome \\
	\valueCTS & Verifica dashboard categorie venditore & Si verifichi che il venditore possa rimuovere una categoria di prodotti \\
	\valueCTS & Verifica dashboard categorie venditore & Si verifichi che il venditore possa modificare il nome di una categoria già esistente nel sistema \\
	\valueCTS & Verifica login venditore & Si verifichi che il venditore possa effettuare il login \\
	\valueCTS & Verifica logout venditore & Si verifichi che il venditore possa effettuare il logout \\
	\hline

	\end{longtable}
\end{center}


\subsection{Tracciamento test di sistema - requisiti}

\begin{center}
	\rowcolors{1}{white}{blue!20}
	\begin{longtable}{|c|c|}
	\hline
	\rowcolor{lighter-grayer}
	\textbf{Test di sistema} & \textbf{Requisiti} \\
	\hline
	\endfirsthead

	% ----- MARCO -----


	\hline
	TS1 & R1F1, R1F1.1, R1F1.2  \\
	TS2 & R1F2, R1F2.1 \\
	TS3 & R1F3, R1F3.1 \\
	TS4 & R1F4, R1F4.1 \\
	TS5 & R1F5, R1F5.1, R1F5.2, R1F5.3, R1F5.4, R1F5.5, R1F5.6 \\
	TS6 & R1F6, R1F6.1, R1F6.2, R1F6.3 \\
	TS7 & R1F7, R1F7.1, R1F7.2, R1F7.3, R1F7.4, R1F7.5, R1F7.6, R1F7.7, R1F7.8 \\
	TS8 & R1F8, R1F8.1, R1F8.2, R1F8.3, R1F8.4, R1F8.5, R1F8.6, R1F8.7, R1F8.8, R1F8.9 \\
	TS9 & R1F9, R1F8.1, R1F8.2, R1F8.3 \\
	TS10 & R1F10, R1F10.1 \\

	
	\hline
	
	

	\hline

	\end{longtable}
\end{center}

\begin{center}
	\rowcolors{1}{white}{blue!20}
	\begin{longtable}{|c|c|}
	\hline
	\rowcolor{lighter-grayer}
	\textbf{Test di sistema} & \textbf{Requisiti} \\
	\hline
	\endfirsthead

	% ----- MICHELE -----

	
	\hline
	TS2.1 & R1F11  \\
	TS2.2 & R1F12 \\
	TS2.3 & R1F13 \\
	TS2.4 & R1F14 \\
	TS2.5 & R1F14 \\
	TS2.6 & R1F15 \\
	TS2.7 & R1F16 \\
	TS2.8 & R1F17 \\
	TS2.9 & R1F18 \\
	TS2.10 & R1F18 \\
	TS2.11 & R1F18 \\
	\hline

	\end{longtable}
\end{center}


\begin{center}
	\rowcolors{1}{white}{blue!20}
	\begin{longtable}{|c|c|}
	\hline
	\rowcolor{lighter-grayer}
	\textbf{Test di sistema} & \textbf{Requisiti} \\
	\hline
	\endfirsthead

	% ----- STEFANO -----

	
	\hline
	TS3.1 &  R1F19 \\
	TS3.2 &  R1F19 \\
	TS3.3 &  R1F20 \\
	TS3.4 &  R1F21 \\
	TS3.5 &  R1F22 \\
	TS3.6 &  R2F22.1 R2F22.2 \\
	TS3.7 &  R1F22.3 \\
	TS3.8 &  R1F22.4 \\
	TS3.9 &  R1F22.5 \\
	TS3.10 & R1F23 \\
	TS3.11 & R2F23.1 \\
	TS3.12 & R2F23.2 \\
	TS3.13 & R1F23.3 \\
	TS3.14 & R1F24 \\
	TS3.15 & R1F24.1 \\
	TS3.16 & R1F24.2 \\
	TS3.17 & R1F24.3 \\
	TS3.18 & R1F24.4 \\
	TS3.19 & R1F25 \\
	TS3.20 & R2F25.1 \\
	TS3.21 & R2F25.2 \\
	TS3.22 & R1F25.4 \\
	TS3.23 & R1F25.6 \\
	TS3.24 & R1F25.7 \\
	TS3.25 & R1F25.8 \\
	TS3.26 & R1F26 \\
	TS3.27 & R1F26.1 \\
	TS3.28 & R1F26.2 \\
	TS3.29 & R1F26.3  \\
	TS3.30 & R1F27 \\
	TS3.31 & R1F28 \\
	\hline

	\end{longtable}
\end{center}



