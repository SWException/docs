\section{Specifica dei test}
\begin{itemize}
    \item \textbf{Test di unità:} si effettua per verificare il corretto comportamento di un singolo componente del programma in modo indipendente da altre unità;
    \item \textbf{Test di integrazione:} si effettua per verificare la corretta integrazione tra due o più unità distinte;
    \item \textbf{Test di sistema:} verificano la funzionalità dell'interno sistema verificando il soddisfacimento dei requisiti obbligatori;
    \item \textbf{Test di accettazione:} vengono eseguiti in collaborazione con il comittente prima di procedere al rilascio.
\end{itemize}

Allo stato attuale non è possibile definire quali test saranno realizzati si rimanda ad una versione del presente documento successiva all'attività di progettazione


\subsection{Test di sistema}

I test di sistema devono assicurare il soddisfacimento dei requisiti individuati nel documento Analisi dei erquisiti. 
Di seguito l'elenco dei test di sistema individuati.

\begin{center}
	\rowcolors{1}{white}{blue!20}
	\begin{longtable}{|p{1cm}|p{6.85cm}|p{7cm}|}
	\hline
	\rowcolor{lighter-grayer}
	\textbf{Codice} & \textbf{Titolo} & \textbf{Descrizione} \\
	\hline
	\endfirsthead

	% ----- MARCO -----


	\hline
	TS1 & Autenticazione utente & Verificare che l'utente non autenticato possa  \\
	TS2 & Autenticazione utente & Verificare che l'utente non autenticato possa  \\

	\hline

	\end{longtable}
\end{center}

\begin{center}
	\rowcolors{1}{white}{blue!20}
	\begin{longtable}{|p{1cm}|p{6.85cm}|p{7cm}|}
	\hline
	\rowcolor{lighter-grayer}
	\textbf{Codice} & \textbf{Titolo} & \textbf{Descrizione} \\
	\hline
	\endfirsthead

	% ----- MICHELE -----


	\hline
	TS2.1 & Apertura pagina carrello & Verificare che: (1) i prodotti inseriti precedentemente vengano visualizzati nella giusta quantità;
													   (2) il totale dell'ordine sia corretto (solo prodotti, senza tasse e spedizione);
													   (3) il totale dell'ordine con le tasse aggiunte sia corretto;
													   (4) si possa vedere il costo dei singoli prodotti;
													   (5) si visualizzi il messaggio di carrello vuoto nel caso in cui lo sia. \\
	TS2.2 & Modifica quantità prodotti nel carrello & Verificare che:
													   (1) si possa aumentare e diminuire la quantità di un singolo articolo nel carrello;
													   (2) in caso di aumento, se la quantità richiesta non è disponibile, venga visualizzato il giusto messaggio di errore;
													   (3) si possa eliminare un singolo articolo.  \\
	TS2.3 & Svuotare il carrello & Verificare che: (1) tutti gli elementi nel carrello possano essere rimossi. \\
	TS2.4 & Processo di checkout - inserimento indirizzi e calcolo spedizione & Verificare che:
												   (1) il messaggio mostri un errore se si avvia il checkout con carrello vuoto;
												   (2) si possano inserire indirizzi di fatturazione e spedizione da form e da quelli salvati (se autenticato);
												   (3) l'indirizzo di spedizione possa essere copiato con un click da quello scelto per la fatturazione;
												   (4) vengano aggiunti e visualizzati i costi di spedizione al totale dell'ordine, dopo aver scelto l'indirizzo di spedizione. \\
	TS2.5 & Processo di checkout - pagamento & Verificare che:
												   (1) il sistema di pagamento con il provider esterno possa essere utilizzato per concludere il checkout;
												   (2) nel caso di fallimento del pagamento, il sistema mostri in modo chiaro che il pagamento non è andato a buon fine e l'ordine non è stato emesso;
												   (3) nel caso di fallimento del pagamento, si possa riprovare immediatamente;
												   (4) in caso di successo del pagamento, si avvisa il cliente dell'avvenuto ordine, vengono scalate le merci acquistate dal magazzino e il venditore possa gestire l'ordine. \\
	TS2.6 & Login cliente & Verificare che: (1) il sistema di login cliente permetta di autenticarsi all'applicazione con un profilo già registrato, inserendo le corrette credenziali;
											(2) il cliente possa navigare tutto il sito mantenendo la sessione di login;
											(3) in caso di credenziali errate, venga mostrato il relativo messaggio di errore e si permetta di riprovare nell'immediato. \\
	TS2.7 & Registrazione cliente & Verificare che: (1) un cliente non autenticato possa avviare il processo di registrazione;
													(2) nel caso in cui il doppio inserimento dell'email fallisca (e.g. non corrisponde) si visualizzi il relativo messaggio d'errore;
													(3) nel caso in cui il doppio inserimento della password fallisca (e.g. non corrisponde) si visualizzi il relativo messaggio d'errore;
													(4) nel caso in cui la password non rispetti i requisiti minimi di sicurezza si visualizzi il relativo messaggio d'errore;
													(5) nel caso in cui esista già un utente con la stessa email si visualizzi il relativo messaggio d'errore;
													(6) nel caso in cui tutto vada a buon fine, arrivi l'email per la verifica dell'indirizzo inserito;
													(7) il click sul link di conferma ricevuto abiliti l'account appena registrato \\
	TS2.8 & Recupero password cliente & Verificare che:
													(1) durante il processo di login si possa avviare quello per il recupero della password;
													(2) si riceva il link per la reimpostazione della password al proprio indirizzo email, nel caso si richieda;
													(3) cliccando il link ricevuto si possa inserire la password con un doppio inserimento;
													(4) nel caso in cui i due inserimenti non corrispondano e/o la password non rispetti i requisiti di sicurezza si mostri il relativo messaggio d'errore;
													(5) nel caso in cui si inseriscano dei dati validi, mostrare il messaggio di conferma cambio password. \\
	TS2.9 & Amministrazione account cliente - gestione indirizzi & Verificare che:
													(1) si possa inserire un nuovo indirizzo e che questo sia utilizzabile durante il processo di checkout; 
													(2) nel caso in cui si inserisca un indirizzo non valido (e.g. campi obbligatori vuoti) si mostri un messaggio di errore e si permetta la modifica immediata dei dati inseriti;
													(3) si possa eliminare un indirizzo precedentemente inserito; \\
	TS2.10 & Amministrazione account cliente - assistenza & Verificare che:
													(1) il cliente possa aprire un ticket di assistenza e che questo sia gestibile dal venditore;
													(2) il cliente possa avviare la richiesta di cancellazione account tramite un apposito tipo di ticket di assistenza. \\
	TS2.11 & Amministrazione account cliente - modifica dati login & Verificare che:
													(1) il cliente possa cambiare indirizzo email;
													(2) il sistema si comporti per il cambio email come nel processo di registrazione;
													(3) il cliente possa cambiare password inserendo quella vecchia;
													(4) il sistema si comporti per il cambio password come nel processo di registrazione. \\
	\hline

	\end{longtable}
\end{center}

\begin{center}
	\rowcolors{1}{white}{blue!20}
	\begin{longtable}{|p{1cm}|p{6.85cm}|p{7cm}|}
	\hline
	\rowcolor{lighter-grayer}
	\textbf{Codice} & \textbf{Titolo} & \textbf{Descrizione} \\
	\hline
	\endfirsthead

	% ----- STEFANO -----


	\hline
	TS1 & Autenticazione utente & Verificare che l'utente non autenticato possa  \\
	TS2 & Autenticazione utente & Verificare che l'utente non autenticato possa  \\

	\hline

	\end{longtable}
\end{center}


\subsection{Tracciamento test di sistema - requisiti}

\begin{center}
	\rowcolors{1}{white}{blue!20}
	\begin{longtable}{|c|c|}
	\hline
	\rowcolor{lighter-grayer}
	\textbf{Test di sistema} & \textbf{Requisiti} \\
	\hline
	\endfirsthead

	% ----- MARCO -----

	
	\hline
	R1F1 & TSA1  \\
	REQ2 & TSA1 \\

	\hline

	\end{longtable}
\end{center}

\begin{center}
	\rowcolors{1}{white}{blue!20}
	\begin{longtable}{|c|c|}
	\hline
	\rowcolor{lighter-grayer}
	\textbf{Test di sistema} & \textbf{Requisiti} \\
	\hline
	\endfirsthead

	% ----- MICHELE -----

	
	\hline
	TS2.1 & R1F11  \\
	TS2.2 & R1F12 \\
	TS2.3 & R1F13 \\
	TS2.4 & R1F14 \\
	TS2.5 & R1F14 \\
	TS2.6 & R1F15 \\
	TS2.7 & R1F16 \\
	TS2.8 & R1F17 \\
	TS2.9 & R1F18 \\
	TS2.10 & R1F18 \\
	TS2.11 & R1F18 \\
	\hline

	\end{longtable}
\end{center}


\begin{center}
	\rowcolors{1}{white}{blue!20}
	\begin{longtable}{|c|c|}
	\hline
	\rowcolor{lighter-grayer}
	\textbf{Test di sistema} & \textbf{Requisiti} \\
	\hline
	\endfirsthead

	% ----- STEFANO -----

	
	\hline
	R1F1 & TSA1  \\
	REQ2 & TSA1 \\

	\hline

	\end{longtable}
\end{center}



