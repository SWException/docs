\section{Qualità del Prodotto}
Per prodotti si intende tutto ciò che è utilizzabile nell'ambito di questo progetto si intendono prodotti i documenti e il software.

\subsection{Documenti}
I documenti vengono rilasciati durante tutto lo sviluppo del progetto, sono ad utilizzo interno o esterno.

\subsubsection{Metriche}
\begin{itemize}
    \item \textbf{Indice di Gulpease:}
    \paragraph{Indice di Gulpease}
\begin{itemize}
\item \textbf{Descrizione:} questo indice, inventato nel 1988, ha come obiettivo descrivere la leggibilità di un testo in lingua italiana in maniera numerica, compresa tra 0 e 100, dove 0 è la leggibilità più bassa e 100 la leggibilità più alta. 
\item \textbf{Formula:} \[\textit{Indice Gulpease} = \frac{\textit{300 * (numero delle frasi) - 10 * (nuemro delle lettere)}}{\textit{(numero delle parole)}}\]
\item \textbf{Valori:}
    \begin{itemize}
        \item \textbf{Accettabile:} $40 \leq$ Indice Gulpease $\leq 10$
        \item \textbf{Otimale:} $80 \leq$ Indice Gulpease $\leq 100$
    \end{itemize}
\end{itemize}
    \item \textbf{Correttezza ortografica:}
\end{itemize}

\subsubsection{Obiettivi}
\begin{itemize}
    \item \textbf{Impaginazione del documento:} i contenuti devono essere inseriti nelle sezioni corrette secondo quanto indicato nell'indice del documento;
    \item \textbf{Correttezza ortografica:} devono essere individuati e corretti tutti gli errori ortografici;
\end{itemize}

\subsection{Software}
La verifica della qualità del software è effettuata mediante le metriche individuate di seguito che sono suddivise in:

\begin{itemize}
    \item \textbf{Metriche interne:} permettono di valutare il comportamento del software dal lato dello sviluppatore;
    \item \textbf{Metriche esterne:} permettono di valutare il comportamento del software dal punto di vista dell'utente.
\end{itemize}


\subsubsection{Metriche interne}
\label{_metricheQualitaCodice}
Per la valutazione qualitativa del codice prodotto si considerano le seguenti metriche interne:
\begin{itemize}
    \item complessità ciclomatica;
    \item complessità delle espressioni booleane;
    \item lunghezza delle righe di codice;
    \item code coverage.
\end{itemize}

\paragraph{Complessità ciclomatica}
\begin{itemize}
    \item \textbf{Codice:} MPD-ART1;
    \item \textbf{Attributo di riferimento:} valore ritornato dal processo automatizzato di analisi statica del codice, il quale includerà l'analisi della complessità ciclomatica;
    \item \textbf{Formula:} \[ v(G)=e-n+2\cdot p \]
    \item \textbf{Range di valori che può assumere:} $\{x \in \mathbb{N} \}$;
    \item \textbf{Range di valori accettabili:} per mantenere una buona qualità del software è necessario che il valore di questa metrica non superi
    10 per ogni modulo.
\end{itemize}

\paragraph{Complessità espressioni booleane}
\begin{itemize}
    \item \textbf{Codice:} MPD-ART2;
    \item \textbf{Attributo di riferimento:} valore ritornato dal processo automatizzato di analisi statica del codice, il quale includerà l'analisi della complessità delle espressioni booleane;
    \item \textbf{Formula:} si contano il numero di operatori logici presenti all'interno di una singola espressione;
    \item \textbf{Range di valori che può assumere:} $\{x \in \mathbb{N} \}$;
    \item \textbf{Range di valori accettabili:} per mantenere una buona qualità del software è necessario che il valore di questa metrica non superi
    3 per ogni espressione presente nel codice.
\end{itemize}

\paragraph{Lunghezza delle righe di codice}
\begin{itemize}
    \item \textbf{Codice:} MPD-ART3;
    \item \textbf{Attributo di riferimento:} valore ritornato dal processo automatizzato di analisi statica del codice, il quale includerà un report sulla lunghezza delle righe di codice
                                            (ovvero segnala quando una o più righe superano la lunghezza massima);
    \item \textbf{Formula:} si contano il numero di caratteri presenti su una singola riga di codice;
    \item \textbf{Range di valori che può assumere:} $\{x \in \mathbb{N}_0 \}$;
    \item \textbf{Range di valori accettabili:} per mantenere una buona qualità del software è necessario che il valore di questa metrica non superi
    80 per ogni riga di codice.
\end{itemize}

\paragraph{Code coverage}
\begin{itemize}
    \item \textbf{Codice:} MPD-ART4;
    \item \textbf{Attributo di riferimento:} valore ritornato dal processo automatizzato di analisi statica del codice, il quale includerà un report sulla lunghezza delle righe di codice
                                            (ovvero segnala quando una o più righe superano la lunghezza massima);
    \item \textbf{Formula:} si contano il numero di caratteri presenti su una singola riga di codice;
    \item \textbf{Range di valori che può assumere:} $\{x \in \mathbb{N}_0 \}$;
    \item \textbf{Range di valori accettabili:} per mantenere una buona qualità del software è necessario che il valore di questa metrica non superi
    80 per ogni riga di codice.
\end{itemize}


\subsubsection{Metriche esterne}

\paragraph{Affidabilità - Maturità dei test}
Per \textit{affidabilità} si intende la capacità del prodotto software di avere un adeguato livello di affidabilità quando opererà.

\begin{itemize}
    \item \textbf{Codice:} 
    \item \textbf{Descrizione:} Misura la percentuale di casi di test eseguiti con successo rispetto al numero totale previsto per garantire una copertura minima dei requisiti;
    \item \textbf{Attributo di riferimento:} Maturità;
    \item \textbf{Sigla:}
    \item \textbf{Formula:}
    \item \textbf{Range di valori che può assumere:}
\end{itemize}


\paragraph{Usabilità - Profondità strutturale dell'interfaccia}
Per \textit{usabilità} si intende la capacità del prodotto software di essere comprensibile, di poter essere usato facilmente in ogni sua parte da qualsiasi utente.

\begin{itemize}
    \item \textbf{Codice:} 
    \item \textbf{Descrizione:} Si valuta la profondità strutturale dell'interfaccia proposta all'utente ;
    \item \textbf{Attributo di riferimento:} Operabilità;
    \item \textbf{Sigla:}
    \item \textbf{Formula:}
    \item \textbf{Range di valori che può assumere:}
\end{itemize}