\section{Qualità del Prodotto}
Per prodotti si intende tutto ciò che è utilizzabile nell'ambito di questo progetto si intendono prodotti i documenti e il software.

\subsection{Prodotti}
\subsubsection{Documenti}
I documenti vengono rilasciati durante tutto lo sviluppo del progetto, sono ad utilizzo interno o esterno.

\subsubsection{Metriche}
\begin{itemize}
    \item \textbf{Indice di Gulpease:}
    \paragraph{Indice di Gulpease}
\begin{itemize}
\item \textbf{Descrizione:} questo indice, inventato nel 1988, ha come obiettivo descrivere la leggibilità di un testo in lingua italiana in maniera numerica, compresa tra 0 e 100, dove 0 è la leggibilità più bassa e 100 la leggibilità più alta. 
\item \textbf{Formula:} \[\textit{Indice Gulpease} = \frac{\textit{300 * (numero delle frasi) - 10 * (nuemro delle lettere)}}{\textit{(numero delle parole)}}\]
\item \textbf{Valori:}
    \begin{itemize}
        \item \textbf{Accettabile:} 40 < Indice Gulpease < 80
        \item \textbf{Otimale:} 80 < Indice Gulpease < 100
    \end{itemize}
\end{itemize}
    \item \textbf{Correttezza ortografica:}
\end{itemize}

\subsubsection{Obiettivi}
\begin{itemize}
    \item \textbf{Impaginazione del documento:} i contenuti devono essere inseriti nelle sezioni corrette secondo quanto indicato nell'indice del documento;
    \item \textbf{Correttezza ortografica:} devono essere individuati e corretti tutti gli errori ortografici;
\end{itemize}

\subsection{Software}
La verifica della qualità del software è effettuata mediante le metriche individuate di seguito che sono suddivise in:

\begin{itemize}
    \item \textbf{Metriche interne:} permettono di valutare il comportamento del software dal lato dello sviluppatore;
    \item \textbf{Metriche esterne:} permettono di valutare il comportamento del software dal punto di vista dell'utente.
\end{itemize}


\subsubsection{Metriche interne}
\paragraph{Funzionalità - Aderenza agli Standard}
Per \textit{Funzionalità} del prodotto software si intende la capacità di soddisfare i requisiti funzionali richiesti e le necessità degli utenti.

\begin{itemize}
    \item \textbf{Codice:} 
    \item \textbf{Descrizione:} Misura il livello di aderenza agli standard delle funzioni ed interfacce sviluppate;
    \item \textbf{Attributo di riferimento:} Aderenza alle funzionalità;
    \item \textbf{Sigla:}
    \item \textbf{Formula:}
    \item \textbf{Range di valori che può assumere:}
\end{itemize}


\paragraph{Affidabilità - Rilevamento dei difetti}
Per \textit{Affidabilità} si intende la capacità di predire se il prodotto software potrà soddisfare i requisti prescritti per l'affidabilità.

\begin{itemize}
    \item \textbf{Codice:} 
    \item \textbf{Descrizione:} Misura in percentuale l'efficacia nel rilevare i difetti presenti nel software durante la fase di sviluppo;
    \item \textbf{Attributo di riferimento:} Maturità;
    \item \textbf{Sigla:}
    \item \textbf{Formula:}
    \item \textbf{Range di valori che può assumere:}
\end{itemize}


\paragraph{Usabilità - Validità dei dati d'input}
Per \textit{usabilità} si iintende la capacità del prodotto sftware di essere comprensibile, di poter essere usato facilmente in ogni sua parte da qualsiasi utente.

\begin{itemize}
    \item \textbf{Codice:} 
    \item \textbf{Descrizione:} Misura in percentuale la correttezza dei dati forniti in input all'applicazione;
    \item \textbf{Attributo di riferimento:} Operabilità;
    \item \textbf{Sigla:}
    \item \textbf{Formula:}
    \item \textbf{Range di valori che può assumere:}
\end{itemize}




\subsubsection{Metriche esterne}

\paragraph{Affidabilità - Maturità dei test}
Per \textit{affidabilità} si intende la capacità del prodotto software di avere un adeguato livello di affidabilità quando opererà.

\begin{itemize}
    \item \textbf{Codice:} 
    \item \textbf{Descrizione:} Misura la percentuale di casi di test eseguiti con successo rispetto al numero totale previsto per garantire una copertura minima dei requisiti;
    \item \textbf{Attributo di riferimento:} Maturità;
    \item \textbf{Sigla:}
    \item \textbf{Formula:}
    \item \textbf{Range di valori che può assumere:}
\end{itemize}


\paragraph{Usabilità - Profondità strutturale dell'interfaccia}
Per \textit{usabilità} si iintende la capacità del prodotto sftware di essere comprensibile, di poter essere usato facilmente in ogni sua parte da qualsiasi utente.

\begin{itemize}
    \item \textbf{Codice:} 
    \item \textbf{Descrizione:} Si valuta la profondità strutturale dell'interfaccia proposta all'utente ;
    \item \textbf{Attributo di riferimento:} Operabilità;
    \item \textbf{Sigla:}
    \item \textbf{Formula:}
    \item \textbf{Range di valori che può assumere:}
\end{itemize}