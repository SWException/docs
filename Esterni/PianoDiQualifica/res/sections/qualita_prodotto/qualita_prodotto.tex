\section{Qualità del Prodotto} \label{_qualitaProdotto}
Per prodotti, nell'ambito di questo progetto, si intendono la documentazione e il software.

\subsection{Documenti}\label{_documenti}
I documenti vengono rilasciati durante tutto lo sviluppo del progetto e sono ad utilizzo interno o esterno.

\subsubsection{Metriche}

\paragraph{MPD-D1: \glock{Indice di Gulpease} (GULP)}
\begin{itemize}
    \item \textbf{Formula:} \[\textit{Indice Gulpease} = \frac{\textit{300 * (numero delle frasi) - 10 * (numero delle lettere)}}{\textit{(numero delle parole)}}\]
    \item \textbf{Valori:}
          \begin{itemize}
              \item \textbf{Accettabile:} $40 \leq Indice Gulpease \leq 100$
              \item \textbf{Ottimale:} $80 \leq Indice Gulpease \leq 100$
          \end{itemize}
\end{itemize}

\paragraph{MPD-D2: Correttezza ortografica (CORT)}
\begin{itemize}
    \item  \textbf{Range di valori accettabili:} 0
\end{itemize}

\subsubsection{Obiettivi}
\begin{itemize}
    \item \textbf{Impaginazione del documento:} i contenuti devono essere inseriti nelle sezioni corrette secondo quanto indicato nell'indice del documento;
    \item \textbf{Correttezza ortografica:} devono essere individuati e corretti tutti gli errori ortografici.
\end{itemize}

\subsection{Software} \label{_metricheQualitaCodice}
La verifica della qualità del software è effettuata mediante le metriche individuate di seguito che sono suddivise in:
\begin{itemize}
    \item \textbf{Metriche interne:} permettono di valutare il comportamento del software dal lato dello sviluppatore;
    \item \textbf{Metriche esterne:} permettono di valutare il comportamento del software dal punto di vista dell'utente.
\end{itemize}

\subsubsection{Metriche interne}
Per la valutazione qualitativa del codice prodotto si considerano le seguenti metriche interne:
\begin{itemize}
    \item \glock{complessità ciclomatica};
    \item complessità delle espressioni booleane;
    \item lunghezza delle righe di codice;
    \item \glock{code coverage}.
\end{itemize}

\paragraph{MPD-S1: Complessità ciclomatica (COCI)}
\begin{itemize}
    \item \textbf{Formula:} \[ v(G)=e-n+2\cdot p \]
    \item \textbf{Range di valori accettabili:} per mantenere una buona qualità del software è necessario che il valore di questa metrica non superi
          10 per ogni modulo.
\end{itemize}

\paragraph{MPD-S2: Complessità espressioni booleane (COEB)}
\begin{itemize}
    \item \textbf{Formula:} si contano il numero di operatori logici presenti all'interno di una singola espressione;
    \item \textbf{Range di valori che può assumere:} $\{x \in \mathbb{N} \}$;
    \item \textbf{Range di valori accettabili:} per mantenere una buona qualità del software è necessario che il valore di questa metrica non superi
          3 per ogni espressione presente nel codice.
\end{itemize}

\paragraph{MPD-S3: Lunghezza delle righe di codice (LRC)}
\begin{itemize}
    \item \textbf{Range di valori accettabili:} per mantenere una buona qualità del software è necessario che il valore di questa metrica non superi
          80 per ogni riga di codice.
\end{itemize}

\paragraph{MPD-S4: Code coverage (CODCO}
\begin{itemize}
    \item \textbf{Formula:} $\frac{righe \ coperte}{righe \ totali} \cdot 100$;
    \item \textbf{Range di valori accettabili:} si punta ad una code coverage di almeno $80\%$;
    \item \textbf{Range di valori ottimali:} l'ideale sarebbe una coverage del $100\%$.
\end{itemize}

\subsubsection{Metriche esterne}

\paragraph{MPD-S5: Maturità dei test (MATE)}
\begin{itemize}
    \item \textbf{Range di valori accettabili:}
          \begin{itemize}
              \item \textbf{Accettabile:} 100%
          \end{itemize}
\end{itemize}

\paragraph{MPD-S6: Profondità strutturale dell'interfaccia (PSI)}
\begin{itemize}
    \item \textbf{Range di valori accettabili:}
          \begin{itemize}
              \item \textbf{Accettabile:} $1 \leq PSI \leq 5$
              \item \textbf{Ottimale:} $1 \leq PSI \leq 3$
          \end{itemize}
\end{itemize}

