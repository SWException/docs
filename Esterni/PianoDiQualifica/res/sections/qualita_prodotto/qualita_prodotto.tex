\section{Qualità del Prodotto}
ISO/IEC 9126
Mi raccomando leggi pagina 8

\subsection{Funzionalità}
\subsubsection{Obiettivo}
\subsubsection{Metriche}
\subsection{Affidabilità}
\subsubsection{Obiettivo}
\subsubsection{Metriche}
\subsection{Usabilità}
L'usabilità rappresenta la capacità di un prodotto di essere comprensibile, facile da usare e da comprendere, di risulatre attraente da parte di un utente sotto determinate condizioni.
\subsubsection{Obiettivo}
\subsubsection{Metriche}
\paragraph{Indice di Gulpease}
\begin{itemize}
\item \textbf{Descrizione:} questo indice, inventato nel 1988, ha come obiettivo descrivere la leggibilità di un testo in lingua italiana in maniera numerica, compresa tra 0 e 100, dove 0 è la leggibilità più bassa e 100 la leggibilità più alta. 
\item \textbf{Formula:} \[\textit{Indice Gulpease} = \frac{\textit{300 * (numero delle frasi) - 10 * (nuemro delle lettere)}}{\textit{(numero delle parole)}}\]
\item \textbf{Valori:}
    \begin{itemize}
        \item \textbf{Accettabile:} 40 < Indice Gulpease < 80
        \item \textbf{Otimale:} 80 < Indice Gulpease < 100
    \end{itemize}
\end{itemize}
\subsection{Efficienza}
\subsubsection{Obiettivo}
\subsubsection{Metriche}
\subsection{Manutenibilità}
\subsubsection{Obiettivo}
\subsubsection{Metriche}
\subsection{Portabilità}
\subsubsection{Obiettivo}
\subsubsection{Metriche}
\subsection{Qualità in uso} ?? Da fare?


