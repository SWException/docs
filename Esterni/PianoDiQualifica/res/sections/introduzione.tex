\section{Introduzione} \label{_introduzione}
\subsection{Scopo del documento}
Lo scopo del documento è quello di descrivere i metodi usati dal gruppo SWException nei processi di verifica e validazione al fine di garantire la qualità preposta sia dei prodotti che dei processi.

\subsection{Riferimenti a glossario e documenti esterni}
Per evitare il più possibile ogni tipo di ambiguità e redarre documenti nel modo più chiaro possibile il gruppo ha creato un Glossario. Al suo interno vi sono tutti i termini presenti in tutti i documenti che potrebbero essere oggetto di ambiguità o non chiarezza. I termini la cui definizione è riportata nel glossario sono seguiti dalla lettera \textit{G} a pedice.\\
I riferimenti ad altri documenti sono segnalati dalla lettera \textit{D} a pedice che segue il nome e la versione del documento riferito.
\subsection{Standard utilizzati}
Per la stesura di questo documento si è fatto ampio uso degli standard \textbf{ISO/IEC 12207:1995} e \textbf{ISO/IEC 9126}. Il primo per quanto riguarda il ciclo di vita del software mentre il secondo per la qualità del prodotto software.
\subsection{Riferimenti}
\subsubsection{Riferimenti normativi}
\begin{itemize}
    \item \dext{NormeDiProgetto\_1.0.0}
\end{itemize}
\subsubsection{Riferimenti informativi}
\begin{itemize}
    \item Qualità del Software: \url{https://www.math.unipd.it/~tullio/IS-1/2020/Dispense/L12.pdf}
    \item Qualità del Processo: \url{https://www.math.unipd.it/~tullio/IS-1/2020/Dispense/L13.pdf}
    \item ISO/IEC 12207:1995: 
    \item Indice Gulpease: \url{https://it.wikipedia.org/wiki/Indice_Gulpease}
\end{itemize}
